\documentclass{article}

    \usepackage{xcolor}
    \definecolor{pf}{rgb}{0.4,0.6,0.4}
    \usepackage[top=1in,bottom=1in, left=0.8in, right=0.8in]{geometry}
    \usepackage{setspace}
    \setstretch{1.2} 
    \setlength{\parindent}{0em}

    \usepackage{paralist}
    \usepackage{cancel}

    % \usepackage{ctex}
    \usepackage{amssymb}
    \usepackage{amsmath}

    \usepackage{tcolorbox}
    \definecolor{Df}{RGB}{0, 184, 148}
    \definecolor{Th}{RGB}{9, 132, 227}
    \definecolor{Rmk}{RGB}{215, 215, 219}
    \newtcolorbox{Df}[2][]{colbacktitle=Df, colback=white, title={\large\color{white}#2},fonttitle=\bfseries,#1}
    \newtcolorbox{Th}[2][]{colbacktitle=Th, colback=white, title={\large\color{white}#2},fonttitle=\bfseries,#1}
    \newtcolorbox{Rmk}[2][]{colbacktitle=Rmk, colback=white, title={\large\color{black}{Remarks}},fonttitle=\bfseries,#1}

    \title{\LARGE \textbf{Limit of Number Sequence}}
    \author{\large Jiawei Hu}

\begin{document}
\maketitle

This is the 1st chapter of Mathematical Analysis, which is about \textbf{the limit of number sequence.} In this course (Mathematical Analysis), only the basic analitic properties and methods in $\mathbb{R}$ are included, and those about complex number are responsible for the course ``3. Complex Analysis''.\\
Here it is necessary to claim a ``definition (Df) -> theorem (Th)'' working cycle, which acts as the writing style throughout this whole course. This working cycle is shown bellow:

\noindent\rule{\textwidth}{2pt}
\begin{Df}{Some Definition}
    The text of this definition.
\end{Df}

\begin{Rmk}{}
    The text of the remarks about the definition just proposed (possibly including what it means and what it is for).\\
    \textcolor{Df}{Some remarks with some incidental definitions.}\\
    \textcolor{Th}{Some remarks with some incidental theorems.}
\end{Rmk}

\begin{Th}{Some Theorem}
    The text of this theorem.
    \tcblower
    \textit{Pf}: The proof of this theorem (is possibly "todo" when the author cannot complete it yet).
\end{Th}

\begin{Rmk}{}
    The text of the remarks about the definition just proposed (possibly including what it means and what it is for).\\
    \textcolor{Df}{Some remarks with some incidental definitions.}\\
    \textcolor{Th}{Some remarks with some incidental theorems.}
\end{Rmk}
\noindent\rule{\textwidth}{2pt}
As for the text of both a definition or a theorem, a common fixed pattern of sentences is adopted, which is ``Suppose \dots (some pre-conditions or background information). Then \dots (the direct text for the definition or the theorem).''. Please identify this pattern later by yourself. 

By the way, we now pre-claim some commonly-used notations:
\begin{compactenum}
    \item $\mathbb{C}$: the set of the complex numbers;
    \item $\mathbb{R}$: the set of the real numbers;
    \item $\mathbb{Q}$: the set of the rational numbers;
    \item $\mathbb{Z}$: the set of the integers;
    \item $\mathbb{N}$: the set of the natural numbers;
    \item $\mathbb{N^\ast}$: the set of the positive integers.
    \item $\sideset{^R}{}{\mathop{D}}$: the set of all functions from $D$ to $R$ (with domain $D$ and range in $R$).
    \item An agreement for the length of a list: if we write $a_1, \dots, a_n$, then we indicate that $n$ is finite and that $n\geq 1$; if we write $a_0, \dots, a_n$, then we indicate that $n$ is finite and that $n\geq 0$.
    \item $\mathbb{R}_\infty = \mathbb{R}\cup\{\infty, -\infty\}$: the set of the extended real numbers.
\end{compactenum} 

Here is the \textbf{Quick Search} for this chapter:
\begin{Th}{Quick Search}
    \begin{compactdesc}
        \item (1.1.*): The limit of a number sequence (finite or infinite limit).
        \item (1.2.*, 1.3.*): The monotonicity, the monotonicity-boundedness theorem.
        \item (1.4.*): The theorem of nested closed intervals.
        \item (1.5.*): The Bolzano-Weierstrass theorem.
        \item (1.6.*, 1.7.*): Cauchy sequence and the Cauchy criterion.
        \item (1.8.*, 1.9.*): Supremum, infimum, and the supremum-infimum principle.
        \item (1.10.*): The finite covering theorem (Heine-Borel theorem).
        \item (1.11.*): Limit superior and limit inferior.
    \end{compactdesc}
\end{Th}

Then with everything prepared, here we go.

\begin{Df}{$\bullet$ Df1.1 (the limit of a number sequence)}
    Suppose $\{a_n: n\in\mathbb{N}\}$ is a sequence in $\mathbb{R}$ and $a\in\mathbb{R}$. Then $a$ is called the limit of the sequence $\{a_n\}$, denoted by $\lim\limits_{n\to\infty}a_n=a$, if: 
    $$\forall \varepsilon>0, \exists N\in\mathbb{N}, \forall n > N, |a_n-a|<\varepsilon.$$
\end{Df}

\begin{Rmk}{}
    \begin{compactenum}
        \item \textcolor{Df}{$\lim\limits_{n\to\infty}a_n=a$ is also called that the sequence $\{a_n\}$ converges to $a$. If no $a\in\mathbb{R}$ s.t. $\lim\limits_{n\to\infty}a_n=a$ exists, then we say that the limit of the sequence does not exist, or that the sequence diverges.}
        \item \textcolor{Th}{(Uniqueness) Easy to see that the limit of a sequence (if exists) is unique.}
        \item \textcolor{Th}{(Boundedness) If $\{a_n\}$ converges to some real number, then there is some $M>0$ s.t. $|a_n|\leq M$ for all $n\in\mathbb{N}$.}
        \item \textcolor{Th}{(Preservation of inequality) If $\lim\limits_{n\to\infty}a_n=a$ and $\lim\limits_{n\to\infty}b_n=b$ and $a_n\leq b_n$ for all $n\in\mathbb{N}$, then $a\leq b$.}
        \item \textcolor{Th}{(Subsequence) If $\{a_n\}$ converges to $a$, then every subsequence of $\{a_n\}$ also converges to $a$.}
        \item \textcolor{Th}{(Arithmetic) If $\{a_n\}$ and $\{b_n\}$ both converges in $\mathbb{R}$, then:
            \begin{compactitem}
                \item $\lim\limits_{n\to\infty}(a_n\pm b_n)=\lim\limits_{n\to\infty}a_n\pm \lim\limits_{n\to\infty}b_n$; 
                \item $\lim\limits_{n\to\infty}(a_n\cdot b_n)=\lim\limits_{n\to\infty}a_n\cdot \lim\limits_{n\to\infty}b_n$, 
                \item If $\lim\limits_{n\to\infty} b_n\neq 0$, then $\lim\limits_{n\to\infty}\frac{a_n}{b_n}=\frac{a}{b}$ (if $b\neq 0$).
            \end{compactitem}
        }
    \end{compactenum}
\end{Rmk}

\begin{Th}{Ex1.1.0.1 (the convergence of $\frac{1}{n}\sum_{k=1}^{n} a_k$)}
    Prove that 
    $$\lim\limits_{n\to\infty}\frac{1}{n}\sum_{k=1}^{n} a_k = a$$
    if $\lim\limits_{n\to\infty}a_n=a$.
    \tcblower
    \textit{Solution}: See the reference book.
\end{Th}

\begin{Th}{Ex1.1.0.2 (the convergence of $\frac{1}{n}\sum_{k=1}^{n} a_k b_{n+1-k}$)}
    Prove that 
    $$\lim\limits_{n\to\infty}\frac{1}{n}\sum_{k=1}^{n} a_k b_{n+1-k} = a b$$
    if $\lim\limits_{n\to\infty}a_n=a$ and $\lim\limits_{n\to\infty}b_n=b$.
    \tcblower
    \textit{Solution}: See the reference book.
\end{Th}

\begin{Df}{$\circ$ Df1.1.1 (infinite small)}
    A sequence $\{a_n\}$ in $\mathbb{R}$ is called an infinite small sequence if $\lim\limits_{n\to\infty}a_n=0$.
\end{Df}

\begin{Df}{$\circ$ Df1.1.2 (infinite large)}
    \begin{compactenum}
        \item A sequence $\{a_n\}$ in $\mathbb{R}$ is called a positive (resp. negative) infinite large sequence, denoted by $\lim\limits_{n\to\infty}a_n=\infty$ (resp. $\lim\limits_{n\to\infty}a_n=-\infty$), if:
        $$\forall M>0, \exists N\in\mathbb{N}, \forall n > N, a_n>M \;(\text{resp. } a_n<-M)$$
        \item A sequence $\{a_n\}$ in $\mathbb{R}$ is called an infinite large sequence, denoted by $\lim\limits_{n\to\infty}a_n=\pm\infty$, if $\{|a_n|\}$ is a positive infinite large sequence.
    \end{compactenum}
\end{Df}

\begin{Rmk}{}
    For a positive (resp. negative) infinite large sequence, we say that it converges to $\infty$ (resp. $-\infty$). But we still say that a positive (resp. negative) infinite large sequence has no limit.
\end{Rmk}

\begin{Th}{$\bullet$ Ex1.1.2.1}
    Some exercises:
    \begin{compactenum}
        \item Prove that $\lim\limits_{n\to\infty} (n!)^{1/n} = \infty$.
    \end{compactenum}
    \tcblower
    \textit{Solution}: 
    \begin{compactenum}
        \item Hint: use the inequality $n!\geq (n/2)^{n/2}$.
    \end{compactenum}
\end{Th}

\begin{Df}{$\bullet$ Df1.2 (the monotonicity of a sequence)}
    Suppose $\{a_n\}$ is a sequence in $\mathbb{R}$. Then $\{a_n\}$ is called:
    \begin{compactenum}
        \item increasing if $a_n\leq a_{n+1}$ for all $n\in\mathbb{N}$;
        \item strictly increasing if $a_n<a_{n+1}$ for all $n\in\mathbb{N}$;
        \item decreasing if $a_n\geq a_{n+1}$ for all $n\in\mathbb{N}$;
        \item strictly decreasing if $a_n>a_{n+1}$ for all $n\in\mathbb{N}$.
        \item monotonic if it is either increasing or decreasing.
    \end{compactenum}
\end{Df}

\begin{Th}{$\bullet$ Th1.2.1 (properties of infinite large sequences)}
    Suppose $\{a_n\}$ is a sequence in $\mathbb{R}$. Then:
    \begin{compactenum}
        \item If $\{a_n\}$ is a infinite large sequence, then $\{a_n\}$ is unbounded; but the converse is not true;
        \item $\{a_n\}$ has no upper bound (resp. $\{a_n\}$ has no lower bound) (resp. $\{a_n\}$ is unbounded) iff there is a positive infinite large (resp. negative infinite large) (resp. infinite large) subsequence of $\{a_n\}$;
        \item $\{a_n\}$ is a infinite large (resp. positive infinite large) (resp. negative infinite large) sequence iff every subsequence of $\{a_n\}$ is a infinite large (resp. positive infinite large) (resp. negative infinite large) sequence;
        \item Suppose $\{b_n\}$ is also a sequence in $\mathbb{R}$. If $\{a_n\}$, $\{b_n\}$ are both positive infinite large sequences, then $\{a_n+b_n\}$ and $\{a_n\cdot b_n\}$ are also positive infinite large sequences.
        \item $\{a_n\}$ has some subsequence that converges to some $a\in\mathbb{R}_\infty$.
        \item If $\{a_n\}$ is monotonic, then $\{a_n\}$ converges to some $a\in\mathbb{R}_\infty$.
        \item (preserve the inequality) Suppose $\{b_n\}$ is also a sequence in $\mathbb{R}$. If $\lim\limits_{n\to\infty}a_n=a\in\mathbb{R}_\infty$ and $\lim\limits_{n\to\infty}b_n=b\in\mathbb{R}_\infty$, then $a\leq b$ if $a_n\leq b_n$ for all $n\in\mathbb{N}$.
    \end{compactenum}
    \tcblower
    \textit{Pf}: 
    \begin{compactenum}
        \item Obvious.
        \item Only prove the ``no upper bound'' part and then the other two parts are similar. The ``if'' is obvious. For the ``only if'', we can first choose a term $a_{k_1}$ s.t. $a_{k_1}>1$. If a sequence has no upper bound, then it still has no upper bound after removing finitely many terms. Then we can keep the terms of $\{a_n\}$ after $a_{k_1}$ and choose a term $a_{k_2}$ s.t. $a_{k_2}>a_{k_1}+1$. So forth, we can choose a positive infinite large subsequence $\{a_{k_n}\}$.
        \item Obvious.
        \item Obvious.
        \item Obvious.
        \item Obvious.
        \item Obvious.
    \end{compactenum}
\end{Th}

\begin{Th}{$\bullet$ Th1.3 (theorem of monotonicity and boundness)}
    Suppose $\{a_n\}$ is a monotonic and bounded sequence in $\mathbb{R}$. Then $\{a_n\}$ converges.
    \tcblower
    \textit{Pf}: Let us say $\{a_n\}$ that is increasing and write the terms in $\{a_n\}$ in decimals:
    $$
    \begin{aligned}
        a_1&=x_{10}.x_{11}x_{12}x_{13}\dots,\\
        a_2&=x_{20}.x_{21}x_{22}x_{23}\dots,\\
        &\dots\\
        a_n&=x_{n0}.x_{n1}x_{n2}x_{n3}\dots,\\
        &\dots
    \end{aligned}
    $$
    Since monotonic $\{a_n\}$ is bounded, we can see that $\{x_{n0}\}$ is a bounded sequence in $\mathbb{N}$, and thus it will finally reach a maximum $x_0\in\mathbb{N}$. Then we can see that $\{x_{n1}\}$ is a bounded sequence in $\mathbb{N}$, and thus it will finally reach a maximum $x_1\in\mathbb{N}$. So forth, we can find a number $x = x_0.x_1x_2x_3\dots$, which can be verified as the limit of $\{a_n\}$.
\end{Th}

\begin{Th}{$\bullet$ Ex1.3.1 ($\sum_{n=1}^{\infty} \frac{1}{n^\alpha}$)}
    Suppose $\alpha>0$ and $a_n = \frac{1}{n^\alpha}$. Let $S_n = \sum_{k=1}^n a_k$. Then does $\{S_n\}$ converge?
    \tcblower
    \textit{Solution}: The conclusion is that $\{S_n\}$ converges iff $\alpha>1$. The proof can be seen in the reference book (see the \verb|exercises.md| file for the reference information).
\end{Th}

\begin{Th}{$\bullet$ Blocks from the file P1}
\end{Th}

\begin{Th}{$\bullet$ Th1.4 (theorem of nested closed intervals)}
    Suppose $\{[a_n, b_n]: n\in\mathbb{N}\}$ is a sequence of closed intervals in $\mathbb{R}$ and $[a_{n+1}, b_{n+1}]\subseteq [a_n, b_n]$ for all $n\in\mathbb{N}$. If the length of the intervals $\{b_n-a_n: n\in\mathbb{N}\}$ converges to $0$, then there is some $x\in\mathbb{R}$ s.t. $x\in[a_n, b_n]$ for all $n\in\mathbb{N}$.
    \tcblower
    \textit{Pf}: From the theorem of monotonicity and boundness, we can see that $\{a_n\}$ and $\{b_n\}$ both converges to some $a$. Then we can see that $a\in[a_n, b_n]$ for all $n\in\mathbb{N}$.
\end{Th}

\begin{Th}{$\circ$ Th1.5.-1}
    In $\mathbb{R}$, any sequence has a monotonic subsequence.
    \tcblower
    \textit{Pf}: Consider some terms named ``pilot'' in a sequence $\{a_n\}$, each of which is defined as a term that is greater than all other terms after. Then:
    \begin{compactitem}
        \item If there are finitely many pilot terms in $\{a_n\}$, then after the last pilot, we can choose an increasing subsequence since every term there has a (not strictly) greater term somewhere after.
        \item If there are infinitely many pilot terms, then the pilots themselves form an decreasing subsequence.
    \end{compactitem}
\end{Th}

\begin{Rmk}{}
    This is a lemma for the proof of the Bolzano-Weierstrass theorem.
\end{Rmk}

\begin{Th}{$\bullet$ Th1.5 (Bolzano-Weierstrass theorem)}
    In $\mathbb{R}$, any bounded sequence has a convergent subsequence.
    \tcblower
    \textit{Pf}: From Th1.5.-1, we can see that any sequence has a monotonic subsequence. Then from Th1.3, we can see that this subsequence converges.
\end{Th}

\begin{Df}{$\bullet$ Df1.6 (Cauchy sequence)}
    Suppose $\{a_n\}$ is a sequence in $\mathbb{R}$. Then $\{a_n\}$ is called a Cauchy sequence if:
    $$\forall \varepsilon>0, \exists N\in\mathbb{N}, \forall m, n > N, |a_m-a_n|<\varepsilon.$$
\end{Df}

\begin{Th}{$\bullet$ Th1.7 (Cauchy criterion)}
    In $\mathbb{R}$, a convergent sequence is equivalent to a Cauchy sequence.
    \tcblower
    \textit{Pf}: 
    \begin{compactenum}
        \item (Convergent $\Rightarrow$ Cauchy) Obvious.
        \item (Cauchy $\Rightarrow$ Convergent) Let $\{a_n\}$ is a Cauchy sequence. Hence $\{a_n\}$ is obviously bounded. Then from the Bolzano-Weierstrass theorem, we choose a convergent subsequence $a_{k_n} \rightarrow a$ as $n\to \infty$. Now we want to show that $\{a_n\}$ converges to $a$. To limit the terms of $\{a_n\}$ arbitrarily close to $a$, we first limit the terms of $\{a_{k_n}\}$ and then restrict the remaining terms near around using the property of Cauchy sequence.
    \end{compactenum}
\end{Th}

\begin{Df}{$\bullet$ Df1.8 (upper bound, lower bound, supremum, infimum)}
    Suppose $A\subseteq\mathbb{R}_\infty$ (\textcolor{Df}{$\mathbb{R}_\infty\triangleq\mathbb{R}\cup\{\infty, -\infty\}$}). Then:
    \begin{compactenum}
        \item An element $u\in\mathbb{R}_\infty$ (resp. $l\in\mathbb{R}_\infty$) is called an upper bound (resp. a lower bound) of $A$ if $a\leq u$ (resp. $a\geq l$) for all $a\in A$.
        \item An element in $\mathbb{R}_\infty$ is called the supremum (resp. infimum) of $A$, denoted by $\sup A$ (resp. $\inf A$), if it is the least upper bound (resp. greatest lower bound) of $A$.
    \end{compactenum}
\end{Df}

\begin{Rmk}{}
    From this definition we see that \textcolor{Th}{for any subset of $\mathbb{R}_\infty$, the supremum and infimum are both unique (if exist).}
\end{Rmk}

\begin{Th}{$\bullet$ Th1.9 (supremum and infimum principle)}
    Suppose $A\subseteq\mathbb{R}_{\infty}$. Then:
    \begin{compactenum}
        \item If $A$ has a finite upper bound (resp. lower bound), then $A$ has a supremum (resp. infimum).
        \item If $A$ has no finite upper bound (resp. no finite lower bound), then $\sup A=\infty$ (resp. $\inf A=-\infty$).
    \end{compactenum}
    \tcblower
    \textit{Pf}: The second statement (the one about no finite bound) is obvious.\\
    Let us say that $A$ has a finite upper bound. To prove the first statement is just to find the supremum of $A$ from the known upper bound $a_0$. Then we can construct a sequence $\{a_n\}$ in the following process:
        \begin{compactenum}
            \item Decrease $\beta_0$ by $10^0$ unless $\beta_{k1}$ is not an upper bound of $A$;
            \item Increase $\beta_{k1}$ by $10^{-1}$ unless $\beta_{k2}$ is an upper bound of $A$;
            \item Decrease $\beta_{k2}$ by $10^{-2}$ unless $\beta_{k3}$ is not an upper bound of $A$;
            \item Increase $\beta_{k3}$ by $10^{-3}$ unless $\beta_{k4}$ is an upper bound of $A$;
            \item $\dots$
        \end{compactenum}
    Then we can easily verify that $\{\beta_n\}$ is Cauchy, and thus converges to a finite number $\beta$, which can then be verified as the supremum of $A$. Actually speaking, we can divide $\{\beta_n\}$ into a ``upper bound subsequence'' and a ``not upper bound subsequence'' according to the process above to show the ``upper bound'' and ``least upper bound'' implication of supremum.
\end{Th}

\begin{Rmk}{}
    We can see that \textcolor{Th}{every subset of $\mathbb{R}_\infty$ has a supremum and an infimum} since it is bounded by $\infty$ and $-\infty$ anyway. And obviously, \textcolor{Th}{the supremum is always greater than or equal to the infimum (except $\varnothing$, with $\sup\varnothing = -\infty$ and $\inf\varnothing = \infty$).}
\end{Rmk}

\begin{Th}{$\bullet$ Th1.10 (finite covering theorem) (Heine-Borel theorem)}
    Suppose $[a,b]\subseteq\mathbb{R}$ is a finite closed interval. If there is a collection $\mathcal{I} = \{I_\lambda: \lambda\in\Lambda\}$ of open intervals (``open interval'' refers to the interval of the form $(a,b)$, $(-\infty, b)$, $(a, \infty)$, or $(-\infty, \infty)$) s.t. $[a,b]\subseteq\bigcup\limits_{\lambda\in\Lambda}I_\lambda$, then there exist finitely many intervals $I_{\lambda_1}, \dots, I_{\lambda_n}$ in $\mathcal{I}$ s.t. $[a,b]\subseteq\bigcup\limits_{i=1}^nI_{\lambda_i}$.
    \tcblower
    \textit{Pf}: We try to prove it with the supremum and infimum principle. Since we try to fill $[a,b]$ with finitely many open intervals, a natural idea is to fill it from $a$ to $b$. \\
    Let $S = \{a<x\leq b: [a,x] \text{ can be covered by finitely many } I_\lambda\}$. First $S$ is obviously not empty. Then we can prove that $\sup S = b$ so that the ``finite coverage'' property can be extended to $b$.
    \begin{compactenum}
        \item $\sup S = b$: If $\sup S < b$, then find an open interval $(\alpha, \beta)$ in $\mathcal{I}$ s.t. $\sup S\in (\alpha, \beta)$. Now take $x_1, x_2$ s.t. $\alpha<x_1<\sup S<x_2<\beta$ and s.t. $x_1\in S$. Then $[a, x_1]$ can be finitely covered by some sub-collection $\pmb{I}$ of $\mathcal{I}$. Hence $[a, x_2]$ can be finitely covered by $\pmb{I}\cup\{(\alpha, \beta)\}$ and thus $x_2\in S$, which contradicts with that $\sup S<x_2$. Hence $\sup S = b$.
        \item $[a,b]$ can be finitely covered: Since $\sup S = b$, any proper sub-interval $[a,x]$ of $[a,b]$ can be finitely covered. Hence we can first cover $b$ with some interval $(\alpha, \beta)$, and then cover $[a, \alpha]$ with finitely many intervals. 
    \end{compactenum}
\end{Th}

\begin{Rmk}{}
    Actually we will propose a more general version of this theorem in the futural discussion about the topology of $\mathbb{R}^n$. There we will define the ``open sets'' and the ``closed sets'', and called a collection of open sets an ``open covering'' of a set $E\in\mathbb{R}^n$ if $E$ is contained in the union of the collection. Since in $\mathbb{R}$, an open (resp. closed) interval is a kind of open set (resp. closed set), and there is no significant topological difference between the open intervals and the open sets, we just proposed this theorem in a more understandable way.\\
    Generally, the ``finite covering theorem'' can be summerized as: \textcolor{Th}{In $\mathbb{R}^n$, any open covering of a bounded closed set has a finite sub-covering.}\\
    Above we have proposed the so called ``six basic theorems'' in $\mathbb{R}$, including (1) the theorem of monotonicity and boundness, (2) the theorem of nested closed intervals, (3) the Bolzano-Weierstrass theorem, (4) the Cauchy criterion, (5) the supremum and infimum principle, and (6) the finite covering theorem. These theorems show the completeness of $\mathbb{R}$ in different aspects, and they are equivalent, as we can see the logic chain $(1)\Rightarrow (2)\Rightarrow (3)\Rightarrow (4)\Rightarrow (5)\Rightarrow (6)$ from the their proofs we have written. Actually, there is still a lacked link $(6)\Rightarrow (1)$ before we truely complete the equivalence logic loop. This link is given by the following proof:
    \tcblower
    \textit{Pf}: Suppose $\{a_n\}$ is an increasing sequence in $\mathbb{R}$ and $\{a_n\}\subseteq [a_0, A]$ for some $A\in\mathbb{R}$. If $\{a_n\}$ has a maximum, then the maximum can be verified as the limit. If $\{a_n\}$ has no maximum, it would difficult to construct a limit with an open covering. Hence in this case, we can prove it by contradiction. Assume that $\{a_n\}$ has no limit in $[a_0, A]$. Then we construct an open covering $\mathcal{I} = \{I_x: x\in [a_0, A]\}$ of $[a_0, A]$ as follows:
    \begin{compactenum}
        \item If $x\in[a_0, A]$ is not an upper bound of $\{a_n\}$, then first check which adjacent terms $a_k, a_{k+1}$ of $\{a_n\}$ $x$ lies between (exactly speaking, if $x=a_k$ for some $k$, it should be which $a_{k-1}, a_{k+1}$) and let $I_x = (a_k, a_{k+1})$. In $\mathcal{I}$, such $I_x$ is called a ``type-1 interval''.
        \item If $x\in[a_0, A]$ is an upper bound of $\{a_n\}$, then find a neighborhood $(x-\varepsilon, x+\varepsilon)$ of $x$ s.t. $\varepsilon>0$ and s.t. no term of $\{a_n\}$ lies in $(x-\varepsilon, x+\varepsilon)$ (such neighborhood exists since $x$ is not the limit of $\{a_n\}$). Let $I_x = (x-\varepsilon, x+\varepsilon)$. In $\mathcal{I}$, such $I_x$ is called a ``type-2 interval''.
    \end{compactenum}
    It is easy to notice that any type 1 interval is disjoint with any type 2 interval, or exactly speaking, the right end point of any type-1 interval is less than the left point of any type-2 interval. Now apply the finite covering theorem to obtain a finite sub-covering. Since $a_0$ can be covered only by a type-1 interval and $A$, only by a type-2 interval, there are both type-1 and type-2 intervals in the sub-covering. But since the sub-covering is finitely many, there is always a ``gap'' between the type-1 and the type-2 intervals, which is impossible for the covering of the entire $[a_0, A]$. Contradiction!
\end{Rmk}

\begin{Df}{$\bullet$ Df1.11 (limit superior, limit inferior)}
    Suppose $\{a_n\}$ is a sequence in $\mathbb{R}$. Consider the set
    $$ E = \{\lim\limits_{n\to\infty} a_{k_n}: a_{k_n} \text{ is a subsequence of } a_n\} $$
    i.e., $E$ is the set of all limits of subsequences of $\{a_n\}$, where the ``limit'' can be $\infty$ or $-\infty$ (but not $\pm\infty$). Then we called $\sup E$ (resp. $\inf E$) the limit superior (resp. limit inferior) of $\{a_n\}$, denoted by $\limsup\limits_{n\to\infty}a_n$ or $\mathop{\overline{\lim}}\limits_{n\to\infty} a_n$ (resp. $\liminf\limits_{n\to\infty}a_n$ or $\mathop{\underline{\lim}}\limits_{n\to\infty} a_n$).
\end{Df}

\begin{Rmk}{}
    \textcolor{Th}{Clearly for every sequence $\{a_n\}$ in $\mathbb{R}$, the set $E$ defined above is non-empty} (since if $\{a_n\}$ is bounded, then $\{a_n\}$ has some convergent subsequence, and if $\{a_n\}$ is unbounded, then $\{a_n\}$ has some infinite large subsequence), \textcolor{Th}{hence $\liminf\limits_{n\to\infty}a_n\leq \limsup\limits_{n\to\infty}a_n$.}
\end{Rmk}

\begin{Th}{$\circ$ Th1.11.0.1 (the set of limits of subsequences is additive)}
    Suppose $\{a_n\}$ is a sequence in $\mathbb{R}$ consisting of finitely many subsequences $\{a_{k_1(n)}\}, \dots, \{a_{k_m(n)}\}$ (e.g. $\{a_n\}$ consists of the ``odd sequence'' $\{a_{2k-1}\}$ and the ``even sequence'' $\{a_{2k}\}$). Let $E$ be the set of limits of subsequences of $\{a_n\}$, and $E_i$ ($i=1, \dots, m$) be the set of limits of subsequences of $\{a_{k_i(n)}\}$. Then
    $$ E = \bigcup\limits_{i=1}^m E_i. $$
    \tcblower
    \textit{Pf}: For any convergent subsequence $\{a_{k(n)}\}$ (convergent to some $l$ in $\mathbb{R}_\infty$) of $\{a_n\}$, it must contain infinitely many terms of some $\{a_{k_i(n)}\}$. These terms form a subsequence of both $\{a_{k(n)}\}$ and $\{a_{k_i(n)}\}$, and thus is convergent to $l$ and $l\in E_i$. 
\end{Th}

\begin{Df}{$\circ$ Df1.11.1.-1 (sufficiently large)}
    Introduce the term ``sufficiently large''. Let $P(n)$ be a statement about $n$. Then we say that \textbf{$P(n)$ is true for sufficiently large $n$} if there is some $N\in\mathbb{N}$ s.t. $P(n)$ is true for all $n>N$. 
\end{Df}

\begin{Th}{$\bullet$ Th1.11.1 (basic properties of $\mathop{\overline{\lim}}$ and $\mathop{\underline{\lim}}$)}
    Suppose $\{a_n\}$ is a sequence in $\mathbb{R}$, and $E$ is the ``set of subsequences' limits'' defined in Df1.11. Then:
    \begin{compactenum}
        \item $\mathop{\overline{\lim}}\limits_{n\to\infty}a_n$ (resp. $\mathop{\underline{\lim}}\limits_{n\to\infty}a_n$) $\in E$;
        \item $\forall x>\mathop{\overline{\lim}}\limits_{n\to\infty}a_n$ (resp. $\forall x<\mathop{\underline{\lim}}\limits_{n\to\infty}a_n$), \bigg[$a_n<x$ (resp. $a_n>x$) for sufficiently large $n$ \bigg];
        \item $\mathop{\overline{\lim}}\limits_{n\to\infty}a_n$ (resp. $\mathop{\underline{\lim}}\limits_{n\to\infty}a_n$) is the only one in $\mathbb{R}_\infty$ that satisfies the properties (1) and (2);
        \item $\mathop{\underline{\lim}}\limits_{n\to\infty}a_n \leq \mathop{\overline{\lim}}\limits_{n\to\infty}a_n$;
        \item Let $l\in\mathbb{R}_\infty$. Then $\lim\limits_{n\to\infty}a_n=l$ iff $\mathop{\overline{\lim}}\limits_{n\to\infty}a_n=\mathop{\underline{\lim}}\limits_{n\to\infty}a_n=l$;
        \item Suppose $\{b_n\}$ is also a sequence in $\mathbb{R}$. If there is some $N\in\mathbb{N}$ s.t. $a_n\leq b_n$ for all $n>N$, then $\mathop{\underline{\lim}}\limits_{n\to\infty}a_n\leq \mathop{\underline{\lim}}\limits_{n\to\infty}b_n$ and $\mathop{\overline{\lim}}\limits_{n\to\infty}a_n\leq \mathop{\overline{\lim}}\limits_{n\to\infty}b_n$.
        \item $\liminf\limits_{n\to\infty}a_n = \lim\limits_{n\to\infty}\left(\inf\limits_{k\geq n}a_k\right)$, $\limsup\limits_{n\to\infty}a_n = \lim\limits_{n\to\infty}\left(\sup\limits_{k\geq n}a_k\right)$ \\
        (here $\inf\limits_{k\geq n}a_k = \inf\{a_n, a_{n+1}, \dots\}$).
    \end{compactenum}
    \tcblower
    \textit{Pf}: See the next page.
\end{Th}

\begin{Th}{$\bullet$ Th1.11.1 (basic properties of $\mathop{\overline{\lim}}$ and $\mathop{\underline{\lim}}$) —  continued}
    \textit{Pf}: 
    \begin{compactenum}
        \item Here we only prove the case of limit superior, and assume the limit superior to be finite (other cases are of the same idea). Let $a^\ast = \mathop{\overline{\lim}}\limits_{n\to\infty}a_n$. To prove $a^\ast\in E$ is to find a subsequence of $\{a_n\}$ that converges to $a^\ast$. The key is ``supremum''. Since $a^\ast = \sup E$, we can find a subsequence $\pmb{a}_1$ that converges to $a^\ast-\frac{1}{2}$, and thus we can choose one term $a_{k_1}$ from $\pmb{a}_1$ s.t. $ a^\ast-\frac{1}{1}< a_{k_1} < a^\ast$. After removing all terms of $\{a_n\}$ before $a_{k_1}$ (which does not affect the limits of all subsequences), we can find a subsequence $\pmb{a}_2$ that converges to $a^\ast-\frac{1}{3}$ so that we can choose one term $a_{k_2}$ from $\pmb{a}_2$ s.t. $ a^\ast-\frac{1}{2}< a_{k_2} < a^\ast$ and $k_2>k_1$. So forth, we can find a subsequence $\{a_{k_n}\}$ of $\{a_n\}$ s.t. $a^\ast-\frac{1}{n}< a_{k_n} < a^\ast$ for all $n$, which converges to $a^\ast$.
        \item Also, we assume the limit superior $a^\ast$ to be finite. Suppose $x>a^\ast$. If, by contradiction, when $n\rightarrow\infty$, $a_n$ jumps so that $a_n\geq x$ from time to time, then we can find a subsequence from these ``jumping terms'' that converges to some $l\geq x$, which contradicts with $a^\ast = \sup E$.
        \item Also, we assume the limit superior $a^\ast$ to be finite. Suppose $A^\ast\in\mathbb{R}_\infty$ also satisfies (1) and (2), then $A^\ast\leq a^\ast$ by (1). If $A^\ast<a$, then by (2) $a_n<A^\ast+\frac{a^\ast-A^\ast}{2}$ holds for all enough large $n$, but still $\{a_n\}$ must yield a subsequence that converges to $a^\ast$, which is impossible. Hence $A^\ast = a^\ast$.
        \item Obvious.
        \item Obvious.
        \item This is straightforward since that $E$ is the range of the map $L(\pmb{a}) = \lim\limits_{n\to\infty} \pmb{a}$ defined for each convergent subsequence $\pmb{a}$ of $\{a_n\}$, and that $\mathop{\overline{\lim}}\limits_{n\to\infty} a_n = \max\limits_{\pmb{a}} L(\pmb{a})$, $\mathop{\underline{\lim}}\limits_{n\to\infty} a_n = \min\limits_{\pmb{a}} L(\pmb{a})$. However we still need a proof since the domains of $L$ for $\{a_n\}$ and $\{b_n\}$ might be different. If, by contradiction, $\mathop{\underline{\lim}}\limits_{n\to\infty}a_n > \mathop{\underline{\lim}}\limits_{n\to\infty}b_n$, then suppose $\lim\limits_{n\to\infty}b_{k_n} = \mathop{\underline{\lim}}\limits_{n\to\infty}b_n$, and we have $a_{k_n}\leq b_{k_n}$ for all $n$. Then 
        $$\mathop{\underline{\lim}}\limits_{n\to\infty}a_n \leq \lim\limits_{n\to\infty}a_{k_n} \leq \lim\limits_{n\to\infty}b_{k_n} = \mathop{\underline{\lim}}\limits_{n\to\infty}b_n.$$
        (Although $\{a_{k_n}\}$ may be not convergent, it has some convergent subsequence $\{a_{k_{l_n}}\}$ (which is also a subsequence of $\{a_n\}$) and we compare $\{a_{k_{l_n}}\}$ and $\{b_{k_{l_n}}\}$ to take the limit.)
        \item Also, we assume the limit superior $a^\ast$ to be finite, and let $\alpha_n = \sup\limits_{k\geq n}a_k$.
        \begin{compactitem}
            \item First, $\lim\limits_{n\to\infty}\alpha_n \leq a^\ast$. For each $k\in\mathbb{N}^\ast$, we can find some $n_k$ s.t. $a_n<a^\ast+\frac{1}{k+1}$ for all $n\geq n_k$ (and we force that $n_1<n_2<\dots$). Then 
            $$\alpha_{n_k} = \sup\limits_{n\geq n_k}a_n\leq a^\ast+\frac{1}{k+1}<a^\ast+\frac{1}{k},$$ 
            and we let $k\rightarrow\infty$ to obtain $\lim\limits_{n\to\infty}\alpha_n \leq a^\ast$.
            \item Second, $\lim\limits_{n\to\infty}\alpha_n \geq a^\ast$. We choose a subsequence $\{a_{k_n}\}$ s.t. $\lim\limits_{n\to\infty}a_{k_n} = a^\ast$. Since that $\alpha_{k_n} \geq a_{k_n}$ for all $n$, and that $\{\alpha_n\}$ is decreasing (hence is convergent), we let $n\rightarrow\infty$ to obtain $\lim\limits_{n\to\infty}\alpha_n = \lim\limits_{n\to\infty}\alpha_{k_n} \geq a^\ast$.
        \end{compactitem}
    \end{compactenum}
\end{Th}

\begin{Rmk}{}
    The 7th property here illustrates why we use $\limsup$ and $\liminf$ to denote limit superior and limit inferior. After all, the readers may first come up with the idea of using $\sup\lim$ instead of $\limsup$, as the limit superior is defined as the supremum of the possible subsequences' limits. \\
    We also desire to construct the arithmetic of $\limsup$ and $\liminf$. Since $\limsup$ and $\liminf$ involve the infinity, we must first supplement the arithmetic for $\infty$ and $-\infty$.
\end{Rmk}

\begin{Df}{$\bullet$ Df1.11.2.-2 (arithmetic in $\mathbb{R}_\infty$)}
    In $\mathbb{R}_\infty$, we define the ``+'', ``-'', ``$\cdot$'' and ``$/$'' (mainly for those rules involving infinity that are not defined before) as follows (the symbol ``$\textcolor{red}{u}$'' means that the we still not define the operation):
    \begin{compactenum}
        \item (+): Suppose $r\in\mathbb{R}$. Then
        $$
        \begin{aligned}
            &r+\infty = \infty, \;\;\quad r+(-\infty) = -\infty,  \\
            &\infty+\infty = \infty, \quad (-\infty)+(-\infty) = -\infty, \quad \infty+(-\infty) = \textcolor{red}{u}.
        \end{aligned}
        $$
        and (+) is defined to be commutative (for those not listed here). \textcolor{Th}{If $a+b+c\triangleq (a+b)+c$ where $a,b,c\in\mathbb{R}_\infty$, then (+) is associative (which means that $a+b+c = a+(b+c)$ for all $a,b,c\in\mathbb{R}_\infty$).}
        \item (-): Suppose $r\in\mathbb{R}$. First define the univariate ``additive inverse'' operation (-) on $\mathbb{R}_\infty$:
        $$ -(r) = -r, \quad -(\infty) = -\infty, \quad -(-\infty) = \infty. $$
        Then the subtraction (-) for any $a,b\in\mathbb{R}_\infty$ is defined as
        $$ a-b = a+(-b). $$
        \item ($\cdot$): Suppose $r\in\mathbb{R}$ and $r>0$. Then
        $$
        \begin{aligned}
            &r\cdot\infty = \infty, \;\;\quad r\cdot(-\infty) = -\infty, \;\;\,\quad (-r)\cdot\infty = -\infty, \quad (-r)\cdot(-\infty) = \infty, \\
            &\infty\cdot\infty = \infty, \quad (-\infty)\cdot(-\infty) = \infty, \quad \infty\cdot(-\infty) = -\infty, \\
            &0\cdot\infty = \textcolor{red}{u}, \;\;\;\quad 0\cdot(-\infty) = \textcolor{red}{u}.
        \end{aligned}
        $$
        and ($\cdot$) is defined to be commutative (for those not listed here). \textcolor{Th}{If $a\cdot b\cdot c\triangleq (a\cdot b)\cdot c$ where $a,b,c\in\mathbb{R}_\infty$, then ($\cdot$) is associative (which means that $a\cdot b\cdot c = a\cdot(b\cdot c)$ for all $a,b,c\in\mathbb{R}_\infty$).}
        \item ($/$): Suppose $r\in\mathbb{R}$ and $r>0$. First define the univariate ``multiplicative inverse'' operation ($1/$) on $\mathbb{R}_\infty$:
        $$ 1/(\pm r) = \pm 1/r, \quad 1/(\infty) = 1/(-\infty) = 0, \quad 1/0 = \textcolor{red}{u} $$
        Then the division ($/$) for any $a,b\in\mathbb{R}_\infty$ is defined as
        $$ a/b = a\cdot(1/b). $$
    \end{compactenum}
\end{Df}

\begin{Rmk}{}
    The infinity is actually a limit, and the arithmetic above is defined to make the limit operation more convenient. This is the following rules which extends the arithmetic of regular limits (those in the Rmk \{, ID: 1.1\})
\end{Rmk}

\begin{Th}{$\bullet$ Th1.11.2.-1 (extended arithmetic of regular limit) (arithmetic of limit)}
    Suppose $\{a_n\}$ and $\{b_n\}$ are sequences in $\mathbb{R}$, and $\lim\limits_{n\to\infty}a_n, \lim\limits_{n\to\infty}b_n$ both exist in $\mathbb{R}_\infty$. Then:
    $$ \lim\limits_{n\to\infty}\Big(a_n (+-\cdot\,/) b_n\Big) = \left(\lim\limits_{n\to\infty}a_n \right)(+-\cdot\,/) \left(\lim\limits_{n\to\infty}b_n\right). $$
    (Here if the right side can be computed in $\mathbb{R}_\infty$, then the sequence of the left side has definition after the $N$-th term for some $N\in\mathbb{N}$, and the limit of the left side exists in $\mathbb{R}_\infty$ and is equal to the result of the right side; if otherwise the right side is ``$\textcolor{red}{u}$'' in Df 1.11.2.-2, then the limit of the left side can exist in $\mathbb{R}_\infty$ or not so that other methods should be used to determine the limit.)
    \tcblower
    \textit{Pf}: Trivial.
\end{Th}

\begin{Th}{$\bullet$ Th1.11.2 (arithmetic of $\mathop{\overline{\lim}}$ and $\mathop{\underline{\lim}}$)}
    Suppose $\{x_n\}$ and $\{a_n\}$ are sequences in $\mathbb{R}$, and $\lim\limits_{n\to\infty}x_n = x\in\mathbb{R}_\infty$. Then:
    \begin{compactenum}
        \item $\mathop{\overline{\lim}}\limits_{n\to\infty}(a_n + x_n) = \left(\mathop{\overline{\lim}}\limits_{n\to\infty}a_n\right) + x$ and $\mathop{\underline{\lim}}\limits_{n\to\infty}(a_n + x_n) = \left(\mathop{\underline{\lim}}\limits_{n\to\infty}a_n\right) + x$.
        \item $\mathop{\overline{\lim}}\limits_{n\to\infty}(a_n - x_n) = \left(\mathop{\overline{\lim}}\limits_{n\to\infty}a_n\right) - x$ and $\mathop{\underline{\lim}}\limits_{n\to\infty}(a_n - x_n) = \left(\mathop{\underline{\lim}}\limits_{n\to\infty}a_n\right) - x$; \\
        $\mathop{\overline{\lim}}\limits_{n\to\infty}(x_n - a_n) = x - \left(\mathop{\underline{\lim}}\limits_{n\to\infty}a_n\right)$ and $\mathop{\underline{\lim}}\limits_{n\to\infty}(x_n - a_n) = x - \left(\mathop{\overline{\lim}}\limits_{n\to\infty}a_n\right)$.
        \item If $x>0$, then $\mathop{\overline{\lim}}\limits_{n\to\infty}(a_n\cdot x_n) = \left(\mathop{\overline{\lim}}\limits_{n\to\infty}a_n\right)\cdot x$ and $\mathop{\underline{\lim}}\limits_{n\to\infty}(a_n\cdot x_n) = \left(\mathop{\underline{\lim}}\limits_{n\to\infty}a_n\right)\cdot x$; \\ 
        if $x<0$, then $\mathop{\overline{\lim}}\limits_{n\to\infty}(a_n\cdot x_n) = \left(\mathop{\underline{\lim}}\limits_{n\to\infty}a_n\right)\cdot x$ and $\mathop{\underline{\lim}}\limits_{n\to\infty}(a_n\cdot x_n) = \left(\mathop{\overline{\lim}}\limits_{n\to\infty}a_n\right)\cdot x$.
        \item If $x>0$, then $\mathop{\overline{\lim}}\limits_{n\to\infty}(a_n/x_n) = \left(\mathop{\overline{\lim}}\limits_{n\to\infty}a_n\right)/x$ and $\mathop{\underline{\lim}}\limits_{n\to\infty}(a_n/x_n) = \left(\mathop{\underline{\lim}}\limits_{n\to\infty}a_n\right)/x$;\\ 
        if $x<0$, then $\mathop{\overline{\lim}}\limits_{n\to\infty}(a_n/x_n) = \left(\mathop{\underline{\lim}}\limits_{n\to\infty}a_n\right)/x$ and $\mathop{\underline{\lim}}\limits_{n\to\infty}(a_n/x_n) = \left(\mathop{\overline{\lim}}\limits_{n\to\infty}a_n\right)/x$. 
        \item[$4^*$] Suppose $\mathop{\underline{\lim}}\limits_{n\to\infty}a_n>0$ or $\mathop{\overline{\lim}}\limits_{n\to\infty}a_n<0$. \\
        If $x>0$, then $\mathop{\overline{\lim}}\limits_{n\to\infty}(x_n/a_n) = x/\left(\mathop{\underline{\lim}}\limits_{n\to\infty}a_n\right)$ and $\mathop{\underline{\lim}}\limits_{n\to\infty}(x_n/a_n) = x/\left(\mathop{\overline{\lim}}\limits_{n\to\infty}a_n\right)$;\\
        if $x<0$, then $\mathop{\overline{\lim}}\limits_{n\to\infty}(x_n/a_n) = x/\left(\mathop{\overline{\lim}}\limits_{n\to\infty}a_n\right)$ and $\mathop{\underline{\lim}}\limits_{n\to\infty}(x_n/a_n) = x/\left(\mathop{\underline{\lim}}\limits_{n\to\infty}a_n\right)$. 
    \end{compactenum}
    (Here each equality, as we have mentioned in Th1.11.2.-1, means that: if the right side can be computed in $\mathbb{R}_\infty$, then the sequence of the left side has definition after the $N$-th term for some $N\in\mathbb{N}$, and the limit superior or limit inferior of the left side exists in $\mathbb{R}_\infty$ and is equal to the result of the right side; if otherwise the right side is ``$\textcolor{red}{u}$'' in Df 1.11.2.-2, then the limit superior or limit inferior of the left side can exist in $\mathbb{R}_\infty$ or not so that other methods should be used to determine it.)
    \tcblower
    \textit{Pf}: See the next page.
\end{Th}

\begin{Th}{$\bullet$ Th1.11.2 (arithmetic of $\mathop{\overline{\lim}}\limits_{n\to\infty}$ and $\mathop{\underline{\lim}}\limits_{n\to\infty}$) —  continued}
    \textit{Pf}: The details of the proof are trivial, and thus we only prove that $\mathop{\overline{\lim}}\limits_{n\to\infty}(a_n + x_n) = \left(\mathop{\overline{\lim}}\limits_{n\to\infty}a_n\right) + x$ to illustrate the idea. \\
    Let $E\{b_n\}$ denote the set of all subsequences' limits of an arbitrary sequence $\{b_n\}$ (as is defined in Df1.11). Then
    $$ E\{b_n\} = \{\lim\limits_{n\to\infty}b_{k_n}: \{b_{k_n}\}\text{ is a subsequence of }\{b_n\}\} $$
    Thus 
    $$ E\{a_n+x_n\} = \{\lim\limits_{n\to\infty}(a_{k_n}+x_{k_n})\}, $$
    where $\{a_{k_n}+x_{k_n}\}$ converges to some element in $\mathbb{R}_\infty$. Here we can further assume that the $\{a_{k_n}\}$ in the expression of $E\{a_n+x_n\}$ above converges to someone in $\mathbb{R}_\infty$ (since if not, we can find a convergent subsequence $\{a_{k_{l_n}}\}$ of $\{a_{k_n}\}$ and then
    $$ \lim\limits_{n\to\infty}(a_{k_{l_n}}+x_{k_{l_n}}) = \lim\limits_{n\to\infty}(a_{k_n} + x_{k_n}), $$
    which does not affect the set $E$.) Furthermore, the sum $\mathop{\overline{\lim}}\limits_{n\to\infty}a_n + x$ is assumed to be computable by the Df \{, ID: 1.11.2.-2\} (since the theorem only claims for this situation). Then let $\{a_{l_n}\}$ be the subsequence s.t. $\lim\limits_{n\to\infty}a_{l_n} = \mathop{\overline{\lim}}\limits_{n\to\infty}a_n$. Then for any $\{a_{k_n}\}$ that satisfies all conditions above, $\lim\limits_{n\to\infty}a_{k_n} = \lim\limits_{n\to\infty}a_{l_n}$ (so that $\lim\limits_{n\to\infty}(a_{k_n} + x_{k_n}) = \left(\mathop{\overline{\lim}}\limits_{n\to\infty}a_n\right) + x$) or $\lim\limits_{n\to\infty}a_{k_n} < \lim\limits_{n\to\infty}a_{l_n}$. Under the latter case, we have $a_{k_n} < a_{l_n}$ (when $n$ is large enough). Then
    $$ a_{k_n} + x_{k_n} < a_{l_n} + x_{k_n}, $$
    and thus after taking the limit, we have
    $$ \lim\limits_{n\to\infty}(a_{k_n} + x_{k_n}) \leq \lim\limits_{n\to\infty}(a_{l_n} + x_{k_n}) = \lim\limits_{n\to\infty}a_{l_n} + \lim\limits_{n\to\infty}x_{k_n} = \left(\mathop{\overline{\lim}}\limits_{n\to\infty}a_n\right) + x. $$
    So far we have verified that $\left(\mathop{\overline{\lim}}\limits_{n\to\infty}a_n\right)+x$ is an upper bound of $E\{a_n+x_n\}$, and clearly this upper bound can be achieved if we choose $\{a_{k_n}\}$ to be $\{a_{l_n}\}$. Hence $\mathop{\overline{\lim}}\limits_{n\to\infty}(a_n+x_n) = \left(\mathop{\overline{\lim}}\limits_{n\to\infty}a_n\right) + x$.
\end{Th}
\end{document}