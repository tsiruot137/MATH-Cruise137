\documentclass{article}

    \usepackage{xcolor}
    \definecolor{pf}{rgb}{0.4,0.6,0.4}
    \usepackage[top=1in,bottom=1in, left=0.8in, right=0.8in]{geometry}
    \usepackage{setspace}
    \setstretch{1.2} 
    \setlength{\parindent}{2em}

    \usepackage{paralist}
    \usepackage{cancel}

    % \usepackage{ctex}
    \usepackage{amssymb}
    \usepackage{amsmath}

    % \usepackage{hyperref}
    % \hypersetup{hidelinks,
	% colorlinks=true,
	% allcolors=black,
	% pdfstartview=Fit,
	% breaklinks=true}

    \usepackage{float}

    \usepackage{tcolorbox}
    \definecolor{Df}{RGB}{0, 184, 148}
    \definecolor{Th}{RGB}{9, 132, 227}
    \definecolor{Rmk}{RGB}{215, 215, 219}
    \newtcolorbox{Df}[2][]{colbacktitle=Df, colback=white, title={\large\color{white}#2},fonttitle=\bfseries,#1}
    \newtcolorbox{Th}[2][]{colbacktitle=Th, colback=white, title={\large\color{white}#2},fonttitle=\bfseries,#1}
    \newtcolorbox{Rmk}[2][]{colbacktitle=Rmk, colback=white, title={\large\color{black}{Remarks}},fonttitle=\bfseries,#1}

    \title{\LARGE \textbf{Natural Logarithm}}
    \author{\large Jiawei Hu}

\begin{document}
\maketitle

This is a problem (that has been solved early but essential for modern mathematics) about the \textbf{Natural Logarithm}. 

\begin{Rmk}{}
    \textbf{Before reading:}
    \begin{compactenum}
        \item Since the discovery of the base of natural logarithm $e$ includes many aspects, we just omit almost all this background information in this file, and just instead list the result of the foundation of $e$ here. As for the background, feel free to learn online by yourself.
        \item Also, we do not have to record the proofs of the theorems in this file, as they can be all found in the reference book (the book [1] in the reference list in the \verb|exercises.md| file). Actually, this file is just a transcription of the corresponding section (section 6 of chapter 1) of that book.
    \end{compactenum}
\end{Rmk}

By the way, we now reiterate some commonly-used notations and conventions:
\begin{compactenum}
    \item $\mathbb{R}$: the set of the real numbers;
    \item $\mathbb{R}^+$: the set of the positive real numbers;
    \item $\mathbb{Z}$: the set of the integers;
    \item $\mathbb{N}$: the set of the natural numbers;
    \item $\mathbb{N^\ast}$ or $\mathbb{N}^+$: the set of the positive integers.
    \item An agreement for the length of a list: if we write $a_1, \dots, a_n$, then we indicate that $n$ is finite and that $n\geq 1$; if we write $a_0, \dots, a_n$, then we indicate that $n$ is finite and that $n\geq 0$.
    \item Continue to use the notations and concepts of functions (see the chapter 1 of course 0).
\end{compactenum} 
Please check the notations and definitions by yourself from the previous chapters or courses. Then with everything prepared, here we go.

\noindent\rule{\textwidth}{2pt}

Many studies show that the limits
$$ \lim_{n\to\infty}\left(1+\frac{1}{n}\right)^n \qquad\text{and}\qquad \lim_{n\to\infty}\left(1+\frac{1}{1!}+\frac{1}{2!}+\cdots+\frac{1}{n!}\right) $$
matter. Then denote
$$ e_n = \left(1+\frac{1}{n}\right)^n\qquad s_n = 1+\frac{1}{1!}+\frac{1}{2!}+\cdots+\frac{1}{n!} $$ 
and we have

\begin{Th}{Th 1\_P1.3.2.1 (monotonicity of $e_n$ and $s_n$)}
    Both $\{e_n\}$ and $\{s_n\}$ are strictly increasing.
\end{Th}

\begin{Th}{Th 1\_P1.3.2.2}
    $e_n \leq s_n < 3$ for all $n\in\mathbb{N}^\ast$. So both $\{e_n\}$ and $\{s_n\}$ have limits.
\end{Th}

\begin{Th}{Th 1\_P1.3.2.3}
    $\lim_{n\to\infty}e_n = \lim_{n\to\infty}s_n \textcolor{Df}{\triangleq e}$.
\end{Th}

\begin{Th}{Th 1\_P1.3.2.4 (error estimate of $s_n$ to $e$)}
    $0<e-s_n<\frac{1}{n!\,n}$ for all $n\in\mathbb{N}^\ast$.
\end{Th} 

\begin{Th}{Th 1\_P1.3.2.5}
    $e$ is irrational.
\end{Th}
\end{document}