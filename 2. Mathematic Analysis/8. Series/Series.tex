\documentclass{article}

    \usepackage{xcolor}
    \definecolor{pf}{rgb}{0.4,0.6,0.4}
    \usepackage[top=1in,bottom=1in, left=0.8in, right=0.8in]{geometry}
    \usepackage{setspace}
    \setstretch{1.2} 
    \setlength{\parindent}{0em}

    \usepackage{paralist}
    \usepackage{cancel}

    % \usepackage{ctex}
    \usepackage{amssymb}
    \usepackage{amsmath}
    \usepackage{extarrows}

    \usepackage{tcolorbox}
    \definecolor{Df}{RGB}{0, 184, 148}
    \definecolor{Th}{RGB}{9, 132, 227}
    \definecolor{Rmk}{RGB}{215, 215, 219}
    \definecolor{P}{RGB}{154, 13, 225}
    \newtcolorbox{Df}[2][]{colbacktitle=Df, colback=white, title={\large\color{white}#2},fonttitle=\bfseries,#1}
    \newtcolorbox{Th}[2][]{colbacktitle=Th, colback=white, title={\large\color{white}#2},fonttitle=\bfseries,#1}
    \newtcolorbox{Rmk}[2][]{colbacktitle=Rmk, colback=white, title={\large\color{black}{Remarks}},fonttitle=\bfseries,#1}

    \title{\LARGE \textbf{Series}}
    \author{\large Jiawei Hu}

    % new commands for formula typying
    \newcommand{\parfrac}[2]{\frac{\partial #1}{\partial #2}}
    \newcommand{\biparfrac}[2]{\frac{\partial^2 #1}{#2}}
    \newcommand{\dif}{\mathop{}\!\mathrm{d}}
    \newcommand{\Dif}{\mathop{}\!\mathrm{D}}
\begin{document}
\maketitle

This is the 8th chapter of Mathematical Analysis, which is about \textbf{Series}. By the way, we now pre-claim some commonly-used notations and terms:
\begin{Df}{Notations and Terms}
    \begin{compactenum}
        \item $\mathbb{R}$: the set of the real numbers; $\mathbb{R}_\infty = \mathbb{R}\cup\{-\infty, \infty\}$;
        \item An agreement for the length of a list: if we write $a_1, \dots, a_n$, then we indicate that $n$ is finite and that $n\geq 1$; if we write $a_0, \dots, a_n$, then we indicate that $n$ is finite and that $n\geq 0$.
        \item Keep coincident in the notions and notations of functions with the chapter 1 of course 0, including the ones of domain, range, restriction, image, pre-image, inverse and composition. Specifically for a function $f: A\rightarrow B$ and some sets $E\subseteq A$ and $F\subseteq B$, the image of $E$ and the pre-image of $F$ under $f$ are just:
        $$f[E] = \{f(x): x\in E\},\quad f^{-1}[F] = \{x\in A: f(x)\in F\}$$
        \item For the existence of a limit, if we have used the symbol $\lim\limits_{x\to x_0} f(x)$ in an expression (such as an equality, an inequality or some expressions involving some other numbers), then without explicitly specification, we imply that the limit exists (``exist'' means finite according to the chapter 1).
        \item A set of sets is called a collection or a family.
    \end{compactenum}
\end{Df}

Here is the \textbf{Quick Search} for this chapter:
\begin{Th}{Quick Search}
    \begin{compactdesc}
        \item (8.1.*): Basic properties of series.
        \item (8.2.*): Positive series.
            \subitem (8.2.1.*): Comparison criterion.
            \subitem (8.2.2.*): Other secondary criteria.
        \item (8.3.*): Criteria for arbitrary series.
            \subitem (8.3.3.*): Criteria for $ \sum_{n=1}^{\infty} a_n b_n $.
        \item (8.4.*): Riemann's rearrangement theorem.
        \item (8.5.*): Product of series.
    \end{compactdesc}
\end{Th}

Then with everything prepared, here we go. 

\begin{Df}{Df8.1 (Series (of real terms))}
    For a sequence $\{a_n\in\mathbb{R}: n\in\mathbb{N}^\ast\}$ in $\mathbb{R}$, the sum $\sum_{n=1}^{\infty} a_n$ is called a \textbf{series}. The series is said to be \textbf{convergent} if the \textcolor{Df}{partial sum sequence 
    $$ \{\sum_{i=1}^{n} a_n: n\in\mathbb{N}^\ast\} $$}
    is convergent (to some $S\in\mathbb{R}$), and this case we define the \textbf{sum of the series} as the limit of the sequence, i.e.,
    $$ \sum_{n=1}^{\infty} a_n \triangleq \lim_{n\to\infty} \sum_{i=1}^{n} a_n; $$
    otherwise, the series is said to be \textbf{divergent}.
\end{Df}

\begin{Rmk}{}
    \textcolor{Th}{Basic properties of series:
    \begin{compactenum}
        \item If the series $\sum_{n=1}^{\infty} a_n$ converges, then $\lim\limits_{n\to\infty} a_n = 0$;
        \item (linearity) If $\sum_{n=1}^{\infty} a_n$ and $\sum_{n=1}^{\infty} b_n$ are convergent, then so are $\sum_{n=1}^{\infty} (a_n\pm b_n)$ and 
        $$ \sum_{n=1}^{\infty} (a_n\pm b_n) = \sum_{n=1}^{\infty} a_n \pm \sum_{n=1}^{\infty} b_n; $$ If $\sum_{n=1}^{\infty} a_n$ is convergent, then so are $\sum_{n=1}^{\infty} ca_n$ for any $c\in\mathbb{R}$, and 
        $$ \sum_{n=1}^{\infty} ca_n = c\sum_{n=1}^{\infty} a_n; $$ 
        \item Changing / Adding / Removing a finite number of terms in a series $\sum_{n=1}^{\infty}$ does not affect its convergence or divergence, but may affect its sum;
    \end{compactenum}
    }
\end{Rmk}

\begin{Th}{Eg8.1.1 (convergence of equal-ratio-series / geometric series)}
    Learn the definition of geometric series by yourself. \\
    For a geometric series $\sum_{n=1}^{\infty} q^{n}$, it converges iff $|q|<1$, and in this case we have 
    $$ \sum_{n=1}^{\infty} q^{n-1} = \frac{1}{1-q}. $$
    \tcblower
    \textit{Solution}: Easy.
\end{Th}

\begin{Th}{Th8.1.2 (associativity of infinite sums)}
    Suppose $\sum_{n=1}^{\infty} a_n$ converges. Arbitrarily combine the terms of the series into a new series
    $$ \sum_{n=0}^{\infty} \left(\sum_{i = k_n+1}^{k_{n+1}} a_{i}\right), $$
    where $0 = k_0 < k_1 < k_2 < \dots$. Then the new series converges and has the same sum as the original series.
    \tcblower
    \textit{Pf}: Obvious.
\end{Th}

\begin{Rmk}{}
    \textcolor{Th}{The converse of this theorem is not true.} For example, the series $\sum_{n=1}^{\infty} (-1)^{n}$ diverges, but combining the $1$'s and $-1$'s in pairs gives the zero series, which converges. However, \textcolor{Th}{if for each $n$, the terms $a_{k_n+1}, \cdots, a_{k_{n+1}}$ are all non-negative (or all non-positive), then the converse is true (see the corresponding position of the reference book).}
\end{Rmk}

\begin{Df}{Df8.2 (Positive series)}
    A series $\sum_{n=1}^{\infty} a_n$ is called \textbf{positive} (resp. \textbf{non-negative}) if $a_n>0$ (resp. $a_n\geq 0$) for all $n\in\mathbb{N}^\ast$.
\end{Df}

\begin{Th}{Th8.2.0.1 (convergence of positive series)}
    Suppose $\sum_{n=1}^{\infty} a_n$ is a non-negative series. Then the series either converges or diverges to $\infty$.
    \tcblower
    \textit{Pf}: Obviously, the series converges iff its partial sum sequence is bounded. 
\end{Th}

\begin{Th}{Eg8.2.0.1 (convergence of $\sum_{n=1}^{\infty} \frac{1}{n^\alpha}$)}
    Let $\alpha>0$. Then the series $\sum_{n=1}^{\infty} \frac{1}{n^\alpha}$ converges iff $\alpha>1$.
    \tcblower
    \textit{Solution}: See the Ex \{, ID: 1.3.1\}.
\end{Th}

\begin{Th}{Th8.2.1 (comparison criterion of positive series)}
    Suppose $\sum_{n=1}^{\infty} a_n$ and $\sum_{n=1}^{\infty} b_n$ are two non-negative series. If
    $$ a_n \leq b_n $$
    holds for sufficiently large $n$, then
    \begin{compactitem}
        \item $\sum_{n=1}^{\infty} b_n$ converges $\Rightarrow$ $\sum_{n=1}^{\infty} a_n$ converges;
        \item $\sum_{n=1}^{\infty} a_n$ diverges $\Rightarrow$ $\sum_{n=1}^{\infty} b_n$ diverges.
    \end{compactitem}
    \tcblower
    \textit{Pf}: Obvious.
\end{Th}

\begin{Th}{Th8.2.1.1 (comparison criterion under self-ratio)}
    Suppose $\sum_{n=1}^{\infty} a_n$ and $\sum_{n=1}^{\infty} b_n$ are two positive series. If
    $$ \frac{a_{n+1}}{a_n} \leq \frac{b_{n+1}}{b_n} $$
    holds for sufficiently large $n$, then
    \begin{compactitem}
        \item $\sum_{n=1}^{\infty} b_n$ converges $\Rightarrow$ $\sum_{n=1}^{\infty} a_n$ converges;
        \item $\sum_{n=1}^{\infty} a_n$ diverges $\Rightarrow$ $\sum_{n=1}^{\infty} b_n$ diverges.
    \end{compactitem}
    \tcblower
    \textit{Pf}: Obvious, using Th \{, ID: 8.2.1\}.
\end{Th}

\begin{Th}{Th8.2.1.2 (comparison criterion under limit)}
    Suppose $\sum_{n=1}^{\infty} a_n$ and $\sum_{n=1}^{\infty} b_n$ are two positive series. If
    $$ \lim_{n\to\infty} \frac{a_n}{b_n} = l, $$
    then:
    \begin{compactitem}
        \item if $0 < l < \infty$, then $\sum_{n=1}^{\infty} b_n$ converges $\Leftrightarrow$ $\sum_{n=1}^{\infty} a_n$ converges;
        \item if $l = 0$, then $\sum_{n=1}^{\infty} b_n$ converges $\Rightarrow$ $\sum_{n=1}^{\infty} a_n$ converges;
        \item if $l = \infty$, then $\sum_{n=1}^{\infty} b_n$ diverges $\Rightarrow$ $\sum_{n=1}^{\infty} a_n$ diverges.
    \end{compactitem}
    \tcblower
    \textit{Pf}: Obvious.
\end{Th}

\begin{Th}{Th8.2.1.3 (Cauchy's integral criterion)}
    Suppose $f$ is a real function defined on $[1, \infty]$. If $f$ is decreasing and non-negative, then the series $\sum_{n=1}^{\infty} f(n)$ converges iff the infinite integral $\int_{1}^{\infty} f(x)\dif x$ converges.
    \tcblower
    \tcblower
    \textit{Pf}: Obvious, as ($S(x) = \int_{1}^{x} f(t)\dif t$, $S_n = \sum_{k=1}^{n} f(k)$):
    $$ S_n \geq S(n) \geq S_{n+1}-S_1. $$
\end{Th}

\begin{Th}{Eg8.2.1.3.1 ($\sum_{n=1}^{\infty} \frac{1}{n^\alpha}$, $\sum_{n=2}^{\infty} \frac{1}{n(\ln n)^\alpha}$)}
    Suppose $\alpha>0$. Then
    \begin{compactenum}
        \item $\sum_{n=1}^{\infty} \frac{1}{n^\alpha}$ converges $\Leftrightarrow$ $\alpha>1$;
        \item $\sum_{n=2}^{\infty} \frac{1}{n(\ln n)^\alpha}$ converges $\Leftrightarrow$ $\alpha>1$.
    \end{compactenum}
    \tcblower
    \textit{Solution}: See the Eg \{, ID: 7.6.4\}.
\end{Th}

\begin{Th}{Th8.2.2.1 (Cauchy's root criterion) ($\sqrt[n]{a_n}$)}
    Suppose $\sum_{n=1}^{\infty} a_n$ is a non-negative series. 
    \begin{compactenum}
        \item If there is some $q<1$ s.t. $\sqrt[n]{a_n} \leq q$ for sufficiently large $n$, then the series converges;
        \item If $\sqrt[n]{a_n} \geq 1$ for infinitely many $n$, then the series diverges.
    \end{compactenum}
    \tcblower
    \textit{Pf}: Compare with the geometric series $\sum_{n=1}^{\infty} q^{n}$, using Th \{, ID: 8.2.1\}.
\end{Th}

\begin{Th}{Th8.2.2.1.1 (Cauchy's root criterion under limit)}
    Suppose $\sum_{n=1}^{\infty} a_n$ is a non-negative series. Let $q = \limsup\limits_{n\to\infty} \sqrt[n]{a_n}$. Then
    \begin{compactenum}
        \item if $q<1$, then the series converges;
        \item if $q>1$, then the series diverges;
        \item if $q = 1$, yet to see.
    \end{compactenum}
    \tcblower
    \textit{Pf}: Obvious. For the case $q = 1$, consider the series $\sum_{n=1}^{\infty} \frac{1}{n}$ and $\sum_{n=1}^{\infty} \frac{1}{n^2}$.
\end{Th}

\begin{Th}{Th8.2.2.2 (d'Alembert's criterion) ($\frac{a_{n+1}}{a_n}$)}
    Suppose $\sum_{n=1}^{\infty} a_n$ is a positive series.
    \begin{compactenum}
        \item If there is some $q<1$ s.t. $\frac{a_{n+1}}{a_n} \leq q$ for sufficiently large $n$, then the series converges;
        \item If $\frac{a_{n+1}}{a_n} \geq 1$ for sufficiently large $n$, then the series diverges.
    \end{compactenum}
    \tcblower
    \textit{Pf}: Compare with the geometric series, using Th \{, ID: 8.2.1.1\}.
\end{Th}

\begin{Th}{Th8.2.2.2.1 (d'Alembert's criterion under limit)}
    Suppose $\sum_{n=1}^{\infty} a_n$ is a positive series. Let
    $$ q = \limsup_{n\to\infty} \frac{a_{n+1}}{a_n} \qquad \text{and} \qquad q^\prime = \liminf_{n\to\infty} \frac{a_{n+1}}{a_n}. $$
    Then
    \begin{compactenum}
        \item if $q < 1$, then the series converges;
        \item if $q^\prime > 1$, then the series diverges;
        \item if $q = 1$ or $q^\prime = 1$, yet to see.
    \end{compactenum}
    \tcblower
    \textit{Pf}: Obvious. For the case $q = 1$ or $q^\prime = 1$, still consider the series $\sum_{n=1}^{\infty} \frac{1}{n}$ and $\sum_{n=1}^{\infty} \frac{1}{n^2}$.
\end{Th}

\begin{Th}{Th8.2.2.2.2 (Cauchy's root criterion is stronger than d'Alembert's criterion)}
    Suppose $a_n>0$. Then
    $$ \liminf_{n\to\infty} \frac{a_{n+1}}{a_n} \leq \liminf_{n\to\infty} \sqrt[n]{a_n} \leq \limsup_{n\to\infty} \sqrt[n]{a_n} \leq \limsup_{n\to\infty} \frac{a_{n+1}}{a_n}. $$
\end{Th}

\begin{Th}{Th8.2.2.3 (Raabe's criterion) ($n(\frac{a_n}{a_{n+1}}-1)$)}
    Suppose $\sum_{n=1}^{\infty} a_n$ is a positive series. 
    \begin{compactenum}
        \item If there is some $\alpha>1$ s.t. $n(\frac{a_n}{a_{n+1}}-1) \geq \alpha$ for sufficiently large $n$, then the series converges;
        \item If $n\left(\frac{a_n}{a_{n+1}}-1\right) \leq 1$ for sufficiently large $n$, then the series diverges.
    \end{compactenum}
    \tcblower
    \textit{Pf}: Compare with the series $\sum_{n=1}^{\infty} \frac{1}{n^\alpha}$, using Th \{, ID: 8.2.1.1\}.
\end{Th}

\begin{Th}{Th8.2.2.3.1 (Raabe's criterion under limit)}
    Suppose $\sum_{n=1}^{\infty} a_n$ is a positive series. Let
    $$ \alpha = \limsup_{n\to\infty} n\left(\frac{a_n}{a_{n+1}}-1\right) \qquad \text{and} \qquad \alpha^\prime = \liminf_{n\to\infty} n\left(\frac{a_n}{a_{n+1}}-1\right). $$
    Then
    \begin{compactenum}
        \item if $\alpha > 1$, then the series converges;
        \item if $\alpha^\prime \leq 1$, then the series diverges.
        \item if $\alpha = 1$ or $\alpha^\prime = 1$, yet to see.
    \end{compactenum}
    \tcblower
    \textit{Pf}: Obvious. For the case $\alpha = 1$ or $\alpha^\prime = 1$, see the exercise below.
\end{Th}

\begin{Rmk}{}
    If $\lim\limits_{n\to\infty} n\left(\frac{a_n}{a_{n+1}}-1\right) = l$, we often write it as 
    $$ \frac{a_n}{a_{n+1}} = 1 + \frac{l}{n} + o\left(\frac{1}{n}\right), $$
    which is convenient for both proofs and calculations.
\end{Rmk}

\begin{Th}{Ex8.2.2.4.-1 (the grey area of Raabe's criterion)}
    Find a series in the grey area of Raabe's criterion.
    \tcblower
    \textit{Solution}: $\sum_{n=1}^{\infty} \frac{1}{n(\ln n)^\alpha}$.
\end{Th}

\begin{Th}{Th8.2.2.4 (Gauss' criterion)}
    Suppose $\sum_{n=1}^{\infty} a_n$ is a positive series. If
    $$ \frac{a_n}{a_{n+1}} = 1 + \frac{1}{n} + \frac{\alpha}{n\ln n} + o\left(\frac{1}{n\ln n}\right), $$
    then
    \begin{compactenum}
        \item if $\alpha > 1$, then the series converges;
        \item if $\alpha < 1$, then the series diverges.
    \end{compactenum}
    \tcblower
    \textit{Pf}: Compare with the series $\sum_{n=1}^{\infty} \frac{1}{n(\ln n)^\alpha}$, using Th \{, ID: 8.2.1.1\}.
\end{Th}

\begin{Th}{Th8.3.1 (Cauchy's criterion)}
    Suppose $\sum_{n=1}^{\infty} a_n$ is a series. Then the series converges iff for any $\varepsilon>0$, there exists an $N\in\mathbb{N}^\ast$ such that
    $$ \left|\sum_{k=n+1}^{n+p} a_k\right| < \varepsilon $$
    holds for all $n>N$ and for all $p\in\mathbb{N}^\ast$.
    \tcblower
    \textit{Pf}: Obviously this is the Cauchy's criterion for the convergence of a sequence.
\end{Th}

\begin{Th}{Th8.3.2 (Leibniz's criterion)}
    Suppose $\sum_{n=1}^{\infty} a_n$ is a non-negative series. If the sequence $\{a_n\}$ is decreasing and $\lim\limits_{n\to\infty} a_n = 0$, then the series
    $$ \sum_{n=1}^{\infty} (-1)^{n} a_n $$
    converges.
\end{Th}

\begin{Rmk}{}
    \textcolor{Df}{A series $\sum_{n=1}^{\infty} (-1)^n a_n$ with $a_n\geq 0$ (or $a_n\leq 0$) is called an \textbf{alternating series}, and the alternating series $\sum_{n=1}^{\infty} (-1)^n a_n$ satisfying the conditions of Leibniz's criterion is called a \textbf{Leibniz series}.}
\end{Rmk}

\begin{Th}{Lma8.3.3.-2 (Abel's partial sum formula)}
    Suppose $\{a_n\}$ and $\{b_n\}$ ($n\in\mathbb{N}^\ast$) are two sequences in $\mathbb{R}$. Then:
    $$ \sum_{k=1}^{n} a_k b_k = A_n b_n - \sum_{k=1}^{n-1} A_k(b_{k+1}-b_k) $$
    holds for all $n\in\mathbb{N}^\ast$, where $A_n = \sum_{k=1}^{n} a_k$.
    \tcblower
    \textit{Pf}: Obvious. This is just the discrete version of the integration by parts.
\end{Th}

\begin{Th}{Lma8.3.3.-1 (Abel's bound for $\sum_{k=1}^{n} a_k b_k$)}
    Suppose $\{a_k\}$ and $\{b_k\}$ ($k=1,\cdots,n$) are two $n$-arrays in $\mathbb{R}$. If
    \begin{compactenum}
        \item $\{A_k\}$ is bounded by $|A_k|\leq M$ ($A_k\triangleq \sum_{i=1}^{k} a_i$) and
        \item $\{b_k\}$ is monotonic,
    \end{compactenum}
    then
    $$ \left|\sum_{k=1}^{n} a_k b_k\right| \leq M(|b_1|+2|b_n|). $$
    \tcblower
    \textit{Pf}: Obvious by the Abel's partial sum formula.
\end{Th}

\begin{Th}{Th8.3.3.1 (Dirichlet's criterion)}
    Suppose $\{a_n\}$ and $\{b_n\}$ ($n\in\mathbb{N}^\ast$) are two sequences in $\mathbb{R}$. Denote $A_n = \sum_{k=1}^{n} a_k$. If
    \begin{compactenum}
        \item $\{A_n\}$ is bounded and
        \item $\{b_n\}$ monotonically converges to $0$,
    \end{compactenum}
    then the series $\sum_{n=1}^{\infty} a_n b_n$ converges.
    \tcblower
    \textit{Pf}: Obvious by the Lma \{, ID: 8.3.3.-1\}
\end{Th}

\begin{Th}{Th8.3.3.2 (Abel's criterion)}
    Suppose $\{a_n\}$ and $\{b_n\}$ ($n\in\mathbb{N}^\ast$) are two sequences in $\mathbb{R}$. Denote $A_n = \sum_{k=1}^{n} a_k$. If
    \begin{compactenum}
        \item $\{A_n\}$ converges and
        \item $\{b_n\}$ is monotonic and bounded,
    \end{compactenum}
    then the series $\sum_{n=1}^{\infty} a_n b_n$ converges.
    \tcblower
    \textit{Pf}: Obvious by the Dirichlet's criterion.
\end{Th}

\begin{Th}{Th8.4 (absolute convergence and conditional convergence)}
    Suppose $\sum_{n=1}^{\infty} a_n$ is a series. Then the convergence of $\sum_{n=1}^{\infty} |a_n|$ implies the convergence of $\sum_{n=1}^{\infty} a_n$, but the converse is false. \textcolor{Df}{The case that both series converge is called the \textbf{absolute convergence}; while the case that $\sum_{n=1}^{\infty} a_n$ converges but $\sum_{n=1}^{\infty} |a_n|$ diverges is called the \textbf{conditional convergence}}.
    \tcblower
    \textit{Pf}: 
    \begin{compactenum}
        \item $\sum_{n=1}^{\infty} |a_n|$ converges $\Rightarrow$ $\sum_{n=1}^{\infty} a_n $ converges: $|\sum_{k=n+1}^{n+p} a_k| \leq \sum_{k=n+1}^{n+p} |a_k| < \varepsilon$ by the Cauchy's criterion;
        \item $\sum_{n=1}^{\infty} |a_n|$ converges $\nLeftarrow$ $\sum_{n=1}^{\infty} a_n $ converges: let $a_n = (-1)^n/n$.
    \end{compactenum}
\end{Th}

\begin{Th}{Th8.4.1 (Riemann's rearrangement theorem)}
    \begin{compactenum}
        \item \textcolor{Df}{A bijection $\varphi: \mathbb{N}^\ast\rightarrow\mathbb{N}^\ast$ is called a \textbf{permutation} (or a \textbf{order}) of $\mathbb{N}^\ast$. \textbf{To rearrange a series } $\sum_{n=1}^{\infty} a_n$ \textbf{by the order $\varphi$} means to compute the sum of the new series $\sum_{n=1}^{\infty} a_{\varphi(n)}$.}
        \item For absolutely convergent series, any rearrangement does not affect the convergence and the exact sum of the series.
        \item For conditionally convergent series, the sum of the series can be changed to any real number, or even to $\infty$ or $-\infty$, by a suitable rearrangement. 
    \end{compactenum}
    \tcblower
    \textit{Pf}: 
    \begin{compactenum}
        \item First verify that rearrangement does not affect a non-negative series. Suppose $\{a_n\}$ is non-negative. Then for the partial sum 
        $$ S_n^\varphi = \sum_{k=1}^{n} a_{\varphi(k)} \leq \sum_{k=1}^{m(n)} a_k = S_{m(n)} \rightarrow S $$
        where $m(n) = \max\{\varphi(1), \cdots, \varphi(n)\}$. Then $S_n^\varphi$, and thus $\sum_{n=1}^{\infty} a_{\varphi(n)}$, converges to some $S^\varphi\leq S$. In the same way we quickly get $S\leq S^\varphi$ so that $S = S^\varphi$, as $S_n = (S^\varphi)_n^{\varphi^{-1}}$.
        \item Next extend to any absolutely convergent series $\sum_{n=1}^{\infty} a_n$. Consider the following two non-negative series:
        $$ a_n^+ = \max\{a_n, 0\}, \quad a_n^- = \max\{-a_n, 0\} $$
        then $a_n = a_n^+ - a_n^-$ and $|a_n| = a_n^+ + a_n^-$. Then the rearrangement of $\sum_{n=1}^{\infty} a_n$ corresponds to the same rearrangement of $\sum_{n=1}^{\infty} a_n^+$ and $\sum_{n=1}^{\infty} a_n^-$.
        \item Finally consider the conditionally convergent $\sum_{n=1}^{\infty} a_n$. As
        $$ a_n^+ = \frac{a_n + |a_n|}{2}, \quad a_n^- = \frac{|a_n| - a_n}{2}, $$
        where $\sum_{n=1}^{\infty} a_n$ converges but $\sum_{n=1}^{\infty} |a_n|$ diverges, both of $\sum_{n=1}^{\infty} a_n^+$ and $\sum_{n=1}^{\infty} a_n^-$ diverge to $\infty$. \\ 
        Let us say $a_n\neq 0$ for convenience. Then the $a_n^+$ and $a_n^-$ are both positive. And let $\{a_{k(n)}\}$ and $\{a_{l(n)}\}$ be the corresponding subsequences of $a_n$ to $\{a_n^+\}$ and $\{a_n^-\}$ respectively, so that $a_{k(n)}>0$ and $a_{l(n)}<0$. Then achieve $\sum_{n=1}^{\infty} a_{\varphi(n)} = S\in (-\infty, \infty)$ inductively by:
        \begin{compactenum}
            \item[(i)] if $S^\varphi_n \leq S$, let $a_{\varphi(n+1)}$ be the next term from $\{a_{k(n)}\}$ that has not been used;
            \item[(ii)] if $S^\varphi_n > S$, let $a_{\varphi(n+1)}$ be the next term from $\{a_{l(n)}\}$ that has not been used. 
        \end{compactenum}
        And achieve $\sum_{n=1}^{\infty} a_{\varphi(n)} = \infty$ (similar for $-\infty$) inductively by:
        \begin{compactenum}
            \item[(i)] Set an ininial ``goal'' $S = 0$;
            \item[(ii)] if $S^\varphi_n < S$, let $a_{\varphi(n+1)}$ be the next term from $\{a_{k(n)}\}$ that has not been used;
            \item[(iii)] if $S^\varphi_n \geq S$, let $a_{\varphi(n+1)}$ be the next term from $\{a_{l(n)}\}$ that has not been used, and set $S = S + 1$.
        \end{compactenum}
        this way we can by rearrangement arbitrarily change the sum of the series (verify the details by yourself).
    \end{compactenum}
\end{Th}

\begin{Df}{Df8.5 (product of series)}
    For two series $\sum_{n=1}^{\infty} a_n$ and $\sum_{n=1}^{\infty} b_n$, their \textbf{product} has many definitions. 
    \begin{compactenum}
        \item Their \textbf{Cauchy product} is defined by 
        $$\Bigg[\left(\sum_{n=1}^{\infty} a_n\right) \left(\sum_{n=1}^{\infty} b_n\right)\Bigg]_\mathsf{C} \xlongequal{\;\vartriangle \;} \sum_{n=1}^{\infty} \left(\sum_{k=1}^{n} a_k b_{n+1-k}\right);$$
        \item Their \textbf{block product} is defined by
        $$\Bigg[\left(\sum_{n=1}^{\infty} a_n\right) \left(\sum_{n=1}^{\infty} b_n\right)\Bigg]_\mathsf{B} \xlongequal{\;\vartriangle \;} \lim\limits_{n\to\infty} \left(\sum_{k=1}^{n} a_k \sum_{k=1}^{n} b_k\right);$$
    \end{compactenum}
\end{Df}

\begin{Rmk}{}
    To define the product of two series $\sum_{n=1}^{\infty} a_n$ and $\sum_{n=1}^{n} b_n$, is just to define an order to sum all the terms $\{a_n b_m: n,m\in\mathbb{N}^\ast\}$. Then the convergence of the product matters.
\end{Rmk}

\begin{Th}{Th8.5.1 (both of the multiplied series converges $\Rightarrow$ the block product converges)}
    $$ \Bigg[\left(\sum_{n=1}^{\infty} a_n\right) \left(\sum_{n=1}^{\infty} b_n\right)\Bigg]_\mathsf{B} = \left(\sum_{n=1}^{\infty} a_n\right) \left(\sum_{n=1}^{\infty} b_n\right), $$
    if both $\sum_{n=1}^{\infty} a_n$ and $\sum_{n=1}^{\infty} b_n$ converge.
    \tcblower
    \textit{Pf}: Obvious.
\end{Th}

\begin{Th}{Th8.5.2 (Cauchy's theorem) (both of the multiplied series absolutely converge $\Rightarrow$ the product by any order converges to the same sum)}
    If both $\sum_{n=1}^{\infty} a_n$ and $\sum_{n=1}^{\infty} b_n$ are absolutely convergent, then by any \textcolor{Df}{order $(\varphi, \psi): \mathbb{N}^\ast \rightarrow \mathbb{N}^\ast\times\mathbb{N}^\ast$ (``order'' here, still, means that $(\varphi, \psi)$ is a bijection)}, the product $ \sum_{n=1}^{\infty} a_{\varphi(n)} b_{\psi(n)} $ absolutely converges and
    $$ \sum_{n=1}^{\infty} a_{\varphi(n)} b_{\psi(n)} = \left(\sum_{n=1}^{\infty} a_n\right) \left(\sum_{n=1}^{\infty} b_n\right). $$
    \tcblower
    \textit{Pf}: For the partial sum $ \sum_{k=1}^{n} a_{\varphi(k)} b_{\psi(k)} $, let $N = \max\{\varphi(1), \cdots, \varphi(n), \psi(1), \cdots, \psi(n)\}$. Then
    $$ \sum_{k=1}^{n} |a_{\varphi(k)} b_{\psi(k)}| \leq \sum_{i=1}^{N} \sum_{j=1}^{N} |a_i| |b_j| = \left(\sum_{i=1}^{N} |a_i|\right) \left(\sum_{j=1}^{N} |b_j|\right) \leq \left(\sum_{n=1}^{\infty} |a_n|\right) \left(\sum_{n=1}^{\infty} |b_n|\right) < \infty. $$
    And thus the product by the order $(\varphi, \psi)$ absolutely converges, and thus the block product — which is the series obtained by combining the terms from $\sum_{n=1}^{\infty} a_{\varphi(n)} b_{\psi(n)}$ — absolutely converges to the same sum.
    $$ \sum_{n=1}^{\infty} a_{\varphi(n)} b_{\psi(n)} = \Bigg[\left(\sum_{n=1}^{\infty} a_n\right) \left(\sum_{n=1}^{\infty} b_n\right)\Bigg]_\mathsf{B} = \left(\sum_{n=1}^{\infty} a_n\right) \left(\sum_{n=1}^{\infty} b_n\right). $$
\end{Th}

\begin{Th}{Th8.5.3 (Mertens' theorem) (both converges and one absolutely converges $\Rightarrow$ Cauchy product converges)}
    $$ \Bigg[\left(\sum_{n=1}^{\infty} a_n\right) \left(\sum_{n=1}^{\infty} b_n\right)\Bigg]_\mathsf{C} = \left(\sum_{n=1}^{\infty} a_n\right) \left(\sum_{n=1}^{\infty} b_n\right), $$
    if both $\sum_{n=1}^{\infty} a_n$ and $\sum_{n=1}^{\infty} b_n$ converge, and one of them absolutely converges.
    \tcblower
    \textit{Pf}: Say $\sum_{n=1}^{\infty} a_n$ absolutely converges. Denote $A_n = \sum_{k=1}^{n} a_k$, $B_n = \sum_{k=1}^{n} b_k$, $C_n = \sum_{k=1}^{n} \sum_{i=1}^{k}a_i b_{k+1-i}$ and $$\lim\limits_{n\to \infty} A_n = A \qquad \lim\limits_{n\to \infty} B_n = B \qquad \lim\limits_{n\to\infty} C_n = C.$$ Then it is to show that $C = AB$. Since $ C_n = a_1 B_n + \cdots + a_n B_1, $ we have
    $$ C_n - A_n B = a_1(B_n - B) + \cdots + a_n(B_1 - B) \triangleq a_1 \beta_n + \cdots + a_n \beta_1 \triangleq \gamma_n, $$
    and it is to show that $\gamma_n \rightarrow 0$, given that $\beta_n \rightarrow 0$. For any $\varepsilon>0$, there is some $N$ s.t. $|\beta_n| < \varepsilon$ for all $n>N$. Then for $n>N$,
    $$ 
    \begin{aligned}
        |\gamma_n| & \leq \left(|a_1| |\beta_n| + \cdots + |a_{n-N}| |\beta_{N+1}|\right) + \left(|a_{n+1-N}| |\beta_{N}| + \cdots + |a_n| |\beta_1|\right) \\
        & < \varepsilon A + |\beta_N| |a_{n+1-N}| + \cdots + |\beta_1| |a_n| 
    \end{aligned}
    $$
    then fix $N$ and let $n\rightarrow \infty$, we have $\limsup\limits_{n\to\infty} |\gamma_n| \leq \varepsilon A$, namely $\gamma_n \rightarrow 0$. Thus $C = AB$.
\end{Th}

\begin{Th}{Th8.5.4 (Abel's theorem) (all converge $\Rightarrow$ equality)}
    $$ \Bigg[\left(\sum_{n=1}^{\infty} a_n\right) \left(\sum_{n=1}^{\infty} b_n\right)\Bigg]_\mathsf{B} = \Bigg[\left(\sum_{n=1}^{\infty} a_n\right) \left(\sum_{n=1}^{\infty} b_n\right)\Bigg]_\mathsf{C} = \left(\sum_{n=1}^{\infty} a_n\right) \left(\sum_{n=1}^{\infty} b_n\right), $$
    if $\sum_{n=1}^{\infty} a_n$, $\sum_{n=1}^{\infty} b_n$ and their Cauchy product all converge.
    \tcblower
    \textit{Pf}: Continue to use the notations in the proof of Th \{, ID: 8.5.3\}. Then
    $$ 
    \begin{aligned}
        C_1 &= a_1 B_1, \\
        C_2 &= a_1 B_2 + a_2 B_1, \\
        C_3 &= a_1 B_3 + a_2 B_2 + a_3 B_1, \\
        & \cdots \\
    \end{aligned}
    $$
    and thus 
    $$ \frac{1}{N} \sum_{n=1}^{N} C_n = \frac{1}{N} \sum_{n=1}^{N} A_n B_{N+1-n}. $$
    Then complete the proof by letting $N\rightarrow \infty$ (as shown in the Ex \{, ID: 1.1.0.2\}).
\end{Th}

\end{document}