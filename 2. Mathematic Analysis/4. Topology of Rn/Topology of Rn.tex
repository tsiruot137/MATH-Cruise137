\documentclass{article}

    \usepackage{xcolor}
    \definecolor{pf}{rgb}{0.4,0.6,0.4}
    \usepackage[top=1in,bottom=1in, left=0.8in, right=0.8in]{geometry}
    \usepackage{setspace}
    \setstretch{1.2} 
    \setlength{\parindent}{0em}

    \usepackage{paralist}
    \usepackage{cancel}

    \usepackage{ctex}
    \usepackage{amssymb}
    \usepackage{amsmath}

    \usepackage{tcolorbox}
    \definecolor{Df}{RGB}{0, 184, 148}
    \definecolor{Th}{RGB}{9, 132, 227}
    \definecolor{Rmk}{RGB}{215, 215, 219}
    \definecolor{P}{RGB}{154, 13, 225}
    \newtcolorbox{Df}[2][]{colbacktitle=Df, colback=white, title={\large\color{white}#2},fonttitle=\bfseries,#1}
    \newtcolorbox{Th}[2][]{colbacktitle=Th, colback=white, title={\large\color{white}#2},fonttitle=\bfseries,#1}
    \newtcolorbox{Rmk}[2][]{colbacktitle=Rmk, colback=white, title={\large\color{black}{Remarks}},fonttitle=\bfseries,#1}

    \title{\LARGE \textbf{Topology of $\mathbb{R}^n$}}
    \author{\large Jiawei Hu}

\begin{document}
\maketitle

This is the 4th chapter of Mathematical Analysis, which is about \textbf{Topology of $\mathbb{R}^n$}. By the way, we now pre-claim some commonly-used notations and terms:
\begin{Df}{Notations and Terms}
    \begin{compactenum}
        \item $\mathbb{R}$: the set of the real numbers; $\mathbb{R}_\infty = \mathbb{R}\cup\{-\infty, \infty\}$;
        \item $\mathbb{R}^n$: without extra specification, $n\in\mathbb{N}^\ast$; 
        \item You may see some statements like ``this Df/Th is merely an extension of the previous one'', which means that the current Df/Th can be reduced to the previous one on some conditions.
        \item An agreement for the length of a list: if we write $a_1, \dots, a_n$, then we indicate that $n$ is finite and that $n\geq 1$; if we write $a_0, \dots, a_n$, then we indicate that $n$ is finite and that $n\geq 0$.
        \item Keep coincident in the notions and notations of functions with the chapter 1 of course 0, including the ones of domain, range, restriction, image, pre-image, inverse and composition. Specifically for a function $f: A\rightarrow B$ and some sets $E\subseteq A$ and $F\subseteq B$, the image of $E$ and the pre-image of $F$ under $f$ are just:
        $$f[E] = \{f(x): x\in E\},\quad f^{-1}[F] = \{x\in A: f(x)\in F\}$$
        \item For the existence of a limit, if we have used the symbol $\lim\limits_{x\to x_0} f(x)$ in an expression (such as an equality, an inequality or some expressions involving some other numbers), then without explicitly specification, we imply that the limit exists (``exist'' means finite according to the chapter 1).
        \item An interval is a subset of $\mathbb{R}$ of one of the following forms: $(a,b)$, $[a,b]$, $(a,b]$, $[a,b)$, $(a, \infty)$, $(-\infty, b)$, $(-\infty, \infty)$, where $a, b\in\mathbb{R}$ and $a<b$. Please identify whether $(a,b)$ stands for a tuple or an open interval from the context by yourself.
        \item Monotonic function: ``increasing'' for ``$\geq$'', ``strictly increasing'' for ``$>$''.
        \item The inner product and norm in $\mathbb{R}^n$ are the typical ones: $\langle \pmb{x}, \pmb{y}\rangle = x_1y_1 + \dots + x_ny_n$, $\Vert \pmb{x}\Vert = \sqrt{\langle \pmb{x}, \pmb{x}\rangle}$.
        \item $E^c$: Let $E\subseteq\mathbb{R}^n$. Then $E^c\triangleq \mathbb{R}^n\setminus E$.
        \item A set of sets is called a collection or a family.
    \end{compactenum}
\end{Df}

Here is the \textbf{Quick Search} for this chapter:
\begin{Th}{Quick Search}
    \begin{compactdesc}
        \item (4.2.*): Open sets and closed sets.
        \item (4.3.*): Sequentially compact $\Leftrightarrow$ compact $\Leftrightarrow$ bounded closed.
        \item (4.4.*): Connected sets.
        \item (4.5.*): Continuity of $n$-real functions.
        \item (4.6.*): Continuity of $n$-real $m$-functions.
    \end{compactdesc}
\end{Th}

When I marched this point, I add the \textbf{Pre \& Post Exercise} pattern to the writing, which is shown as the following Th block:

\begin{Th}{Some theorem}
    The text.
    \tcblower
    \textcolor{P}{\textit{Analytically}: The pre-exercise part.}\\
    \textit{Pf}: The formal proof.\\
    \textcolor{P}{\textit{Thoughtfully}: The post-exercise part.}
\end{Th}

Then with everything prepared, here we go.

\begin{Df}{Df4.1 (limit of sequence) (extension of Df \{, ID: 1.1\})}
    Suppose $\{\pmb{x}_n: n\in\mathbb{N}\}$ is a sequence in $\mathbb{R}^n$ and $\pmb{l}\in\mathbb{R}^n$. Then $\pmb{l}$ is called the \textbf{limit} of the sequence $\{\pmb{x}_n\}$ (or $\{\pmb{x}_n\}$ \textbf{converges} to $\pmb{l}$), denoted by $\lim\limits_{n\to\infty} \pmb{x}_n = \pmb{l}$, if
    $$\forall \varepsilon > 0, \exists N\in\mathbb{N}^\ast, \forall n\geq N, \Vert \pmb{x}_n - \pmb{l}\Vert < \varepsilon.$$
\end{Df}

\begin{Rmk}{}
    We can easily verify the arithmetics:
    \textcolor{Th}{Suppose $\{\pmb{x}_n\}$ and $\{\pmb{y}_n\}$ are two sequences in $\mathbb{R}^n$, both with their limits in $\mathbb{R}^n$. Then:
    \begin{compactenum}
        \item $\lim\limits_{n\to\infty} (\pmb{x}_n\pm\pmb{y}_n) = \lim\limits_{n\to\infty} \pmb{x}_n \pm \lim\limits_{n\to\infty} \pmb{y}_n$;
        \item $\lim\limits_{n\to\infty} c\pmb{x}_n = c\lim\limits_{n\to\infty} \pmb{x}_n$ for any $c\in\mathbb{R}$.
    \end{compactenum}}
    And we naturally have:
    \textcolor{Th}{Suppose $\{\pmb{x}_n\}$ is a sequence in $\mathbb{R}^n$ and $\pmb{l}\in\mathbb{R}^n$. Then $\lim\limits_{n\to\infty} \pmb{x}_n = \pmb{l}$ iff every subsequence of $\{\pmb{x}_n\}$ converges to $\pmb{l}$.}
\end{Rmk}

\begin{Th}{Th4.1.1 (points sequence converges by components)}
    Suppose $\{\pmb{x}_n\}$ is a sequence in $\mathbb{R}^n$ and $\pmb{l}\in\mathbb{R}^n$. Then $\lim\limits_{n\to\infty} \pmb{x}_n = \pmb{l}$ iff $\lim\limits_{n\to\infty} x^{(i)}_n = l^{(i)}$ for each $i = 1, \dots, n$, where the superscript ``$(i)$'' denotes the $i$-th component of a vector.
    \tcblower
    \textit{Pf}: Obvious.
\end{Th}

\begin{Df}{Df4.1.2 (Cauchy sequence) (extension of Df \{, ID: 1.6\})}
    Suppose $\{\pmb{x}_n\}$ is a sequence in $\mathbb{R}^n$. Then $\{\pmb{x}_n\}$ is called a \textbf{Cauchy sequence} if
    $$\forall \varepsilon > 0, \exists N\in\mathbb{N}^\ast, \left(\forall m,n\geq N\right), \Vert \pmb{x}_m - \pmb{x}_n\Vert < \varepsilon.$$
\end{Df}

\begin{Th}{Th4.1.2.1 (Cauchy criterion) (extension of Th \{, ID: 1.7\})}
    Suppose $\{\pmb{x}_n\}$ is a sequence in $\mathbb{R}^n$. Then $\{\pmb{x}_n\}$ has a limit in $\mathbb{R}^n$ iff $\{\pmb{x}_n\}$ is a Cauchy sequence.
    \tcblower
    \textit{Pf}: Obvious if we reduce it to the 1-dimensional case, aware of the ``convergence by components''.
\end{Th}

\begin{Df}{Df4.1.3.-1 (bounded set)}
    Suppose $E\subseteq\mathbb{R}^n$. Then $E$ is called \textbf{bounded} if there is some positive real number $R$ s.t. $\Vert\pmb{x}\Vert\leq R$ for every $\pmb{x}\in E$. Otherwise, $E$ is called \textbf{unbounded}.
\end{Df}

\begin{Th}{Th4.1.3 (Bolzano-Weierstrass theorem) (extension of Th \{, ID: 1.5\})}
    Suppose $\{\pmb{x}_n\}$ is a bounded sequence in $\mathbb{R}^n$. Then $\{\pmb{x}_n\}$ has a convergent subsequence.
    \tcblower
    \textit{Pf}: Obvious if we reduce it to the 1-dimensional case, aware of the ``convergence by components''. Actually, for the 1st component of the sequence, it is bounded and thus has a convergent subsequence; then for the 2nd component of this subsequence, it is bounded and thus has a convergent sub-subsequence; and so on.
\end{Th}

\begin{Df}{Df4.2 (open ball) (extension of Df \{, ID: 2.1.-1\})}
    Suppose $\pmb{a}\in\mathbb{R}^n$ and $r$ is a positive real number. Then:
    \begin{compactenum}
        \item $B_r(\pmb{a})\triangleq \{\pmb{x}\in\mathbb{R}^n: \Vert \pmb{x}-\pmb{a}\Vert < r\}$ is called the \textbf{open ball} of radius $r$ centered at $\pmb{a}$.
        \item $\bar{B}_r(\pmb{a})\triangleq \{\pmb{x}\in\mathbb{R}^n: \Vert \pmb{x}-\pmb{a}\Vert \leq r\}$ is called the \textbf{closed ball} of radius $r$ centered at $\pmb{a}$.
        \item $\check{B}_r(\pmb{a})\triangleq \{\pmb{x}\in\mathbb{R}^n: 0 < \Vert \pmb{x}-\pmb{a}\Vert < r\}$ is called the \textbf{hollow open ball} of radius $r$ centered at $\pmb{a}$.
    \end{compactenum}
\end{Df}

\begin{Df}{Df4.2.1 (open set)}
    Suppose $E\subseteq\mathbb{R}^n$. Then 
    \begin{compactenum}
        \item If $\pmb{a}\in E$, then $\pmb{a}$ is called an \textbf{interior point} of $E$ if there exists an open ball $B_r(\pmb{a})$ such that $B_r(\pmb{a})\subseteq E$;
        \item The set of all interior points of $E$ is called the \textbf{interior} of $E$, denoted by $E^\circ$ (or $\text{int}(E)$);
        \item $E$ is called an \textbf{open set} if $E^\circ = E$;
        \item $E$ is called a \textbf{closed set} if $E^c$ is open.
    \end{compactenum}
\end{Df}

\begin{Rmk}{}
    We find that \textcolor{Th}{an open ball is an open set, a hollow open ball is an open set, a closed ball is a closed set}. Here the ``open'' and ``closed'' are not disjoint, since \textcolor{Th}{$\mathbb{R}^n$ and $\varnothing$ are both open and closed}, and \textcolor{Th}{there are sets that are neither open nor closed (e.g. $(0,1]$).}
\end{Rmk}

\begin{Th}{Th4.2.2 (union and intersection of open sets and closed sets)}
    In $\mathbb{R}^n$, we have:
    \begin{compactenum}
        \item The union of any collection of open sets is open;
        \item The intersection of finitely many open sets is open;
        \item The union of finitely many closed sets is closed;
        \item The intersection of any collection of closed sets is closed.
    \end{compactenum}
    \tcblower
    \textit{Pf}: (1) is natural. For (2), suppose $E_1, \dots, E_n$ are open. Then for any $\pmb{a}\in E_1\cap\dots\cap E_n$, there exist $r_1, \dots, r_n$ such that $B_{r_1}(\pmb{a})\subseteq E_1, \dots, B_{r_n}(\pmb{a})\subseteq E_n$. Let $r = \min\{r_1, \dots, r_n\}$, so that $B_r(\pmb{a})\subseteq E_1\cap\dots\cap E_n$. (3) and (4) are natural by De Morgan's laws. 
\end{Th}

\begin{Rmk}{}
    Let us talk about why (2) is not true for an infinite collection of open sets. Consider the collection $\{B_{1/n}(0): n\in\mathbb{N}^\ast\}$. Then $\bigcap\limits_{n=1}^\infty B_{1/n}(0) = \{0\}$, which is not open.
\end{Rmk}

\begin{Df}{Df4.2.3 (limit point) (focal point) (extension of Df \{, ID: 2.1.-2\})}
    Suppose $E\subseteq\mathbb{R}^n$ and $\pmb{a}\in\mathbb{R}^n$. Then $\pmb{a}$ is called a \textbf{limit point} (or \textbf{focal point}) of $E$ if $E$ intersects every hollow open ball centered at $\pmb{a}$.
\end{Df}
    
\begin{Rmk}{}
    The equivalent definition of a limit point is: \textcolor{Df}{Suppose $E\subseteq\mathbb{R}^n$ and $\pmb{a}\in\mathbb{R}^n$. Then $\pmb{a}$ is a limit point of $E$ if there is a sequence $\{\pmb{x}_n\}$ in $E\setminus\{\pmb{a}\}$ s.t. $\lim\limits_{n\to\infty} \pmb{x}_n = \pmb{a}$.}\\
    \textcolor{Df}{Let $E\subseteq\mathbb{R}^n$. Then the set of all limit points of $E$, denoted by $E^\prime$, is called the \textbf{derived set} of $E$. If a point $\pmb{a}$ is not a limit point of $E$, then $\pmb{a}$ is called an \textbf{isolated point} of $E$. Call the set $E\cup E^\prime$ the \textbf{closure} of $E$, denoted by $\overline{E}$.} We can then see these natural facts: 
    \textcolor{Th}{The following $E$, $r$ and $\pmb{a}$ are all arbitrary:
    \begin{compactenum}
        \item $E^\prime$ and $\overline{E}$ are both closed;
        \item $[B_r(\pmb{a})]^\prime = [\check{B}_r(\pmb{a})]^\prime = [\bar{B}_r(\pmb{a})]^\prime = \bar{B}_r(\pmb{a})$;
        \item $E^\circ$ is the largest open set contained in $E$, $\overline{E}$ is the smallest closed set containing $E$.
    \end{compactenum}}
    \textcolor{Df}{For a subset $E$ of $\mathbb{R}^n$, we call those points in $(E^c)^\circ$ the \textbf{exterior points} of $E$, the set of all outer points of $E$ the \textbf{exterior} of $E$, denoted by $\text{ext}(E)$; call those points that are neither interior points nor exterior points of $E$ the \textbf{boundary points} of $E$, and the set of all boundary points of $E$ the \textbf{boundary} of $E$, denoted by $\partial E$.} 
\end{Rmk}

\begin{Th}{Th4.2.4 ($E$ is closed iff $E = \overline{E}$)}
    Suppose $E\subseteq\mathbb{R}^n$. Then $E$ is closed iff $E = \overline{E}$.
    \tcblower
    \textit{Pf}: Obvious.
\end{Th}

\begin{Th}{Ex4.2.5 ($\varnothing$ and $\mathbb{R}^n$ are the only two sets that are both open and closed)}
    In $\mathbb{R}^n$, are $\varnothing$ and $\mathbb{R}^n$ the only two sets that are both open and closed?
    \tcblower
    \textit{Answer}: Yes.\\
    \textit{Solution}: If some set $E$ (neither $\varnothing$ nor $\mathbb{R}^n$) is both open and closed, then both $E$ and $E^c$ are open. To search for idea and simplify the thought, we first consider $E$ to be an open ball. Then we have known that $E^c$ cannot be open, which is due to the boundary of the ball $E$. \\
    Then we think of that the an arbitrary set $E$ cannot be both open and closed if it has some boundary points. Actually, if $E$ is both open and closed, and $E$ has some boundary points $\pmb{x}$, then $\pmb{x}\notin E^\circ = E$ and $\pmb{x}\notin (E^c)^\circ = E^c$, impossible. \\
    Then how to prove that except for $\varnothing$ and $\mathbb{R}^n$, any other subset $E$ of $\mathbb{R}^n$ has some boundary points? Since $E$ and $E^c$ are both non-empty, we find a point $\pmb{x}\in E$ and another point $\pmb{y}\in E^c$. As we can imagine, if moving from $\pmb{x}$ to $\pmb{y}$ along the line segment gradually, from $E^\circ$ to $(E^c)^\circ$, we must pass through some boundary points somewhere. \\
    Now is the formal proof. Suppose $\varnothing\subsetneq E\subsetneq\mathbb{R}^n$. Then there are some $\pmb{x}\in E$ and $\pmb{y}\in E^c$. Consider the set
    $$\Lambda = \{\lambda\in [0,1]: \pmb{x}+\lambda(\pmb{y}-\pmb{x})\in E^\circ\}.$$
    If $\Lambda = \varnothing$, then $\pmb{x} = \pmb{x}+0(\pmb{y}-\pmb{x})\notin E^\circ$ and thus $\pmb{x}\in\partial E$. Suppose $\Lambda\neq\varnothing$, then $0\leq\inf\Lambda\leq\sup\Lambda\leq 1$. Consider the critical point $\pmb{z} = \pmb{x}+\lambda_0(\pmb{y}-\pmb{x})$ where $\lambda_0 = \inf\Lambda$. Then we can show that $\pmb{z}\in\partial E$:
    \begin{compactenum}
        \item $\pmb{z}\notin E^\circ$: If otherwise $\pmb{z}\in E^\circ$, then there is some $B_r(\pmb{z})\subseteq E$, and thus $\pmb{z}^\prime\triangleq \pmb{z}+\frac{r}{2}\frac{\pmb{y}-\pmb{x}}{\Vert\pmb{y}-\pmb{x}\Vert}\in E^\circ$, which contradicts that $\lambda_0$ is an upper bound of $\Lambda$;
        \item $\pmb{z}\notin (E^c)^\circ$: If otherwise $\pmb{z}\in (E^c)^\circ$, then there is some $B_r(\pmb{z})\subseteq E^c$. But since $\lambda_0$ is the least upper bound of $\Lambda$, there is some $\lambda^\prime\in\Lambda$ s.t. $\lambda^\prime>\lambda_0-\frac{r}{\Vert\pmb{y}-\pmb{x}\Vert}$ (namely $\pmb{z}^\prime \triangleq \pmb{x}+\lambda^\prime(\pmb{y}-\pmb{x})\in E^\circ\subseteq E$), which contradicts that $\pmb{z}^\prime\in B_r(\pmb{z})\subseteq E^c$.
    \end{compactenum}
\end{Th} 

\begin{Rmk}{}
    This exercise obtain a byproduct: \textcolor{Th}{Every non-empty proper subset of $\mathbb{R}^n$ has some boundary points.}
\end{Rmk}

\begin{Df}{Df4.2.6.-1 (diameter)}
    Suppose $E\subseteq\mathbb{R}^n$. Then the \textbf{diameter} of $E$, denoted by $\text{diam}(E)$, is defined as
    $$\text{diam}(E) = \sup\{\Vert\pmb{x}-\pmb{y}\Vert: \pmb{x}, \pmb{y}\in E\}.$$
\end{Df}

\begin{Th}{Th4.2.6 (theorem of nested closed sets) (extension of Th \{, ID: 1.4\})}
    Suppose $\{F_i: i\in\mathbb{N}\}$ is a sequence of non-empty closed sets in $\mathbb{R}^n$ and $F_1\supseteq F_2\supseteq F_3\supseteq \cdots$. Suppose also $\text{diam}(F_i)\rightarrow 0$. Then there is exactly one point in $\bigcap\limits_{i=1}^\infty F_i$.
    \tcblower
    \textit{Pf}: Since each $F_i$ is non-empty, find some $\pmb{x}_i\in F_i$ for each $i$. Clearly the sequence $X_i\triangleq\{\pmb{x}_i, \pmb{x}_{i+1}, \dots\}$ is in $F_i$. Since $\text{diam}(F_i)\rightarrow 0$, each $X_i$ is a Cauchy sequence, and different $X_i$ converge to some common $\pmb{x}$. Then for each $i$,
    \begin{compactenum}
        \item If all terms (or all terms after some term) in $X_i$ are $\pmb{x}$, then $\pmb{x}\in F_i$;
        \item If there are always some term in $X_i$ that is not $\pmb{x}$, then choose these points as a subsequence $Y_i$ of $X_i$, and thus $Y_i$ is a sequence in $F_i\setminus\{\pmb{x}\}$ convergent to $\pmb{x}$, indicating that $\pmb{x}\in F_i^\prime\subseteq F_i$;
    \end{compactenum}
    that is, $\pmb{x}\in F_i$. Hence $\pmb{x}\in\bigcap\limits_{i=1}^\infty F_i$.\\
    Easy to verify that $\bigcap\limits_{i=1}^\infty F_i$ contains only $\pmb{x}$.
\end{Th}

\begin{Df}{Df4.3.1 (sequentially compact sets (列紧集))}
    Suppose $E\subseteq\mathbb{R}^n$. Then $E$ is said to be sequentially compact if every sequence in $E$ has some subsequence who converges to some limit in $E$.
\end{Df}

\begin{Th}{Th4.3.1.1 (sequentially compact sets are exactly bounded closed sets)}
    Suppose $E\subseteq\mathbb{R}^n$. Then $E$ is sequentially compact iff $E$ is a bounded closed set.
    \tcblower
    \textit{Pf}: Obvious.
\end{Th}

\begin{Df}{Df4.3.2.-1 (open cover (开覆盖))}
    Suppose $E\subseteq\mathbb{R}^n$ and $\{A_\lambda: \lambda\in\Lambda\}$ is a collection of open sets in $\mathbb{R}^n$. Then $\{A_\lambda\}$ is called an \textbf{open cover} of $E$ if $E\subseteq\bigcup\limits_{\lambda\in\Lambda} A_\lambda$.
\end{Df}

\begin{Df}{Df4.3.2 (compact sets (紧致集))}
    Suppose $E\subseteq\mathbb{R}^n$. Then $E$ is said to be compact if: \\
    For every open cover $\{A_\lambda: \lambda\in\Lambda\}$ of $E$, there are finitely many $\lambda_1, \dots, \lambda_m$ s.t. $E\subseteq \bigcup_{i=1}^m A_{\lambda_i}$.
\end{Df}

\begin{Rmk}{}
    In a word, a compact set is a set such that every open cover of it has a finite subcover. In $\mathbb{R}^n$, we have defined a more general concept of open cover than that in $\mathbb{R}$, the latter of which is just a collection of open intervals. And with this more general concept, we can extend the finite-covering theorem (or the Heine-Borel theorem) in $\mathbb{R}$ to $\mathbb{R}^n$ as follows.
\end{Rmk}

\begin{Th}{Th4.3.2.1 (compact sets are exactly bounded closed sets) (finite-covering theorem) (Heine-Borel theorem)}
    Suppose $E\subseteq\mathbb{R}^n$. Then $E$ is compact iff $E$ is a bounded closed set.
    \tcblower
    \textit{Pf}: 
    \begin{compactenum}
        \item ``only if'': Suppose $E$ is compact. Then $E$ is obviously bounded as we can use finitely many open balls in the open cover $\{B_N(\pmb{0}): N\in\mathbb{N}^\ast\}$ to cover $E$. To prove that $E$ is closed is to prove that $E^c$ is open, namely, to find an open ball centered at $\pmb{y}$ for each point $\pmb{y}\in E^c$. Since for every point $\pmb{x}\in E$, there is an appropriate radius $r(\pmb{x})$ s.t. $B_{r(\pmb{x})}(\pmb{x})\cap B_{r(\pmb{x})}(\pmb{y}) = \varnothing$, we collect all these balls to form an open cover
        $$\{B_{r(\pmb{x})}(\pmb{x}): \pmb{x}\in E\}$$
        of $E$. Then there is an finite subcover $\{B_{r_1}(\pmb{x}_1), \cdots, B_{r_m}(\pmb{x}_m)\}$ of $E$. Hence we can then choose the minimum radius $r = \min\{r_1, \dots, r_m\}$ to ensure that $B_r(\pmb{y})\cap \bigcup_{i=1}^m B_{r_i}(\pmb{x}_i) = \varnothing$, which means that $B_r(\pmb{y})\subseteq E^c$.
        \item ``if'': Obviously we need to prove with contradiction. Suppose $E$ is bounded and closed, but there is an open cover $\{A_\lambda: \lambda\in\Lambda\}$ of $E$ that has no finite subcover. \\
        To search for contradiction, we can try to prove that $E$ is not closed, i.e., to find a point $\pmb{x}$ s.t. $\pmb{x}\in E^\prime$ but $\pmb{x}\notin E$ (of course this is just a try which is not necessary the final solution). Then it naturally comes to the nested closed sets theorem since this theorem is powerful in finding some singular point. Then we just assume that $E\subseteq \mathbb{R}^2$ for simplicity. \\
        First find a Rectangle $H_1$ s.t. $H_1\supseteq E$. Then divide $H_1$ into 4 equal rectangles (just cut $H_1$ with a cross), and find one of them, say $H_2$, s.t. $H_2\cap E$ cannot be covered by finitely many $A_\lambda$. Then divide $H_2$ into 4 equal rectangles, and find one of them, say $H_3$, s.t. $H_3\cap E$ cannot be covered by finitely many $A_\lambda$. And so on. Then we can find a sequence of nested closed sets $(H_1\cap E)\supseteq (H_2\cap E)\supseteq (H_3\cap E)\supseteq \cdots$ s.t. 
        \begin{compactenum}
            \item $\text{diam}(H_i\cap E)\rightarrow 0$;
            \item $H_i\cap E$ cannot be covered by finitely many $A_\lambda$ (let alone $H_i$).
        \end{compactenum}
        Then there is exactly one point $\pmb{x}\in\bigcap_{i=1}^\infty (H_i\cap E) = E\cap \bigcap_{i=1}^\infty H_i$. Since $\pmb{x}\in E$, there is some $A_{\lambda_0}$ s.t. $\pmb{x}\in A_{\lambda_0}$, and thus there is some open ball $B_r(\pmb{x})\subseteq A_{\lambda_0}$. Then those $H_i$ small enough will be contained in $B_r(\pmb{x})$ and thus covered by $A_{\lambda_0}$, which contradicts that $H_i$ cannot be covered by finitely many $A_\lambda$.
    \end{compactenum}
\end{Th}

\begin{Df}{Df4.4.-1 (non-empty-disjoint partition)}
    Suppose $E$ is a set. If $E$ can be written as the union $E = \bigcup_{\lambda\in\Lambda} E_\lambda$ where $E_\lambda\neq\varnothing$ and $E_\lambda\cap E_\mu = \varnothing$ for any $\lambda\neq\mu$, then $E = \bigcup_{\lambda\in\Lambda} E_\lambda$ is called a \textbf{non-empty-disjoint partition} of $E$, and the union $\bigcup_{\lambda\in\Lambda} E_\lambda$ is called a \textbf{non-empty-disjoint union}.
\end{Df}

\begin{Rmk}{}
    This very literal definition is proposed just for the convenience of the following narrative.
\end{Rmk}

\begin{Df}{Df4.4 (connected sets (连通集))}
    Suppose $E\subseteq\mathbb{R}^n$ and there is at least two distinct points in $E$. Then $E$ is said to be connected if for any non-empty-disjoint partition $E = A\cup B$ of $E$, \\
    $A\cap B^\prime\neq\varnothing$ or $A^\prime\cap B\neq\varnothing$.
\end{Df}

\begin{Rmk}{}
    The definition is constructed as an extension of the one of intervals in $\mathbb{R}$, as we will see that many properties of real functions hold on intervals, i.e., on some kinds of sets that are ``connected'' or ``not separated''. To depict this property of ``connected'' mathematicians proposed this definition.\\
    The premise that there are at least two distinct points in $E$ is for the convenience that there must exist some non-empty-disjoint partition $E = A\cup B$ of $E$.
\end{Rmk}

\begin{Df}{Df4.4.1.-1 (region (区域))}
    In $\mathbb{R}^n$, a connected open set is called a \textbf{region}, and the closure of a region is called a \textbf{closed region}.
\end{Df}

\begin{Th}{Th4.4.1 (equivalent definition for regions)}
    Suppose $E\subseteq\mathbb{R}^n$ is an open set that contains at least two distinct points. Then $E$ is connected iff for any non-empty-disjoint partition $E = A\cup B$ of $E$, \\
    one of $A$ and $B$ is not open.
    \tcblower
    \textit{Pf}: Obvious.
\end{Th}

\begin{Th}{Th4.4.2 (in $\mathbb{R}$, connected sets are exactly intervals)}
    Suppose $E\subseteq\mathbb{R}$. Then $E$ is connected iff $E$ is an interval.
    \tcblower
    \textit{Pf}: Obviously \textcolor{Th}{$E$ is an interval iff $\inf E<\sup E$ and $(\inf E, \sup E)\subseteq E$}.
    \begin{compactenum}
        \item ``only if'': Suppose $E$ is connected. If otherwise $E$ is not an interval, then there are some $x\in (\inf E, \sup E)$ s.t. $x\notin E$. Let $A = E\cap (-\infty, x)\cap E$ and $B = E\cap (x, \infty)$, then $E = A\cup B$ is a non-empty-disjoint partition of $E$ and thus, say, $A^\prime\cap B\neq\varnothing$. Let $a\in A^\prime\cap B$, then it is clear that $a\leq x$ since $a\in A^\prime$. But $a>x$ since $a\in B$, which is a contradiction.
        \item ``if'': Suppose $E$ is an interval. If otherwise a non-empty-disjoint partition $E = A\cup B$ of $E$ satisfies $A\cap B^\prime = A^\prime\cap B = \varnothing$. \\ 
        First take $a_0\in A$ and $b_0\in B$, then start the dichotomy from the interval $[a_0, b_0]$ (let us say $a_0<b_0$). If the midpoint of the current interval is in $A$ (resp. in $B$), then let the next interval be the left half (resp. right half) of the current interval. Then we obtain the nested closed intervals $\{[a_i, b_i]\}$ s.t. $a_i\in A$ and $b_i\in B$ for each $i$, and thus there is a point $x\in\bigcap_i [a_i, b_i]$. Since $E$ is an interval and $x\in [a_0, b_0]$, $x\in E$. Now $x$ must in one of $A$ and $B$. Say $x\in A$, then $x\notin B$ so that $b_i\rightarrow x$ and $b_i\neq x$, and thus $x\in A\cap B^\prime$, which contradicts $A\cap B^\prime = \varnothing$.
    \end{compactenum}
    \textcolor{P}{\textit{Thoughtfully}: To find a point, consider the nested closed sets theorem.}
\end{Th}

\begin{Df}{Df4.4.3.-1 (continuous curve)}
    \begin{compactenum}
        \item Suppose $l\subseteq\mathbb{R}^n$. Then $l$ is called a \textbf{continuous curve} in $\mathbb{R}^n$ if there are continuous functions $\varphi_i: [a,b]\rightarrow\mathbb{R}$ ($i = 1, \cdots, n$) s.t. $l = \{\pmb{\varphi}(t): t\in [a,b]\}$ (where $\pmb{\varphi}(t) = (\varphi_1(t), \cdots, \varphi_n(t))$).
        \item Suppose $l$ is a continuous curve in $\mathbb{R}^n$ with the parameterization $\pmb{\varphi}: [a,b]\rightarrow\mathbb{R}^n$ defined above. Then $l$ is said to \textbf{connect} $\pmb{\varphi}(a)$ and $\pmb{\varphi}(b)$.
    \end{compactenum}
\end{Df}

\begin{Rmk}{}
    Here each of the $\varphi_i$ is defined on and is continuous on $[a,b]$. Actually it is more appropriate to say ``interval-continuous'' according to our convention, but never mind since for a real function defined on a closed interval, ``continuous'' is the same as ``interval-continuous''.
\end{Rmk}

\begin{Df}{Df4.4.3 (path-connected sets (道路连通集))}
    Suppose $E\subseteq\mathbb{R}^n$ and there are at least two distinct points in $E$. Then $E$ is said to be path-connected if for any two distinct points $\pmb{x}, \pmb{y}\in E$, there is a continuous curve $l\subseteq E$ that connects $\pmb{x}$ and $\pmb{y}$.
\end{Df}

\begin{Th}{Th4.4.3.1 (path-connected $\Rightarrow$ connected)}
    Suppose $E\subseteq\mathbb{R}^n$. Then $E$ is connected if $E$ is path-connected.
    \tcblower
    \textcolor{P}{\textit{Analytically}: To verify that either $A$ or $B$ contains a limit point of the other given $E = A\cup B$, the same idea with the proof of Th \{, ID: 4.4.2\} — dichotomy and nested closed intervals — can be adopted.}\\
    \textit{Pf}: Obvious.
\end{Th}

\begin{Rmk}{}
    This definition also reveals some kind of ``connectedness'' of a set, and we can see that the path-connectedness is much more intuitive than connectedness defined before. Actually this concept is proposed as an identification tool for connected sets since we are generally much more able to judge if a set is path-connected.
\end{Rmk}

\begin{Th}{Th4.4.3.2 (for open sets, path-connected $\Leftrightarrow$ connected)}
    Suppose $E\subseteq\mathbb{R}^n$ is an open set. Then $E$ is connected iff $E$ is path-connected.
    \tcblower
    \textit{Pf}: If otherwise $E$ is connected but not path-connected, then there are two points $\pmb{x}, \pmb{y}\in E$ which cannot be connected by a continuous curve in $E$. Consider the set $P(\pmb{x})$ of all points in $E$ that can be path-connected to $\pmb{x}$. Then for each $\pmb{x}^\prime\in P(\pmb{x})$, there is some $B_r(\pmb{x}^\prime)\subseteq E$. Hence $\pmb{x}$ can be path-connected to every points in $B_r(\pmb{x}^\prime)$ and thus $B_r(\pmb{x}^\prime)\subseteq P(\pmb{x})$, implying that $P(\pmb{x})$ is open. Also, we know that for each $\pmb{a}\in E$, the set $P(\pmb{a})$ defined similarly is open. \\
    Now consider the non-empty-disjoint partition $E = P(\pmb{x})\cup \bigcup_{\pmb{a}\in E\setminus P(\pmb{x})}P(\pmb{a})\triangleq A\cup B$, where both $A$ and $B$ are open. Contradicts that $E$ is connected.\\
    \textcolor{P}{\textit{Thoughtfully}: Firstly, prove by contradiction to convert the constructive proof to verifiable proof. Secondly, to find a path starting from $\pmb{x}$ to the destination $\pmb{y}$, do not stick to paving the path with open balls, try other possible larger paving stones instead by making use of the conditions.}
\end{Th}

\begin{Th}{Ex4.4.3.3 (connected $\nRightarrow$ path-connected)}
    Find an example of a connected set that is not path-connected.
    \tcblower
    \textit{Solution}: In $\mathbb{R}^2$, consider the set $E = \{(x, \sin\frac{1}{x}): x\in (0,\frac{2}{\pi}]\}$. Then we prove that $\overline{E}$ is connected but not path-connected. \\
    (Note: here $\sin$ is the sine function. Although we have not learned the triangular functions for the sake of rigor (as we need power series to define them), we have learned their basic properties in high school. Just work with them now as there is no cycle reasoning.)\\
    A trivial lemma: \textcolor{Th}{Suppose $E\subseteq\mathbb{R}^n$ is connected. Then $\overline{E}$ is connected.} Then since $E$ defined here is path-connected, $\overline{E}$ is connected. \\
    Now we prove that $\overline{E}$ is not path-connected. Since $\overline{E} = E\cup\{(0, y): y\in [-1,1]\}$, we claim that there is no continuous curve in $\overline{E}$ that connects $(0,0)$ and $(2/\pi, 1)$. \\
    If otherwise $l = \{(x(t), y(t)): t\in [a,b]\}\subseteq \overline{E}$ connects $(0,0)$ and $(2/\pi, 1)$, say $a=0$, $b=1$, $(x(0), y(0)) = (0,0)$ and $(x(1), y(1)) = (2/\pi, 1)$, then 
    \begin{compactenum}
        \item $\lim\limits_{t\to 0^+} x(t) = x(0) = 0$;
        \item $\lim\limits_{t\to 0^+} y(t) = y(0) = 0$;
    \end{compactenum}
    Now let $m_0 = 0$, $t_0=1$ and $x_0 = 2/[(1+4m_0)\pi] = 2/\pi$, then $x(t_0) = x_0$. 
    \begin{compactenum}
        \item Since $x(\cdot)$ is continuous on $[0,1/2]$, the range of $x(\cdot)$ on $[0,1/2]$ is $[0, M_1]$ ($M_1 = \max_{t\in[0,1/2]}x(t)$). Then choose an integer $m_1$ s.t. $m_1>m_0$ and $x_1\triangleq 2/[(1+4m_1)\pi]\in [0, M_1]$, so that $x_1 = x(t_1)$ for some $t_1\in (0, 1/2]$.
        \item Since $x(\cdot)$ is continuous on $[0,1/4]$, the range of $x(\cdot)$ on $[0,1/4]$ is $[0, M_2]$ ($M_2 = \max_{t\in[0,1/4]}x(t)$). Then choose an integer $m_2$ s.t. $m_2>m_1$ and $x_2\triangleq 2/[(1+4m_2)\pi]\in [0, M_2]$, so that $x_2 = x(t_2)$ for some $t_2\in (0, 1/4]$.
        \item $\cdots$
    \end{compactenum}
    Then we obtain a sequence $\{t_i\}$ s.t.
    \begin{compactenum}
        \item $t_i\rightarrow 0$ and $t_i\neq 0$;
        \item $x_i = x(t_i) = 2/[(1+4m_i)\pi]\rightarrow 0$ (since $m_i\rightarrow\infty$);
    \end{compactenum}
    Hence we find the following contradiction:
    $$ 0 = \lim\limits_{t\to 0^+} y(t) = \lim\limits_{i\to\infty} y(t_i) \overset{x_i\neq 0}{=} \lim\limits_{i\to\infty} \sin\frac{1}{x_i} = \lim\limits_{i\to\infty} \sin\left(\frac{\pi}{2}+2m_i\pi\right) = 1. $$
    \textcolor{P}{\textit{Thoughtfully}: See the next page.}
\end{Th}

\begin{Th}{Ex4.4.3.3 (connected $\nRightarrow$ path-connected) — continued}
    \textcolor{P}{\textit{Thoughtfully}: it is how to find such a set $E$ that troubles us.\\
    To find and prove such a set $E$, we cannot avoid such statement ``there is no path in $E$ connecting $\pmb{x}$ and $\pmb{y}$'' or ``if otherwise there is a path in $E$ connecting $\pmb{x}$ and $\pmb{y}$'', which would brought immense troubles as we often cannot enumerate all the possible paths connecting $\pmb{x}$ and $\pmb{y}$, of a huge variety of shapes especially in some trivial suspects (balls, rectangles, \dots). But what if we can limit the shape of the path? The ideal case is that $E$ is a curve, say, a line segment, so that every path in $E$ becomes very trivial.\\
    A line-segment $AB$ is path-connected, so we need some modification for it. A set $E$ which contains some path-separated pair $(x,y)$ of points must have:
    \begin{compactenum}
        \item Every path connecting $x$ and $y$ is cut off by some point not in $E$;
        \item Every path in $E$ starting from $x$ can never reaches $y$.
    \end{compactenum}
    If we construct $E$ by (1), we are to remove some points from the line segment $AB$. Removing one point, finitely many points, or a convergent sequence of points will not work, as this will directly lead to an unconnected $E$, neither will removing all rational points.
    If we construct $E$ by (2), we are to add some points to $AB$ which cannot be reached by any path starting from $AB$. Sadly the newly added points cannot be isolated (i.e., not in the derived set of $AB$) as otherwise $E$ is again unconnected, and then we have no idea since no extra point is a limit point of $AB$.\\
    But can we find such an extra limit point if $AB$ is not a line segment? Or can we twist the shape of $AB$ to make some other point $y$ approachable in any extent but never reachable? Imagine drawing $AB$ horizontally on a piece of paper, say $A=(0,0)$ and $B=(1,0)$, and choosing $y=(1,1)$. Then we can make the line segment repeatedly bulge upwards as moving from $A$ to $B$, in every effort, for infinitely many times, to approach $y$. This way we finally achieve the goal.\\
    So far we have see the key to find such a set $E$: a curve, unlimitedly twisted or oscillating in a limited space. And there is a typical example of such characteristic: the curve $\{(x,y): y=\sin(1/x)\}$.}
\end{Th}

\begin{Df}{Df4.5.-1 ($n$-real function)}
    Suppose $f$ is a function. Then $f$ is called an $n$-real function if $\text{dom}(f)\subseteq\mathbb{R}^n$ and $f: \text{dom}(f)\rightarrow\mathbb{R}$.
\end{Df}

\begin{Rmk}{}
    $1$-real function is just the real function.
\end{Rmk}

\begin{Df}{Df4.5 (limit of $n$-real function)}
    Suppose $f$ is an $n$-real function and $\pmb{x}_0\in(\text{dom}(f))^\prime$. Suppose also $l\in\mathbb{R}$. Then $l$ is called the limit of $f$ at $\pmb{x}_0$, denoted by $\lim\limits_{\pmb{x}\to\pmb{x}_0}f(\pmb{x}) = l$, if:
    $$ \forall\varepsilon>0, \exists\delta>0, \forall\pmb{x}\in\text{dom}(f)\cap \check{B}_\delta(\pmb{x}_0), f(\pmb{x})\in B_\varepsilon(l). $$
\end{Df}

\begin{Rmk}{}
    This definition is merely an extension of the limit of real function.\\
    \textcolor{Df}{For an $n$-real function $f$, we say that $\lim\limits_{\pmb{x}\to \pmb{x}_0} f(\pmb{x})$ exists if there is some real number $l$ such that $\lim\limits_{\pmb{x}\to \pmb{x}_0} f(\pmb{x}) = l$.}\\
    \textcolor{Th}{From this definition, we can easily prove the following basic properties, just copy all the proof of those of real function:
    \begin{compactenum}
        \item (Uniqueness)
        \item (Local boundness)
        \item (Heine's theorem) (resolution principle) Suppose $f$ is an $n$-real function, $x_0\in(\text{dom}(f))^\prime$ and $l\in\mathbb{R}$. Then $\lim\limits_{\pmb{x}\rightarrow \pmb{x}_0}f(\pmb{x}) = l$ if and only if:\\
        For any sequence $\{\pmb{x}_n:n\in\mathbb{N^\ast}\}$ in $\text{dom}(f)$ s.t. (1) $\pmb{x}_n\neq \pmb{x}_0$ for all $n\in\mathbb{N^\ast}$ and (2) $\lim\limits_{n\rightarrow\infty}\pmb{x}_n = \pmb{x}_0$, we have $\lim\limits_{n\rightarrow\infty}f(\pmb{x}_n) = l$.
        \item (Arithmetics of limits) If both $\lim\limits_{\pmb{x}\rightarrow \pmb{x}_0} f(\pmb{x})$ and $\lim\limits_{\pmb{x}\rightarrow \pmb{x}_0} g(\pmb{x})$ exist, and $\pmb{x}_0$ is a limit point of each function below, then:
        \begin{compactitem}
            \item $\lim\limits_{\pmb{x}\rightarrow \pmb{x}_0} (f(\pmb{x})\pm g(\pmb{x})) = \lim\limits_{\pmb{x}\rightarrow \pmb{x}_0} f(\pmb{x}) \pm \lim\limits_{\pmb{x}\rightarrow \pmb{x}_0} g(\pmb{x})$;
            \item $\lim\limits_{\pmb{x}\rightarrow \pmb{x}_0} (f(\pmb{x})g(\pmb{x})) = \lim\limits_{\pmb{x}\rightarrow \pmb{x}_0} f(\pmb{x}) \cdot \lim\limits_{\pmb{x}\rightarrow \pmb{x}_0} g(\pmb{x})$;
            \item If $\lim\limits_{\pmb{x}\rightarrow \pmb{x}_0} g(\pmb{x})\neq 0$, then $\lim\limits_{\pmb{x}\rightarrow \pmb{x}_0} \frac{f(\pmb{x})}{g(\pmb{x})} = \frac{\lim\limits_{\pmb{x}\rightarrow \pmb{x}_0} f(\pmb{x})}{\lim\limits_{\pmb{x}\rightarrow \pmb{x}_0} g(\pmb{x})}$.
        \end{compactitem}
        \item (Limit of composite function) Suppose $\varphi$ is a real function, $f$ is an $n$-real function and $\varphi\circ f$ is composible. If:
        \begin{compactenum}
            \item $\lim\limits_{\pmb{x}\to\pmb{x}_0} f(\pmb{x}) = t_0$;
            \item $\lim\limits_{t\to t_0} \varphi(t) = l$;
            \item $\exists \check{B}_\eta(\pmb{x}_0)$ s.t. $f(\pmb{x})\neq t_0$ for all $\pmb{x}\in\check{B}_\eta(\pmb{x}_0)$,
        \end{compactenum}
        Then $\lim\limits_{\pmb{x}\to\pmb{x}_0} \varphi(f(\pmb{x}_0)) = l$.
        \item (Cauchy criterion) Suppose $f$ is an $n$-real function and $\pmb{x}_0\in(\text{dom}{f})^\prime$. Then $\lim\limits_{\pmb{x}\to\pmb{x}_0} f(\pmb{x})$ exists iff:
        $$\forall \varepsilon>0, \exists \delta>0, \forall \pmb{x}_1, \pmb{x}_2\in\text{dom}(f)\cap \check{B}_\delta(\pmb{x}_0), |f(\pmb{x}_1)-f(\pmb{x}_2)|<\varepsilon.$$
    \end{compactenum}}
    These are merely the trivial extension of those real-function counterparts.
\end{Rmk}

\begin{Df}{Df4.5.1 (continuity of $n$-real function)}
    Suppose $f$ is an $n$-real function and $\pmb{x}_0\in\text{dom}(f)$. Then $f$ is called continuous at $\pmb{x}_0$ (or, $\pmb{x}_0$ is called a continuity point of $f$) if:
    $$\forall\varepsilon>0, \exists\delta>0, \forall \pmb{x}\in\text{dom}(f)\cap B_\delta(\pmb{x}_0), f(\pmb{x})\in B_\varepsilon(f(\pmb{x}_0)).$$
\end{Df}

\begin{Rmk}{}
    This definition is merely a trivial extension of the continuity of real function, and so are the following definitions and theorems (along their proofs).
    \begin{compactenum}
        \item \textcolor{Df}{Suppose $f$ is an $n$-real function and $D\subseteq\text{dom}(f)$. Then $f$ is called continuous on $D$ if $f$ is continuous at every point in $D$. Suppose also $\pmb{y}\in\mathbb{R}^n$, then $f$ is called discontinuous at $\pmb{y}$ (or $\pmb{y}$ is called a discontinuous point of $f$) if $f$ is not continuous at $\pmb{y}$.}
        \item \textcolor{Th}{Suppose $f$ is an $n$-real function and $\pmb{y}\notin (\text{dom}(f))^\prime$. Then $f$ is continuous at $\pmb{y}$.}
        \item \textcolor{Th}{Suppose $f$ is an $n$-real function and $\pmb{x}_0\in\text{dom}(f)\cap (\text{dom}(f))^\prime$. Then $f$ is continuous at $\pmb{x}_0$ iff $\lim\limits_{\pmb{x}\to\pmb{x}_0} f(\pmb{x}) = f(\pmb{x}_0)$.}
        \item \textcolor{Th}{(Arithmetics) If $n$-real functions $f$ and $g$ are both continuous at $\pmb{x}_0$, then: 
        \begin{compactenum}
            \item $f\pm g$, $f\cdot g$ are all continuous at $\pmb{x}_0$;
            \item if $g(\pmb{x}_0)\neq 0$, then $f/g$ is also continuous at $\pmb{x}_0$. 
        \end{compactenum}}
        \item \textcolor{Th}{(Continuity of composite function) Suppose real function $\varphi$ and $n$-real function $f$ are composable ($\varphi\circ f$). If $f$ is continuous at $\pmb{x}_0$ and $\varphi$ is continuous at $f(\pmb{x}_0)$, then $\varphi\circ f$ is continuous at $\pmb{x}_0$.}
    \end{compactenum}
\end{Rmk}

\begin{Df}{Df4.5.2 (uniform continuity)}
    Suppose $f$ is an $n$-real function and $D\subseteq\text{dom}(f)$. Then $f$ is called uniformly continuous on $D$ if:
    $$\forall\varepsilon>0, \exists\delta>0, \left(\forall \pmb{x}_1, \pmb{x}_2\in D\text{ and } \Vert\pmb{x}_1-\pmb{x}_2\Vert <\delta\right), |f(\pmb{x}_1)-f(\pmb{x}_2)|<\varepsilon.$$
\end{Df}

\begin{Rmk}{}
    This definition is merely a trivial extension of the uniform continuity of real function, and so are the following definitions and theorems (along their proofs).
    \begin{compactenum}
        \item \textcolor{Th}{Suppose $f$ is an $n$-real function. If $f$ is uniformly continuous on $\text{dom}(f)$, then $f$ is continuous on $\text{dom}(f)$.}
    \end{compactenum}
\end{Rmk}

\begin{Th}{Th4.5.3.1 (continuous function on a compact set is uniformly continuous)}
    Suppose $f$ is an $n$-real function. If
    \begin{compactenum}
        \item $\text{dom}(f)$ is compact and
        \item $f$ is continuous on $\text{dom}(f)$,
    \end{compactenum}
    then $f$ is uniformly continuous on $\text{dom}(f)$.
    \tcblower
    \textit{Pf}: Completely the same method with the proof of \{, ID: 2.4.1\}.
\end{Th}

\begin{Th}{Th4.5.3.2 (continuous function maps a compact set to a compact set)}
    Suppose $f$ is an $n$-real function. If
    \begin{compactenum}
        \item $\text{dom}(f)$ is compact and
        \item $f$ is continuous on $\text{dom}(f)$,
    \end{compactenum}
    then $\text{range}(f)$ is also compact.
    \tcblower
    \textit{Pf}: To prove that $\text{range}(f)$ is compact, namely, sequentially compact, we find a limit for some subsequence of an arbitrary sequence $\{y_k\}\subseteq\text{range}(f)$. Let $y_k = f(\pmb{x}_k)$ for some sequence $\{\pmb{x}_k\}\subseteq\text{dom}(f)$. Since $\text{dom}(f)$ is compact, a subsequence $\{\pmb{x}_{m_k}\}$ of $\{\pmb{x}_k\}$ converges to some $\pmb{x}_0\in\text{dom}(f)$.
    \begin{compactenum}
        \item If $\pmb{x}_{m_k} = \pmb{x}_0$ for $k>N$, then $y_{m_k} = f(\pmb{x}_{m_k}) = f(\pmb{x}_0)\rightarrow f(\pmb{x}_0)\in\text{range}(f)$;
        \item If for every $N$ there is always $k>N$ s.t. $\pmb{x}_{m_k}\neq \pmb{x}_0$, let us say that $\pmb{x}_{m_k}\neq \pmb{x}_0$ for all $k$. Then $\pmb{x}_0\in\text{dom}(f)\cap (\text{dom}(f))^\prime$ and
        $$ \lim\limits_{k\to\infty} y_{m_k} = \lim\limits_{k\to\infty} f(\pmb{x}_{m_k}) = \lim\limits_{\pmb{x}\to\pmb{x}_0} f(\pmb{x}) = f(\pmb{x}_0)\in\text{range}(f). $$
    \end{compactenum}
    And thus $\text{range}(f)$ is sequentially compact.
\end{Th}

\begin{Th}{Th4.5.3.3 (continuous function maps a connected set to a connected set)}
    Suppose $f$ is an $n$-real function. If
    \begin{compactenum}
        \item $\text{dom}(f)$ is connected and $\text{range}(f)$ has at least 2 distinct points;
        \item $f$ is continuous on $\text{dom}(f)$;
    \end{compactenum}
    then $\text{range}(f)$ is also connected.
    \tcblower
    \textit{Pf}: For every non-empty-disjoint partition $\text{range}(f) = A\cup B$, it is clear that $\text{dom}(f) = f^{-1}[A]\cup f^{-1}[B]$ is a non-empty-disjoint partition of $\text{dom}(f)$ (here $f^{-1}[A] = \{\pmb{x}\in\text{dom}(f): f(\pmb{x})\in A\})$ (see \{course: 0, ID: 1.5.2\}). Let us say that some sequence $\{\pmb{a}_k\}\subseteq f^{-1}[A]$ converges to some point $\pmb{b}\in f^{-1}[B]$, then $\pmb{a}_k\rightarrow\pmb{b}$ so that $\pmb{b}\in\text{dom}(f)\cap(\text{dom}(f))^\prime$, so that $\lim\limits_{\pmb{x}\to\pmb{b}} f(\pmb{x}) = f(\pmb{b})$ and so that:
    $$ \lim\limits_{k\to\infty} f(\pmb{a}_k) = \lim\limits_{\pmb{x}\to\pmb{b}} f(\pmb{x}) = f(\pmb{b})$$
    and thus $f(\pmb{b})\in A^\prime\cap B$.
\end{Th}

\begin{Th}{Th4.5.3.3.1 (intermediate value property)}
    Suppose $f$ is an $n$-real function. If
    \begin{compactenum}
        \item $\text{dom}(f)$ is connected and
        \item $f$ is continuous on $\text{dom}(f)$,
    \end{compactenum}
    then for any $a, b\in\text{range}(f)$ and any $c$ s.t. $a<c<b$, there is some $\pmb{x}_0\in\text{dom}(f)$ s.t. $f(\pmb{x}_0) = c$.
    \tcblower
    \textit{Pf}: Obvious by \{, ID: 4.5.3.3\}
\end{Th}

\begin{Df}{Df4.5.4.-1 (convex $n$-real function)}
    Suppose $f$ is an $n$-real function and $\varnothing\neq A\subseteq\text{dom}(f)$ where $A$ is convex. Then the restriction of $f$ on $A$ is called a convex (resp. strict convex) $n$-real function or $f$ is said to be convex on $A$ (resp. strict convex on $A$) if: \\
    $\forall \pmb{x}_1, \pmb{x}_2\in A$, $\forall \lambda_1, \lambda_2\in (0,1) \text{ s.t. } \lambda_1+\lambda_2=1$, we have $f(\lambda_1 \pmb{x}_1 + \lambda_2 \pmb{x}_2) \leq \lambda_1 f(\pmb{x}_1) + \lambda_2 f(\pmb{x}_2)$ (resp. $f(\lambda_1 \pmb{x}_1 + \lambda_2 \pmb{x}_2) < \lambda_1 f(\pmb{x}_1) + \lambda_2 f(\pmb{x}_2)$).
\end{Df}

\begin{Rmk}{}
    This definition is merely a trivial extension of the one of convex real function.
\end{Rmk}

\begin{Th}{Ex4.5.4 (convex functions on convex regions are continuous)}
    Suppose $f$ is an $n$-real function. If
    \begin{compactenum}
        \item $\text{dom}(f)$ is a convex region (i.e. convex, open and connected) and
        \item $f$ is convex on $\text{dom}(f)$,
    \end{compactenum}
    then $f$ is continuous on $\text{dom}(f)$.
    \tcblower
    \textit{Pf}: First we prove that $f$ is bounded on every compact subset $K$ of $\text{dom}(f)$. 
    \begin{compactenum}
        \item $f$ has an upper bound on $K$. Take arbitrarily a closed rectangle $H\subseteq\text{dom}(f)$, then we claim that $f$ has an upper bound on $H$: since $f(\lambda\pmb{x}+(1-\lambda)\pmb{y})\leq\max\{f(\pmb{x}), f(\pmb{y})\}$ holds for each line segment with endpoints $\pmb{x}, \pmb{y}\in H$, we can see that $f(\pmb{x})$ is not greater than the maximum of $f$ at the $2^n$ vertices of $H$. Hence $f$ has an upper bound on every closed rectangle $H$ in $\text{dom}(f)$, and thus on $K$ by the Heine-Borel theorem (just choose a open rectangle centered at each point in $K$, join these rectangles to form an open cover of $K$, and then a finite subcover can upper bound $f$ on $K$).
        \item $f$ has a lower bound on every compact subset $K$ of $\text{dom}(f)$. If otherwise $f$ has no lower bound on $H$, then there is a sequence $\{\pmb{x}_k\}\subseteq K$ convergent to some $\pmb{x}_0\in K$ s.t. $f(\pmb{x}_k)\rightarrow -\infty$. Since $\text{dom}(f)$ is open, we choose some $B_\eta(\pmb{x}_0)\subseteq\text{dom}(f)$, and inspect those $\pmb{x}_k$ in $B_\eta(\pmb{x}_0)$. For each $\pmb{x}_k$, consider its symmetric point $2\pmb{x}_0-\pmb{x}_k\in B_\eta(\pmb{x}_0)$ with respect to $\pmb{x}_0$, then we have
        $$ f(\pmb{x}_0)\leq \frac{1}{2}f(\pmb{x}_k)+\frac{1}{2}f(2\pmb{x}_0-\pmb{x}_k), $$
        which is a contradiction. That is because $\{2\pmb{x}_0-\pmb{x}_k\}$ is a bounded sequence (and thus is a compact subset of $\text{dom}(f)$), so that $f(2\pmb{x}_0-\pmb{x}_k)$ has an upper bound, but $\{f(\pmb{x}_k)\}$ has no lower bound. Hence $f$ has a lower bound on $K$.
    \end{compactenum}
    Second we prove that $f$ is uniformly continuous on $\bar{B}_r(\pmb{x}_0)$ for every $\pmb{x}_0\in\text{dom}(f)$ so that $f$ is continuous on each $\pmb{x}_0\in\text{dom}(f)$. Let us say that $\bar{B}_{2r}(x_0)\subseteq\text{dom}(f)$. For every distinct $\pmb{x}, \pmb{y}\in \bar{B}_r(\pmb{x}_0)$, say $f(\pmb{x})\leq f(\pmb{y})$, we extend the line segment $\pmb{x}\pmb{y}$ intersecting the boundary of $\bar{B}_{2r}(\pmb{x}_0)$ at $\pmb{z}$. Then $\pmb{y} = (1-\lambda)\pmb{x}+\lambda\pmb{z}$ for some $\lambda\in(0,1)$, and thus
    $$ |f(\pmb{y}) - f(\pmb{x})| = f(\pmb{y}) - f(\pmb{x}) \leq \lambda [f(\pmb{z})-f(\pmb{x})] \leq \lambda M = \frac{\Vert\pmb{y}-\pmb{x}\Vert}{\Vert\pmb{z}-\pmb{x}\Vert}M\leq \frac{M}{r}\Vert\pmb{y}-\pmb{x}\Vert, $$
    (where $M$ is an upper bound for $\{f(\pmb{z}) - f(\pmb{x}): \pmb{z}, \pmb{x}\in\bar{B}_r(\pmb{x}_0)\}$) which is a Lipschitz condition, implying that $f$ is uniformly continuous on $\bar{B}_r(\pmb{x}_0)$.\\
    \textcolor{P}{\textit{Thoughtfully}: Normally we prove the continuity of $f$, at every point $\pmb{x}_0$, by verifying the limit $\lim\limits_{\pmb{x}\to\pmb{x}_0} f(\pmb{x}) = f(\pmb{x}_0)$, and by using the resolution principle since no explicit formula for $f$ is given. But here we have only the convexity inequality which can only exemplify the behavior of $f$ on line segments, and thus cannot deal with the limit $\lim\limits_{i\to \infty} f(\pmb{x}_i)$ directly. So we have to change our mind to prove by uniform continuity, which is sometimes easier if we can find a Lipschitz condition.\\
    \textbf{Keep in mind: to prove continuity, use definition, arithmetics or uniform continuity.}}
\end{Th}

\begin{Df}{Df4.6.-1 ($n$-real $m$-function)}
    Suppose $\pmb{f}$ is a function. Then $\pmb{f}$ is called an $n$-real $m$-function if $\text{dom}(\pmb{f})\subseteq\mathbb{R}^n$ and $\pmb{f}: \text{dom}(\pmb{f})\rightarrow\mathbb{R}^m$.
\end{Df}

\begin{Rmk}{}
    $n$-real $1$-function is just the $n$-real function.
\end{Rmk}

\begin{Df}{Df4.6 (limit of $n$-real $m$-function)}
    Suppose $\pmb{f}$ is an $n$-real $m$-function and $\pmb{x}_0\in(\text{dom}(\pmb{f}))^\prime$. Suppose also $\pmb{l}\in\mathbb{R}^m$. Then $\pmb{l}$ is called the limit of $\pmb{f}$ at $\pmb{x}_0$, denoted by $\lim\limits_{\pmb{x}\to\pmb{x}_0}\pmb{f}(\pmb{x}) = \pmb{l}$, if:
    $$ \forall\varepsilon>0, \exists\delta>0, \forall\pmb{x}\in\text{dom}(\pmb{f})\cap \check{B}_\delta(\pmb{x}_0), \pmb{f}(\pmb{x})\in B_\varepsilon(\pmb{l}). $$
\end{Df}

\begin{Rmk}{}
    This definition is merely an extension of the limit of $n$-real function.\\
    \textcolor{Df}{For an $n$-real $m$-function $\pmb{f}$, we say that $\lim\limits_{\pmb{x}\to \pmb{x}_0} \pmb{f}(\pmb{x})$ exists if there is some $\pmb{l}\in\mathbb{R}^m$ such that $\lim\limits_{\pmb{x}\to \pmb{x}_0} \pmb{f}(\pmb{x}) = \pmb{l}$}\\
    \textcolor{Th}{From this definition, we can copy all the properties (along their proofs) below from those of $n$-real function, just bold $f$ to $\pmb{f}$:
    \begin{compactenum}
        \item (Uniqueness)
        \item (Converge by components) Suppose $\pmb{f}$ is an $n$-real $m$-function with $\pmb{f} = (f_1, \cdots, f_m)$, and $\pmb{l} = (l_1, \cdots, l_m)$. Then $\lim\limits_{\pmb{x}\to\pmb{x}_0}\pmb{f}(\pmb{x}) = \pmb{l}$ iff $\lim\limits_{\pmb{x}\to\pmb{x}_0}f_i(\pmb{x}) = l_i$ for all $i\in\{1,\cdots,m\}$.
        \item (Local boundness)
        \item (Heine's theorem) (resolution principle) 
        \item (Arithmetics of limits) Let $\pmb{f}$ and $\pmb{g}$ are both $n$-real $m$-functions. If both $\lim\limits_{\pmb{x}\rightarrow \pmb{x}_0} \pmb{f}(\pmb{x})$ and $\lim\limits_{\pmb{x}\rightarrow \pmb{x}_0} \pmb{g}(\pmb{x})$ exist, and $\pmb{x}_0$ is a limit point of the domain of each function below, then:
        \begin{compactitem}
            \item $\lim\limits_{\pmb{x}\rightarrow \pmb{x}_0} (\pmb{f}(\pmb{x})\pm \pmb{g}(\pmb{x})) = \lim\limits_{\pmb{x}\rightarrow \pmb{x}_0} \pmb{f}(\pmb{x}) \pm \lim\limits_{\pmb{x}\rightarrow \pmb{x}_0} \pmb{g}(\pmb{x})$;
            \item $\lim\limits_{\pmb{x}\rightarrow \pmb{x}_0} \lambda\pmb{f}(\pmb{x}) = \lambda\lim\limits_{\pmb{x}\rightarrow \pmb{x}_0} \pmb{f}(\pmb{x})$ for any $\lambda\in\mathbb{R}$;
        \end{compactitem}
        \item (Limit of composite function) Suppose $\pmb{f}$ is an $n$-real $m$-function, $\pmb{g}$ is a $p$-real $n$-function and $\pmb{f}\circ\pmb{g}$ is composible. If:
        \begin{compactenum}
            \item $\lim\limits_{\pmb{t}\to\pmb{t}_0} \pmb{g}(\pmb{t}) = \pmb{x}_0$;
            \item $\lim\limits_{\pmb{x}\to\pmb{x}_0} \pmb{f}(\pmb{x}) = \pmb{l}$;
            \item $\exists \check{B}_\eta(\pmb{t}_0)$ s.t. $\pmb{g}(\pmb{t})\neq \pmb{x}_0$ for all $\pmb{t}\in\check{B}_\eta(\pmb{t}_0)$,
        \end{compactenum}
        Then $\lim\limits_{\pmb{t}\to\pmb{t}_0} \pmb{f}(\pmb{g}(\pmb{t}_0)) = \pmb{l}$.
        \item (Cauchy criterion) 
    \end{compactenum}}
    These are merely the trivial extension of those $n$-real function counterparts.
\end{Rmk}

\begin{Df}{Df4.6.1 (continuity of $n$-real $m$-function)}
    Suppose $\pmb{f}$ is an $n$-real $m$-function and $\pmb{x}_0\in\text{dom}(\pmb{f})$. Then $\pmb{f}$ is called continuous at $\pmb{x}_0$ (or, $\pmb{x}_0$ is called a continuity point of $\pmb{f}$) if:
    $$\forall\varepsilon>0, \exists\delta>0, \forall \pmb{x}\in\text{dom}(\pmb{f})\cap B_\delta(\pmb{x}_0), \pmb{f}(\pmb{x})\in B_\varepsilon(\pmb{f}(\pmb{x}_0)).$$
\end{Df}

\begin{Rmk}{}
    This definition is merely a trivial extension of the continuity of $n$-real function, and so are the following definitions and theorems (along their proofs). Just copy all those in the Rmk \{, ID: 4.5.1\}.
    \begin{compactenum}
        \item \textcolor{Df}{Suppose \dots}
        \item \textcolor{Th}{Suppose \dots}
        \item \textcolor{Th}{Suppose \dots}
        \item \textcolor{Th}{(Arithmetics) If \dots}
        \item \textcolor{Th}{(Continuity of composite function) Suppose $n$-real $m$-function $\pmb{f}$ and $p$-real $n$-function $\pmb{g}$ are composable ($\pmb{f}\circ\pmb{g}$). If $\pmb{g}$ is continuous at $\pmb{t}_0$ and $\pmb{f}$ is continuous at $\pmb{g}(\pmb{t}_0)$, then $\pmb{f}\circ\pmb{g}$ is continuous at $\pmb{t}_0$.}
        \item \textcolor{Th}{(Continuous by components) Suppose $\pmb{f}$ is an $n$-real $m$-function with $\pmb{f} = (f_1, \cdots, f_m)$, and $\pmb{x}_0\in\text{dom}(\pmb{f})$. Then $\pmb{f}$ is continuous at $\pmb{x}_0$ iff each $f_i$ is continuous at $\pmb{x}_0$.}
    \end{compactenum}
\end{Rmk}

\begin{Df}{Df4.6.2 (uniform continuity)}
    Suppose $\pmb{f}$ is an $n$-real $m$-function and $D\subseteq\text{dom}(\pmb{f})$. Then $\pmb{f}$ is called uniformly continuous on $D$ if:
    $$\forall\varepsilon>0, \exists\delta>0, \left(\forall \pmb{x}_1, \pmb{x}_2\in D\text{ and } \Vert\pmb{x}_1-\pmb{x}_2\Vert <\delta\right), \Vert\pmb{f}(\pmb{x}_1)-\pmb{f}(\pmb{x}_2)\Vert<\varepsilon.$$
\end{Df}

\begin{Rmk}{}
    This definition is merely a trivial extension of the uniform continuity of $n$-real function, and so are the following definitions and theorems (along their proofs).
    \begin{compactenum}
        \item \textcolor{Th}{Suppose \dots}
        \item \textcolor{Th}{(Uniformly continuous by components) Suppose $\pmb{f}$ is an $n$-real $m$-function with $\pmb{f} = (f_1, \cdots, f_m)$, and $D\subseteq\text{dom}(\pmb{f})$. Then $\pmb{f}$ is uniformly continuous on $D$ iff each $f_i$ is uniformly continuous on $D$.}
    \end{compactenum}
\end{Rmk}

\begin{Th}{Th4.6.3}
    Suppose $\pmb{f}$ is an $n$-real $m$-function. If $\text{dom}(f)$ is open, then $\pmb{f}$ is continuous on $\text{dom}(f)$ iff: \\
    for every open set $G$ in $\mathbb{R}^m$, $\pmb{f}^{-1}[G]$ is open in $\mathbb{R}^n$.
    \tcblower
    \textit{Pf}: Obvious.
\end{Th}

\begin{Th}{Th4.6.4 (copy of Th \{, ID: 4.5.3.1\}, Th \{, ID:4.5.3.2\}, Th \{, ID: 4.5.3.3\} to $n$-real $m$-function)}
    Suppose $\pmb{f}$ is an $n$-real $m$-function. 
    \begin{compactenum}
        \item (4.5.3.1) \dots
        \item (4.5.3.2) \dots
        \item (4.5.3.3) \dots
    \end{compactenum}
    \tcblower
    \textit{Pf}: \dots
\end{Th}
\end{document}