\documentclass{article}

    \usepackage{xcolor}
    \definecolor{pf}{rgb}{0.4,0.6,0.4}
    \usepackage[top=1in,bottom=1in, left=0.8in, right=0.8in]{geometry}
    \usepackage{setspace}
    \setstretch{1.2} 
    \setlength{\parindent}{0em}

    \usepackage{paralist}
    \usepackage{cancel}

    % \usepackage{ctex}
    \usepackage{amssymb}
    \usepackage{amsmath}

    \usepackage{tcolorbox}
    \definecolor{Df}{RGB}{0, 184, 148}
    \definecolor{Th}{RGB}{9, 132, 227}
    \definecolor{Rmk}{RGB}{215, 215, 219}
    \newtcolorbox{Df}[2][]{colbacktitle=Df, colback=white, title={\large\color{white}#2},fonttitle=\bfseries,#1}
    \newtcolorbox{Th}[2][]{colbacktitle=Th, colback=white, title={\large\color{white}#2},fonttitle=\bfseries,#1}
    \newtcolorbox{Rmk}[2][]{colbacktitle=Rmk, colback=white, title={\large\color{black}{Remarks}},fonttitle=\bfseries,#1}

    \title{\LARGE \textbf{Determinants}}
    \author{\large Jiawei Hu}

\begin{document}
\maketitle

This is the 8th chapter of advanced algebra, which is about \textbf{Determinants}.\\ 
Determinants is a powerful tool for understanding the behaviours of linear operators.\\
By the way, we now reiterate some commonly-used notations and conventions:
\begin{compactenum}
    \item $\mathbb{C}$: the set of the complex numbers;
    \item $\mathbb{R}$: the set of the real numbers;
    \item $\mathbb{R}^+$: the set of the positive real numbers;
    \item $\mathbb{Q}$: the set of the rational numbers;
    \item $\mathbb{Z}$: the set of the integers;
    \item $\mathbb{N}$: the set of the natural numbers;
    \item $\mathbb{N^\ast}$ or $\mathbb{N}^+$: the set of the positive integers.
    \item $\sideset{^R}{}{\mathop{D}}$: the set of all functions from $D$ to $R$ (with domain $D$ and range in $R$).
    \item An agreement for the length of a list: if we write $a_1, \dots, a_n$, then we indicate that $n$ is finite and that $n\geq 1$; if we write $a_0, \dots, a_n$, then we indicate that $n$ is finite and that $n\geq 0$.
    \item $A\times B$: the Cartesian product of $A$ and $B$.
    \item $\mathbb{F}$: a number field.
    \item Continue to use the notations and concepts of functions (see the chapter 1 of course 0).
    \item The matrix product $KA$ is referred as ``$A$ left-multiplied by $K$'' or ``left-multiply $A$ by $K$''; $AK$ is similar.
\end{compactenum} 
Please check the notations and definitions by yourself from the previous chapters or courses. Then with everything prepared, here we go.

\begin{Th}{Quick Search}
    \begin{compactdesc}
        \item (8.1, 8.2) Definitions of determinants.
        \item (8.3) Properties of determinants.
        \item (8.5) Characteristic polynomial.
    \end{compactdesc}
\end{Th}

\begin{Df}{Df8.1.-1.1 (swap)}
    Given the set $\mathcal{P}_n$ of all permutations of $(1,2,\dots,n)$, a \textbf{swap} $S(i,j; \cdot): \mathcal{P}_n\to\mathcal{P}_n$ is a function that swaps the $i$-th and $j$-th elements (say, $i\neq j$) of a permutation. 
\end{Df}

\begin{Rmk}{}
    \textcolor{Th}{A swap is a bijection.} Although a swap, in its natural sense, is just an operation that exchanges the positions of two elements, we limit the domain of a swap to $\mathcal{P}_n$ to normalize the discussion later.
\end{Rmk}

\begin{Df}{Df8.1.-1.2 (the number of reversals)}
    Suppose $\pmb{p} = (p_1, \cdots, p_n)$ is a permutation of $(1,\cdots,n)$. Among all the pairs $(p_i, p_j)$ with $i<j$, those with $p_i>p_j$ are called \textbf{reversals}. We denote the number of reversals in $\pmb{p}$ as $r(\pmb{p})$. \\
    We agree that $r(\pmb{p}) = 0$ when $n=1$.
\end{Df}

\begin{Df}{Df8.1.-1.3 (the parity of a permutation)}
    A permutation $\pmb{p}$ is called \textbf{odd} (resp. \textbf{even}) if $r(\pmb{p})$ is odd (resp. even). 
\end{Df}

\begin{Th}{Lma8.1.-1.3 (a swap changes the parity of a permutation)}
    Suppose $\pmb{p} = (p_1, \cdots, p_n)$ is a permutation of $(1,\cdots,n)$. Then for any swap $S$, the parity of $S(\pmb{p})$ is different from that of $\pmb{p}$.
    \tcblower
    \textit{Pf}: The result $S(\pmb{p})$ is something like ($i<j$):
    $$ \bullet\bullet\bullet\bullet p_j\circ\circ\circ p_i\bullet\bullet\bullet\bullet $$
    then the reversals among those pairs $(p_i,\bullet)$, $(p_j,\bullet)$ does not change; the reversals' changes among those pairs $(p_i, \circ)$ are each matched with $(p_j, \circ)$, resulting in a totally even number of change in the number of reversals. Thus the parity is changed by the last remaining pair swapped $p_i\leftrightarrow p_j$. 
\end{Th}

\begin{Th}{Th8.1.-1.4 (the parity and the number of swaps)}
    The odd (resp. even) permutations are exactly those which can be obtained by an odd (resp. even) number of swaps of the identity permutation $(1,\cdots,n)$. Furthermore, given two permutations $\pmb{p}_1$ and $\pmb{p}_2$, the parity of the number of swaps needed to transform $\pmb{p}_1$ to $\pmb{p}_2$ must be the same as the parity of $r(\pmb{p}_1)+r(\pmb{p}_2)$, and is independent of the exact sequence of swaps.
    \tcblower
    \textit{Pf}: Obvious by the Lma \{, ID: 8.1.-1.3\}.
\end{Th}

\begin{Df}{Df8.1 (determinants)}
    Suppose $A = (a_{ij})\in \mathbb{F}^{n,n}$ is a square matrix over a number field $\mathbb{F}$. Let denotes a permutation of $(1,\cdots,n)$. Then the \textbf{determinant} of $A$, denoted as $\det A$ or $|A|$, is defined as the number:
    \begin{equation}
        \det A = \sum_{\pmb{p}} (-1)^{r(\pmb{p})} (a_{1p_1}\cdots a_{np_n})\qquad \tag{1}
    \end{equation}
    where $\pmb{p} = (p_1, \cdots, p_n)$ denotes the permutation of $(1,\cdots,n)$.
\end{Df}

\begin{Th}{Th8.2 (equivalent definition of determinants: expansion by a row or a column)}
    Suppose $A = (a_{ij})$ is a square matrix of $n$ orders. Then:
    \begin{compactenum}
        \item If $n=1$, then $|A| = a_{11}$;
        \item If $n\geq 2$, then for any $i,j\in \{1,\cdots,n\}$, we have
        $$ |A| = \sum_{l=1}^{n} (-1)^{i+l} a_{il}|A_{il}| = \sum_{k=1}^{n} (-1)^{k+j} a_{kj}|A_{kj}|, $$
        where $A_{kl}$ denotes the remaining matrix after removing the row and the column where $a_{kl}$ lies on.
    \end{compactenum}
    \tcblower
    \textit{Pf}: Trivial, as we can group the terms (1) in Df \{, ID: 8.1\} into $n$ groups (for example, fix $i$):
    $$
    \begin{aligned}
        \sum a\cdots a\, & a_{i1}\, a\cdots a \\
        \sum a\cdots a\, & a_{i2}\, a\cdots a \\
        &\vdots \\
        \sum a\cdots a\, & a_{in}\, a\cdots a
    \end{aligned}
    $$
    and immediately get the result.
\end{Th}

\begin{Th}{Th8.3 (the properties of determinants)}
    Let $A$ and $B$ be $n$-order matrices over $\mathbb{F}$. Then:
    \begin{compactenum}
        \item $|A^\prime| = |A|$, $|A^\ast| = |A|^\ast$;
        \item Exchange $A$ its two rows (resp. two columns), $|A|$ turns to $-|A|$;
        \item Multiply $A$ a row (resp. column) by a scalar $c$, $|A|$ turns to $c|A|$;
        \item Add one multiple of some row (resp. column) to another row (resp. column), $|A|$ remains the same;
        \item $\det\, [\pmb{a}_1 + \pmb{b}_1, \pmb{a}_2,\cdots, \pmb{a}_n] = \det\,[\pmb{a}_1, \pmb{a}_2\cdots,\pmb{a}_n] + \det\,[\pmb{b}_1,\pmb{a}_2,\cdots,\pmb{a}_n]$ (similar for the cases of other columns and other rows);
        \item $|AB| = |A|\cdot |B|$;
        \item If $A$ is triangular, then $|A| = \prod_{k=1}^{n} a_{kk}$;
        \item If $A$ is block-triangular with the diagonal be the matrices $A_{11}, \cdots, A_{mm}$, then $|A| = \prod_{k=1}^{m} |A_{kk}|$.
    \end{compactenum}
    \tcblower
    \textit{Pf}: 
    \begin{compactenum}
        \item Obvious;
        \item Obvious;
        \item Obvious;
        \item Trivial;
        \item Trivial;
        \item Trivial;
        \item Obvious;
        \item Apply Gaussian elimination for each $A_{kk}$, obvious.
    \end{compactenum}
\end{Th}

\begin{Th}{Th8.4 (Cramer's rule)}
    Learn by yourself.
\end{Th}

\begin{Df}{Df8.5 (characteristic polynomial)}
    Suppose $A\in\mathbb{F}^{n,n}$. Then the characteristic polynomial $p(z)$ of $A$ is defined as:
    $$ p(z) = \det (A-zI), $$
    where $I$ is the identity matrix.
\end{Df}

\begin{Rmk}{}
    \begin{compactenum}
        \item Since we can easily verify that \textcolor{Th}{Similar square-matrices share the same determinant}, \textcolor{Df}{we can define the \textbf{characteristic polynomial of a linear map} $T$ as $p(z) = \det (\mathcal{M}(T)-zI)$}, as the polynomial is independent of the choice of $\mathcal{M}(T)$.
        \item \textcolor{Th}{Suppose $T$ is an operator over $\mathbb{F}$ (resp. $A$ is a square-matrix over $\mathbb{F}$). A number $\lambda\in\mathbb{F}$ is an eigenvalue of $T$ (resp. an eigenvalue of $A$) iff $\lambda$ is a root of the characteristic polynomial of $T$ (resp. $A$)}
        \item \textcolor{Df}{Say a polynomial $p(z)$ ``splits'' if it can be factored as the product of degree-1 polynomials.} For example, \textcolor{Th}{all polynomial in $\mathbb{C}$ split.} Moreover, \textcolor{Th}{we can weaken the condition ``$V$ is over $\mathbb{C}$'' in the Th \{, ID: 6.3.2\} and the Schur's theorem (the Th \{, ID: 7.4.3\}) with ``the characteristic polynomial of $T$ splits''.}
    \end{compactenum}
\end{Rmk}

\end{document}