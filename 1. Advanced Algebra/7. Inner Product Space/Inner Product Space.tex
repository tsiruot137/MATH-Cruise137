\documentclass{article}

    \usepackage{xcolor}
    \definecolor{pf}{rgb}{0.4,0.6,0.4}
    \usepackage[top=1in,bottom=1in, left=0.8in, right=0.8in]{geometry}
    \usepackage{setspace}
    \setstretch{1.2} 
    \setlength{\parindent}{0em}

    \usepackage{paralist}
    \usepackage{cancel}

    \usepackage{ctex}
    \usepackage{amssymb}
    \usepackage{amsmath}

    \usepackage{tcolorbox}
    \definecolor{Df}{RGB}{0, 184, 148}
    \definecolor{Th}{RGB}{9, 132, 227}
    \definecolor{Rmk}{RGB}{215, 215, 219}
    \newtcolorbox{Df}[2][]{colbacktitle=Df, colback=white, title={\large\color{white}#2},fonttitle=\bfseries,#1}
    \newtcolorbox{Th}[2][]{colbacktitle=Th, colback=white, title={\large\color{white}#2},fonttitle=\bfseries,#1}
    \newtcolorbox{Rmk}[2][]{colbacktitle=Rmk, colback=white, title={\large\color{black}{Remarks}},fonttitle=\bfseries,#1}

    \title{\LARGE \textbf{Inner Product Space}}
    \author{\large Jiawei Hu}

\begin{document}
\maketitle

This is the 7th chapter of advanced algebra, which is about \textbf{Inner Product Space}\\
Here it is necessary to claim a ``definition (Df) -> theorem (Th)'' working cycle, which acts as the writing style throughout this whole course. This working cycle is shown below:

\noindent\rule{\textwidth}{2pt}
\begin{Df}{Some Definition}
    The text of this definition.
\end{Df}

\begin{Rmk}{}
    The text of the remarks about the definition just proposed (possibly including what it means and what it is for).\\
    \textcolor{Df}{Some remarks with some incidental definitions.}\\
    \textcolor{Th}{Some remarks with some incidental theorems.}
\end{Rmk}

\begin{Th}{Some Theorem}
    The text of this theorem.
    \tcblower
    \textit{Pf}: The proof of this theorem (is possibly "todo" when the author cannot complete it yet).
\end{Th}

\begin{Rmk}{}
    The text of the remarks about the definition just proposed (possibly including what it means and what it is for).\\
    \textcolor{Df}{Some remarks with some incidental definitions.}\\
    \textcolor{Th}{Some remarks with some incidental theorems.}
\end{Rmk}
\noindent\rule{\textwidth}{2pt}
As for the text of both a definition or a theorem, a common fixed pattern of sentences is adopted, which is ``Suppose \dots (some pre-conditions or background information). Then \dots (the direct text for the definition or the theorem).''. Please identify this pattern later by yourself. 

By the way, we now reiterate some commonly-used notations and conventions:
\begin{compactenum}
    \item $\mathbb{C}$: the set of the complex numbers;
    \item $\mathbb{R}$: the set of the real numbers;
    \item $\mathbb{R}^+$: the set of the positive real numbers;
    \item $\mathbb{Q}$: the set of the rational numbers;
    \item $\mathbb{Z}$: the set of the integers;
    \item $\mathbb{N}$: the set of the natural numbers;
    \item $\mathbb{N^\ast}$ or $\mathbb{N}^+$: the set of the positive integers.
    \item $\sideset{^R}{}{\mathop{D}}$: the set of all functions from $D$ to $R$ (with domain $D$ and range in $R$).
    \item An agreement for the length of a list: if we write $a_1, \dots, a_n$, then we indicate that $n$ is finite and that $n\geq 1$; if we write $a_0, \dots, a_n$, then we indicate that $n$ is finite and that $n\geq 0$.
    \item $A\times B$: the Cartesian product of $A$ and $B$.
    \item $\mathbb{F}$: a number field.
    \item Continue to use the notations and concepts of functions (see the chapter 1 of course 0).
    \item The matrix product $KA$ is referred as ``$A$ left-multiplied by $K$'' or ``left-multiply $A$ by $K$''; $AK$ is similar.
\end{compactenum} 
Please check the notations and definitions by yourself from the previous chapters or courses. Then with everything prepared, here we go.

\begin{Df}{$\bullet$ Df7.1 (inner product)}
    Suppose $V$ is a vector space over $\mathbb{F}$. An \textbf{inner product} on $V$ is a function $\langle \cdot, \cdot \rangle: V\times V\to \mathbb{F}$ that satisfies the following properties:
    \begin{compactenum}
        \item (positivity) $\forall v\in V$, $\langle v, v\rangle$ is a non-negative real number;
        \item (definiteness) $\forall v\in V$: $\langle v, v\rangle = 0$ iff $v=0$;
        \item (additivity of the 1st slot) $\forall u, v, w\in V$, $\langle u+v, w\rangle = \langle u, w\rangle + \langle v, w\rangle$;
        \item (homogeneity of the 1st slot) $\forall u, v\in V$ and $\forall \lambda\in \mathbb{F}$, $\langle \lambda u, v\rangle = \lambda \langle u, v\rangle$;
        \item (conjugate symmetry) $\forall u, v\in V$, $\langle u, v\rangle = \overline{\langle v, u\rangle}$.
    \end{compactenum}
\end{Df}

\begin{Rmk}{}
    The concept of inner product is a generalization of the dot product in $\mathbb{F}^n$. We can then derive the following basic properties of the inner product $\langle \cdot, \cdot\rangle: V\times V\rightarrow \mathbb{F}$:
    \textcolor{Th}{
    \begin{compactenum}
        \item $\forall u$, the map $\langle \cdot, u\rangle$ is a linear functional on $V$;
        \item $\forall v\in V$, $\langle v, 0\rangle = \langle 0, v\rangle = 0$;
        \item (additivity of the 2nd slot) $\forall u, v, w\in V$, $\langle u, v+w\rangle = \langle u, v\rangle + \langle u, w\rangle$;
        \item (homogeneity of the 2nd slot) $\forall u, v\in V$ and $\forall \lambda\in \mathbb{F}$, $\langle u, \lambda v\rangle = \overline{\lambda}\langle u, v\rangle$.
    \end{compactenum}
    }
\end{Rmk}

\begin{Df}{$\bullet$ Df7.1.1 (inner product space)}
    An inner product space is a vector space equipped with an inner product.
\end{Df}

\begin{Df}{$\bullet$ Df7.2 (norm)}
    Suppose $V$ is a inner product space. The norm of a vector $v\in V$ is defined as $\|v\| = \sqrt{\langle v, v\rangle}$.
\end{Df}

\begin{Rmk}{}
    Norm is a generalization of the length of a vector in euclidean space. 
    \textcolor{Th}{Suppose $V$ is an inner product space. Then the basic properties of norm:
    \begin{compactenum}
        \item $\forall v\in V$, $\|v\|=0$ iff $v=0$;
        \item $\forall v\in V, \lambda\in\mathbb{F}$, $\|\lambda v\| = |\lambda|\cdot \|v\|$.
    \end{compactenum}}
\end{Rmk}

\begin{Df}{$\bullet$ Df7.3 (orthogonality)}
    Suppose $V$ is an inner product space and $u, v\in V$. We say $u$ and $v$ are orthogonal if $\langle u, v\rangle = 0$.
\end{Df}

\begin{Rmk}{}
    The concept of orthogonality is a generalization of the concept of perpendicularity in euclidean space. We can see that \textcolor{Th}{in an inner product space, $0$ is orthogonal to all vectors, and $0$ is the only one in $V$ that is orthogonal to itself.}
\end{Rmk}

\begin{Df}{$\bullet$ Df7.3.1 (parallelism)}
    Suppose $V$ is an inner product space and $u, v\in V$. We say that:
    \begin{compactenum}
        \item $u$ and $v$ are parallel if $\exists \lambda\in\mathbb{F}$ such that $u = \lambda v$ or $v = \lambda u$.
        \item $u$ and $v$ share the same orientation if $\exists \lambda\in\mathbb{R}^+\cup \{0\}$ s.t. $u = \lambda v$ or $v = \lambda u$.
        \item $u$ and $v$ are of the opposite orientation if $\exists \lambda\in\mathbb{R}^+\cup \{0\}$ s.t. $u = -\lambda v$ or $v = -\lambda u$. 
    \end{compactenum}
\end{Df}

\begin{Th}{$\bullet$ Th7.3.2 (Pythagorean theorem)}
    Suppose $V$ is an inner product space and $u, v\in V$. If $u$ and $v$ are orthogonal, then $\|u+v\|^2 = \|u\|^2 + \|v\|^2$.
    \tcblower
    \textit{Pf}: Obvious using the properties of the inner product.
\end{Th}

\begin{Rmk}{}
    We also have the ``parallelogram law'' in the inner product space: \textcolor{Th}{For any $u, v\in V$, $\|u+v\|^2 + \|u-v\|^2 = 2\|u\|^2 + 2\|v\|^2$.} This is a generalization of the fact: for a parallelogram, the sum of the squared lengths of the diagonals is equal to the sum of the squared lengths of the sides.
\end{Rmk}

\begin{Th}{$\bullet$ Th7.3.3 (orthogonal projection)}
    Suppose $V$ is an inner product space. Suppose also $u, v\in V$ and $u\neq 0$. \textcolor{Df}{Define the orthogonal projection of $v$ onto $u$ as the vector $\text{Proj}(v|u)$ such that $\text{Proj}(v|u)$ is parallel to $u$ and $v-\text{Proj}(v|u)$ is orthogonal to $u$.} Then we have:
    \begin{compactenum}
        \item $\text{Proj}(v|u)$ exists and is unique;
        \item $\text{Proj}(v|u) = \frac{\langle v, u\rangle}{\langle u, u\rangle}u$.
    \end{compactenum}
    \tcblower
    \textit{Pf}: Trivial.
\end{Th}

\begin{Th}{$\bullet$ Th7.3.4 (Cauchy-Schwarz inequality)}
    Suppose $V$ is an inner product space. Then $\forall u, v\in V$: 
    $$|\langle u, v\rangle | \leq \|u\|\cdot \|v\|.$$
    And the equality holds iff $u$ and $v$ are parallel.
    \tcblower
    \textit{Pf}: Trivial if you notice that a vector is never shorter than its orthogonal projection.
\end{Th}

\begin{Th}{$\bullet$ Th7.3.5 (triangle inequality)}
    Suppose $V$ is an inner product space. Then $\forall u, v\in V$:
    $$\|u+v\|\leq \|u\| + \|v\|.$$
    And the equality holds iff $u$ and $v$ share the same orientation.
    \tcblower
    \textit{Pf}: Trivial.
\end{Th}

\begin{Rmk}{}
    This theorem is a generalization of the fact that ``the sum of the lengths of two sides of a triangle is greater than the length of the third side''.
\end{Rmk}

\end{document}