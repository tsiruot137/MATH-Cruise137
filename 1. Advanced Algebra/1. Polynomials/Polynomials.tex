\documentclass{article}

    \usepackage{xcolor}
    \definecolor{pf}{rgb}{0.4,0.6,0.4}
    \usepackage[top=1in,bottom=1in, left=0.8in, right=0.8in]{geometry}
    \usepackage{setspace}
    \setstretch{1.2} 
    \setlength{\parindent}{0em}

    \usepackage{paralist}
    \usepackage{cancel}

    \usepackage{ctex}
    \usepackage{amssymb}
    \usepackage{amsmath}

    \usepackage{tcolorbox}
    \definecolor{Df}{RGB}{0, 184, 148}
    \definecolor{Th}{RGB}{9, 132, 227}
    % \definecolor{Rdf}{RGB}{34, 166, 179}
    % \definecolor{Rth}{RGB}{86, 66, 143}
    \definecolor{Rmk}{RGB}{215, 215, 219}
    \newtcolorbox{Df}[2][]{colbacktitle=Df, colback=white, title={\large\color{white}#2},fonttitle=\bfseries,#1}
    \newtcolorbox{Th}[2][]{colbacktitle=Th, colback=white, title={\large\color{white}#2},fonttitle=\bfseries,#1}
    \newtcolorbox{Rmk}[2][]{colbacktitle=Rmk, colback=white, title={\large\color{black}{Remarks}},fonttitle=\bfseries,#1}

    \title{\LARGE \textbf{Polynomial}}
    \author{\large Jiawei Hu}

\begin{document}
\maketitle

This chapter is the 1st one of the course \textbf{Adavanced Algebra}, which is about the polynomials. Here it is necessary to claim a ``definition (Df) -> theorem (Th)'' working cycle, which acts as the writing style throughout this whole course. This working cycle is shown bellow:

\noindent\rule{\textwidth}{2pt}
\begin{Df}{Some Definition}
    The text of this definition.
\end{Df}

\begin{Rmk}{}
    The text of the remarks about the definition just proposed (possibly including what it means and what it is for).\\
    \textcolor{Df}{Some remarks with some incidental definitions.}\\
    \textcolor{Th}{Some remarks with some incidental theorems.}
\end{Rmk}

\begin{Th}{Some Theorem}
    The text of this theorem.
    \tcblower
    \textit{Pf}: The proof of this theorem (is possibly "todo" when the author cannot complete it yet).
\end{Th}

\begin{Rmk}{}
    The text of the remarks about the definition just proposed (possibly including what it means and what it is for).\\
    \textcolor{Df}{Some remarks with some incidental definitions.}\\
    \textcolor{Th}{Some remarks with some incidental theorems.}
\end{Rmk}
\noindent\rule{\textwidth}{2pt}
As for the text of both a definition or a theorem, a common fixed pattern of sentences is adopted, which is ``Suppose \dots (some pre-conditions or background information). Then \dots (the direct text for the definition or the theorem).''. Please identify this pattern later by yourself. 

By the way, we now pre-claim some commonly-used notations:
\begin{compactenum}
    \item $\mathbb{C}$: the set of the complex numbers;
    \item $\mathbb{R}$: the set of the real numbers;
    \item $\mathbb{Q}$: the set of the rational numbers;
    \item $\mathbb{Z}$: the set of the integers;
    \item $\mathbb{N}$: the set of the natural numbers;
    \item $\mathbb{N^\ast}$: the set of the positive integers.
\end{compactenum} 
Then with everything prepared, here we go.

\begin{Df}{$\bullet$ Df1.1 (number field (数域))}
    Suppose $\mathbb{F}\subseteq\mathbb{C}$ and $0,1\in\mathbb{F}$. Then $\mathbb{F}$ is called a number field if:
    \begin{compactenum}
        \item $\forall (a,b\in \mathbb{F}), (a+b, a-b, a\cdot b \in \mathbb{F})$;
        \item $\forall (a,b\in \mathbb{F} \text{ and } b\neq 0), (a/b\in \mathbb{F})$.
    \end{compactenum}
\end{Df}

\begin{Rmk}{}
    In other word, a subset $\mathbb{F}$ of $\mathbb{C}$ that contains 0 and 1 is called a number field if it is closed under addition, subtraction, multiplication and division (the divider is non-zero).\\
    \textcolor{Th}{For example, $\mathbb{Q}$, $\mathbb{R}$ and $\mathbb{C}$ are number field, while $\mathbb{Z}$ is not.}
\end{Rmk}

\begin{Th}{$\bullet$ Th1.1.1 ($\mathbb{Q}$ is the minimal number field)}
    \begin{compactenum}
        \item $\mathbb{Q}$ is a number field;
        \item For any number field $\mathbb{F}$, $\mathbb{F}\supseteq\mathbb{Q}$.
    \end{compactenum}
    \tcblower
    \textit{Pf}: Since any number field $\mathbb{F}$ contains 0 and 1, it contains all integers due to the closure of addition and subtraction; and it further contains all fractions due to the closure of division.
\end{Th}

\begin{Df}{$\bullet$ Df1.2 (polynomial)}
    Suppose $\mathbb{F}$ is a number field and $\pmb{a} = \{a_n: n\in\mathbb{Z}\}$ is a sequence of numbers in $\mathbb{F}$ such that
    \begin{compactenum}
        \item $a_n = 0$ for all negative $n$;
        \item There are finitely many $a_n$'s that are non-zero.
    \end{compactenum}
    Suppose also $x$ is a charactor. Then the tuple $(\pmb{a}, x)$ is called a \textbf{polynomial} about $x$ in $\mathbb{F}$ (or, a polynomial in the variable $x$ with coefficients in $\mathbb{F}$)
\end{Df}

\begin{Rmk}{}
    \begin{compactenum}
        \item This definition of polynomial is an abstraction of the polynomials we learnt before. What we learnt is that a polynomial is a function about $x$ of the form $f(x) = a_nx^n+\dots+a_1x+a_0$, where $x$ is engaged in some operations of addition, multiplication and power and we can see that such a polynomial is determined by a sequence of numbers $(a_0, \dots, a_n)$. The definition here generalizes the concept of addition, multiplication and power, and is regardless of what $x$ is. Instead, it is just a tuple $(\pmb{a}, x)$ and can be denoted in a function-like way $f(x)$ where $f$ wraps the information of the coefficients $\pmb{a}$ and $x$ tells us what the polynomial is about. Actually we make this abstraction aimed at explore the common properties of polynomials.
        \item To link this kind of general polynomial to the one we learnt before, we first try to write it as the form $f(x) = a_nx^n + \dots + a_1x + a_0$, which requires the specification of the addition and multiplication for polynomials. \textcolor{Df}{For any two polynomials $f(x) = (\pmb{a}, x)$ and $g(x) = (\pmb{b}, x)$ about $x$ in $\mathbb{F}$, we define the addition and multiplication of them as:
        \begin{compactitem}
            \item $f(x)+g(x) = (\pmb{a}+\pmb{b}, x)$;
            \item $f(x)\cdot g(x) = (\{c_n: n\in\mathbb{Z}\}, x)$ where $c_n = \sum_{i+j = n} a_ib_j$.
        \end{compactitem}}
        And we can see that this definition of addition and multiplication is consistent with the one we learnt before. \textcolor{Th}{Now if for a sequence $\{a_n\}$ that has only one non-zero element $a_i$, we denote the polynomial $(\{a_n\}, x)$ by $a_ix^i$, we can see that any polynomial $f(x) = (\{a_n\}, x)$ where $n$ is an integer such that $a_N=0$ for all $N>n$, can be written as $f(x) = a_nx^n + \dots + a_0x^0$ or $f(x) = \sum_{i=0}^{n} a_ix^i$} \textcolor{Df}{Here we define the largest $n$ such that $a_n\neq 0$ (if there are any non-zero elements in $\{a_n$\}) as the degree of $f(x)$, denoted by $\deg f(x) = n$; if all elements of $\{a_n\}$ is $0$, we call $f(x)$ the zero polynomial, whose degree is defined as $\deg f(x) = -\infty$.}
        \item For a polynomial $f(x) = (\{a_n\}, x)$, the sequence of numbers $\{a_n\}$ is called the coefficients of $f(x)$ and $a_k$ is called the $k$-th coefficient.
        \item \textcolor{Df}{In convention, we write $a_0x^0$ as $a_0$ as if $x^0 = 1$. Although inaccurate since $0^0$ is meaningless, this handling shows its convenience in algebra.}
        \item From the definition of polynomial addition and multiplication, we should clearly understand that \textcolor{Th}{any number field $\mathbb{F}$ makes the collection of all polynomials in it closed under polynomial addition, subtraction and multiplication.}
        \item And we can easily verify \textcolor{Th}{the commutativity, associativity of both addition and multiplication among polynomials, along with the distributive law of multiplication over addition.}
        \item By simple subtraction, we can also easily prove the cancellation of multiplication:\\
        \textcolor{Th}{Suppose $f(x)$, $g(x)$ and $h(x)$ are all polynomials in $\mathbb{F}$, and $h(x)\neq 0$. Then $h(x)f(x) = h(x)g(x)$ implies $f(x) = g(x)$.}
    \end{compactenum}
\end{Rmk}

\begin{Th}{$\bullet$ Th1.2.1 (the degree of polynomial sum and product)}
    Suppose $f(x)$ and $g(x)$ are both polynomials in $\mathbb{F}$. Then:
    \begin{compactenum}
        \item $\deg(f(x)\pm g(x))\leq \max(\deg f(x), \deg g(x))$;
        \item $\deg(f(x)g(x)) = \deg f(x)+\deg g(x)$.
    \end{compactenum}
    \tcblower
    \textit{Pf}: Obvious.
\end{Th}

\begin{Rmk}{}
    Obviously this theorem still holds for the case of arbitrarily finitely many polynomials.
\end{Rmk}

\begin{Df}{$\circ$ Df1.3 (unary (一元的) polynomial ring on $\mathbb{F}$)}
    Suppose $\mathbb{F}$ is a number field and $x$ is a charactor. Then the collection of all polynomials about $x$ in $\mathbb{F}$ is called a unary polynomial ring on $\mathbb{F}$, denoted by $\mathcal{P}_\mathbb{F}(x)$. And we denote a polynomial $f(x)$ in this ring by $f(x)\in \mathcal{P}_\mathbb{F}(x)$.
\end{Df}

\begin{Rmk}{}
    This definition is something involving the group theory we will learn in the future. Now let us just treat it as a pre-condition when we discuss the polynomials.
\end{Rmk}

\begin{Th}{$\bullet$ Th1.4 (division with remainder in $\mathcal{P}_\mathbb{F}(x)$)}
    Suppose $\mathbb{F}$ is a number field. Then for any polynomials $f(x), g(x)\in \mathcal{P}_\mathbb{F}(x)$ such that $g(x)\neq 0$, there exist unique $q(x), r(x)\in \mathcal{P}_\mathbb{F}(x)$ s.t. 
    $$f(x) = q(x)g(x)+r(x) \text{ and } \deg r(x)<\deg g(x)$$
    \tcblower
    \textit{Pf}: \begin{compactenum}
        \item[(I)] Existence: for any such given $f(x)$ and $g(x)$, we can apply the polynomial division algorithm (please search online by yourself), and then yield the quotient $q(x)$ and the remainder $r(x)$ with $\deg r(x)<\deg g(x)$.
        \item[(II)] Uniqueness: we can first assume that $f(x) = 0$ and prove that $q(x) = r(x) = 0$. Then we assume that there are $g_1(x), r_1(x)$ and $g_2(x), r_2(x)$ satisfying the condition, and obtain that $q(x)[g_1(x)-g_2(x)]+[r_1(x)-r_2(x)] = 0$, namely, $g_1(x) = g_2(x)$, $r_1(x) = r_2(x)$.
    \end{compactenum}
\end{Th}

\begin{Rmk}{}
    The division with remainder here is similar to the one of integers. \textcolor{Df}{Here $q(x)$ and $r(x)$ are again called the quotient and the remainder respectively.} And we will see that the exact division theory of polynomials (talked later) can be found on the same logic with that in the number theory.
\end{Rmk}

\begin{Df}{$\bullet$ Df1.5 (exact division (整除))}
    Suppose $f(x), g(x)\in \mathcal{P}_\mathbb{F}(x)$. Then we say $g(x)$ exactly divides $f(x)$ (or $g(x)$ divides $f(x)$, or $f(x)$ is divisible by $g(x)$) if:
    $$\exists h(x) \text{ s.t. } f(x) = h(x)g(x).$$
\end{Df}

\begin{Rmk}{}
    \textcolor{Df}{``$g(x)$ exactly divides $f(x)$'' is denoted by ``$g(x)\mid f(x)$'', and ``$g(x)$ does not exactly divides $f(x)$'' is by ``$g(x)\nmid f(x)$''. If $g(x)\mid f(x)$, then $g(x)$ is called a factor of $f(x)$ and $f(x)$ is called a multiple of $g(x)$.} Note that the definition of exact division does not force $g(x)$ to be non-zero.
\end{Rmk}

\begin{Th}{$\bullet$ Th1.5.1}
    Suppose $f(x), g(x)\in \mathcal{P}_\mathbb{F}(x)$ and $g(x)\neq 0$. Then:\\
    $g(x)$ exactly divides $f(x)$ iff the remainder of $f(x)$ divided by $g(x)$ is $0$.
    \tcblower
    \textit{Pf}: Obvious.
\end{Th}

\begin{Rmk}{}
    From the definition of exact division, we can see that: \textcolor{Th}{
    \begin{compactenum}
        \item Any polynomial divides itself;
        \item Any polynomial divides $0$;
        \item Any 0-degree polynomial divides any polynomial.
    \end{compactenum}}
    When $g(x)\mid f(x)$, \textcolor{Df}{we can also denote the quotient of $f(x)$ divided by $g(x)$ by $f(x)/g(x)$ or $\frac{f(x)}{g(x)}$}
\end{Rmk}

\begin{Th}{$\bullet$ Th1.5.2 (some basic properties of exact division)}
    Suppose $f(x), g(x)\in\mathcal{P}_\mathbb{F}(x)$. Then:
    \begin{compactenum}
        \item If $f(x)\mid g(x)$ and $g(x)\mid f(x)$, then $f(x) = cg(x)$ for some non-zero number $c$ in $\mathbb{F}$;
        \item (Transitive:) If $h(x)\mid g(x)$ and $g(x)\mid f(x)$, then $h(x)\mid f(x)$;
        \item Suppose also $f_1(x), \dots, f_n(x), u_1(x), \dots, u_n(x)\in \mathcal{P}_\mathbb{F}(x)$ where $n$ is a positive integer. If $g(x)\mid f_i(x)$ for $i=1,\dots,n$, then $g(x)\mid\sum_{i=1}^{n} u_i(x)f_i(x)$.
    \end{compactenum}
    \tcblower
    \textit{Pf}: Obvious.
\end{Th}

\begin{Rmk}{}
    \textcolor{Df}{About the 3rd term of the theorem above, we called $\sum_{i=1}^{n} u_i(x)f_i(x)$ a combination of the polynomials $f_1(x),\dots, f_n(x)$.}
    \textcolor{Th}{Also from the above theorem, we can see that: if $f(x) = cg(x)$ for some non-zero number $c\in\mathbb{F}$, then $f(x)$ and $g(x)$ share the common factors and have the common multiples. In other word, $f(x) = cg(x)$ indicates that $f(x)$ and $g(x)$ are identical in the sense of exact division.}
\end{Rmk}

\begin{Th}{$\circ$ Th1.5.3 (the exact division is invariant of the enlargement of $\mathbb{F}$)}
    Suppose $\mathbb{F}, \mathbb{G}$ are both number fields and $\mathbb{G}\supseteq\mathbb{F}$. Suppose also $f(x), g(x)\in \mathbb{F}$. Then:\\
    $f(x), g(x)\in\mathbb{G}$, and $f(x), g(x)$ keep all relation of exact division when condidered as polynomials in $\mathbb{G}$.
    \tcblower
    \textit{Pf}: Obvious. 
\end{Th}

\begin{Df}{$\bullet$ Df1.6 (the greatest common factor (gcd)) (最大公因式)}
    \begin{compactitem}
        \item Suppose $f(x), g(x), d(x)\in \mathcal{P}_\mathbb{F}(x)$ and $f(x), g(x)$ are not both $0$. Then $d(x)$ is called the greatest common factor of $f(x)$ and $g(x)$ if:
        \begin{compactenum}
            \item $d(x)\mid f(x), g(x)$;
            \item For any factor $\varphi(x)$ of both $f(x)$ and $g(x)$, $d(x)\mid \varphi(x)$.
        \end{compactenum}
        \item We define the only greatest common factor of two zero-polynomials as also $0$.
        \item For a single polynomial $f(x)$ in $\mathcal{P}_\mathbb{F}(x)$, we define its greatest common factor(s) as the ones of $f(x)$ and $f(x)$.
    \end{compactitem}
\end{Df}

\begin{Rmk}{}
    This definition of gcd derives from the one in the number theory. Obviously one thing proceeds all the discussions about it — the discussion about existence and the uniqueness of gcd.
    \begin{compactenum}
        \item \textcolor{Th}{(Existence): for any polynomials $f(x)$ and $g(x)$, their gcd exists.} This can be clarified by the Euclidean algorithm (辗转相除法).
        \item \textcolor{Th}{(Uniqueness): for any polynomials $f(x)$ and $g(x)$, their gcd is unique regardless of a non-zero constant multiplier.} Namely, if $d_1(x)$ and $d_2(x)$ are both the gcd of $f(x)$ and $g(x)$, then $d_1(x) = cd_2(x)$ for some non-zero $c$ in $\mathbb{F}$. This fact is from the definition of gcd itself.
    \end{compactenum}
    \textcolor{Df}{Then we denote the monic (首项系数为一的) gcd of $f(x)$ and $g(x)$ by $\gcd(f(x), g(x))$.}
\end{Rmk}

\begin{Df}{$\bullet$ Df1.6.1 (coprime (互素的) polynomial)}
    Suppose $f(x), g(x)\in\mathcal{P}_\mathbb{F}(x)$. Then $f(x)$ and $g(x)$ are called coprime if $\gcd(f(x), g(x)) = 1$.
\end{Df}

\begin{Rmk}{}
    This definition is also from the elementary number theory.
\end{Rmk}

\begin{Th}{$\bullet$ Th1.6.2 (Bezout's theorem)}
    Suppose $f(x), g(x)\in\mathcal{P}_\mathbb{F}(x)$. Then:
    \begin{compactenum}
        \item $\exists u(x), v(x)\in \mathcal{P}_\mathbb{F}(x)$ such that $u(x)f(x)+v(x)g(x) = \gcd(f(x), g(x))$;
        \item $f(x)$ and $g(x)$ are coprime $\Leftrightarrow $ $\exists u(x), v(x)\in \mathcal{P}_\mathbb{F}(x) \text{ such that } u(x)f(x)+v(x)g(x) = 1$.
    \end{compactenum}
    \tcblower
    \textit{Pf}: \begin{compactenum}
        \item Consider the Euclidean algorithm and let the remainders of the sequence of division equations $E_1, E_2, \dots, E_n$ are $r_1(x), r_2(x), \dots, r_n(x)$. Then by replacing $r_1$ in $E_2$ using $E_1$, replacing $r_2$ in $E_3$ using $E_2$, and so forth, we obtain $u(x)$ and $v(x)$.
        \item `only if' now follows 1., and we now prove `if'. If $u(x)f(x)+v(x)g(x) = 1$ for some $u(x)$ and $v(x)$, then the gcd $d(x)$ of $f(x)$ and $g(x)$ divides both of them, and thus divides their combination $u(x)f(x)+v(x)g(x)$, namely, divides $1$. Therefore $d(x) = 1$
    \end{compactenum}
\end{Th}

\begin{Th}{$\circ$ Th1.6.3 (some useful conclusions)}
    Suppose the polynomials mentioned below are all in $\mathcal{P}_\mathbb{F}(x)$. Then:
    \begin{compactenum}
        \item If $g(x)\mid f(x)h(x)$ and $\gcd(f(x), g(x)) = 1$, then $g(x)\mid h(x)$;
        \item If $g_1(x)\mid f(x), g_2(x)\mid f(x)$ and $\gcd(g_1(x), g_2(x)) = 1$, then $g_1(x)g_2(x)\mid f(x)$
    \end{compactenum}
    \tcblower
    \textit{Pf}: Obvious using the Bezout's theorem.
\end{Th}

\begin{Df}{$\bullet$ Df1.7 (gcd in the multi-dimensional cases)}
    \begin{compactitem}
        \item Suppose $f_1(x), \dots, f_n(x)\in\mathcal{P}_\mathbb{F}(x)$ where $n\in\mathbb{N}^\ast$ and $f_1, \dots, f_n$ are not all zero. Then a polynomial $d(x)\in\mathcal{P}_\mathbb{F}(x)$ is called the greatest common factor of $f_1(x), \dots, f_n(x)$ if:
        \begin{compactenum}
            \item $d(x)\mid f_1(x), \dots, f_n(x)$;
            \item For any factor $\varphi(x)$ of all $f_1(x), \dots, f_n(x)$, we have $d(x)\mid \varphi(x)$.
        \end{compactenum}
        \item We define the only gcd of arbitrarily finitely many zero-polynomials as also $0$.
    \end{compactitem}
\end{Df}

\begin{Rmk}{}
    \begin{compactitem}
        \item This definition is a natural extension to the multi-dimensional cases. \textcolor{Th}{Here we can still clearly confirm the existence and uniqueness (regardless of a non-zero constant multiplier) of the gcd.}
        \item As we cannot carry on the Euclidean algorithm in multi-dimensional cases, the computation of multi-polynomial-gcd is proposed as the theorem here:\\
        \textcolor{Th}{Suppose $F$ is a list of some polynomials in $\mathcal{P}_\mathbb{F}(x)$. If we partition $F(x)$ by two non-empty and disjoint lists $F_1, F_2$, then $\gcd(F) = \gcd[\gcd(F_1), \gcd(F_2)]$}\\
        The proof of this theorem is also easy just using the definition. And this theorem indicates that we can obtain the gcd of many polynomials, just by computing for any two of them each time and aggregating the results in arbitrary order.
        \item \textcolor{Df}{We still say $f_1(x), \dots, f_n(x)$ are coprime if $\gcd(f_1(x), \dots, f_n(x)) = 1$.} And we can see that \textcolor{Th}{the Bezout theorem proposed above still holds for the multi-dimensional cases.}
    \end{compactitem}
\end{Rmk}

\begin{Df}{$\bullet$ Df1.8 (irreducible (不可约的) polynomial)}
    \begin{compactitem}
        \item Suppose $p(x)\in\mathcal{P}_\mathbb{F}(x)$ and $\deg p(x)\geq 1$. Then $p(x)$ is called reducible if $\exists g_1(x), g_2(x)\in\mathcal{P}_\mathbb{F}(x)$ such that:
        \begin{compactenum}
            \item $p(x) = g_1(x)g_2(x)$;
            \item $\deg g_1(x), \deg g_2(x) < \deg p(x)$.
        \end{compactenum}
        \item Suppose $p(x)\in\mathcal{P}_\mathbb{F}(x)$ and $\deg p(x)\geq 1$. Then $p(x)$ is called irreducible if it is not reducible.
    \end{compactitem}
\end{Df}

\begin{Rmk}{}
    \begin{compactenum}
        \item The discussion about reducibility of polynomials also derives from elementary number theory.
        \item Generally whether a polynomial is reducible depends on which $\mathcal{P}_\mathbb{F}(x)$ it lies, since we can often see a polynomials in $\mathbb{R}$ cannot be decomposed of two factors in $\mathbb{R}$, but can be decomposed instead in $\mathbb{C}$.
        \item \textcolor{Th}{According to the definition, a polynomial with degree 1 is always irreducible.}
    \end{compactenum}
\end{Rmk}

\begin{Th}{$\bullet$ Th1.8.1 (the only factors of an irreducible polynomial are $1$ and itself)}
    Suppose $p(x)\in\mathcal{P}_\mathbb{F}(x)$ and $\deg p(x)\geq 1$. Then:\\
    $p(x)$ is irreducible $\Leftrightarrow$ A factor of $p(x)$ is either $1$ or $p(x)$ itself (regardless of a non-zero constant multiplier).
    \tcblower
    \textit{Pf}: Obvious.
\end{Th}

\begin{Rmk}{}
    \begin{compactenum}
        \item \textcolor{Th}{Suppose $p(x)\in\mathcal{P}_\mathbb{F}(x)$ is irreducible. Then for any $f(x)\in\mathcal{P}_\mathbb{F}(x)$, $p(x)\mid f(x)$ or $\gcd(p(x), f(x)) = 1$.}
    \end{compactenum}
\end{Rmk}

\begin{Th}{$\bullet$ Th1.8.2}
    Suppose $p(x)\in\mathcal{P}_\mathbb{F}(x)$ is irreducible and $f(x), g(x)\in\mathcal{P}_\mathbb{F}(x)$. Then:\\
    $p(x)\mid f(x)g(x)$ $\Leftrightarrow$ $p(x)\mid f(x)$ or $p(x)\mid g(x)$.
    \tcblower
    \textit{Pf}: Obvious.
\end{Th}

\begin{Rmk}{}
    Obviously, the above theorem can be recursively extended to the multi-dimensional cases.
\end{Rmk}

\begin{Th}{$\bullet$ Th1.8.3 (the irreducible factorization of a polynomial)}
    Suppose $f(x)\in\mathcal{P}_\mathbb{F}(x)$ and $\deg f(x)\geq 1$. Then $f(x)$ can be uniquely factorized into a product of irreducible polynomials in $p_1(x), \dots, p_n(x)\in\mathcal{P}_\mathbb{F}(x)$. \\
    Here the uniqueness means that:
    if $f(x) = p_1(x)\dots p_n(x) = q_1(x)\dots q_m(x)$ for some irreducible polynomials $p_1(x), \dots, p_n(x), q_1(x), \dots, q_m(x)$, then $n = m$ and $p_i(x)\overset{c}{=}q_i(x)$ for all $i$ after a proper permutation of the indices (``$\overset{c}{=}$'' means the left and the right sides are equal regardless of a non-zero constant multiplier).
    \tcblower
    \textit{Pf}: 
    \begin{compactitem}
        \item Existence: if $f(x)$ is irreducible, then we are done; otherwise, factor $f(x)$. Then we can further factorize the factors of $f(x)$ just yielded, and so forth, until all the factors are irreducible.
        \item Uniqueness: assume that $f(x) = p_1(x)\dots p_n(x) = q_1(x)\dots q_m(x)$ for some irreducible $p_i(x)$'s and $q_j(x)$'s. Then we can see that $p_1(x)\mid f(x) = q_1(x)\dots q_m(x)$, and thus $p_1(x)\mid q_j(x)$ for some $j$. Since $q_j(x)$ is also irreducible, we have $q_j(x)\mid p_1(x)$ (as it is impossible that $\gcd(p_1(x), q_j(x)) = 1$). Hence $p_1(x)\overset{c}{=}q_i(x)$, and then we can remove $p_1(x)$ and $q_i(x)$ from the equation. Next is $p_2(x)$, which is paired in this way with some $q_k(x)$, and so forth. Finally we can obtain that $n = m$ and $p_i(x)\overset{c}{=}q_i(x)$ for all $i$ after a proper permutation of the indices.
    \end{compactitem}
\end{Th}

\begin{Rmk}{}
    Although this theorem claims the existence of this factorization, it does not give a method to compute it. A universal approach for this factorization does not exist.\\
    \textcolor{Df}{This factorization is called the irreducible factorization of polynomials.}\\
    \textcolor{Df}{For convenience, in the factorization we often extract the constant multipliers of all the irreducible factors $p_1(x), \dots, p_n(x)$ to reduce them to monic polynomials, and then merge the same factors. This way we write $f(x)$ as the form like $f(x) = c[p_1(x)]^{n_1}\dots [p_m(x)]^{n_m}$, which is called the canonical factorization of $f(x)$}
\end{Rmk}

\begin{Df}{$\bullet$ Df1.9 (multiplicative factor (重因式))}
    Suppose $f(x)\in\mathcal{P}_\mathbb{F}(x)$. Then an irreducible $p(x)$ in $\mathcal{P}_\mathbb{F}(x)$ is called a $k$-multiplicative factor of $f(x)$ if $[p(x)]^k\mid f(x)$ and $[p(x)]^{k+1}\nmid f(x)$.
\end{Df}

\begin{Rmk}{}
    \begin{compactitem}
        \item \textcolor{Df}{A 1-multiplicative factor is also called a simple factor and a $k$-multiplicative factor is called as a multiple factor. If $p(x)$ is a $k$-multiplicative factor of $f(x)$, then $k$ is called the multiplicity of $p(x)$ (for $f(x)$).}
        \item \textcolor{Th}{For a polynomial $f(x)$ with $\deg f(x)\geq 1$, its canonical factorization exactly marks the orders of each multiplicative factors. For example, if $f(x) = c p_1(x)^{n_1}\dots p_m(x)^{n_m}$, then $p_1(x)$ is a $n_1$-multiplicative factor of $f(x)$, and so forth; as for those factors that do not appear in the canonical factorization, they are $0-$ multiplicative factors of $f(x)$.}
    \end{compactitem}
\end{Rmk}

\begin{Df}{$\bullet$ Df1.10 (derivatives(微商))}
    Suppose $f(x)\in\mathcal{P}_\mathbb{F}(x)$. Then the derivative of $f(x)$, denoted by $f'(x)$, is defined as the polynomial obtained by the derivative rule in analysis.
\end{Df}

\begin{Rmk}{}
    \begin{compactitem}
        \item To compute the derivative, just like $(a_nx^n)^\prime = a_nnx^{n-1}$;
        \item \textcolor{Th}{$\deg f'(x) = \deg f(x) -1$ (1 minus 1 is said to be $-\infty$ here).}
    \end{compactitem}
\end{Rmk}

\begin{Th}{$\bullet$ Th1.10.1 (multiplicative factors in $f(x)$ and $f'(x)$)}
    Suppose in $\mathcal{P}_\mathbb{F}(x)$, irreducible $p(x)$ is a $k$-multiplicative factor of $f(x)$ where $k\geq 1$. Then $p(x)$ is a $(k-1)$-multiplicative factor of $f'(x)$.
    \tcblower
    \textit{Pf}: Obvious.
\end{Th}

\begin{Rmk}{}
    \textcolor{Th}{Suppose $f(x), p(x)\in \mathcal{P}_\mathbb{F}(x)$ with $\deg f(x)\geq 1$ and $p(x)$ irreducible. Then:
    \begin{compactitem}
        \item $p(x)$ is a multiple factor of $f(x)$ $\Leftrightarrow$ $p(x)$ is a common factor of $f(x)$ and $f'(x)$;
        \item $f(x)$ and $f'(x)$ have no common multiple factors $\Leftrightarrow$ $f(x)$ and $f'(x)$ are coprime.
    \end{compactitem}}
    This theorem can be used to remove the multiplicities of a polynomial. Actually, if $f(x) = c[p_1(x)]^{n_1} \dots [p_m(x)]^{n_m}$, then $\frac{f(x)}{\gcd(f(x), f'(x))} = cp_1(x)\dots p_m(x)$, where the multiplicity of each factor is reduced to 1, and we keep all the original factors of $f(x)$.
\end{Rmk}

\begin{Df}{$\bullet$ Df1.11 (polynomial functions)}
    Suppose $f(x)\in\mathcal{P}_\mathbb{F}(x)$. If we let $x$ take values in $\mathbb{F}$, then $f(x)$ defines a function $f(x): \mathbb{F}\rightarrow\mathbb{F}$ (according to the arithmetic rules we have learnt before) which is called a polynomial function in $\mathbb{F}$. And we call the collection of all polynomial functions in $\mathbb{F}$ as $\mathcal{P}(\mathbb{F})$.
\end{Df}

\begin{Th}{$\bullet$ Th1.11.1 (remainder theorem)}
    Suppose $f(x)\in\mathcal{P}_\mathbb{F}(x)$ and $\alpha\in\mathbb{F}$. Then the remainder of $f(x)$ divided by polynomial $x-\alpha$ is the constant polynomial $f(\alpha)$.
    \tcblower
    \textit{Pf}: Obvious.
\end{Th}

\begin{Rmk}{}
    Although this theorem is trivial, it is very fundamental in the discussions of the polynomial functions. \textcolor{Df}{Now we call a point $\alpha$ subject to $f(\alpha)=0$ a root of this polynomial function $f$,} and we clearly have the following theorem.
\end{Rmk}

\begin{Th}{$\bullet$ Th1.11.2 (root and factor)}
    Suppose $f(x)\in\mathcal{P}_\mathbb{F}(x)$ and $\alpha\in\mathbb{F}$. Then $\alpha$ is a root of $f$ $\Leftrightarrow$ $(x-\alpha)\mid f(x)$.
\end{Th}

\begin{Rmk}{}
    For convenience, we call $\alpha$ the $k-$ multiplicative root of $f$ if $x-\alpha$ is a $k-$ multiplicative factor of $f(x)$. Accordingly, we still have the termilogy of ``simple root'' and ``multiple root''.
\end{Rmk}

\begin{Th}{$\bullet$ Th1.12 (the number of roots)}
    Suppose $f(x)\in\mathcal{P}_\mathbb{F}(x)$ and $f(x)\neq 0$. Then $f$ has at most $\deg f$ roots (repeatedly counted for a multiple root by its multiplicity) in $\mathbb{F}$.
    \tcblower
    \textit{Pf}: \begin{compactenum}
        \item If $\deg f(x) = 1 $, obvious.
        \item If $\deg f(x)>1$, then we consider the canonical factorization $f(x) = c\prod_{i=1}^{m}[p_i(x)]^{n_i}$. For each $p_i(x)$, if $\deg p_i(x) = 1$, then it contributes exactly one root to $f$; if $\deg p_i(x)>1$, then it has no root (otherwise it can be further factored). Hence, the number of roots of $f$ $=\sum_{i:\deg p_i(x) = 1} n_i \leq \sum_{i=1}^{m} n_i = \deg f(x)$.
    \end{compactenum} 
\end{Th}

\begin{Th}{$\bullet$ Th1.13 (polynomial $f(x)$ has a freedom of $\deg f(x)+1$)}
    Suppose $f(x), g(x)\in\mathcal{P}_\mathbb{F}(x)$ and $\deg f(x)=\deg g(x)= n\geq 0$. Then:\\
    $f(\alpha) = g(\alpha)$ for $n+1$ different $\alpha\in\mathbb{F}$ $\Rightarrow$ $f(x) = g(x)$
    \tcblower
    \textit{Pf}: If there are $n+1$ different $\alpha\in\mathbb{F}$ such that $f(\alpha) = g(\alpha)$, namely, $f(\alpha)-g(\alpha) = 0$, then $f-g$ has $n+1$ or more roots. If $f(x)-g(x)\neq 0$, then $\deg(f(x)-g(x))\leq n$, which indicates that the roots of $f-g$ are no more than $n$, contradicting the theorem \{course: 1, ID: 1.12\}. Hence $f(x)-g(x) = 0$, namely, $f(x) = g(x)$.
\end{Th}

\begin{Rmk}{}
    This theorem indicates that an $n$-degree polynomial function is determined by exactly $n+1$ different points. (As for an $-\infty$-degree polynomial, namely, $0$, it is determined by (let us say) $-infty+1=0$ point, which is also true.) \\ 
    We then have a natural but essential collary: \textcolor{Th}{For any number field $\mathbb{F}$, two polynomial functions are equal iff they correspond to the same polynomial,} i.e., the map $f(x)\mapsto f$ from $\mathcal{P}_\mathbb{F}(x)$ to $\mathcal{P}(\mathbb{F})$ is a bijection, and thus everything about the polynomials (including the exact division, gcd, irreducible factorization and other properties we have discussed) hold for the corresponding polynomial functions. To further illustrate this fact, we assume that $f(x) = g(x)$ for all $x\in\mathbb{F}$. Since there must be infinitely many $x$ in $\mathbb{F}$ (as $\mathbb{F}\supseteq\mathbb{Q}$), we can see that $f(x) = g(x)$ for infinitely many $x$, and thus the two polynomials are equal.
\end{Rmk}

\begin{Th}{$\bullet$ Th1.14 (the fundamental theorem of algebra)}
    Suppose $f\in\mathcal{P}(\mathbb{C})$ and $\deg f(x)\geq 1$. Then $f$ has at least one root in $\mathbb{C}$.
    \tcblower
    \textit{Pf}: Without causing cycle reasoning, we will prove this theorem in the future using complex analysis.
\end{Th}

\begin{Th}{$\bullet$ Th1.14.1 (roots of $f\in\mathcal{P}(\mathbb{C})$)}
    Suppose $f\in\mathcal{P}(\mathbb{C})$ and $\deg f(x)\geq 1$. Then $f$ has exactly $\deg f$ roots in $\mathbb{C}$ (repeatedly counted for a multiple root by its multiplicity).
    \tcblower
    \textit{Pf}: By the fundamental theorem of algebra, we can see that every $p_i(x)$ in the canonical factorization $c[p_1(x)]^{n_1}\dots[p_m(x)]^{n_m}$ of $f(x)$ is of 1-degree, (otherwise $p_i(x)$ is reducible since it has at least one root) and thus contributes exactly $1$ roots to $f(x)$. Hence, $f$ has exactly $\deg f$ roots in $\mathbb{C}$, with each distinct root repeated by its multiplicity. 
\end{Th}

\begin{Rmk}{}
    This theorem can be equivalently stated as the theorem \{course: 1, ID: 1.14.2\}.
\end{Rmk}

\begin{Th}{$\bullet$ Th1.14.2 (factorization of $f\in\mathcal{P}(\mathbb{C})$)}
    Suppose $f\in\mathcal{P}(\mathbb{C})$ and $\deg f(x)\geq 1$. Then every irreducible factor in the irreducible factorization of $f(x)$ is of 1-degree.
    \tcblower
    \textit{Pf}: Obvious.
\end{Th}

\begin{Th}{$\bullet$ Th1.14.3 (imaginary roots of a real-coefficient polynomial appear in pairs)}
    Suppose $f\in\mathcal{P}(\mathbb{C})$ where $\deg f(x)\geq 1$, and the coefficients of $f(x)$ are all in $\mathbb{R}$. \textcolor{Df}{(Here we call such $f(x)$ as a real-coefficient polynomial, and we use $\bar{\alpha}$ to denote the conjugate of $\alpha\in\mathbb{C}$.)} Then:\\
    If $\alpha\in\mathbb{C}$ is a root of $f$, then $\bar{\alpha}$ is also a root of $f$.
    \tcblower
    \textit{Pf}: Obvious.
\end{Th}

\begin{Th}{$\bullet$ Th1.14.4 (factorization of $f\in\mathcal{P}(\mathbb{R})$)}
    Suppose $f\in\mathcal{P}(\mathbb{R})$ and $\deg f(x)\geq 1$. Then every irreducible factor in the irreducible factorization of $f(x)$ is of 1-degree or 2-degree.
    \tcblower
    \textit{Pf}: Consider $f(x)$ as a real-coefficient polynomial in $\mathcal{P}(\mathbb{C})$. Then we can take the irreducible factorization for it in the complex way, say $f(x) = c(x-\alpha_1)\dots (x-\alpha_n)$. Now for those $x-\alpha_i$ where $\alpha_i\notin\mathbb{R}$, we paired it with $x-\bar{\alpha_i}$ and merge this couple of factors as a 2-degree one, $(x-\alpha_i)(x-\bar{\alpha_i})$. Hence we can see that this polynomial $(x-\alpha_i)(x-\bar{\alpha_i})$ is real-coefficient, and is irreducible in $\mathbb{R}$ (since its only irreducible factorization is imaginary-coefficient). Therefore, after merging for every pair of imaginary roots, we obtain the irreducible factorization of $f(x)$ in $\mathbb{R}$, which contains only 1-degree or 2-degree factors.
\end{Th}

\begin{Df}{$\bullet$ Df1.15 (primitive polynomial (本源多项式))}
    Suppose $g(x)=\sum_{i=0}^{n} a_ix^i\in\mathcal{P}_\mathbb{\mathbb{Q}}(x)$ where $g(x)$ is non-zero and integer-coefficients (namely, the coefficients of $g(x)$ are all integers). Then $g(x)$ is called a primitive polynomial if the coefficients $a_0, \dots, a_n$ have no common factor except for $\pm 1$.
\end{Df}

\begin{Rmk}{}
    This definition is actually prompted by the research on the factorization of \textcolor{Df}{the rational-coefficient polynomials (the polynomials with all the coefficients in $\mathbb{Q}$)}. We can see that \textcolor{Th}{a non-zero rational coefficients polynomial $f(x)$ can be uniquely (regardless of whether a plus sign or a minus sign ahead of $r$ and $g(x)$) factored as $f(x) = r\cdot g(x)$ where $r\in\mathbb{Q}$ and $g(x)$ is a primitive polynomial.} Hence, the discussion about the primitive polynomials is actually a discussion about the factorization of the rational-coefficient polynomials.
\end{Rmk}

\begin{Th}{$\circ$ Th1.15.1 (Gauss Lemma)}
    Suppose $f(x), g(x)\in\mathcal{P}_\mathbb{\mathbb{Q}}(x)$ are both primitive polynomials. Then $f(x)g(x)$ is also a primitive polynomial.
    \tcblower
    \textit{Pf}: Let $f(x) = \sum_{k=0}^{\infty} a_kx^k$ and $g(x) = \sum_{k=0}^{\infty} b_kx^k$. Then the $s$-th coefficient of $f(x)g(x)$ is $c_s = \sum_{i+j = s} a_ib_{j}$. Suppose $f(x)g(x)$ is not primitive, then there exists a prime number $p$ such that $p\mid c_s$ for all $s$. Since $f(x)$ and $g(x)$ are both primitive, $p$ can not exactly divides some $a_i$ and $b_j$. Let us say $a_i$ is the first one in $a_0, a_1, \dots$ that $p$ does not divide, and $b_j$ is the first one in $b_0, b_1, \dots$ that $p$ does not divide. Since $p\mid c_{i+j} = a_ib_j+\sum_{k}a_{i+k}b_{j-k}$, where $p$ divides $\sum_{k}a_{i+k}b_{j-k}$ (as $p$ divides either $a$ or $b$ of every term in this sum), we know that $p\mid a_ib_j$, which is a contradiction. Hence $f(x)g(x)$ is primitive.
\end{Th}

\begin{Th}{$\bullet$ Th1.15.2 (factor integer-coefficients polynomial into integer-coefficients polynomials)}
    Suppose $f(x)$ is a non-zero and integer-coefficients polynomial. If $f(x) = g(x)h(x)$ for some rational-coefficients $g(x)$ and $h(x)$ such that $\deg g(x), \deg h(x)<\deg f(x)$, then $f(x) = p(x)q(x)$ for some integer-coefficients $p(x)$ and $q(x)$ such that $\deg p(x), \deg q(x)<\deg f(x)$.
    \tcblower
    \textit{Pf}: Perform the primitive factorizations (remark \{, ID: 1.15\}) $f(x) = af_1(x)$, $g(x) = rg_1(x)$ and $h(x) = sh_1(x)$. Then $af_1(x) = rsg_1(x)h_1(x)$, where both sides represent a primitive factorization of $f(x)$ (since $g_1(x)h_1(x)$ is also primitive). Thus $a = \pm rs$, namely, $rs\in\mathbb{Z}$. And we can let $p(x) = rsg_1(x)$ and $q(x) = h_1(x)$. 
\end{Th}

\begin{Rmk}{}
    This theorem has a collary. \textcolor{Th}{Suppose $f(x)$, $g(x)$ are integer-coefficients, and $g(x)$ is primitive. If $f(x) = g(x)h(x)$ and $h(x)$ is rational-coefficients, then $h(x)$ is integer-coefficients.}
\end{Rmk}

\begin{Th}{$\bullet$ Th1.16 (compute the rational roots of a integer-coefficients polynomial)}
    Suppose $f(x)=a_nx^n + \dots + a_1x +a_0$ is a integer-coefficients polynomial. If $\frac{r}{s}$ is a rational root of $f$ where $r,s$ are coprime integers (two integers are called coprime if they share no common factor except $\pm 1$), then $r\mid a_0$ and $s\mid a_n$.
    \tcblower
    \textit{Pf}: If $\frac{r}{s}$ is a rational root of $f$ where $r$, $s$ are coprime, then $(x-\frac{r}{s}\mid f(x))$, namely, $(sx-r)\mid f(x)$. Since $(sx-r)$ is primitive and $f(x) = (sx-r)h(x)$ for some rational-coefficients $h(x)$, then $h(x)$ is integer-coefficients (according to the collary in the remark \{, ID: 1.15.2\}). By comparing the coefficients on the both sides of $f(x) = (sx-r)h(x)$, we get the result.  
\end{Th}

\begin{Rmk}{}
    This theorem offers a rough method to solve the rational roots of a integer-coefficients polynomial $f(x)$. Actually, we first list all possible candidates $r_1, \dots, r_N$ and $s_1, \dots, s_M$ for a rational root $\frac{r}{s}$, then we can verify the validity of each choice of $r$ and $s$.
\end{Rmk}

\begin{Th}{$\bullet$ Th1.17 (Eisenstein's discriminant method)}
    Suppose $f(x) = a_nx^n + a_{n-1}x^{n-1}\dots + a_0$ where $f(x)$ is a integer-coefficients polynomial and $\deg f(x)\geq 1$. If there is some prime number $p$ such that:
    \begin{compactitem}
        \item $p\nmid a_n$;
        \item $p\mid a_0, \dots, a_{n-1}$;
        \item $p^2\nmid a_0$,
    \end{compactitem}
    then $f(x)$ is irreducible in $\mathcal{P}_\mathbb{Q}(x)$.
    \tcblower
    \textit{Pf}:
    By contradiction, assume that $f(x)$ is reducible. Then factor $f(x)$ as $f(x) = g(x)h(x)$, where $g(x) = b_kx^k + \dots + b_1x + b_0$ and $h(x) = c_lx^l + \dots + c_1x + c_0$ are both integer-coefficients (according to theorem \{, ID: 1.15.2\}). By comparing the coefficients on the both sides of $f(x) = g(x)h(x)$, we have $a_0 = b_0c_0$ and $a_n = b_kc_l$. Since $p\mid a_0$ but $p^2\nmid a_0$, $p$ must divide either $b_0$ or $c_0$ (but not both), thus let us say $p\mid b_0$ but $p\nmid c_0$. Since $p\nmid a_n = b_kc_l$, $p\nmid b_k, c_l$. Then we let $b_s$ is the first one in $b_0, \dots, b_k$ that is not exactly divided by $p$, and we consider $a_s = b_sc_0 + \sum_{t} b_{s-t}c_{0+t}$. Since $p\mid \sum_{t} b_{s-t}c_{0+t}$ (as $p$ divides every $b_{s-t}$) and $p\mid a_s$, we have $p\mid b_sc_0$, and thus $p\mid b_s$ or $p\mid c_0$, which is a contradiction. 
\end{Th}

\begin{Rmk}{}
    According to this theorem, we would like to show an interesting fact about $\mathcal{P}_\mathbb{Q}(x)$. That is: \textcolor{Th}{there are irreducible polynomials of any degree in $\mathcal{P}_\mathbb{Q}(x)$. For example, we can easily verify that $x^n+2$ is irreducible for any $n\geq 1$.} This is a contrast to the situations in $\mathcal{P}_\mathbb{R}(x)$ where the irreducible polynomials are all of 1-degree, and in $\mathcal{P}_\mathbb{C}(x)$ where the irreducible polynomials are all of 1-degree or 2-degree.
\end{Rmk}

\begin{Df}{$\bullet$ Df1.18 (multi-variable polynomial)}
    Suppose $\mathbb{F}$ is a number field, $n\in\mathbb{N}^\ast$, and $\varphi$ is a function s.t. 
    \begin{compactenum}
        \item $\varphi: \mathbb{N}^n\rightarrow\mathbb{F}$;
        \item there are finitely many tuples $(\alpha_1, \cdots, \alpha_n)$ in $\mathbb{N}^n$ s.t. $\varphi(\alpha_1, \cdots, \alpha_n)\neq 0$. 
    \end{compactenum}
    Suppose also $x_1, \cdots, x_n$ are $n$ charactors, then the tuple $(\varphi, x_1, \dots, x_n)$ is called an $n$-variate polynomial in $\mathbb{F}^n$ about $x_1, \cdots, x_n$.
\end{Df}

\begin{Rmk}{}
    \textcolor{Th}{This is just an extension of Df \{, ID: 1.2\},} \textcolor{Df}{and we can define the addition and multiplication in the same way, so that we can write an $n$-variate polynomial $(\varphi, x_1, \cdots, x_n)$ in the so-called \textbf{``$n$-variate polynomial ring on $\mathbb{F}$''} $\mathcal{P}_\mathbb{F}(x_1,\cdots, x_n)$ as
    $$\sum_{(\alpha_1, \cdots, \alpha_n)\in\mathbb{N}^n}\varphi(\alpha_1, \cdots, \alpha_n)\, x_1^{\alpha_1}\cdots x_n^{\alpha_n}$$}
\end{Rmk}
\end{document}