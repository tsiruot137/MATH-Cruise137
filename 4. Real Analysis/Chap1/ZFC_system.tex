\documentclass{article}

    \usepackage{xcolor}
    \definecolor{pf}{rgb}{0.4,0.6,0.4}
    \usepackage[top=1in,bottom=1in, left=0.8in, right=0.8in]{geometry}
    \usepackage{setspace}
    \setstretch{1.2} 
    \setlength{\parindent}{0em}
    \usepackage{amssymb}
    \usepackage{amsmath}
    \usepackage[UTF8]{ctex}

\begin{document}

{\Large $\bullet $Ax0 (ZFC0)}\par
\begin{itemize}
    \item[]
    {\large \textcolor{blue}{$\exists x$}}\par
    \textcolor{pf}{This axiom guarantees that what we are discussing about is not nihility, 
    that is, there exists some set $x$ in the discourse universe.}
\end{itemize}\par
\quad

{\Large $\bullet $Ax0.1 (ZFC1) (the axiom of extensionality(外延))}\par
\begin{itemize}
    \item[]
    {\large \textcolor{blue}{$\forall a\forall b[\forall x(x\in a)\leftrightarrow (x\in b)]\leftrightarrow (a=b)$}}\par
    \textcolor{pf}{Two sets are equal if and only if (iff) they have the same members.}
\end{itemize}\par
\quad

{\Large $\bullet $Df0.1.1}\par
\begin{itemize}
    \item[]
    {\large Suppose $a$ and $b$ are two sets, we say:
    \begin{itemize}
        \item[] 
        $a$ is contained in $b$, denoted by $a\subseteq b$, if $\forall x(x\in a)\rightarrow (x\in b)$ \\
        $a$ contains $b$, denoted by $a\supseteq b$, if $\forall x(x\in b)\rightarrow (x\in a)$ \\
        $a$ is properly contained in $b$, denoted by $a\subsetneq b$, if $(a\subseteq b)\land (\exists y(y\in b)\land (y\notin a))$ \\
        $a$ properly contains $b$, denoted by $a\supsetneq b$, if $(a\supseteq b)\land (\exists y(y\in a)\land (y\notin b))$ 
    \end{itemize}}
\end{itemize}\par
\quad

{\Large $\bullet $Th0.1.2 (Two sets are equal iff they include each other.)}\par
\begin{itemize}
    \item[]
    {\large \textcolor{blue}{$\forall a\forall b(a=b)\leftrightarrow(a\subseteq b)\land(a\supseteq b)$}}\par
    {\textcolor{pf}{Obvious.}}
\end{itemize}\par
\quad

{\Large $\bullet $Ax0.2 (ZFC2) (the axiom schema of comprehension (内涵公理模式))}\par
\begin{itemize}
    \item[]
    {\large \textcolor{blue}{Suppose $p$ is an individual-predicate, then:\\
        $\forall s\exists y\forall x(x\in y)\leftrightarrow(x\in s)\land p(x)$}}\par
    \textcolor{pf}{For a given set $s$, $\left \{ x\in s:p(x) \right \} $ is a set.}
\end{itemize}\par
\quad

{\Large $\bullet $Th0.2.1 ($\left \{ x\in s:p(x) \right \} $ is unique)}\par
\begin{itemize}
    \item[]
    {\large \textcolor{blue}{$\forall s\exists!y\forall x(x\in y)\leftrightarrow(x\in s)\land p(x)$}}\par
    {\textcolor{pf}{Obvious.}}
\end{itemize}\par
\quad

{\Large $\bullet $Df2.2 ($\varnothing $)}\par
\begin{itemize}
    \item[]
    {\large Since there exists at least one set $s$, according to ZFC2, 
    $\left \{ x\in s: x\neq x \right \} $ is a set, denoted by $\varnothing $}\par
\end{itemize}\par
\quad

{\Large $\bullet $Th0.2.3 ($\varnothing $ is unique (corresponding to different $s$),and it has NO members)}\par
\begin{itemize}
    \item[]
    {\large \textcolor{blue}{$(\forall s (\{x\in s:x\neq x\}=\varnothing))\wedge(\forall x(x\notin\varnothing))$}}\par
    {\textcolor{pf}{Obvious.}}
\end{itemize}\par
\quad

{\Large Ax0.0 (ZFC0)}\par
\begin{itemize}
    \item[]
    {\large \textcolor{blue}{$\exists x$}}\par
    {\textcolor{pf}{}}
\end{itemize}\par
\quad

{\Large Ax0.0 (ZFC0)}\par
\begin{itemize}
    \item[]
    {\large \textcolor{blue}{$\exists x$}}\par
    {\textcolor{pf}{}}
\end{itemize}\par
\quad

{\Large Ax0.0 (ZFC0)}\par
\begin{itemize}
    \item[]
    {\large \textcolor{blue}{$\exists x$}}\par
    {\textcolor{pf}{}}
\end{itemize}\par
\quad

{\Large Ax0.0 (ZFC0)}\par
\begin{itemize}
    \item[]
    {\large \textcolor{blue}{$\exists x$}}\par
    {\textcolor{pf}{}}
\end{itemize}\par
\quad

{\Large Ax0.0 (ZFC0)}\par
\begin{itemize}
    \item[]
    {\large \textcolor{blue}{$\exists x$}}\par
    {\textcolor{pf}{}}
\end{itemize}\par
\quad

{\Large Ax0.0 (ZFC0)}\par
\begin{itemize}
    \item[]
    {\large \textcolor{blue}{$\exists x$}}\par
    {\textcolor{pf}{}}
\end{itemize}\par
\quad

{\Large Ax0.0 (ZFC0)}\par
\begin{itemize}
    \item[]
    {\large \textcolor{blue}{$\exists x$}}\par
    {\textcolor{pf}{}}
\end{itemize}\par
\quad

{\Large Ax0.0 (ZFC0)}\par
\begin{itemize}
    \item[]
    {\large \textcolor{blue}{$\exists x$}}\par
    {\textcolor{pf}{}}
\end{itemize}\par
\quad

{\Large Ax0.0 (ZFC0)}\par
\begin{itemize}
    \item[]
    {\large \textcolor{blue}{$\exists x$}}\par
    {\textcolor{pf}{}}
\end{itemize}\par
\quad

{\Large Ax0.0 (ZFC0)}\par
\begin{itemize}
    \item[]
    {\large \textcolor{blue}{$\exists x$}}\par
    {\textcolor{pf}{}}
\end{itemize}\par
\quad

{\Large Ax0.0 (ZFC0)}\par
\begin{itemize}
    \item[]
    {\large \textcolor{blue}{$\exists x$}}\par
    {\textcolor{pf}{}}
\end{itemize}\par
\quad

{\Large Ax0.0 (ZFC0)}\par
\begin{itemize}
    \item[]
    {\large \textcolor{blue}{$\exists x$}}\par
    {\textcolor{pf}{}}
\end{itemize}\par
\quad





{\Large $\bullet $ Df1 (limit of sequence)}\par
\begin{itemize}
    \item[]
    {\large Suppose $\{a_n\}$ is a sequence of real numbers and $a\in \mathbb{R} $. If
    $$\forall\varepsilon>0, \exists N \in \mathbb{N}^{\ast}, \forall n>N, \left|a_n-a\right|<\varepsilon, $$
    then $a$ is called the limit of sequence $\{a_n\}$ (or we say $\{a_n\}$ converges to $a$), 
    denoted by $\lim\limits_{n \to \infty} a_n=a $.}
\end{itemize}\par
\quad

{\Large $\circ $ Df1.1 (limit of sequence of complex numbers)}\par
\begin{itemize}
    \item[]
    {\large If $\{a_n\}$ and $a$ above are complex numbers, 
    just make an analogy on the definition of complex sequence 
    simply with $|\cdot |$ being the operation of absolute values of complex numbers.}
\end{itemize}\par
\quad

{\Large $\circ $ Df1.1.1 (neighborhood (邻域) of a real number)}\par
\begin{itemize}
    \item[]
    {\large Suppose $x\in \mathbb{R}$ and $r>0$, 
    then the interval $(x-r, x+r)$ is called a neighborhood of x, denoted by $U_r(x)$;
    and $U_r(x)\setminus \{x\}$ is called the corresponding hollow neighborhood, denoted by $\hat{U_r}(x)$.} 
\end{itemize}\par
\quad

{\Large $\circ $ Df1.1.2 (limit point (极限点) of a set of real numbers)}\par
\begin{itemize}
    \item[]
    {\large Suppose $A\subseteq \mathbb{R}$ and $a\in A$. Then we call $a$ a limit point of $A$ if
    $$\forall r>0, \exists x \text{ s.t. } x\in \hat{U_r}(x)\cap A$$}
    {\textcolor{orange}{This definition has an equivalent version: 
    Suppose $A\subseteq \mathbb{R}$ and $a\in A$, then we call $a$ a limit point of $A$ if
    we can select a sequence that has the limit a from the set $A\setminus\{a\}$}}
\end{itemize}\par
\quad

{\Large $\bullet $ Df1.2 (limit of function)}\par
\begin{itemize}
    \item[]
    {\large Suppose $f$ is a function with $\text{dom}(f), \text{range}(f)\subseteq \mathbb{R}$. 
    Suppose also $a, x_0\in\mathbb{R}$ and $x_0$ is a limit point of $\text{dom}(f)$. If
    $$\forall\varepsilon>0, \exists \delta >0, \forall x\in \text{dom}(f)\cap \hat{U_{\delta }}(x), |f(x)-a|<\varepsilon, $$
    then $f(x)$ is said to converge to $a$ at point $x_0$, denoted by $\lim\limits_{x \to x_0} f(x)=a $.}
\end{itemize}\par
\quad

{\Large $\bullet $ Df1.3 (metric (度量))}\par
\begin{itemize}
    \item[]
    {\large Suppose $X$ is a non-empty set and $\rho: X\times X\rightarrow [0, \infty)$. 
    Then $\rho$ is called a metric on $X$ if $\forall x, y, z\in X,$
    \begin{enumerate}
        \item[(a)] $\rho(x, y) = 0 $ iff $x = y;$
        \item[(b)] $\rho(x, y) = \rho(y, x);$
        \item[(c)] $\rho(x, z) \leqslant \rho(x, y) + \rho(y, z).$   
    \end{enumerate}}
    {\textcolor{pf}{A metric on a set is a way of calculating a certain type of distance of any two objects in $X$.
    And for such set $X$ equipped with metric $\rho$, we call it a metric space, denoted by metr($X, \rho$).}}
    {\textcolor{orange}{For example, $\mathbb{R}$ is a metric space with metric $|\cdot|$ (the absolute-value operation). 
    \textcolor{pf}{And we consider $\mathbb{R}$ as metr($\mathbb{R},|\cdot|$) by default later.} 
    Naturally, if metr($X, \rho$) and $A \subseteq X$, 
    then the restriction of $\rho$ on $A\times A$, or $\rho \upharpoonright (A\times A)$, is a metric on $A$.}}
\end{itemize}\par
\quad

{\Large $\circ $ Df1.3.1 (balls in metric space)}\par
\begin{itemize}
    \item[]
    {\large Suppose metr($X, \rho$) and $x \in X, r>0$. Then the set
    $$B_r(x)\triangleq \{y\in X: \rho (x, y)<r\}$$
    is called the (open) ball in $X$ with center $x$ and radius $r$.}
\end{itemize}\par
\quad

{\Large $\circ $ Df1.3.2 (product metric)}\par
\begin{itemize}
    \item[]
    {\large Suppose for $i=1,...,n$, metr($X_i, \rho_i$). Then the product metric of 
    metrics $\rho_1,...,\rho_n$ is defined to be a metric $\rho$ on $\prod_{i=1}^n X_i$.
    The form of $\rho$ is defined as:
    $$\rho[(x_1,...,x_n),(y_1,...,y_n)]=\max(\rho_1(x_1,y_1),...,\rho_n(x_n,y_n))
    (\text{for } (x_1,...,x_n),(y_1,...,y_n)\in \prod_{i=1}^n X_i)$$
    \textcolor{orange}{(It's obvious that such $\rho$ is indeed a metric on $\prod_{i=1}^n X_i$.)}}
\end{itemize}\par
\quad

{\Large $\bullet $ Df1.4 (open and closed sets in metric space)}\par
\begin{itemize}
    \item[]
    {\large Suppose metr($X, \rho$) and $E\subseteq X$. Then we say $E$ is open (in X) if
    $$\forall x\in E, \exists r>0 \text{ s.t. } B_r(x)\subseteq E. $$
    And we say $E$ is closed (in X) if its complement in X, or $X\setminus E$, is open.}  
    {\textcolor{orange}{Obviously every open ball in a metric space $X$ is open. And we can easily verify that:
    \begin{enumerate}
        \item[(a)] The union (resp. intersection) of any family of open (resp. closed) sets is open (resp. closed).
        \item[(b)] The intersection (resp. union) of any finite family of open (resp. closed) sets is open (resp. closed).
    \end{enumerate}}}
\end{itemize}\par
\quad

{\Large $\circ $ Df1.4.1 (interior (内部) and closure (闭包))}\par
\begin{itemize}
    \item[]
    {\large Suppose metr($X, \rho$) and $E\subseteq X$. 
    Then the union of all open sets contained in $E$ is the largest open set contained in $E$, 
    and we call this open set the interior of $E$, denoted by $E^o $.
    Likewise, the intersection of all closed sets containing $E$ is the smallest closed set containing $E$,
    and we call this closed set the closure of $E$, denoted by $\overline{E} $.}\par
    \textcolor{orange}{Clearly according to this definition, for a subset $E$ of $X$,  
    \begin{enumerate}
        \item[(a)] $E$ is open iff $E^o = E$;
        \item[(b)] $E$ is closed iff $\overline{E} = E$. 
    \end{enumerate}}
\end{itemize}\par
\quad

{\Large $\circ $ Df1.4.2 (dense (稠密) or nowhere dense (疏朗))}\par
\begin{itemize}
    \item[]
    {\large Suppose metr($X, \rho$) and $E\subseteq X$. 
    Then $E$ is said to be dense if $\overline{E}=X$. And $E$ is said to be nowhere dense if $(\overline{E})^o=\varnothing$}
\end{itemize}\par
\quad

{\Large $\circ $ Df1.4.3 (separable)}\par
\begin{itemize}
    \item[]
    {\large Suppose metr($X, \rho$). Then X is called separable if it has a countable dense subset.}
\end{itemize}\par
\quad

{\Large $\circ $ Df1.4.4 (convergent sequence in metric space)}\par
\begin{itemize}
    \item[]
    {\large Suppose metr($X, \rho$) and $x\in X$. Suppose also $\{x_n\}$ is a sequence in $X$. 
    Then $\{x_n\}$ is said to converge to $x$ if $\lim\limits_{n \to \infty} \rho(x_n,x)=0 $}\par
    {\textcolor{orange}{It is clear that when $X=\mathbb{R}$ and $\rho=|\cdot|$, this definition
    of sequence convergence accords to Df1.}}
\end{itemize}\par
\quad

{\Large $\circ $ Th1.4.4.1 (equivalent conditions of $x\in \overline{E}$)}\par
\begin{itemize}
    \item[]
    {\large Suppose metr($X, \rho$), $E\subseteq X$ and $x\in X$. Then the following are equivalent:
    \begin{enumerate}
        \item[(a)] $x\in \overline{E}$.
        \item[(b)] $\forall r>0, B_r(x)\cap E\neq \varnothing$.
        \item[(c)] There is a sequence $\{x_n\}$ in $E$ that converges to $x$.
    \end{enumerate}}
    {\textcolor{pf}{
        \begin{enumerate}
            \item[(a)$\rightarrow$(b)]: Assume by contradiction that there is some ball $B_{r_0}(x)$
            that has no element of $E$. Then we can verify that $\overline{E}\setminus B_{r_0}(x)$ is
            a closed set containing $E$, but it is strictly smaller than $\overline{E}$, contradicting
            the minimality of $\overline{E}$.  
            \item[(b)$\rightarrow$(c)]: Obvious.
            \item[(c)$\rightarrow$(a)]: It is equivalent to verifying that $x$ is in every closed set  
            containing $E$. Assume by contradiction that there exists a closed set $A$ containing $E$
            such that $x\notin A$, then $x\in X\setminus A$. Since $X\setminus A$ is open, there is a
            ball $B_{r_0}(x) \subseteq X\setminus A$. Hence,
            $$\exists y \text{ in the sequence } \{x_n\} \text{ s.t. } y\in B_{r_0}(x) \subseteq X\setminus A
            \Rightarrow y\notin A \Rightarrow y\notin E,$$ which contradicts with $y\in \{x_n\}\subseteq E$.
        \end{enumerate}
    }}
\end{itemize}\par
\quad

{\Large $\bullet $ Df1.4.5 (continuity of function in metric space)}\par
\begin{itemize}
    \item[]
    {\large Suppose metr($X, \rho_X$) and metr($Y, \rho_Y$). Suppose also $f:X\rightarrow Y$ and $x_0\in X$.
    Then $f$ is said to be continuous at $x_0$ if 
    $$\forall \varepsilon>0, \exists \delta>0, \forall x\in B_{\delta}(x_0), f(x_0)\in B_{\varepsilon}(f(x_0))$$}
    {\textcolor{pf}{In other word, $f$ is said to be continuous at $x_0$ if
    $$\forall \text{ open ball } B_Y \text{ with center } f(x_0),\exists \text{ open ball } B_X 
    \text{ with center } x_0 \text{ such that } f(B_X)\subseteq B_Y.$$}}
    {\large And suppose again metr($X, \rho_X$), metr($Y, \rho_Y$) and $f:X\rightarrow Y$. 
    Then $f$ is said to be continuous if it is continuous at every $x_0$ in $X$.}
\end{itemize}\par
\quad

{\Large $\bullet $ Th1.4.5.1 (equivalent condition of continuity)}\par
\begin{itemize}
    \item[]
    {\large Suppose metr($X, \rho_X$), metr($Y, \rho_Y$) and $f:X\rightarrow Y$. Then\\
        $f$ is continuous iff for any open set $U$ contained in Y,
        $f^{-1}(U)$ is an open set contained in $X$.}
    {\textcolor{pf}{Obvious.}}
\end{itemize}\par
\quad

{\Large $\bullet $ Df1.5 (Cauchy (柯西) sequences)}\par
\begin{itemize}
    \item[]
    {\large Suppose metr($X, \rho$) and $\{x_n\}$ is a sequence in $X$. 
    Then $\{x_n\}$ is called a Cauchy sequence if
    $$\forall \varepsilon>0, \exists N\in \mathbb{N}^{\ast}, \forall m,n>N, \rho(m, n)<\varepsilon.$$}
\end{itemize}\par
\quad

{\Large $\circ $ Df1.5.1 (complete)}\par
\begin{itemize}
    \item[]
    {\large Suppose metr($X, \rho$) and $E\subseteq X$. 
    Then $E$ is called complete if every Cauchy sequence in $E$ 
    converges and its limit is in $E$.}
\end{itemize}\par
\quad

{\Large $\circ $ Th1.5.2 (completeness and closedness)}\par
\begin{itemize}
    \item[]
    {\large A closed subset of a complete metric space is complete, 
    and a complete subset of an arbitrary metric space is closed.}\par
    {\textcolor{pf}{Obvious.}}
\end{itemize}\par
\quad

{\Large $\bullet $ Df1.6 (the distance between two subsets)}\par
\begin{itemize}
    \item[]
    {\large Suppose metr($X, \rho$) and $E, F\subseteq X$. 
    Then the distance between $E$ and $F$ is defined as:
    $$\rho(E,F) \triangleq \text{inf} \{\rho(x,y):x\in E, y\in F\}.$$}
    {\textcolor{orange}{It is clear that this definition is compactble 
    with the previous one of $\rho(x,y)$ since when $E$ has a single 
    point $x$ and $F$ has a single point $y$, $\rho(E,F)$ indeed equals $\rho(x,y)$.}}
\end{itemize}\par
\quad

{\Large $\circ $ Df1.6.1 (diameter of a metric subset)}\par
\begin{itemize}
    \item[]
    {\large Suppose metr($X, \rho$) and $E\subseteq X$. Then the diameter of $E$ is defined as
    $$\text{diam}(E)\triangleq \text{sup}\{\rho(x,y):x,y\in E\}.$$
    And $E$ is said to be bounded if $\text{diam}(E)<\infty$.}
\end{itemize}\par
\quad

{\Large $\bullet $ Df1.7 (cover (覆盖))}\par
\begin{itemize}
    \item[]
    {\large Suppose metr($X, \rho$) and $E\subseteq X$. 
    Suppose also $\{V_\alpha\}_{\alpha \in A}$ is a family of subsets of $X$ 
    (Here $A$ is the index set). If $E\subseteq \bigcup_{\alpha \in A}V_\alpha$, then 
    $\{V_\alpha\}_{\alpha \in A}$ is called a cover of $E$, 
    and $E$ is said to be covered by $\{V_\alpha\}_{\alpha \in A}$.}
\end{itemize}\par
\quad

{\Large $\circ $ Df1.7.1 (totally covered)}\par
\begin{itemize}
    \item[]
    {\large Suppose metr($X, \rho$) and $E\subseteq X$. Then $E$ is called totally bounded if
    $$\forall\varepsilon>0, E \text{ can be covered by finitely many open balls of radius }\varepsilon. $$}
    {\textcolor{orange}{Obviously, a totally bounded set is bounded, while the converse is false.}}
\end{itemize}\par
\quad

{\Large $\bullet $ Th1.8 (compact sets (紧致集))}\par
\begin{itemize}
    \item[]
    {\large Suppose metr($X, \rho$) and $E\subseteq X$. Then the following are equivalent:
    \begin{enumerate}
        \item[(a)] $E$ is complete and totally bounded.
        \item[(b)] (\textbf{The Bolzano-Weierstrass Property}) Every sequence in $E$ has a subsequence 
        that converges to a point of $E$.
        \item[(c)] (\textbf{The Heine-Borel Property}) If $\{V_\alpha\}_{\alpha \in A}$ is a family of open subsets
        of $X$, and $E$ is covered by $\{V_\alpha\}_{\alpha \in A}$, 
        then there exists a finite set $F\subseteq A$ such
        that $\{V_\alpha\}_{\alpha \in F}$ covers $E$. 
    \end{enumerate}}
    {\textcolor{pf}{
        \begin{enumerate}
            \item[(a)$\leftrightarrow $(b):] \textcolor{magenta}{todo}
            \item[(a)$\land $(b)$\rightarrow $(c):] \textcolor{magenta}{todo}
            \item[(c)$\rightarrow $(b):] \textcolor{magenta}{todo} 
        \end{enumerate}
        A subset $E$ of $X$ that possesses the properties (a)-(c) is said to be compact. And further more, 
        \textcolor{orange}{every compact set is closed (by Th1.5.2) and bounded; 
        the converse is false in general, but true in every $\mathbb{R}^n$.}}}
\end{itemize}\par
\quad

{\Large $\circ $ Df1.8.1 (equivalent metrics)}\par
\begin{itemize}
    \item[]
    {\large Suppose $\rho_1$ and $\rho_2$ are both metrics on $X$. 
    Then $\rho_1$ and $\rho_2$ are called equivalent if
    $$\exists C,C'>0, \forall x,y\in X, C\rho_1\leqslant \rho_2\leqslant C'\rho_1.$$}
    {\textcolor{pf}{}}
\end{itemize}\par
\quad

{\Large $\bullet $ Th1.8.2 (equivalent metrics delimit the same open subsets)}\par
\begin{itemize}
    \item[]
    {\large Suppose $\rho_1$ and $\rho_2$ are equivalent metrics on $X$. Then every
    open subset of $X$ under $\rho_1$ (resp. under $\rho_2$) 
    is also open under $\rho_2$ (resp. under $\rho_1$)}\par
    {\textcolor{pf}{Obvious following the definitions related.}}
    {\textcolor{orange}{And likewise, equivalent metrics also delimit the same closed subsets.}}
\end{itemize}\par
\quad

{\Large $\bullet $ Df1.9 (algebra of sets)}\par
\begin{itemize}
    \item[]
    {\large Suppose $X$ is a nonempty set and $\mathcal{A} \subseteq \mathcal{P}(X)$. 
    Then $\mathcal{A} $ is called an algebra of sets on $X$ if:
    \begin{enumerate}
        \item[(a)] $\forall n\in \mathbb{N}^{\ast}, \forall E_1,...,E_n \in \mathcal{A},\bigcup_{i=1}^{n}E_i \in \mathcal{A} $;
        \item[(b)] $\forall E\in \mathcal{A} , X\setminus E \in \mathcal{A} $.  
    \end{enumerate}
    }
    {\textcolor{pf}{In short, an algebra of sets on $X$ is a nonempty collection $\mathcal{A}$ of subsets of $X$ that
    is closed under the operation of finite union and complement.}}
\end{itemize}\par
\quad

{\Large $\circ $ Df1.9.1 ($\sigma $-algebra of sets)}\par
\begin{itemize}
    \item[]
    {\large Suppose $X$ is a nonempty set and $\mathcal{A} $ is an algebra of sets on $X$. 
    Then $\mathcal{A}$ is called a $\sigma $-algebra of sets on $X$ if:
    $$\forall \text{ sequence } \{E_i: i=1,2,\dots\} \text{ in } \mathcal{A}(\text{ that is, every } E_i \text{ is in } \mathcal{A}), 
    \bigcup_{i=1}^{\infty}E_i \in \mathcal{A}.$$}
    {\textcolor{pf}{In short, a $\sigma $-algebra of sets is an algebra 
    that is closed additionally under the operation of countable union. }}
    {\textcolor{orange}{Clearly an algebra of sets (resp. a $\sigma$-algebra of sets) 
    is also closed under the operation of finite intersection (resp. countable intersection). Moreover, if $\mathcal{A} $ is an
    algebra of sets on $X$, then $\varnothing$ and $X$ must be both in $\mathcal{A} $. Furthermore, if $X$ is a nonempty set, then:
    $\{\varnothing, X\}$ is the smallest $\sigma$-algebra of sets on $X$, and $\mathcal{P}(X) $ is the largest one.}}\par
    {\textcolor{orange}{It is worth noting that an algebra $\mathcal{A} $ is a $\sigma$-algebra provided that 
    $\mathcal{A} $ is closed under the operation of countable disjoint union.}}
\end{itemize}\par
\quad

{\Large $\circ $ Th1.9.1.1 ($\sigma $-algebra of \textbf{countable or co-countable} sets)}\par
\begin{itemize}
    \item[]
    {\large Suppose $X$ is an uncountable set. Then the collection
    $$\mathcal{A} \triangleq \{E\subseteq X: E \text{ is countable or } X\setminus E \text{ is countable }\}$$
    is a $\sigma $-algebra on $X$, called the $\sigma $-algebra of countable or co-countable sets.}\par
    {\textcolor{pf}{Obvious.}}
\end{itemize}\par
\quad

{\Large $\circ $ Th1.9.2 (the $\sigma $-algebra generated by a collection)}\par
\begin{itemize}
    \item[]
    {\large Suppose $X$ is a nonempty set. Then the intersection of any collection of $\sigma $-algebras on $X$
    is also a $\sigma $-algebra.}\par
    {\textcolor{pf}{Obvious.}}\par
    {\textcolor{pf}{Suppose $X$ is a nonempty set and $\mathcal{E}\subseteq \mathcal{P}(X)$. Then the intersection
    of all $\sigma $-algebras containing $\mathcal{E}$, called the $\sigma $-algebra generated by $\mathcal{E}$, 
    \textcolor{orange}{is the unique smallest $\sigma $-algebra that contains $\mathcal{E} $.} 
    We denote this intersection by $\mathcal{M}(\mathcal{E} ) $. }}
\end{itemize}\par
\quad

{\Large $\bullet $ Th1.10 (a useful lemma)}\par
\begin{itemize}
    \item[]
    {\large Suppose $X$ is a nonempty set and $\mathcal{E} ,\mathcal{F} \subseteq \mathcal{P}(X)$. Then:\\
    if $\mathcal{E} \subseteq \mathcal{M}(\mathcal{F}) $, 
    then $\mathcal{M} (\mathcal{E}) \subseteq \mathcal{M}(\mathcal{F}) $.}\par
    {\textcolor{pf}{Obvious.}}
\end{itemize}\par
\quad

{\Large $\bullet $ Th1.10.1 (the expression of open sets in term of open balls)}\par
\begin{itemize}
    \item[]
    {\large Suppose metr$(X,\rho)$ and $X$ has a dense subset $C$. Let
    $$\mathcal{E}=\{B_r(x):r\in\mathbb{Q}, x\in C\},$$
    then any open subset of $X$ is the union of some members of $\mathcal{E}$.}\par
    {\textcolor{pf}{For any open subset $F$ of $X$, let 
    $$\mathcal{F}=\{B_r(x)\subseteq F:r\in\mathbb{Q}, x\in F\cap C\}.$$ Then $\mathcal{F}\subseteq\mathcal{E}$.
    Now we claim that $F=\bigcup\mathcal{F}$ by mutual inclusion.\\
    Firstly, it is clear that $F\supseteq\bigcup\mathcal{F}$ 
    since every member of $\mathcal{F}$ is contained in $\mathcal{F}$.\\
    Next, for any $x\in F$, there is a sequence $\{c_n\}$ that converges to $x$ since $C$ is dense.
    Also, there is an open ball $B_r(x)\subseteq F$ (let us say $r$ is rational). Therefore, for a $c_i$ in 
    both that sequence and $B_r(x)$, $x$ must be in the ball $B_r(c_i)$, which is a member of $\mathcal{F}$.
    Thus, $F\subseteq\bigcup\mathcal{F}$.}}
\end{itemize}\par
\quad

{\Large $\bullet $ Df1.11 (Borel $\sigma$-algebra)}\par
\begin{itemize}
    \item[]
    {\large Suppose $X$ is a metric space. 
    Then the $\sigma$-algebra generated by the collection of all open subsets of $X$ 
    \textcolor{orange}{(or, equivalently, by the collection of all closed subsets of $X$)} 
    is called the Borel $\sigma$-algebra on $X$ and is denoted by $\mathcal{B}_X$. 
    Its members are called Borel sets.}\par
    {\textcolor{orange}{It can be seen that under different but equivalent two metrics 
    $\rho_1$ and $\rho_2$ on a same metric space $X$, the $\mathcal{B}_X$ is the same. (This
    is obvious using Th1.8.2.)}}
\end{itemize}\par
\quad

{\Large $\circ $ Th1.11.1 ($\mathcal{B}_{\mathbb{R}}$)}\par
\begin{itemize}
    \item[]
    {\large The Borel $\sigma$-algebra of $\mathbb{R}$ can be generated by each of the following
    (i.e. $\mathcal{B}_{\mathbb{R}} = \mathcal{M}(\mathcal{E}_j)$ for all $j\in \{1,2,3,...,8\}$):
    \begin{enumerate}
        \item[(a)] the open intervals: $\mathcal{E}_1 = \{(a,b):a<b\}$,
        \item[(b)] the closed intervals: $\mathcal{E}_2 = \{[a,b]:a<b\}$,
        \item[(c)] the half-open intervals: $\mathcal{E}_3 = \{[a,b):a<b\}$ or $\mathcal{E}_4 = \{(a,b]:a<b\}$,
        \item[(d)] the open rays: $\mathcal{E}_5 = \{(a,\infty):a\in \mathbb{R}\}$ or 
        $\mathcal{E}_6 = \{(-\infty,b):b\in \mathbb{R}\}$,
        \item[(e)] the closed rays: $\mathcal{E}_7 = \{[a,\infty):a\in \mathbb{R}\}$ or 
        $\mathcal{E}_8 = \{(-\infty,b]:b\in \mathbb{R}\}$. 
    \end{enumerate}}
    {\textcolor{pf}{Obvious using Th1.10 and Th1.10.1.}}
\end{itemize}\par
\quad

{\Large $\bullet $ Df1.12 (the product of $\sigma$-algebras)}\par
\begin{itemize}
    \item[]
    {\large Suppose $\{X_\alpha\}_{\alpha\in A}$ is an indexed collection of nonempty sets 
    and $X=\prod_{\alpha\in A}X_\alpha$. Suppose also for each $\alpha$ in $A$, 
    $\pi_\alpha:X\rightarrow X_\alpha$ is the coordinate map and $\mathcal{M}_\alpha$ is a $\sigma$-algebra
    on $X_\alpha$. Then the product of $\mathcal{M}_\alpha$'s is defined to be the 
    $\sigma$-algebra $\mathcal{M}$ generated by the set
    $$\{\pi_{\alpha}^{-1}(E_\alpha):E_\alpha \in \mathcal{M}_\alpha, \alpha\in A\}$$
    (here this expression of this set actually means 
    $\{\mathcal{E}:\exists\alpha\in A,\exists E_\alpha\in\mathcal{M}_\alpha,\mathcal{E}=\pi_{\alpha}^{-1}(E_\alpha)\}$).
    We denote the product $\mathcal{M}$ of $\mathcal{M}_\alpha$'s 
    by $\mathcal{M}=\bigotimes_{\alpha\in A}\mathcal{M}_\alpha$}
\end{itemize}\par
\quad

{\Large $\bullet $ Th1.12.1 (the product of at-most-countably-many $\sigma$-algebras)}\par
\begin{itemize}
    \item[]
    {\large Suppose $\{X_\alpha\}_{\alpha\in A}$ is an indexed collection of nonempty sets 
    and $X=\prod_{\alpha\in A}X_\alpha$. Suppose also $A$ is an at-most-countable set and for each $\alpha$ in $A$, 
    $\mathcal{M}_\alpha$ is a $\sigma$-algebra on $X_\alpha$. 
    Then $\bigotimes_{\alpha\in A}\mathcal{M}_\alpha$ is the $\sigma$-algebra 
    generated by the set 
    $\{\prod_{\alpha\in A}E_\alpha: \text{ for each } \alpha\in A, E_\alpha\in\mathcal{M}_\alpha\}$}\par
    {\textcolor{pf}{We will complete the proof using Th1.10 to show by mutual inclusion.\\
    On one hand, for any set $E$ in $\{\pi_{\alpha}^{-1}(E_\alpha):E_\alpha \in \mathcal{M}_\alpha, \alpha\in A\}$,
    there exists $\alpha\in A$ s.t. $E=\pi_{\alpha}^{-1}(E_\alpha)$. 
    Clearly $\pi_{\alpha}^{-1}(E_\alpha) = \prod_{\beta\in A}E_\beta$ 
    where $E_\beta = X_\beta$ for all $\beta\neq \alpha$. On the other hand, 
    $\prod_{\alpha\in A}E_\alpha = \bigcap_{\alpha\in A}\pi_{\alpha}^{-1}(E_\alpha)$.}}
\end{itemize}\par
\quad

{\Large $\bullet $ Th1.12.2}\par
\begin{itemize}
    \item[]
    {\large Suppose $\{X_\alpha\}_{\alpha\in A}$ is an indexed collection of nonempty sets,
    $X=\prod_{\alpha\in A}X_\alpha$ and for each $\alpha$ in $A$, 
    $\mathcal{E}_\alpha \subseteq\mathcal{P}(X_\alpha)$. Then:
    $$\bigotimes_{\alpha\in A}\mathcal{M}(\mathcal{E}_\alpha) = 
    \mathcal{M}(\{\pi_{\alpha}^{-1}(E_\alpha):E_\alpha \in \mathcal{E}_\alpha, \alpha\in A\}).$$
    Furthermore, if $A$ is countable and $X_\alpha\in \mathcal{E}_\alpha$ for each $\alpha$, then:
    $$\bigotimes_{\alpha\in A}\mathcal{M}(\mathcal{E}_\alpha) = \mathcal{M}(
    \{\prod_{\alpha\in A}E_\alpha: \text{ for each } \alpha\in A, E_\alpha\in\mathcal{E}_\alpha\})$$}
    {\textcolor{pf}{Let $\mathcal{F} = 
    \mathcal{M}(\{\pi_{\alpha}^{-1}(E_\alpha):E_\alpha \in \mathcal{E}_\alpha, \alpha\in A\})$.
    Obviously $\mathcal{M}(\mathcal{F}_1)\subseteq\bigotimes_{\alpha\in A}\mathcal{M}(\mathcal{E}_\alpha)$.
    On the other hand, for each $\alpha$, 
    $\{E\subseteq X_\alpha:\pi_\alpha^{-1}(E)\in \mathcal{M}(\mathcal{F}_1)\}$ is seen to be a $\sigma$-algebra
    on $X_\alpha$ that contains $\mathcal{E}_\alpha$ and hence $\mathcal{M}(\mathcal{E}_\alpha)$. In other word,
    $\pi_\alpha^{-1}(E)\in \mathcal{M}(\mathcal{F}_1)$ for all $E\in\mathcal{M}(\mathcal{E}_\alpha)$, $\alpha\in A$
    and hence $\mathcal{M}(\mathcal{F}_1)\supseteq\bigotimes_{\alpha\in A}\mathcal{M}(\mathcal{E}_\alpha)$. The second
    assertion follows from the first as in the proof of Th1.12.1.}}
\end{itemize}\par
\quad

{\Large $\circ $ Th1.12.3 ($\bigotimes_{i=1}^n \mathcal{B}_{X_i}$ and $\mathcal{B}_{\prod_{i=1}^n X_i}$)}\par
\begin{itemize}
    \item[]
    {\large Suppose $X_1,...,X_n$ are metric spaces and $X=\prod_{i=1}^n X_i$, equipped with the product metric.
    Then $\bigotimes_{i=1}^n \mathcal{B}_{X_i} \subseteq \mathcal{B}_X$. Furthermore, if the $X_i$'s are all 
    separable, then $\bigotimes_{i=1}^n \mathcal{B}_{X_i} = \mathcal{B}_X$.}\par
    {\textcolor{pf}{By Th1.12.3, $\bigotimes_{i=1}^n \mathcal{B}_{X_i}$ is generated by 
    $\{\pi_i^{-1}(U_i):U_i \text{ is open subset of } X_i, 1\leqslant i\leqslant n\}$. 
    Since such $\pi_i^{-1}(U_i)$'s are all open subsets of $X$ (refer to the formula in the proof of Th1.12.1),
    Th1.10 implies that $\bigotimes_{i=1}^n \mathcal{B}_{X_i} \subseteq \mathcal{B}_X$. Suppose now that for
    each $i$, $C_i$ is a countable dense subset of $X_i$, and let $\mathcal{E}_i$ be the collection of open balls
    in $X_i$ with rational radius and centers in $C_i$. Then every open subset of $X_i$ can be written as a union
    of some members of $\mathcal{E}_i$ (according to Th1.10.1)---actually a countable union since $\mathcal{E}_i$
    itself is countable. Moreover, it is easy to verify that $\prod_{i=1}^n C_i$ is also a dense subset of $X$, 
    and the open balls of radius $r$ in $X$ are merely Cartisian products of balls of radius $r$ in the $X_i$'s.
    It follows that $\mathcal{B}_{X_i}$ is generated by $\mathcal{E}_i$ for all $i$, and $\mathcal{B}_X$ is 
    generated by $\{\prod_{i=1}^n E_i: \text{ for each } 1\leqslant i\leqslant n, E_i \in\mathcal{E}_i\}$.
    Therefore, $\bigotimes_{i=1}^n \mathcal{B}_{X_i} = \mathcal{B}_X$ by Th1.12.2.}}
\end{itemize}\par
\quad

{\Large $\circ $ Df1.13 (elementary family)}\par
\begin{itemize}
    \item[]
    {\large Suppose $X$ is a nonempty set and $\mathcal{E}\subseteq \mathcal{P}(X)$. Then $\mathcal{E}$ is called
    an elementary family on $X$ if:
    \begin{enumerate}
        \item[(a)] $\varnothing\in \mathcal{E}$;
        \item[(b)] $\forall E,F\in \mathcal{E}, E\cap F\in\mathcal{E}$;
        \item[(c)] $\forall E\in \mathcal{E}, E^c (\text{ namely, }X\setminus E) 
                    \text{ is a finite disjoint union of some members of }\mathcal{E}.$ 
    \end{enumerate}}
\end{itemize}\par
\quad

{\Large $\circ $ Th1.13.1}\par
\begin{itemize}
    \item[]
    {\large Suppose $X$ is a nonempty set and $\mathcal{E}$ is an elementary family on $X$. Then the collection
    $\mathcal{A}$ of finite disjoint unions of members of $\mathcal{E}$ 
    (namely, $\mathcal{A}=\{E\in \mathcal{P}(X):\exists n\in\mathbb{N}^\ast, \exists E_1,...,E_n\in \mathcal{E} 
    \text{ s.t. } E \text{ is the disjoint union of } E_1,...,E_n \}$) is an algebra of sets on $X$.}
    {\textcolor{pf}{We firstly verify that $\forall A,B\in \mathcal{E}, A\cup B\text{ and } A\setminus B 
    \text{ are both in } \mathcal{A}$. Then we claim from mathematic induction that 
    $$\forall n\in\mathbb{N}^\ast ,\forall A_1,...,A_n\in\mathcal{E}, \bigcup_{i=1}^n A_i \in \mathcal{A}$$
    so that we can easily verify that $\mathcal{A}$ is closed under finite unions. Next we show that $\mathcal{A}$
    is closed under complements. For any $A\in\mathcal{A}$, suppose $A=\bigcup_{m=1}^n A_m$ where $A_m\in\mathcal{E}$
    for all $1\leqslant m\leqslant n$. Then for each $m$, let $A_m^c = \bigcup_{j=1}^{J_m} B_m^j $ where 
    $B_m^1,...,B_m^{J_m}$ are disjoint members of $\mathcal{E}$. Hence, 
    $$A^c = \left(\bigcup_{m=1}^{n}A_{m}\right)^{c}=\bigcap_{m=1}^{n}\left(\bigcup_{j=1}^{J_{m}}B_{m}^{j}\right)
    =\bigcup\{B_{1}^{j_{1}}\cap\cdots\cap B_{n}^{j_{n}}:\forall m\in\{1,...,n\},j_{m}\in \{1,...,J_m\}\}\in \mathcal{A}.$$
    (The third equation comes from a result similar to the expansion of a product of sums of numbers:
    $$\prod_{m=1}^n(x_m^1+\cdots+x_m^{J_m})=\sum_{
        \begin{array}
            {c}j_1\in\{1,\cdots,J_1\}\\\vdots\\j_n\in\{1,\cdots,J_n\}
        \end{array}
        }(x_1^{j_1}\cdots x_n^{j_n}).$$
    And this expansion only relies on the commutative, associative and distributive laws of number-addition and 
    number-multiplication. Thus an analogy can be made by just considering the union and intersection 
    as addition and multiplication respectively.)}}
\end{itemize}\par
\quad

{\Large $\bullet $ Df1.14 (measure(测度))}\par
\begin{itemize}
    \item[]
    {\large Suppose $X$ is a nonempty set and $\mathcal{M}$ is a $sigma$-algebra of sets on $X$. Suppose also
    $\mu:\mathcal{M}\rightarrow [0,\infty]$. Then $\mu$ is called a measure on $(X, \mathcal{M})$ 
    (or a measure on $X$ (resp. $\mathcal{M}$) when $\mathcal{M}$ (resp. $X$) is clear) if:
    \begin{enumerate}
        \item[(i)] $\mu(\varnothing)=0$,
        \item[(ii)] For any sequence $\{E_i\}_{i=1}^\infty$ of disjoint sets in $\mathcal{M}$,
                    $\mu(\bigcup_{i=1}^\infty E_i) = \sum_{i=1}^\infty \mu(E_i)$.
    \end{enumerate}
    (Property (ii) is called countable additivity (可列可加性). Also, the one below is called finite additivity
    (有限可加性):
    \begin{enumerate}
        \item[(ii')] $\forall n\in \mathbb{N}^\ast, \forall E_1,...,E_n\in \mathcal{M}, \text{ if } E_1,...,E_n
        \text{ are disjoint, then } \mu(\bigcup_{i=1}^n E_i) = \sum_{i=1}^n \mu(E_i).$
    \end{enumerate}
    {Clearly the property (ii) is well defined since: the series on the right is a positive series (which is 
    either convergent or equal to $\infty$) and additionally its value is irrelevant to the order of $\mu(E_i)$'s
    no matter whether it converges or not.}\\
    \textcolor{orange}{And it is also clear that countable additivity implies finite additivity, while the converse fails.}
    If a function $\mu:\mathcal{M}\rightarrow [0,\infty]$ satisfies (i) and (ii') above, then $\mu$ is called a
    finite additive measure.)\\
    For a nonempty set $X$ and a $\sigma$-algebra $\mathcal{M}$ on it, the ordered pair 
    $(X, \mathcal{M})$ is called a measurable space (可测空间) and the set in $\mathcal{M}$ are called 
    measurable sets (可测集). And if $\mu$ is a measure on $(X, \mathcal{M})$, then we say the tuple 
    $(X, \mathcal{M}, \mu)$ is a measure space (测度空间). \textcolor{orange}{Obviously, there exists indeed a
    measure $\mu$ on a measurable space $(X, \mathcal{M})$ (as for the most trivial one, 
    $\mu(E)=0$ for all $E\in \mathcal{M}$)}.
    }
\end{itemize}\par
\quad

{\Large $\circ $ Th1.14.1 (basic properties of a measure)}\par
\begin{itemize}
    \item[]
    {\large Suppose $(X, \mathcal{M}, \mu)$ is a measure space, then we have the following properties.
    \begin{enumerate}
        \item[(a)] (monotonicity (单调性)): If $E, F\in\mathcal{M}$ and $E\subseteq F$, then $\mu(E)\leqslant\mu(F)$.
        \item[(b)] (subadditivity (弱可加性)): If $\{E_i\}_{i=1}^\infty$ is a sequence in $\mathcal{M}$, then
                    $\mu(\bigcup_{i=1}^\infty E_i)\leqslant \sum_{i=1}^{\infty} E_i$.
        \item[(c)] (continuity from below (自下的连续性)): If $\{E_i\}_{i=1}^\infty$ is a sequence in $\mathcal{M}$
                    and $E_1\subseteq E_2\subseteq...$, then $\mu(\bigcup_{i=1}^\infty E_i) = \lim_{i\to\infty} \mu(E_i)$.
        \item[(d)] (continuity from above (自上的连续性)): If $\{E_i\}_{i=1}^\infty$ is a sequence in $\mathcal{M}$,
                    $E_1\supseteq E_2\supseteq...$ and $\exists j\in \mathbb{N}^\ast \text{ s.t. }
                    \mu(E_j)<\infty$, then $\mu(\bigcap_{i=1}^\infty E_i) = \lim_{i\to\infty} \mu(E_i)$.  
    \end{enumerate}}
    {\textcolor{pf}{Obvious.}}
\end{itemize}\par
\quad

{\Large $\circ $ Df1.14.2 (finite, $\sigma$-finite and semifinite)}\par
\begin{itemize}
    \item[]
    {\large Suppose $(X, \mathcal{M}, \mu)$ is a measure space, then we have following terminology.\\
    If $\mu(X)<\infty$, then $\mu$ is called \textbf{finite}; If $\exists \text{ a sequence } \{E_i\}_{i=1}^\infty$
    in $\mathcal{M}$ s.t. $(X=\bigcup_{i=1}^\infty E_i)\land (\forall i\in\mathbb{N}^\ast,\mu(E_i)<\infty)$, then
    $\mu$ is called \textbf{$\sigma$-finite} (More generally, if $E$ is a set satisfying that 
    $\exists \text{ a sequence } \{E_i\}_{i=1}^\infty$ in $\mathcal{M}$ s.t. 
    $(E=\bigcup_{i=1}^\infty E_i)\land (\forall i\in\mathbb{N}^\ast,\mu(E_i)<\infty)$, then $E$ is said
    to be $\sigma$-finite for $\mu$); If for each $E\in\mathcal{M}$ with $\mu(E)=\infty$ 
    there exists $F\in\mathcal{M}$ s.t. $F\subseteq E$ and $0<\mu(F)<\infty$, 
    then $\mu$ is called \textbf{semifinite.}}\\
    {\textcolor{orange}{It is obvious that these "finite"'s have the relationships below.\\
    Suppose $(X, \mathcal{M}, \mu)$ is a measure space, then:
    $$\mu\text{ is finite }\rightharpoonup \mu\text{ is }\sigma\text{-finite }\rightharpoonup \mu\text{ is semifinite }$$
    where the notation "$\rightharpoonup$" means that the leftside 
    can imply the rightside but the right cannot imply the left.}}
\end{itemize}\par
\quad

{\Large $\bullet$ Df1.14.3 (null sets (零测度集))}\par
\begin{itemize}
    \item[]
    {\large Suppose $(X, \mathcal{M}, \mu)$ is a measure space and $E\in\mathcal{M}$. If $\mu(E)=0$, we call $E$ a null set.
    Furthermore, suppose $p(x)$ is a predicate about the points in $X$, then $p(x)$ is said to 
    \textbf{be true almost everywhere (abbreviated a.e.)} if $p(x)$ is false for only the points in some null set.}\\
    {\textcolor{orange}{Let $(X, \mathcal{M}, \mu)$ be a measure space, then any countable union of null sets is
    again a null set.}}
\end{itemize}\par
\quad

{\Large $\circ $ Df1.14.3.1 (complete measure)}\par
\begin{itemize}
    \item[]
    {\large Suppose $(X, \mathcal{M}, \mu)$ is a measure space. If for any null set $N$, every subset of $N$ is in $\mathcal{M}$,
    then $\mu$ is said to be complete.}\\
    {\textcolor{pf}{This completeness can sometimes obviate annoying technical points, and it can always be achieved by 
    enlarging the domain of $\mu$, as follows.}}
\end{itemize}\par
\quad

{\Large $\circ $ Th1.14.3.2 (the completion of a measure)}\par
\begin{itemize}
    \item[]
    {\large Suppose $(X, \mathcal{M}, \mu)$ is a measure space. Let $\mathcal{N}=\{N\in \mathcal{M}: \mu(N)=0\}$ and 
    $\overline{\mathcal{M}}=\{E\cup F: E\in\mathcal{M} \text{ and } \exists N\in \mathcal{N}, F\subseteq N\}$. Then
    $\mathcal{M}$ is a $\sigma$-algebra, and there is a unique extension $\overline{\mu}$ of $\mu$ to a measure on
    $\overline{\mathcal{M}}$. Additionally, this unique extended measure $\overline{\mu}$, considering measure space
    $(X, \overline{\mathcal{M}}, \overline{\mu})$, is complete.}\\
    {\textcolor{pf}{Just verify that $\mathcal{M}$ is a $\sigma$-algebra by the definition.\\
    Define $\overline{\mu}$ as $\overline{\mu}(E\cup F)=\mu(E)$. (This is well defined since we can show that if a set
    $A$ in $\overline{\mathcal{M}}$ can be written as $A=E_1\cup F_1=E_2\cup F_2$, then $\mu(E_1)=\mu(E_2)$.) Next it is
    easy to verify that $\overline{\mu}$ is the only measure on $\overline{\mathcal{M}}$ that extends $\mu$, and that
    $\overline{\mu}$ is complete w.r.t. measure space $(X, \overline{\mathcal{M}}, \overline{\mu})$.}}\\
    \textcolor{orange}{Obviously, for the operation above that extends $\mathcal{M}$ to $\overline{\mathcal{M}}$, we can find
    that $\overline{\overline{\mathcal{M}}}=\overline{\mathcal{M}}$ for any $\sigma$-algebra $\mathcal{M}$.}
    {For a measure space $(X, \mathcal{M}, \mu)$, the measure $\overline{\mu}$ in this theorem is called the completion
    of $\mu$, and $\overline{\mathcal{M}}$ is called the completion of $\mathcal{M}$ w.r.t. $\mu$.}
\end{itemize}\par
\quad

{\Large $\circ $ Df1.4.3 (separable)}\par
\begin{itemize}
    \item[]
    {\large }
    {\textcolor{pf}{}}
\end{itemize}\par
\quad

{\Large $\circ $ Df1.4.3 (separable)}\par
\begin{itemize}
    \item[]
    {\large }
    {\textcolor{pf}{}}
\end{itemize}\par
\quad

{\Large $\circ $ Df1.4.3 (separable)}\par
\begin{itemize}
    \item[]
    {\large }
    {\textcolor{pf}{}}
\end{itemize}\par
\quad

{\Large $\circ $ Df1.4.3 (separable)}\par
\begin{itemize}
    \item[]
    {\large }
    {\textcolor{pf}{}}
\end{itemize}\par
\quad

{\Large $\circ $ Df1.4.3 (separable)}\par
\begin{itemize}
    \item[]
    {\large }
    {\textcolor{pf}{}}
\end{itemize}\par
\quad

{\Large $\circ $ Df1.4.3 (separable)}\par
\begin{itemize}
    \item[]
    {\large }
    {\textcolor{pf}{}}
\end{itemize}\par
\quad

{\Large $\circ $ Df1.4.3 (separable)}\par
\begin{itemize}
    \item[]
    {\large }
    {\textcolor{pf}{}}
\end{itemize}\par
\quad

{\Large $\circ $ Df1.4.3 (separable)}\par
\begin{itemize}
    \item[]
    {\large }
    {\textcolor{pf}{}}
\end{itemize}\par
\quad

{\Large $\circ $ Df1.4.3 (separable)}\par
\begin{itemize}
    \item[]
    {\large }
    {\textcolor{pf}{}}
\end{itemize}\par
\quad

{\Large $\circ $ Df1.4.3 (separable)}\par
\begin{itemize}
    \item[]
    {\large }
    {\textcolor{pf}{}}
\end{itemize}\par
\quad

{\Large $\circ $ Df1.4.3 (separable)}\par
\begin{itemize}
    \item[]
    {\large }
    {\textcolor{pf}{}}
\end{itemize}\par
\quad

{\Large $\circ $ Df1.4.3 (separable)}\par
\begin{itemize}
    \item[]
    {\large }
    {\textcolor{pf}{}}
\end{itemize}\par
\quad

{\Large $\circ $ Df1.4.3 (separable)}\par
\begin{itemize}
    \item[]
    {\large }
    {\textcolor{pf}{}}
\end{itemize}\par
\quad

{\Large $\circ $ Df1.4.3 (separable)}\par
\begin{itemize}
    \item[]
    {\large }
    {\textcolor{pf}{}}
\end{itemize}\par
\quad

{\Large $\circ $ Df1.4.3 (separable)}\par
\begin{itemize}
    \item[]
    {\large }
    {\textcolor{pf}{}}
\end{itemize}\par
\quad

{\Large $\circ $ Df1.4.3 (separable)}\par
\begin{itemize}
    \item[]
    {\large }
    {\textcolor{pf}{}}
\end{itemize}\par
\quad

{\Large $\circ $ Df1.4.3 (separable)}\par
\begin{itemize}
    \item[]
    {\large }
    {\textcolor{pf}{}}
\end{itemize}\par
\quad

{\Large $\circ $ Df1.4.3 (separable)}\par
\begin{itemize}
    \item[]
    {\large }
    {\textcolor{pf}{}}
\end{itemize}\par
\quad

{\Large $\circ $ Df1.4.3 (separable)}\par
\begin{itemize}
    \item[]
    {\large }
    {\textcolor{pf}{}}
\end{itemize}\par
\quad

{\Large $\circ $ Df1.4.3 (separable)}\par
\begin{itemize}
    \item[]
    {\large }
    {\textcolor{pf}{}}
\end{itemize}\par
\quad

{\Large $\circ $ Df1.4.3 (separable)}\par
\begin{itemize}
    \item[]
    {\large }
    {\textcolor{pf}{}}
\end{itemize}\par
\quad

{\Large $\circ $ Df1.4.3 (separable)}\par
\begin{itemize}
    \item[]
    {\large }
    {\textcolor{pf}{}}
\end{itemize}\par
\quad

{\Large $\circ $ Df1.4.3 (separable)}\par
\begin{itemize}
    \item[]
    {\large }
    {\textcolor{pf}{}}
\end{itemize}\par
\quad

{\Large $\circ $ Df1.4.3 (separable)}\par
\begin{itemize}
    \item[]
    {\large }
    {\textcolor{pf}{}}
\end{itemize}\par
\quad

{\Large $\circ $ Df1.4.3 (separable)}\par
\begin{itemize}
    \item[]
    {\large }
    {\textcolor{pf}{}}
\end{itemize}\par
\quad

{\Large $\circ $ Df1.4.3 (separable)}\par
\begin{itemize}
    \item[]
    {\large }
    {\textcolor{pf}{}}
\end{itemize}\par
\quad

{\Large $\circ $ Df1.4.3 (separable)}\par
\begin{itemize}
    \item[]
    {\large }
    {\textcolor{pf}{}}
\end{itemize}\par
\quad

{\Large $\circ $ Df1.4.3 (separable)}\par
\begin{itemize}
    \item[]
    {\large }
    {\textcolor{pf}{}}
\end{itemize}\par
\quad

{\Large $\circ $ Df1.4.3 (separable)}\par
\begin{itemize}
    \item[]
    {\large }
    {\textcolor{pf}{}}
\end{itemize}\par
\quad

{\Large $\circ $ Df1.4.3 (separable)}\par
\begin{itemize}
    \item[]
    {\large }
    {\textcolor{pf}{}}
\end{itemize}\par
\quad

{\Large $\circ $ Df1.4.3 (separable)}\par
\begin{itemize}
    \item[]
    {\large }
    {\textcolor{pf}{}}
\end{itemize}\par
\quad

{\Large $\circ $ Df1.4.3 (separable)}\par
\begin{itemize}
    \item[]
    {\large }
    {\textcolor{pf}{}}
\end{itemize}\par
\quad

{\Large $\circ $ Df1.4.3 (separable)}\par
\begin{itemize}
    \item[]
    {\large }
    {\textcolor{pf}{}}
\end{itemize}\par
\quad

{\Large $\circ $ Df1.4.3 (separable)}\par
\begin{itemize}
    \item[]
    {\large }
    {\textcolor{pf}{}}
\end{itemize}\par
\quad

{\Large $\circ $ Df1.4.3 (separable)}\par
\begin{itemize}
    \item[]
    {\large }
    {\textcolor{pf}{}}
\end{itemize}\par
\quad

{\Large $\circ $ Df1.4.3 (separable)}\par
\begin{itemize}
    \item[]
    {\large }
    {\textcolor{pf}{}}
\end{itemize}\par
\quad

{\Large $\circ $ Df1.4.3 (separable)}\par
\begin{itemize}
    \item[]
    {\large }
    {\textcolor{pf}{}}
\end{itemize}\par
\quad

{\Large $\circ $ Df1.4.3 (separable)}\par
\begin{itemize}
    \item[]
    {\large }
    {\textcolor{pf}{}}
\end{itemize}\par
\quad

{\Large $\circ $ Df1.4.3 (separable)}\par
\begin{itemize}
    \item[]
    {\large }
    {\textcolor{pf}{}}
\end{itemize}\par
\quad

{\Large $\circ $ Df1.4.3 (separable)}\par
\begin{itemize}
    \item[]
    {\large }
    {\textcolor{pf}{}}
\end{itemize}\par
\quad

{\Large $\circ $ Df1.4.3 (separable)}\par
\begin{itemize}
    \item[]
    {\large }
    {\textcolor{pf}{}}
\end{itemize}\par
\quad

{\Large $\circ $ Df1.4.3 (separable)}\par
\begin{itemize}
    \item[]
    {\large }
    {\textcolor{pf}{}}
\end{itemize}\par
\quad

{\Large $\circ $ Df1.4.3 (separable)}\par
\begin{itemize}
    \item[]
    {\large }
    {\textcolor{pf}{}}
\end{itemize}\par
\quad

{\Large $\circ $ Df1.4.3 (separable)}\par
\begin{itemize}
    \item[]
    {\large }
    {\textcolor{pf}{}}
\end{itemize}\par
\quad



\end{document}