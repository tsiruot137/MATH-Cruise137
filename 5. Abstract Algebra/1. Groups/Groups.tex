\documentclass{article}

    \usepackage{xcolor}
    \definecolor{pf}{rgb}{0.4,0.6,0.4}
    \usepackage[top=1in,bottom=1in, left=0.8in, right=0.8in]{geometry}
    \usepackage{setspace}
    \setstretch{1.2} 
    \setlength{\parindent}{0em}

    \usepackage{paralist}
    \usepackage{cancel}

    % \usepackage{ctex}
    \usepackage{amssymb}
    \usepackage{amsmath}
    \usepackage{extarrows}
    \usepackage{tikz-cd}

    \usepackage{tcolorbox}
    \definecolor{Df}{RGB}{0, 184, 148}
    \definecolor{Th}{RGB}{9, 132, 227}
    \definecolor{Rmk}{RGB}{215, 215, 219}
    \definecolor{P}{RGB}{154, 13, 225}
    \newtcolorbox{Df}[2][]{colbacktitle=Df, colback=white, title={\large\color{white}#2},fonttitle=\bfseries,#1}
    \newtcolorbox{Th}[2][]{colbacktitle=Th, colback=white, title={\large\color{white}#2},fonttitle=\bfseries,#1}
    \newtcolorbox{Rmk}[2][]{colbacktitle=Rmk, colback=white, title={\large\color{black}{Remarks}},fonttitle=\bfseries,#1}

    \title{\LARGE \textbf{Groups}}
    \author{\large Jiawei Hu}

    % new commands for formula typing
    \newcommand{\lcm}{\text{lcm}}
    \newcommand{\cycl}{\text{cycl}}
    \newcommand{\nles}{\vartriangleleft}
    \newcommand{\notnles}{\ntriangleleft}
    \newcommand{\Ker}{\text{Ker}\,}
    \newcommand{\Ima}{\text{Im}\,}
    \newcommand{\Endo}{\text{End}\,}
    \newcommand{\Auto}{\text{Aut}\,}
    \newcommand{\hooktwoheadrightarrow}{%
        \hookrightarrow\mathrel{\mspace{-15mu}}\rightarrow}

\begin{document}
\maketitle

This is the 1st chapter of Abstract Algebra, which is about the \textbf{Groups}. By the way, we now pre-claim some commonly-used notations and terms:
\begin{Df}{Notations and Terms}
    \begin{compactenum}
        \item An agreement for the length of a list: if we write $a_1, \dots, a_n$, then we indicate that $n$ is finite and that $n\geq 1$; if we write $a_0, \dots, a_n$, then we indicate that $n$ is finite and that $n\geq 0$.
    \end{compactenum}
\end{Df}

Here is the \textbf{Quick Search} for this chapter:
\begin{Th}{Quick Search}
    \begin{compactdesc}
        \item (1.2.*): The order of a group; the order of an element of a group.
        \item (1.4.*): Dihedral group.
        \item (1.5.*): Cycles, transpositions and cycle-decomposition.
        \item (1.6.*): Homomorphisms.
        \item (1.7.*): Subgroups and generated subgroups.
        \item (1.8.*): Cyclic groups.
        \item (1.9.*): Cosets.
        \item (1.10.*): $HK$.
        \item (1.11.*): Normal subgroups.
        \item (1.12.*): Homomorphisms involving quotient groups.
        \item (1.13.*): The structure of symmetric groups, alternating groups.
        \item (1.14.*): Simple groups.
        \item (1.15.*): Categories (included in the P1 file).
        \item (1.16.*): Direct products (included in the P1 file).
        \item (1.17.*): Free groups (included in the P1 file).
        \item (1.18.*): A summary of the examples of concrete groups (included in the P2 file).
        \item (1.19.*): Group actions and Sylow theorem (included in the P3 file).
    \end{compactdesc}
\end{Th}

Then with everything prepared, here we go. 

\begin{Df}{Df1.1 (groups)}
    A \textbf{group} is a non-empty set $G$ with a binary operation $*$ (or denoted as $\times$) on it (i.e. for any $a, b\in G$, $a*b\in G$) that satisfies the following properties:
    \begin{compactenum}
        \item (Associative) For any $a, b, c\in G$, $(a*b)*c=a*(b*c)$.
        \item (Two-sided Identity) There exists an element $1\in G$ ($1$ here is some element in $G$, not the number $1$) such that for any $a\in G$, $1*a=a*1=a$. \textcolor{Th}{Obviously, such $1$ is unique, } and we call it the \textbf{identity element} (or \textbf{identity} for short) of $G$.
        \item (Two-sided Inverse) For any $a\in G$, there exists an element $a^{-1}\in G$ such that $a*a^{-1}=a^{-1}*a=1$. \textcolor{Th}{Obviously, for any $a\in G$, such $a^{-1}$ is unique, } and we call it the \textbf{inverse element} (or \textbf{inverse} for short) of $a$. 
    \end{compactenum}
\end{Df}

\begin{Rmk}{}
    \begin{compactenum}
        \item \textcolor{Df}{The identity element of a group is also denoted as $e$ or $\mathrm{id}$ in some cases, as long as it is clear from the context.}
        \item \textcolor{Df}{A non-empty set $G$ with a binary operation $*$ (write as a tuple $(G, \ast)$) that satisfies the first property (resp. the first two properties) (resp. all the three properties) is called a \textbf{semigroup} (resp. a \textbf{monoid}) (resp. a \textbf{group}).}
        \item \textcolor{Df}{For the binary operation $*$ on $G$, we interpret it as ``multiplication'' and write $a*b$ as $a\cdot b$ or $ab$. This is called the \textbf{multiplicative notation}. There is also the \textbf{additive notation}, which is often used for Abelian groups (talked about later).}
        \item \textcolor{Th}{A single element itself can be defined as a group}, \textcolor{Df}{called the \textbf{trivial group}. In this case, the group is said to be trivial.}
        \item \textcolor{Th}{Obviously, for a group $G$ we have
        \begin{compactenum}
            \item $(a^{-1})^{-1}=a$ for any $a\in G$;
            \item $(a*b)^{-1}=b^{-1}*a^{-1}$ for any $a, b\in G$;
            \item Define $a_1a_2a_3\cdots a_n = (\cdots ((a_1a_2)a_3)\cdots a_{n-1})a_n$. Then the associative parentheses can be arbitrarily placed in $a_1\cdots a_n$ without changing the result.
            \item (Two-sided cancellation) For any $a, b, c\in G$, if $c*a=c*b$, (resp. $a*c=b*c$), then $a=b$.
        \end{compactenum}}
        \item \textcolor{Th}{Some examples of groups include:
        \begin{compactenum}
            \item $(\mathbb{Z}, +)$, $(\mathbb{Q}, +)$, $(\mathbb{R}, +)$, $(\mathbb{C}, +)$, each with the identity $0$ and the inverse $-a$;
            \item $(\mathbb{Q}\backslash\{0\}, \cdot)$, $(\mathbb{R}\backslash\{0\}, \cdot)$, $(\mathbb{C}\backslash\{0\}, \cdot)$, each with the identity $1$ and the inverse $1/a$;
            \item $(\mathbb{F}^n, +)$, $(\mathbb{F}^{m,n}, +)$ where $\mathbb{F}$ is a number field, each with the identity $\mathbf{0}$ (the zero vector or matrix) and the inverse $-\mathbf{a}$;
            \item $(\mathcal{A}(S), \circ)$ where $S$ is a non-empty set and $\mathcal{A}(S)$ is the set of all bijections $f$ from $S$ to $S$, with the identity $\text{id}_S$ (the identity function) and the inverse $f^{-1}$.
        \end{compactenum}}
        \textcolor{Df}{For the group $\mathcal{A}(S)$ here, the elements in $\mathcal{A}(S)$ are called \textbf{permutations} of $S$. If $S = \{1, \cdots, n\}$, then $\mathcal{A}(S)$ called the \textbf{symmetric group on $n$ letters}, denoted by $\mathbb{S}_n$.} \textcolor{Df}{For $f\in\mathbb{S}_n$ and $a\in\{1,\cdots,n\}$, we call $a$ a \textbf{fixed point} (resp. \textbf{non-fixed point}) of $f$ if $f(a) = a$ (resp. $f(a) \neq a$).}
    \end{compactenum}
\end{Rmk}

\begin{Df}{Df1.2 (order)}
    Let $G$ be a group ($(G, \ast)$). Then:
    \begin{compactenum}
        \item The \textbf{order} of an element $a\in G$, denoted as $|a|$, is the smallest positive integer $n$ such that $a^n=1$ (here $a^n = \underbrace{a\ast a\ast \cdots \ast a}_{n\text{ times}}$). If no such $n$ exists, then we say that the order of $a$ is infinite, denoted as $|a|=\infty$;
        \item The \textbf{order} of a group $G$ is the Cardinality of $G$, denoted as $|G|$. $G$ is said to be \textbf{finite} (resp. \textbf{infinite}) if $G$ is a finite (resp. infinite) set.
    \end{compactenum}
\end{Df}

\begin{Rmk}{}
    \textcolor{Df}{Define the \textbf{integer powers} of an element $a$ in a group $G$ as $a^0\triangleq 1$, $a^1\triangleq a$, $a^2\triangleq a\ast a$, $a^3\triangleq a\ast a\ast a$, $\cdots$, and $a^{-1}\triangleq a^{-1}$, $a^{-2}\triangleq (a^{-1})^2 = (a^2)^{-1}$, $a^{-3}\triangleq (a^{-1})^3 = (a^3)^{-1}$, $\cdots$.}
\end{Rmk}

\begin{Th}{Th1.2.1}
    Finite groups' elements all have finite orders.
    \tcblower
    \textit{Pf}: If otherwise some element $a$ in the finite group $G$ has infinite order, namely, the sequence $a, a^2, a^3, \cdots$ would not go back to $1$, then it must have a repeated element (since $G$ contains only finitely many elements), say, $a^i=a^{i+k}$ for some $i, k>0$. Then $a^k = 1$, contradiction.
\end{Th}

\begin{Df}{Df1.3 (Abelian group)}
    A group $G$ is said to be \textbf{Abelian} or \textbf{commutative} if the binary operation is commutative, i.e. 
    $$ ab = ba $$
    for any $a, b\in G$.
\end{Df}

\begin{Rmk}{}
    \begin{compactenum}
        \item \textcolor{Th}{The group $(\mathbb{S}_n, \circ)$ is not Abelian for $n\geq 3$.}
        \item \textcolor{Df}{We often use the additive notation for Abelian groups, that is, we denote the binary operation as $+$ and write $a+b$ instead of $a\ast b$. Hence, for general groups, we use the multiplicative notation, and denote the identity as $1$ and the inverse as $a^{-1}$; for Abelian groups, we use the additive notation, and denote the identity as $0$ and the inverse as $-a$.}
    \end{compactenum}
\end{Rmk}

\begin{Th}{Ex1.3.1}
    Let $G$ be a group. Prove that:
    \begin{compactenum}
        \item If $a^2=1$ for any $a\in G$, then $G$ is Abelian.
        \item If $|G|$ is finite and even, then there exists $a\in G$ such that $a\neq 1$ and $a^2=1$.
        \item For elements $a, b\in G$, we have $|a^{-1}| = |a|$, $|ab| = |ba|$ and $|bab^{-1}| = |a|$.
    \end{compactenum}
    \tcblower
    \textit{Solution}:
    \begin{compactenum}
        \item Since $a^2=1$ $\Leftrightarrow$ $a=a^{-1}$, we have
        $$ ab = (ab)^{-1} = b^{-1}a^{-1} = ba. $$
        \item Pair the non-desired elements $a$ with their inverses $a^{-1}$, and remove all these ``bad'' pairs $(a, a^{-1})$ from $G$.
        \item Obvious. 
    \end{compactenum}
\end{Th}

\begin{Th}{Ex1.4 (dihedral group)}
    Let $P_n$ be the regular $n$-gon in the $\mathbb{R}^2$ plane ($n\geq 3$) whose center is placed at the origin. Consider the set $\mathbb{D}_n$ of all the linear isometries $S$ on $\mathbb{R}^2$ that keep $P_n$ invariant (i.e. $S(P_n)=P_n$). Then
    \begin{compactenum}
        \item Is $\mathbb{D}_n$ a group?
        \item Which elements are in $\mathbb{D}_n$? 
        \item What are the orders of the group and the elements?
    \end{compactenum}
    \tcblower
    \textit{Solution}:
    \begin{compactenum}
        \item $(\mathbb{D}_n, \circ)$ is a group, \textcolor{Df}{called the \textbf{dihedral group}}. This can be answered after we figure out the elements in $\mathbb{D}_n$ (see below).
        \item The discussion about determinants (in the I1 file of the chapter 8, course 1) gives us the very hint about the answer, which is that $\mathbb{D}_n$ contains and only contains $n$ rotation operators and $n$ flipping operators, $2n$ elements in total. Actually, let $r$ be the rotation operator that rotates the regular $n$-gon by $2\pi/n$ counterclockwise, and let $s$ be the flipping operator that flips the $n$-gon with respect to the axis passing through the vertex $1$ (let us mark the vertices of the $n$-gon as $1, 2, \cdots, n$ beforehand) and the center. Then we know that these $2n$ elements 
        $$ r^0, r, \cdots, r^{n-1}, s, sr, \cdots, sr^{n-1} $$
        are all in $\mathbb{D}_n$, and it seems that $r^0 = r^n = s^2 = s^0 = \mathrm{id}$ for $\mathbb{D}_n$ to be a group. Then we can easily check that this set (with these $2n$ elements) is a group. Besides, $\mathbb{D}_n$ only contains these $2n$ elements. Actually, since a linear isometry preserves the norms of vectors, a vertex of the $n$-gon can only be mapped to again a vertex; and since an isometry also preserves the distances between points, two adjacent vertices of the $n$-gon can only be mapped to again two adjacent vertices. Hence an element in $\mathbb{D}_n$ is determined by which vertices it maps the vertices $1$ and $2$ to, and there are exactly $2n$ such possibilities. Thus $|\mathbb{D}_n|=2n$ and there is no other element except for these $2n$ elements in $\mathbb{D}_n$.
        \item From 2, we know that $|D_n|=2n$, and
        $$ |r^k| = \frac{\lcm(k, n)}{k}, \quad |sr^k| = 2 $$
        for any $k=1,\cdots, n$ (where $\lcm$ is the least common multiple).
        Actually, $|r^k|$ is obvious, and we can verify that $|s| = 2$ and $sr = r^{-1}s$, so that 
        $$ sr^k sr^k = sr^{k-1}r(sr)r^{k-1} = sr^{k-1}r(r^{-1}s)r^{k-1} = sr^{k-1}sr^{k-1} = \cdots = ss = \mathrm{id}. $$
    \end{compactenum}
\end{Th}

\begin{Df}{Df1.5 (cycle)}
    A permutation $\sigma\in\mathbb{S}_n$ (of course $n\geq 2$) is called a \textbf{cycle} in $\mathbb{S}_n$ if there exist distinct $a_1, a_2, \cdots, a_m\in \{1,2,\cdots,n\}$ ($m\geq 2$) such that
    $$ \sigma(a_1) = a_2, \quad \sigma(a_2) = a_3, \quad \cdots \quad ,\sigma(a_{m-1}) = a_m, \quad \sigma(a_m) = a_1 $$
    and $\sigma(a_i) = a_i$ for any $a_i\notin \{a_1, a_2, \cdots, a_m\}$. In this case:
    \begin{compactenum}
        \item We write $\sigma = \cycl (a_1, a_2, \cdots, a_m)$ (or $\sigma = (a_1, a_2, \cdots, a_m)$, or $\sigma = (a_1a_2\cdots a_m)$, if no ambiguity), and call $(a_1, a_2, \cdots, a_m)$ a \textbf{cycle expression} of $\sigma$, and call $a_1, a_2, \cdots, a_m$ the \textbf{elements} of $\sigma$;
        \item \textcolor{Th}{The non-fixed points of $\sigma$ are exactly the elements of $\sigma$, and the fixed points of $\sigma$ are exactly the others.} Thus \textcolor{Th}{the elements of $\sigma$ are unique (that is, must be the set of non-fixed points of $\sigma$)}, and we call $m$ the \textbf{length} of $\sigma$, or say that $\sigma$ is a $m$-cycle.
        \item \textcolor{Th}{Obviously there is no $1$-cycle. And the $0$-cycle is exactly the identity map.}
        \item \textcolor{Th}{Different expressions of the same cycle is of cyclic symmetry, that is, all possible expressions of the cycle $(a_1, a_2, \cdots, a_m)$ are exactly $(a_1, a_2, \cdots, a_m)$, $(a_2, \cdots, a_m, a_1)$, $\cdots$, $(a_m, a_1, a_2, \cdots )$.}
        \item $\sigma$ is called a \textbf{transposition} if $m=2$.
    \end{compactenum}
\end{Df}

\begin{Rmk}{}
    \begin{compactenum}
        \item \textcolor{Th}{Not all permutations are cycles.} For example, the permutation $f\in\mathbb{S}_4$ such that $f(1,2,3,4) = (3,4,1,2)$ is not a cycle. Actually, if otherwise $f$ is a cycle, then we write down $f$ beginning with $1$:
        $$ f = \left(1, \right. = \left(1, 3, \right. = \left(1, 3\right) $$
        which means that $f(2) = 2$ and $f(4) = 4$, contradiction.
        \item \textcolor{Th}{(The inverse of a cycle) In $\mathbb{S}_n$, the inverse of a cycle is still a cycle:
        $$ (a_1, a_2, \cdots, a_m)^{-1} = (a_m, \cdots, a_2, a_1). $$}
        \item (The order of cycles) \textcolor{Th}{The order of a cycle is equal to its length (expect for the $0$-cycle, whose order is $1$).} 
    \end{compactenum}
\end{Rmk}

\begin{Th}{Th1.5.1 (disjoint permutations are commutative)}
    \begin{compactenum}
        \item \textcolor{Df}{A list of permutations $\sigma_1, \sigma_2, \cdots, \sigma_m\in\mathbb{S}_n$ are said to be \textbf{disjoint} if the sets $N_1, N_2, \cdots, N_m$ of their non-fixed points are pairwise disjoint.} 
        \item Disjoint permutations are commutative.
    \end{compactenum}
    \tcblower
    \textit{Pf}: Obvious. 
\end{Th}

\begin{Th}{Th1.5.2 (the cycle-decomposition of a permutation)}
    Any $f\in\mathbb{S}_n\setminus \{\mathrm{id}\}$ (that is, $f$ is a permutation in $\mathbb{S}_n$ and $f$ is not the identity map) can be written as a product of disjoint cycles in $\mathbb{S}_n$, and this product is unique up to the order of the cycles (of course, the identity map is omitted in this product expression). 
    \tcblower
    \textit{Pf}: 
    \begin{compactenum}
        \item (Existence) Easy to check that the following algorithm works and finds a cycle decomposition of any $f$ as desired:
        \begin{compactenum}
            \item[(1)] Initially, mark all the elements $1, \cdots, n$ as unvisited.
            \item[(2)] Choose the smallest unvisited number $a$ and compute to obtain the cycle $\sigma = (a, f(a), f^2(a), \cdots)$. Use $\sigma$ as a factor in the cycle decomposition.
            \item[(3)] Mark all the elements in $\sigma$ as visited. If there is still unvisited elements, go back to step (2); otherwise, terminate.
        \end{compactenum}
        Obviously, the step (2) will end up with a cycle $\sigma$ since the $|a|$ is finite, and the next $\sigma$ obtained from the next step (2) will be disjoint with the previous ones.
        \item (Uniqueness) Suppose 
        $$ (\quad)\cdots(\quad) = [\quad]\cdots[\quad] $$
        where ``$(\quad)$'' and ``$[\quad]$'' are all cycles, and these ``$(\quad)$'' (resp. these ``$[\quad]$'') are disjoint. Then we can compare a pair of ``$(\quad)$'' and ``$[\quad]$'' via the behavior of $f$ on a non-fixed point $a$ (i.e. $f(a)\neq a$), which resulted in that $(\cdots, a, \cdots) = [\cdots, a, \cdots]$. 
    \end{compactenum}
\end{Th}

\begin{Th}{Th1.5.3 (the order of a permutation)}
    The order of a permutation equals the least common multiple of the orders of its cycle factors (in the expression of cycle decomposition (Th \{, ID: 1.5.2\})). In particular, the order of the identity map $\mathrm{id}$ is $1$, namely, $|\mathrm{id}|=|\text{the } 0\text{-cycle}| = \lcm(1) = 1$.
    \tcblower
    \textit{Pf}: Obvious by the cycle decomposition.
\end{Th}

\begin{Th}{Th1.5.4 (the transposition-decomposition of a permutation)}
    Any $f\in\mathbb{S}_n\setminus\{\mathrm{id}\}$ can be written as a product of transpositions. This factorization is not necessarily unique, and the factors (as cycles) are not necessarily disjoint.
    \tcblower
    \textit{Pf}: One can easily prove it based on the cycle decomposition, but we here prove it in an easier and more general way. Actually we do not need to formally prove it, but just need to clarify it by an exercise. That is, for an arbitrary given permutation, say, $f = (3, 4, 2, 1)^\mathrm{T}$ (here the notation means a column vector, representing that $f(1) = 3$, $f(2) = 4$, $f(3) = 2$, $f(4) = 1$), please write it as a product of transpositions. Then it is naturally to think of:
    $$ \mathrm{id} = \begin{bmatrix}
        1 \\ 2 \\ 3 \\ 4
    \end{bmatrix} \xlongrightarrow[\text{place } 3\text{ to the correct position}]{\text{multiply }(1,3) \text{ on the left to}} \begin{bmatrix}
        3 \\ 2 \\ 1 \\ 4
    \end{bmatrix} \xlongrightarrow[\text{place } 4\text{ to the correct position}]{\text{multiply }(2,4) \text{ on the left to}} \begin{bmatrix}
        3 \\ 4 \\ 1 \\ 2
    \end{bmatrix} \xlongrightarrow[\text{to place } 2\text{ to }\cdots]{\text{multiply }(1,2)\, \cdots} \begin{bmatrix}
        3 \\ 2 \\ 4 \\ 1
    \end{bmatrix} = f. $$
    Thus $f = (1,2)(2,4)(1,3)$. \\
    Now we see that this expression is not unique, as also $f = (3,4)(2,3)(1,4)$. Also, a general permutation may fail to be expressed as a product of disjoint transpositions, otherwise it would violates the uniqueness of the cycle decomposition. 
\end{Th}

\begin{Th}{Th1.5.5 (quickly factorizing a cycle into transpositions)}
    For any cycle $(a_1, a_2, \cdots, a_m)$, we have
    $$ (a_1, a_2, \cdots, a_m) = (a_1, a_m)(a_1, a_{m-1})\cdots(a_1, a_3)(a_1, a_2). $$
    \tcblower
    \textit{Pf}: Obvious.
\end{Th}

\begin{Df}{Df1.6 (homomorphisms)}
    Let $G$ and $H$ are semigroups. Then a function $f: G\to H$ is called a \textbf{homomorphism} (from $G$ to $H$) if 
    $$ f(ab) = f(a)f(b) $$
    for any $a, b\in G$. In this case, if further:
    \begin{compactenum}
        \item $f$ is injective, then $f$ is called a \textbf{monomorphism}.
        \item $f$ is surjective, then $f$ is called an \textbf{epimorphism}.
        \item $f$ is bijective, then $f$ is called an \textbf{isomorphism}, denoted as $f: G\overset{\sim}{\to} H$. In this case, $G$ and $H$ are called \textbf{isomorphic}, denoted as $G\simeq H$.
        \item $H = G$, then $f$ is called an \textbf{endomorphism}.
        \item $H = G$ and $f$ is bijective, then $f$ is called an \textbf{automorphism}.
    \end{compactenum}
    The set of all endomorphisms (resp. automorphisms) of a group $G$ is denoted as $\Endo G$ (resp. $\Auto G$).
\end{Df}

\begin{Rmk}{}
    \textcolor{Df}{In this definition, if the semigroups $G$ and $H$ are groups, then we call $f$ a \textbf{group homomorphism}.}
    \textcolor{Th}{Let $G$, $H$ are groups and $f: G\to H$ is a homomorphism. Then:
    \begin{compactenum}
        \item $f(1_G) = 1_H$;
        \item $f(a^{-1}) = f(a)^{-1}$;
    \end{compactenum}}
    One can prove the first statement by first showing that $f(G)$ is a subgroup (subgroups are talked later, but here is no cyclic reasoning) of $H$, and that $f(1_G) = 1_{f(G)}$ (the identity element of the group $f(G)$), and then using the fact that $1_{f(G)} = 1_H$. The second statement can be directly proved by the first statement.
    \textcolor{Th}{Some examples of these ``morphisms'':
    \begin{compactenum}
        \item The function $f: (\mathbb{Z}/10\mathbb{Z})\to (\mathbb{Z}/10\mathbb{Z})$ defined as
        $$ f(\bar{x}) = 2\bar{x} $$
        is a homomorphism, but is neither a monomorphism nor an epimorphism. If $f$ is instead defined as
        $$ f(\bar{x}) = 3\bar{x}, $$
        then it is an isomorphism (and thus an automorphism).
        \item The function $f: G\to G$ on an Abelian group $G$ defined as $f(a) = a^{-1}$ is an automorphism. However, this is not true for non-Abelian groups. A counterexample is that $G = \mathbb{S}_3$.
    \end{compactenum}}
\end{Rmk}

\begin{Df}{Df1.6.1 (kernel and image of a homomorphism)}
    Let $G$ and $H$ are groups, and $f: G\to H$ be a homomorphism. Then:
    \begin{compactenum}
        \item the \textbf{(homomorphism) kernel} of $f$, denoted as $\Ker f$, is defined as $\Ker f = \{a\in G: f(a) = 1_H\}$;
        \item the \textbf{(homomorphism) image} of $f$, denoted as $\Ima f$, is defined as $\Ima f = \{f(a): a\in G\}$.
    \end{compactenum}
\end{Df}

\begin{Th}{Th1.6.2}
    Let $f: G\to H$ be a group homomorphism. Then:
    \begin{compactenum}
        \item $f$ is an monomorphism iff $\Ker f = \{1_G\}$;
        \item $f$ is an epimorphism iff $\Ima f = H$;
        \item $f$ is an isomorphism iff 
        $$ ff^{-1} = \mathrm{id}_H \quad \text{and} \quad f^{-1}f = \mathrm{id}_G $$
        for some group homomorphism $f^{-1}: H\to G$ (here the $\mathrm{id}$ is the identity function).
    \end{compactenum}
    \tcblower
    \textit{Pf}: Obvious.
\end{Th}

\begin{Df}{Df1.7 (subgroups)}
    Let $(G, \cdot)$ be a group. Then a subset $H$ of $G$ is called a \textbf{subgroup} of $G$ if $(H, \cdot)$ is a group. In this case, we write $H < G$.
\end{Df}

\begin{Rmk}{}
    \begin{compactenum}
        \item \textcolor{Th}{Given a group $G$, we have $\{1\} < G$ and $G < G$.} \textcolor{Df}{The former is called the \textbf{trivial subgroup} of $G$.} And \textcolor{Df}{for $H < G$, we say $H$ is a \textbf{proper subgroup} of $G$, if $H$ is neither $\{1\}$ nor $G$.}
    \end{compactenum}
\end{Rmk}

\begin{Th}{Th1.7.1 (how to check a subgroup)}
    Let $(G, \cdot)$ be a group and $H\subseteq G$. Then $H < G$ iff
    \begin{compactenum}
        \item[(i)] $H$ is closed under $\cdot$\; (i.e. for any $a, b\in H$, $a\cdot b\in H$);
        \item[(ii)] $1\in H$;
        \item[(iii)] For any $a\in H$, $a^{-1}\in H$.
    \end{compactenum}
    \tcblower
    \textit{Pf}: Obvious.
\end{Th}

\begin{Th}{Th1.7.1.1 (Th \{, ID: 1.7.1\} for short)}
    Let $(G, \cdot)$ be a group and $\varnothing\neq H\subseteq G$. Then
    $$ \begin{aligned}
        H < G \quad & \Longleftrightarrow \quad \forall a, b\in H, \quad ab^{-1}\in H \\ \quad & \Longleftrightarrow \quad \forall a, b\in H, \quad a^{-1}b\in H. 
    \end{aligned} $$
    \tcblower
    \textit{Pf}: Obvious.
\end{Th}

\begin{Rmk}{}
    This theorem is only looking easier than Th \{, ID: 1.7.1\} for checking a subgroup, but is not different much in use. With these two theorems, \textcolor{Th}{we can easily check the following examples and facts of subgroups:
    \begin{compactenum}
        \item For any $n\in\mathbb{N}$, $n\mathbb{Z}$ is a subgroup of $(\mathbb{Z}, +)$;
        \item Let $H = \{f\in\mathbb{S}_n: f(n) = n\}$. Then $H<\mathbb{S}_n$, and in fact, $H\simeq\mathbb{S}_{n-1}$.
        \item (Homomorphisms) Let $f: G\to H$ be a group homomorphism. Then 
        \begin{compactenum}
            \item For any $A < G$, $f(A) < H$ (here $f(A) = \{f(a): a\in A\}$). In particular, $\Ima f < H$;
            \item For any $B < H$, $f^{-1}(B) < G$ (here $f^{-1}(B) = \{a\in G: f(a)\in B\}$). In particular, $\Ker f < G$.
        \end{compactenum}
        \item (Intersection) The intersection of (any non-empty family of) subgroups is still a subgroup.
        \item (Union) The union of subgroups is not necessarily a subgroup. For example, $2\mathbb{Z}\cup 3\mathbb{Z}$ is not a subgroup of $(\mathbb{Z}, +)$.
    \end{compactenum}}
\end{Rmk}

\begin{Df}{Df1.7.2 (generated subgroup)}
    Let $G$ be a group and $\varnothing\neq X\subseteq G$. Then the \textbf{subgroup generated by $X$}, denoted as $\langle X\,\rangle$, is the smallest subgroup of $G$ that contains $X$. In other words, 
    $$ \langle X\,\rangle \triangleq \bigcap_{H\leq G,\, X\subseteq H} H. $$
    In this case, we say that $X$ \textbf{generates} $\langle X\,\rangle$, and the elements in $X$ are called the \textbf{generators} of $\langle X\,\rangle$. If $X$ is finite, with $X = \{a_1, \cdots, a_n\}$, then we write $\langle X\,\rangle = \langle a_1, \cdots, a_n\,\rangle$ for convenience.
\end{Df}

\begin{Rmk}{}
    \textcolor{Df}{Define the generated subgroup by a (arbitrary) family of subsets $\{X_i: i\in I\}$ as $\langle\; \bigcup_{i\in I} X_i\;\rangle$. In this case, if $I$ is finite, with $I = \{1, \cdots, n\}$, then we write
    $$ \langle\; \bigcup_{i\in I} X_i\;\rangle = X_1\vee\cdots\vee X_n = \bigvee_{i=1}^n X_i. $$}
\end{Rmk}

\begin{Th}{Th1.7.2.1 (which elements $\langle X\,\rangle$ contains)}
    Let $G$ be a group and $\varnothing\neq X\subseteq G$. Then
    $$ \langle X\,\rangle = \Big\{a_1^{n_1}\cdots a_m^{n_m}: \quad m\in\mathbb{N}^\ast, a_i\in X, n_i\in\mathbb{Z}\Big\}. $$
    \tcblower
    \textit{Pf}: Obvious.
\end{Th}

\begin{Df}{Df1.7.2.2 (cyclic subgroup)}
    \begin{compactenum}
        \item A subgroup $H$ of a group $G$ is called \textbf{cyclic} if 
        $$ H = \langle a\rangle \quad \Big( = \langle \{a\} \rangle \Big) $$
        for some $a\in G$. 
        \item A group $G$ is called \textbf{cyclic} if $G = \langle a\rangle$ for some $a\in G$;
        \item A group $G$ is called \textbf{finitely generated} if 
        $$ G = \langle a_1, \cdots, a_n\,\rangle $$
        for some (finitely many elements) $a_1, \cdots, a_n\in G$.
    \end{compactenum}
\end{Df}

\begin{Rmk}{}
    \textcolor{Th}{Some examples of cyclic subgroups:
    \begin{compactenum}
        \item For any $n\in\mathbb{Z}$, we have $\langle n\rangle = n\mathbb{Z}$; conversely, if $n\mathbb{Z} = \langle a\rangle$, then $a = \pm n$.
        \item For any $n\in\mathbb{N}$ and $n\geq 2$, we have $\mathbb{S}_n = \langle \text{transpositions in }\mathbb{S}_n\,\rangle$.
        \item For any $n\in\mathbb{N}$ and $n\geq 2$, we have $(\mathbb{Z}/n\mathbb{Z}, +) = \langle \bar{1}\rangle$. Actually, $\mathbb{Z}/n\mathbb{Z} = \langle \bar{x}\rangle$ iff $\gcd(x, n) = 1$.
    \end{compactenum}}
\end{Rmk}

\begin{Th}{Th1.8 (what cyclic groups exactly are)}
    A cyclic group $G$ is either isomorphic to $(\mathbb{Z}/n\mathbb{Z}, +)$ for some $n\in\mathbb{N}$ (say $\mathbb{Z}/1\mathbb{Z} = \{\bar{0}\}$), or is isomorphic to $(\mathbb{Z}, +)$.
    \tcblower
    \textit{Pf}: Write $G = \langle a\rangle = \{a^n: n\in\mathbb{Z}\}$. Then:
    \begin{compactenum}
        \item If $G$ is finite, then 
        $$ a^j = a^{j+n} $$
        for some $j\in\mathbb{Z}$ and $n\in\mathbb{N}^\ast$. Say this $n$ is the smallest one (that makes the above equation hold). Then $1 = a^n$, and thus
        $$ G = \{1, a, \cdots, a^{n-1}\} \simeq \mathbb{Z}/n\mathbb{Z}, $$
        via the isomorphism $a^j\mapsto \bar{j}$.
        \item If $G$ is infinite, then 
        $$ \cdots, a^{-2}, a^{-1}, 1, a, a^2, \cdots $$
        has no repeated elements (since otherwise we fall back into the case of finite $G$), and thus $G\simeq\mathbb{Z}$, via the isomorphism $a^n\mapsto n$.
    \end{compactenum}
\end{Th}

\begin{Th}{Th1.8.1 (the subgroups of $(\mathbb{Z}, +)$)}
    The subgroups of $(\mathbb{Z}, +)$ are exactly the $n\mathbb{Z}$'s ($n\in\mathbb{Z}$).
    \tcblower
    \textit{Pf}: Clearly any $n\mathbb{Z}$ ($n\in\mathbb{Z}$) is a subgroup of $(\mathbb{Z}, +)$. \\
    Conversely, let $H$ be a subgroup of $(\mathbb{Z}, +)$, then $H$ is trivial ($H = \{0\} = 0\mathbb{Z}$) or non-trivial. If $H$ is non-trivial, then $H$ contains a non-negative element, and thus a positive element, and thus a minimal positive element, say, $n$. Then clearly
    $$ n\mathbb{Z} \subseteq H. $$
    Now for any $m\in H$, we claim that $n\mid m$ so that $m\in n\mathbb{Z}$. Otherwise, $\gcd(n, m) = d < n$, and thus by the Bezout identity:
    $$ xn + ym = d $$
    for some $x, y\in\mathbb{Z}$. Since $n, m\in H$, we have $d\in H$. But $d<n$, contradicting the minimality of $n$. Hence $H\subseteq n\mathbb{Z}$, and thus $H = n\mathbb{Z}$.
\end{Th}

\begin{Th}{Th1.8.2 (the subgroups of $(\mathbb{Z}/n\mathbb{Z}, +)$ (the group of congruence classes modulo $n$))}
    Let $n\geq 2$ be an integer. If $H<(\mathbb{Z}/n\mathbb{Z}, +)$, then $H$ is cyclic. Exactly:
    \begin{compactenum}
        \item $H = \langle d\rangle$, where $d$ is the smallest integer in $\{1, \cdots, n\}$ such that $\overline{d}\in H$ (here we write $\overline{n} = \overline{0}$, and so throughout this theorem);
        \item Consider the $d$ described above. Then any $\overline{d^\prime}$ s.t. $\gcd(d^\prime, n) = d$ can be a generator of $H$;
        \item For any $d^\prime\in\{1, \cdots, n\}$, we have
        $ \big|\langle \overline{d^\prime}\rangle\big| = \frac{n}{\gcd(d^\prime, n)} $.
    \end{compactenum}
    \tcblower
    \textcolor{P}{\textit{Analytically}: To figure out what the subgroups of $\mathbb{Z}/n\mathbb{Z}$ are, we can try the cyclic subgroups, say, $\langle \overline{1}\rangle < \mathbb{Z}/10\mathbb{Z}$, $\langle \overline{2}\rangle < \mathbb{Z}/10\mathbb{Z}$, $\langle \overline{3}\rangle < \mathbb{Z}/10\mathbb{Z}$ and so on, by which we would derive this theorem.} \\
    \textit{Pf}: 
    \begin{compactenum} 
        \item Let $H<\mathbb{Z}/n\mathbb{Z}$. First prove that $H = \langle \overline{d}\rangle$. Clearly $\langle \overline{d}\rangle < H$. Then consider $\overline{h}\in H$ ($h\in\{1,\cdots,n\}$). We claim that $d\mid h$, and then we complete the proof of this part (since if $d\mid h$, then $\overline{h}\in\langle\overline{d}\rangle$). Consider $\gcd(h, d)$, we first have $\gcd(h, d) \leq d$. Now write the Bezout identity:
        $$ xh + yd = \gcd(h, d), $$
        and thus $\overline{xh} + \overline{yd} = \overline{\gcd(h, d)} $. Since $\overline{h}, \overline{d}\in H$, we have $\overline{\gcd(h, d)}\in H$, and thus by the minimality of $d$ we get $\gcd(h, d) \geq d$, and thus $\gcd(h, d) = d$, and thus $d\mid h$.
        \item If we have some $d^\prime\in\{1,\cdots,n\}$ s.t. $\gcd(d^\prime, n) = d$, then we claim that $\langle \overline{d^\prime}\rangle = \langle \overline{d}\rangle$. We first have $\langle \overline{d^\prime}\rangle < \langle d \rangle$. For the other direction, write again the Bezout identity:
        $$ d = xd^\prime + yn, $$
        and thus $d\equiv xd^\prime$ modulo $n$, and thus $\overline{d} = \overline{xd^\prime}\in\langle \overline{d^\prime}\rangle$, and thus $\langle \overline{d}\rangle = \langle \overline{d^\prime}\rangle = H$.
        \item Clearly $\langle \overline{d^\prime}\rangle$ consists of $\overline{d^\prime}, \overline{2d^\prime}, \cdots$, and the list stops exactly when $\overline{kd^\prime}$ reaches a multiple of $n$ for the first time. Thus it is obvious that:
        $$ \big|\langle \overline{d^\prime}\rangle\big| = \frac{\lcm(d^\prime, n)}{d^\prime} = \frac{d^\prime n}{\gcd(d^\prime, n)d^\prime} = \frac{n}{\gcd(d^\prime, n)}. $$
    \end{compactenum}
\end{Th}

\begin{Th}{Th1.8.3 ($\big|\langle a\rangle\big| = |a|$)}
    Let $G$ be a cyclic group with $G = \langle a\rangle$. Then $|G| = |a|$. Actually, if $|a|$ is finite, then
    $$ G = \{a, a^2, \cdots, a^{|a|}(=\text{identity})\}. $$
    \tcblower
    \textit{Pf}: Obvious.
\end{Th}

\begin{Th}{Th1.8.4}
    Every homomorphic image and every subgroup of a cyclic group is cyclic.
    \tcblower
    \textit{Pf}: For the group homomorphism $f:\langle a\rangle\to K$, we have $\text{Im} f = f(\langle a \rangle) = \langle f(a)\rangle$. For a subgroup of a cyclic group, we can treat it as a subgroup of $\mathbb{Z}$ or a subgroup of $\mathbb{Z}/n\mathbb{Z}$, which has been thouroughly discussed in the Th \{, ID: 1.8.1\} and the Th \{, ID: 1.8.2\}.
\end{Th}

\begin{Th}{Th1.8.5}
    A group is finite iff it has only finitely many subgroups.
    \tcblower
    \textit{Pf}: The ``only if'' is obvious. Now suppose a group $G$ has only finitely many subgroups. If otherwise $G$ is infinite, then select $1\neq a\in G$, and consider $\langle a\rangle$. If $\langle a\rangle \simeq \mathbb{Z}$, then (just like $\mathbb{Z}$) $\langle a\rangle$ has infinitely many subgroups (also subgroups of $G$), contradiction. Thus $\langle a\rangle$ is a finite subgroup of $G$. Then select $1\neq b \in G$ s.t. $b\notin\langle a\rangle$, and still, $\langle b\rangle$ is finite, and it is another subgroup of $G$. By repeating this process we would get infinitely many subgroups of $G$, contradiction.
\end{Th}

\begin{Th}{Th1.8.6}
    An infinite group is cyclic iff it it isomorphic to each of its proper subgroups.
    \tcblower
    \textit{Pf}: The ``Only if'' is obvious. Now suppose $G$ is infinite and isomorphic to each of its proper subgroups. Then $G\simeq \langle a\rangle$ for some $a\in G$, indicating that $G$ is cyclic.
\end{Th}

\begin{Th}{Ex1.8.7}
    Suppose $G$ is an Abelian group, and $a, b\in G$, $|a|, |b| < \infty$. Prove that there is an element in $G$ whose order is $\lcm(|a|, |b|)$.
    \tcblower
    \textit{Solution}: 
    \begin{compactenum}
        \item First assume that $\gcd(|a|, |b|) = 1$. Then we claim that $|ab| = |a||b| = \lcm(|a|, |b|)$. On one hand, $|ab|$ divides $|a||b|$ since 
        $$ (ab)^{|a||b|} \xlongequal{G \text{ is Abelian}} a^{|a||b|}b^{|b||a|} = 1. $$
        On the other hand, $|a||b|$ divides $|ab|$. Actually, since $(ab)^{|ab|} = 1$, we have
        $$ a^{|b||ab|} = a^{|b||ab|}b^{|b||ab|} = (ab)^{|b||ab|} = 1, $$
        and thus $|a|$ divides $|b||ab|$. By the Bezout identity $x|a| + y|b| = 1$, we have
        $$ x|a||ab| + y|b||ab| = |ab|. $$
        Since $|a|$ divides $|b||ab|$, and thus divides the left side of the identity, we obtain that $a$ divides $|ab|$. By a similar argument we have that $|b|$ divides $|ab|$. Thus $|ab|$ is a common multiple of $|a|$ and $|b|$, and thus $|a||b| = \lcm(|a|, |b|)$ divides $|ab|$.
        \item Then consider the general case $\gcd(|a|, |b|) = d$. Denote $|a| = m$, $|b| = n$. A natural idea is to reduce it to the previous case, which requires us to write 
        $$ \lcm(m,n) = m^\prime n^\prime $$
        for some $m^\prime$, $n^\prime$ s.t. $m^\prime|m$, $n^\prime|n$ and $\gcd(m^\prime, n^\prime) = 1$. This can be achieved by first performing the prime-factorization:
        $$ \begin{aligned}
            &\lcm(m,n) = \frac{m}{d}\cdot d\cdot\frac{n}{d} = p\cdot d\cdot q \\
            &= \underbrace{\prod_{i} p_i}_p \underbrace{\prod_{i} p_{k_i} \prod_{i} r_i \prod_{i} q_{k_i}}_d \underbrace{\prod_{i} q_i}_q \\
            &= \Pi_1\Pi_2\Pi_3\Pi_4\Pi_5,
        \end{aligned} $$
        where the $p_i$'s, $r_i$'s, $q_i$'s are primes, and $\{p_i\}$, $\{q_i\}$, $\{r_i\}$ are disjoint. Now let $m^\prime = \Pi_1\Pi_2\Pi_3$, $n^\prime = \Pi_4\Pi_5$, and we can easily check that $m^\prime$ and $n^\prime$ satisfy the requirements. Therefore by the previous case, we have $|a^{m/m^\prime}| = m^\prime$ and $|b^{n/n^\prime}| = n^\prime$, and thus
        $$ |a^{m/m^\prime}b^{n/n^\prime}| = m^\prime n^\prime = \lcm(m,n). $$
    \end{compactenum}
\end{Th}

\begin{Df}{Df1.9 (left and right congruence)}
    Let $G$ be a group and $H<G$. Then for $a, b\in G$, we say that $a$ is \textbf{right congruent} (resp. \textbf{left congruent}) to $b$ modulo $H$, denoted as $a\equiv_r b\pmod{H}$ (resp. $a\equiv_l b\pmod{H}$), if
    $$ ab^{-1}\in H \quad (\text{resp. } a^{-1}b\in H). $$
\end{Df}

\begin{Rmk}{}
    \begin{compactenum}
        \item The concept of right (and left) congruence comes from the congruence in number theory. \textcolor{Th}{In number theory, the congruence modulo $n$ is both a left congruence and a right congruence modulo $n\mathbb{Z}$, in the group $G = (\mathbb{Z}, +)$.}
        \item Unlike the literal meaning, \textcolor{Th}{$a\equiv_r b\pmod{H}$ is not equivalent to $b\equiv_l a\pmod{H}$}. 
        \item Obviously, \textcolor{Th}{in Abelian groups, the left and right congruences coincides. But this is not true in general.} 
    \end{compactenum}
\end{Rmk}

\begin{Th}{Th1.9.1 (the right (resp. left) congruence is an equivalence relation)}
    Let $G$ be a group and $H<G$. Then:
    \begin{compactenum}
        \item The right (resp. left) congruence modulo $H$ is an equivalence relation on $G$.
        \item The equivalence class of $a\in G$ under the right (resp. left) congruence modulo $H$ is $Ha \triangleq \{ha: h\in H\}$ (resp. $aH \triangleq \{ah: h\in H\}$).
        \item For any $a\in G$, $|Ha| = |aH| = |H|$.
    \end{compactenum}
    \textcolor{Df}{And we call $Ha$ (resp. $aH$) a \textbf{right coset} (resp. \textbf{left coset}) of $H$ in $G$.}
    \tcblower
    \textit{Pf}:
    \begin{compactenum}
        \item Obvious.
        \item The verification is obvious. About how to derive what the equivalence class is, we think like this:
        $$ \text{The equivalence class of } a = \{b\in G: ab^{-1}\in H\} = A(a, H), $$
        that is, the equivalence class of $a$ should be a set only depends on $a$ and $H$. How can a $b$ makes $ab^{-1}\in H$? Since we have little information about how $a$ is related to $H$ (that is, we can not expect whether or not $a\in H$), we can only let $b$ first eliminate the $a$ in $ab^{-1}$, and then let the product fall into $H$. Thus $b^{-1} = a^{-1}h$ for some $h\in H$ is suitable, namely, $b = ha$ for some $h\in H$. Thus $A(a, H) = Ha$.
        \item Obvious, as we can immediately construct a bijection from $H$ to $Ha$:
        $$ h\mapsto ha. $$
    \end{compactenum}
\end{Th}

\begin{Th}{Th1.9.2}
    Suppose $G$ is a group and $H<G$. Then:
    \begin{compactenum}
        \item $G = \bigcup_{a\in G} Ha = \bigcup_{a\in G} aH$;
        \item Let $\mathcal{R} = \{Ha: a\in G\}$ and $\mathcal{L} = \{aH: a\in G\}$. Then $|\mathcal{R}| = |\mathcal{L}|$. 
    \end{compactenum}
    \tcblower
    \textit{Pf}:
    \begin{compactenum}
        \item Obvious.
        \item Just verify the map $Ha\mapsto a^{-1}H$ is a bijection from $\mathcal{R}$ to $\mathcal{L}$.
    \end{compactenum}
    \textcolor{P}{\textit{Thoughtfully}: To verify $|\mathcal{R}| = |\mathcal{L}|$, the map is constructed as $Ha\mapsto a^{-1}H$, but not $Ha\mapsto aH$. This is to make the map well-defined.}
\end{Th}

\begin{Df}{Df1.9.3 (index of subgroup, and representatives of cosets)}
    Let $G$ be a group and $H<G$. Then 
    \begin{compactenum}
        \item The \textbf{index} of $H$ in $G$, denoted as $[G:H]$, is the cardinality of the family of right cosets of $H$ in $G$ (recall the Th \{, ID: 1.9.2\}, we have $[G:H] = |\mathcal{R}| = |\mathcal{L}|$ of course).
        \item A \textbf{complete set of right (resp. left) coset representatives} of $H$ in $G$ is a set contains and only contains exactly one element from each right (resp. left) coset of $H$ in $G$. This definition is well-defined according to the axiom of choice.
    \end{compactenum}
\end{Df}

\begin{Th}{Th1.9.4 (the transitivity of the index)}
    Let $G$ be a group and $K<H<G$. Then
    $$ [G:K] = [G:H][H:K]. $$
    \tcblower
    \textcolor{P}{\textit{Analytically}: To prove the product of cardinalities equals another cardinality, we can only construct a bijection from a Cartisian product to another set.} \\
    \textit{Pf}: Consider the map $f: \{Kh\}\times\{Hg\}\to \{Kg\}$ defined as
    $$ (Kh, Hg)\mapsto Khg. $$
    We have to constrain the values of $h$ and $g$. Actually, we specify that $I$ (resp. $J$) is a complete set of right coset representatives of $H$ in $G$ (resp. of $K$ in $H$), and thus
    $$ |I| = [G:H], \quad |J| = [H:K], $$
    and the map $f$ is actually defined as 
    $$ (Kh, Hg)\mapsto Khg \quad \text{for } h\in I, g\in J. $$
    \begin{compactenum}
        \item $f$ is well-defined, as the specification of $I$ and $J$ solves the ambiguity of the product $Khg$.
        \item $f$ is injective. If $Khg = Kh^\prime g^\prime$, then $hgg^{\prime-1}h^{\prime-1}\in K\subseteq H$, and thus (by cancelling $h$ and $h^{\prime-1}$) $gg^{\prime-1}\in H$. Since $g, g^\prime\in J$, we have $g = g^\prime$. Then $K\ni hgg^{\prime-1}h^{\prime-1} = hgg^{-1}h^{\prime-1} = hh^{\prime-1}$, and thus $h = h^\prime$, and thus $(Kh, Hg) = (Kh^\prime, Hg^\prime)$.
        \item $f$ is surjective. For $Kg^\prime\in\{Kg\}$, we choose a $g\in J$ s.t. $Hg = Hg^\prime$. Then $g^\prime g^{-1}\in H$, and thus $g^\prime g^{-1} = h^\prime \in Kh$ for some $h\in I$, and thus $g^\prime g^{-1} = kh$ for some $k\in K$, and thus
        $$ Kg^\prime = Khg. $$
    \end{compactenum}
\end{Th}

\begin{Th}{Clry1.9.4.1 (Lagrange theorem)}
    Let $G$ be a group and $H<G$. Then
    $$ |G| = [G:H]|H|. $$
    \tcblower
    \textit{Pf}: Obvious, as we let $K = \{1\}$ in the Th \{, ID: 1.9.4\}.
\end{Th}

\begin{Df}{Df1.10 ($HK$)}
    Let $G$ be a group and $H, K\subseteq G$. Then the \textbf{product} of $H$ and $K$ is defined as
    $$ HK \triangleq \{hk: h\in H, k\in K\}. $$
    \textcolor{Th}{In particular, $\varnothing K = H\varnothing = \varnothing$ by the set theory.}
\end{Df}

\begin{Th}{Th1.10.1 (properties of $HK$)}
    Let $G$ be a group.
    \begin{compactenum}
        \item (Cosets) For $H<G$, we have $Ha = H\{a\}$ and $aH = \{a\}H$;
        \item (Associative) For $A, B, C\subseteq G$, we have $(AB)C = A(BC)$;
        \item ((Two-sided) Distributive over Union) Suppose $A\subseteq G$ and $\{B_i: i\in I\}$ is a family of subsets $B_i$ of $G$. Then
        $$ A\Big(\bigcup_{i\in I} B_i\Big) = \bigcup_{i\in I} (AB_i), \quad \Big(\bigcup_{i\in I} B_i\Big)A = \bigcup_{i\in I} (B_iA). $$
        \item (Absorption) If $K<H<G$, then $HK = H$.
        \item (Subgroup) For $H, K<G$, their product $HK$ is not necessarily a subgroup of $G$. But we have
        $$ HK < G \quad \Longleftrightarrow \quad HK = KH \quad \Longleftrightarrow \quad KH < G. $$
    \end{compactenum}
    \tcblower
    \textit{Pf}:
    \begin{compactenum}
        \item Obvious.
        \item Obvious.
        \item Obvious.
        \item Obvious.
        \item The second equivalence can be implied by the first one. Now prove the ``$\Rightarrow$'' of the first equivalence. We have:
        $$ \begin{aligned}
            & h_1k_1 k_2^{-1}h_2^{-1} \in HK \quad (\text{by Th \{, ID: 1.7.1\}}) \\
            \Longrightarrow \quad & h_1k_1 k_2^{-1}h_2^{-1} = h_3k_3 \\
            \Longrightarrow \quad & (k_1^{-1}h_1^{-1})^{-1}k_2^{-1}h_2^{-1} = h_3k_3 = (h_4k_4)^{-1} = k_4^{-1}h_4^{-1} = k_5h_5 \in KH \\
            \Longrightarrow \quad & (k_1h_1)^{-1}(k_2h_2) \in KH \quad (\text{replace } h_1, k_1, h_2, k_2 \text{ by their inverses}) \\
            \Longrightarrow \quad & KH < G \quad (\text{by Th \{, ID: 1.7.1\}}).
        \end{aligned} $$
        Then we can show that $HK$ and $KH$ contain each other, one direction of which is given by:
        $$ \forall\; hk = (k^{-1}h^{-1})^{-1} = (k^\prime h^\prime)^{-1} = k^{\prime\prime} h^{\prime\prime} \in KH. $$
        Then for the $\Leftarrow$ of the first equivalence, we have:
        $$ \forall\; h_1k_1(h_2k_2)^{-1} = h_1(k_1k_2^{-1})h_2^{-1} = h_1(k_3h_2^{-1}) = (h_1h_3)k_4 = h_4k_4 \in KH $$
    \end{compactenum}
\end{Th}

\begin{Th}{Th1.10.2 (cardinalities involving $HK$)}
    Let $G$ be a group and $H, K<G$. Then
    \begin{compactenum}
        \item $|HK|\cdot |H\cap K| = |H|\cdot |K|$.
        \item $[H:H\cap K] \leq [G:K]$. If $[G:K]$ is finite, then the equality holds iff $G = HK$ (resp. $G = KH$).
        \item $[G:H\cap K] \leq [G:H][G:K]$. If $[G:H]$ and $[G:K]$ are both finite, then the equality holds iff $G = HK$ (resp. $G = KH$).
    \end{compactenum}
    \tcblower
    \textit{Pf}:
    \begin{compactenum}
        \item Denote $C = H\cap K$, then $C<H, K$, and thus by the Lagrange theorem we have $|K| = [K:C]\cdot |C|$, and thus:
        $$ |H|\cdot |K| = |H|\cdot [K:C]\cdot |C|, $$
        and thus it suffices to show that $|H|\cdot [K:C] = |HK|$. Now consider showing this map is bijective: $f: H\times I \to HK$ defined as
        $$ (h, k)\mapsto hk, $$
        where $I$ is a complete set of right coset representatives of $C$ in $K$. If $h_1k_1 = h_2k_2$, then $k_1k_2^{-1} = h_1^{-1}h_2\in H$, and thus $k_1k_2^{-1} \in H\cap K = C$, and thus $k_1 = k_2$, and thus $h_1 = h_2$, and thus $f$ is injective. For surjectivity, we affirm it by the fact that $HK = \bigcup_{k\in I} Hk$. Actually, since $K = \bigcup_{k\in I} Ck$, we can multiply $H$ on both sides to get the desired result, using the absorption and distributive properties of subsets product.
        \item Consider showing the map $f: A\to B$ is injective, where
        $$ A = \{(H\cap K)h: h\in H\}, \quad B = \{Kg: g\in G\} \quad\text{and}\quad f((H\cap K)h) = Kh. $$
        This is obvious. If the cardinalities are finite, then the equality holds iff $|A| = |B|$, and thus iff $f$ is surjective (note that the second ``iff'' is not true if the cardinalities are infinite). Then $f$ is surjective 
        $$ \begin{aligned}
            \Longleftrightarrow \quad & \forall g\in G,\; \exists h\in H,\; Kg = Kh, \\
            \Longleftrightarrow \quad & \forall g\in G,\; \exists h\in H,\; gh^{-1}\in K, \\
            \Longleftrightarrow \quad & \forall g\in G,\; \exists h\in H,\; \exists k\in K,\; gh^{-1} = k, \\
            \Longleftrightarrow \quad & \forall g\in G,\; \exists h\in H,\; \exists k\in K,\; g = hk, \\
            \Longleftrightarrow \quad & \forall g\in G,\; g\in HK \\
            \Longleftrightarrow \quad & G = HK.
        \end{aligned} $$
        And by the Th \{, ID: 1.10.1\}, we know that if $G = HK$, then $HK < G$ and thus $KH = HK = G$.
        \item We have:
        $$ \begin{aligned}\relax % \relax is used to prevent the warning of "Bracket group [xxx] at formula start!"
            [G:H\cap K] &= [G:H][H:H\cap K] \quad (\text{by the Th \{, ID: 1.9.4\}}) \\
            &\leq [G:H][G:K] \quad (\text{by 2.}) 
        \end{aligned} $$
    \end{compactenum}
\end{Th}

\begin{Df}{Df1.11 (normal subgroup)}
    Let $G$ be a group and $N<G$. Then $N$ is called a \textbf{normal subgroup} of $G$ (or briefly we say, $N$ is \textbf{normal} in $G$), denoted as $N\vartriangleleft G$, if the left and right congruences modulo $N$ coincide.
\end{Df}

\begin{Rmk}{}
    This definition says that $N$ is said to be normal if for all $a, b\in G$:
    $$ a \equiv_l b\pmod{N} \quad \Longleftrightarrow \quad a \equiv_r b\pmod{N}. $$
    \textcolor{Df}{For $N<G$ but $N$ is not normal, we write $N\notnles G$.}
\end{Rmk}

\begin{Th}{Th1.11.1 (equivalent definitions of normal subgroup)}
    Let $G$ be a group and $N<G$. Then the following statements are equivalent:
    \begin{compactenum}
        \item $N\vartriangleleft G$;
        \item Every left coset of $N$ in $G$ is a right coset of $N$ in $G$;
        \item Every right coset of $N$ in $G$ is a left coset of $N$ in $G$;
        \item $Ng = gN$ for all $g\in G$;
        \item $gNg^{-1} = N$ for all $g\in G$;
        \item $gNg^{-1} \subseteq N$ for all $g\in G$.
    \end{compactenum}
    \tcblower
    \textit{Pf}: Obvious.
\end{Th}

\begin{Rmk}{}
    Among all the equivalent statements in this theorem, only the last two bring us something new. Since the statement 6 is literally weaker than the statement 5, it is more convenient for judging whether a subgroup is normal. \\
    Now for some examples of the normality in a group $G$:
    \begin{compactenum}
        \item \textcolor{Th}{$\{1\}, G\;\nles\; G$};
        \item \textcolor{Th}{If $G$ is Abelian, then every subgroup of $G$ is normal in $G$};
        \item \textcolor{Th}{If $N\nles G$ and $N < H < G$, then $N\nles H$};
        \item \textcolor{Th}{In the symmetric group $\mathbb{S}_3$, recall that such stuff: $(1,2)$, $(1,2,3)$, are some cycles. Then we have
        $$ \langle (1,2) \rangle \notnles \,\mathbb{S}_3, \quad \langle (1,2,3) \rangle \nles \,\mathbb{S}_3. $$}
        \item \textcolor{Th}{In gereral the transitivity of normality is not true, that is, 
        $$ \left.\begin{matrix}
            K \nles H \; \\
            H \nles G \;
        \end{matrix}\right\} \quad\nRightarrow\quad K \nles G. $$}
        \item Obviously, \textcolor{Th}{every subgroup of index 2 is normal.}
    \end{compactenum}
    \textcolor{Df}{Also, the subset of the form $gNg^{-1}$, for $N<G$, (clearly $gNg^{-1} < G$) is called a \textbf{conjugate} of $N$ in $G$.} Thus \textcolor{Th}{the normality of $N$ is equivalent to the fact that every conjugate of $N$ is $N$ itself.}
\end{Rmk}

\begin{Th}{Th1.11.2 ($N, K$) (What happens to the subgroup-union and the subgroup-intersection with normality involved?)}
    Let $G$ be a group and $K<G$, $N\nles G$. Then
    \begin{compactenum}
        \item $N\cap K \nles K$;
        \item $N\nles\, N\vee K = NK = KN < G$; (When discussing the union, we are instead discussing the generated subgroup of the union, as the union itself is not a subgroup in general.)
    \end{compactenum}
    \tcblower
    \textit{Pf}: 
    \begin{compactenum}
        \item Obvious.
        \item The ``$\nles$'' is obvious. Now for the rest of the statement, we already have
        $$ N, K \subseteq N\cup K \subseteq NK \subseteq \langle N\cup K\rangle = N\vee K. $$
        We now claim that $NK < G$, and then by the minimality of $\langle N\cup K\rangle$ we have $NK = \langle N\cup K\rangle = N\vee K$, and by the Th \{, ID: 1.10.1\} we have $NK = KN$. Then verify the claim:
        $$ n_1k_1(n_2k_2)^{-1} = n_1k_1(k_2^{-1}n_2^{-1}) \xlongequal[]{kN = Nk} n_1(k_1n_3)k_2^{-1} \overset{\cdots}{=} (n_1n_4)(k_1k_2^{-1}) \in NK. $$ 
    \end{compactenum}
\end{Th}

\begin{Th}{Th1.11.3 ($n_1n_2 = n_2n_1$)}
    Let $G$ be a group. If $N_1, N_2\nles G$ and $N_1\cap N_2 = \{1\}$, then
    $$ n_1n_2 = n_2n_1, \quad \forall n_1\in N_1, n_2\in N_2. $$
    \tcblower
    \textit{Pf}: To ask this equality is to ask whether
    $$ n_1n_2n_1^{-1}n_2^{-1} = 1. $$
    This is true, as
    $$ N_2 \ni n_2^\prime n_2^{-1} = (n_1n_2n_1^{-1})n_2^{-1} = n_1(n_2n_1^{-1}n_2^{-1}) = n_1n_1^\prime \in N_1 $$
    and thus $n_1n_2n_1^{-1}n_2^{-1} \in N_1\cap N_2 = \{1\}$.
\end{Th}

\begin{Df}{Th1.11.4 (quotient group)}
    Let $G$ be a group and $N\nles G$. \textcolor{Th}{Then the set
    $$ G/N \triangleq \{Ng: g\in G\} = \{gN: g\in G\} $$
    is a group under the subset product defined in Df \{, ID: 1.10\}. Actually we have:
    $$ g_1N\cdot g_2N = g_1Ng_2N = g_1g_2N. $$}
    This group is called the \textbf{quotient group} (or \textbf{factor group}) of $G$ by $N$.
\end{Df}

\begin{Rmk}{}
    \begin{compactenum}
        \item It is trivial to check that $G/N$ is a group, \textcolor{Th}{with the identity $N$ and the inverse of $gN$ being $g^{-1}N$}. And we should notice that it is the normality of $N$ that makes sure that $g_1Ng_2N = g_1g_2N$.
        \item \textcolor{Th}{$|G/N| = [G:N]$}.
        \item \textcolor{Th}{$G/\{1\} = \{G\}$, $G/G = \{1\}$}.
        \item \textcolor{Th}{The notation $\mathbb{Z}/n\mathbb{Z}$ is of what it literally means: it is exactly the quotient group of $\mathbb{Z}$ by $n\mathbb{Z}$. Clearly, for $m,n\in\mathbb{N}$, we have
        $$ m\mathbb{Z}<n\mathbb{Z} \quad \Longleftrightarrow \quad n|m. $$
        Also we have (assume $n\mid k\mid m$):
        $$ (n\mathbb{Z}/m\mathbb{Z})\Big/ (k\mathbb{Z}/m\mathbb{Z}) \simeq n\mathbb{Z}/k\mathbb{Z} $$
        (one can understand this by trying $(2\mathbb{Z}/12\mathbb{Z})\big/ (4\mathbb{Z}/12\mathbb{Z})$).}
    \end{compactenum}
\end{Rmk}

\begin{Th}{Th1.12.1 (the canonical epimorphism)}
    The kernel of a group homomorphism is a normal subgroup (of the domain); and conversely, a normal subgroup is the kernel of a group homomorphism (from the ``supgroup'').
    \tcblower
    \textit{Pf}: 
    \begin{compactenum}
        \item Let $f: G\to H$ be a group homomorphism. Then we can verify that $\Ker f \,\nles\, G$.
        \item Let $N\nles G$. Then the map $\pi_N^G: G\to G/N$ defined as $\pi_N^G(g) = gN$ is a group homomorphism, with $\Ker \pi_N^G \,=\, N$.
    \end{compactenum}
\end{Th}

\begin{Rmk}{}
    \begin{compactenum}
        \item \textcolor{Th}{The homomorphism $\pi_N^G$ defined here is epimorphic, } \textcolor{Df}{and we call $\pi_N^G$ (or $\pi_N$, $\pi$ for short if it is clear from the context) the \textbf{canonical epimorphism} (or the \textbf{canonical projection}) (or \textbf{quotient map}) (or \textbf{projection map}) from $G$ to $G/N$.}
        \item The canonical epimorphism is a common homomorphism from $G$ to $G/N$, and thus when a epimorphism $G\twoheadrightarrow H$ is needed to be constructed, we usually consider $\pi_N^G$.
        \item Due to the introduction of normal subgroups, and then of the quotient groups, we have more ways to construct new groups. Then for these new groups the study of their correspondences (namely, the homomorphisms between them) is important, which leads to the following several theorems.
        \item \textcolor{Df}{We now introduce a less formal term, that is, we often \textbf{induce} a map $\overline{f}$ from a given map $f$, by defining $\overline{f}$ based on $f$.} This is less formal, as $\overline{f}$ can be defined in different ways.
    \end{compactenum}
\end{Rmk}

\begin{Th}{Ex1.12.2.-1}
    Suppose $f: G\to H$ is a group homomorphism. Given a normal subgroup $N\nles G$, please induce a group homomorphism $\overline{f}: G/N\to H$. To do this, please figure out what other conditions are needed, and please study the (1) kernel, (2) image, (3) mono(morphism) and (4) epi(morphism) of $\overline{f}$.
    \tcblower
    \textit{Solution}: To induce $\overline{f}$ is to realize what exactly the dashed arrow in the following commutative diagram is for, relying on the existing arrows:
    $$ \begin{tikzcd}
        G \arrow[rr, "f"] \arrow[rd, two heads] & & H \\
        & G/N \arrow[ru, dashed, "\overline{f}"'] &
    \end{tikzcd} $$
    Then it is natural to define $\overline{f}$ as $\overline{f} = f\circ(\pi_N^G)^{-1}$ ($\pi_N^G$ is actually not invertible in general, we is just introducing the idea). Then we define
    $$ \overline{f}(gN) = f(g), \quad \forall g\in G. $$
    Then:
    \begin{compactenum}
        \item Is $\overline{f}$ well-defined? Does $g_1N = g_2N$ imply $f(g_1) = f(g_2)$? Does $g_1g_2^{-1}\in N$ imply $g_1g_2^{-1}\in\Ker f$? \textcolor{P}{This suffices to that $N<\Ker f$}, which should be a premise before we define $\overline{f}$.
        \item Is $\overline{f}$ a homomorphism? Does $\overline{f}(abN) = f(ab) = f(a)f(b) = \overline{f}(aN)\overline{f}(bN)$? Yes.
        \item $ \Ker \overline{f} = \{gN\in G/N: \overline{f}(gN) = 1\} = \{gN\in G/N: f(g) = 1\} = \{gN\in G/N: g\in\Ker f\} = (\Ker f)\,/\,N $.
        \item $ \Ima \overline{f} = \Ima f $.
        \item $\overline{f}$ is mono $\Longleftrightarrow$ $\Ker \overline{f} = \{1\}$ $\Longleftrightarrow$ $(\Ker f)\,/\,N = \{N\}$ $\Longleftrightarrow$ $\Ker f = N$ (by taking union on the both sides).
        \item $\overline{f}$ is epi $\Longleftrightarrow$ $f$ is epi. 
    \end{compactenum}
\end{Th}

\begin{Th}{Clry\,1.12.2 ($f: G\to H$, $\overline{f}: G/N\to H$)}
    Suppose $f: G\to H$ is a group homomorphism. If $N\nles G$ and $N<\Ker f$, then the map $\overline{f}: G/N\to H$ defined as $\overline{f}(gN) = f(g)$ is a group homomorphism, with:
    \begin{compactenum}
        \item $\Ker \overline{f} = (\Ker f)\,/\,N$;
        \item $\Ima \overline{f} = \Ima f$;
        \item $\overline{f}$ is a monomorphism iff $N = \Ker f$;
        \item $\overline{f}$ is an epimorphism iff $f$ is an epimorphism.
    \end{compactenum}
    \tcblower
    \textit{Pf}: This is just the Ex \{, ID: 1.12.2.-1\}.
\end{Th}

\begin{Th}{Ex1.12.3.-1}
    For the group homomorphism $f: G\to H$, please find an group isomorphism by inducing $f$ as $\overline{f}$. To do this, please figure out what other conditions are needed.
    \tcblower
    \textit{Solution}: Still searching idea by first drawing the commutative diagram. Since we have no other information, we should consider $\Ker f$ and $\Ima f$ besides the given groups $G$ and $H$. Notice that $\Ker f\nles G$, and $\Ima f$ makes $f$ epi, so we can think of:
    $$ \begin{tikzcd}
        G \arrow[dd, two heads] \arrow[rr, "f"] \arrow[rd, two heads] & & H \\
        & \Ima f \arrow[ru, hook] & \\
        G/\Ker f \arrow[ru, dashed, "\overline{f}"'] & &
    \end{tikzcd} $$
    where we expect the dashed arrow $\overline{f}$ to be an isomorphism. It is natural to define $\overline{f}: G/\Ker f\to\Ima f$ as:
    $$ \overline{f}(g\,\Ker f) = f(g), \quad \forall g\in G. $$
    Then $\overline{f}$ is actually induced by $f: G\to \Ima f$ in the sense of the one in Clry \{, ID: 1.12.2\}, where we can know that $\overline{f}$ is an isomorphism, and we do not need extra conditions to complete this exercise.
\end{Th}

\begin{Th}{Clry 1.12.3 (the 1st isomorphism theorem) ($G/\Ker f \simeq \Ima f$)}
    For a group homomorphism $f: G\to H$, we have
    $$ G/\Ker f \simeq \Ima f. $$
    \tcblower
    \textit{Pf}: This is just the Ex \{, ID: 1.12.3.-1\}. And the isomorphism desired is $g\Ker f \mapsto f(g)$.
\end{Th}

\begin{Th}{Ex1.12.4.-1}
    Suppose $f: G\to H$ is a group homomorphism and $N\nles G$, $M\nles H$. Please induce a group homomorphism $\overline{f}: G/N\to H/M$. To do this, please figure out what other conditions are needed, and please study the (1) kernel, (2) image, (3) mono and (4) epi of $\overline{f}$.
    \tcblower
    \textit{Solution}: To induce $\overline{f}$ (which involves the quotient spaces), we should first bridge the original groups and their quotient spaces by the canonical epimorphisms. Then from the following commutative diagram
    $$ \begin{tikzcd}
        G \arrow[r, "f"] \arrow[d, two heads] & H \arrow[d, two heads] \\
        G/N \arrow[r, dashed, "\overline{f}"] & H/M
    \end{tikzcd} $$
    it is natural to think of defining $\overline{f}$ as $\overline{f} = \pi_M^H\circ f\circ(\pi_N^G)^{-1}$. Is $(\pi_N^G)^{-1}$ here legal? Although $\pi_N^G$ is not invertible in general, we can resort to the Ex \{, ID: 1.12.2.-1\}, which inspires us to define $\overline{f}$ as:
    $$ \begin{aligned}
        \alpha &= \pi_M^H\circ f, \\
        \overline{f} &= \overline{\alpha}, 
    \end{aligned} $$
    where we induce $\alpha$ to $\overline{\alpha}$ in the way of Ex \{, ID: 1.12.2.-1\}. According to that exercise, we should add an extra condition that $N<\Ker\alpha$. Since 
    $$ \Ker\alpha = \{g\in G: \pi_M^H(f(g)) = 1\} = \{g\in G: f(g)\in M\} = f^{-1}(M), $$
    this extra condition is equivalent to $N<f^{-1}(M)$, namely, \textcolor{P}{$f(N)<M$}. With this premise holding, $\overline{f}$ is well-defined as:
    $$ \overline{f}(gN) = f(g)M, \quad \forall g\in G, $$
    with:
    \begin{compactenum}
        \item $\Ker \overline{f} = (\Ker \alpha)\,/\,N = f^{-1}(M)\,/\,N$;
        \item $\Ima \overline{f} = \Ima \alpha = \{f(g)M: g\in G\} = \{hM: h\in\Ima f\}$;
        \item $\overline{f}$ is mono $\Longleftrightarrow$ $\overline{\alpha}$ is mono $\Longleftrightarrow$ $\Ker\alpha = N$ $\Longleftrightarrow$ $f^{-1}(M) = N$ $\Longleftrightarrow$ $f^{-1}(M) < N$;
        \item $\overline{f}$ is epi $\Longleftrightarrow$ $\alpha$ is epi $\Longleftrightarrow$ $\Ima\alpha = H/M$. Since $\Ima\alpha = \{hM: h\in\Ima f\} < \{hM: h\in H\} = H/M$, the ``$<$'' becomes ``$=$'' iff:
        $$ \bigcup_{h\,\in\,\Ima f} hM = \bigcup_{h\,\in\, H} hM, $$
        which is equivalent to
        $$ H = \bigcup_{h\,\in\, \Ima f} hM = \bigcup_{h\,\in\, \Ima f} \{h\} M = \left(\bigcup_{h\,\in\, \Ima f} \{h\}\right) M = (\Ima f) M. $$
        Thus, $\overline{f}$ is epi $\Longleftrightarrow$ $(\Ima f) M = H$.
    \end{compactenum}
\end{Th}

\begin{Th}{Clry 1.12.4 ($f: G\to H$, $\overline{f}: G/N\to H/M$)}
    Suppose $f: G\to H$ is a group homomorphism and $N\nles G$, $M\nles H$. If $f(N)<M$, then the map $\overline{f}: G/N\to H/M$ defined as $\overline{f}(gN) = f(g)M$ is a group homomorphism, with:
    \begin{compactenum}
        \item $\Ker \overline{f} = f^{-1}(M)\,/\,N$;
        \item $\Ima \overline{f} = \{hM: h\in\Ima f\}$;
        \item $\overline{f}$ is a monomorphism iff $f^{-1}(M) < N$;
        \item $\overline{f}$ is an epimorphism iff $(\Ima f) M = H$.
    \end{compactenum}
    \tcblower
    \textit{Pf}: This is just the Ex \{, ID: 1.12.4.-1\}.
\end{Th}

\begin{Th}{Ex1.12.5.-1}
    Let $G$ be a group and $H, K< G$. Please find an isomorphism (according to what is given, that is, consider an isomorphism between two of these groups: $H$, $K$, $H\cap K$, $H\vee K$, $H/(H\cap K)$, $K/(H\cap K)$ and all others you can think of). To do this, please figure out what other conditions are needed.
    \tcblower
    \textit{Solution}: If it is not that clear that which two groups (among all that we can think if) are isomorphic, we can first finding an equality of cardinalities, as an isomorphism is first a bijection. Then for this what comes to our mind is that
    $$ |HK|\cdot |H\cap K|= |H|\cdot |K|. $$
    As we now know little about the Cartisian product of groups, we consider dividing the two sides of the equality instead, by
    $$ \frac{|HK|}{|H|} = \frac{|K|}{|H\cap K|}, $$
    and interpreting the division of cardinalities as the cardinality of the quotient group. Then it is a problem that $HK$ is not a subgroup. Of course this can be solved by assuming that one of $H$ and $K$ is normal in $G$. Then by the Th \{, ID: 1.11.2\}, we should let \textcolor{P}{$N\nles G$} instead of $K\nles G$, so that the problem becomes comes to that to find a homomorphism between two quotient groups, which inspires us to apply the Clry \{, ID: 1.12.4\} by considering $f: K\to NK$ as 
    $$ f(k) = k. $$
    And it works for this exercise. Then the isomorphism desired is $k(N\cap K)\mapsto kN$.
\end{Th}

\begin{Th}{Clry 1.12.5 (the 2nd isomorphism theorem) ($ K/(N\cap K) \simeq NK/N $)}
    Let $G$ be a group and $N\nles G$, $K<G$. Then
    $$ K/(N\cap K) \simeq NK/N. $$
    \tcblower
    \textit{Pf}: This is just the Ex \{, ID: 1.12.5.-1\}. And the isomorphism desired is $k(N\cap K)\mapsto kN$.
\end{Th}

\begin{Th}{Ex1.12.6.-1}
    Recall that the equality
    $$ (n\mathbb{Z}/m\mathbb{Z})\Big/ (k\mathbb{Z}/m\mathbb{Z}) \simeq n\mathbb{Z}/k\mathbb{Z} $$
    holds, provided that $n\mid k\mid m$. Can this equality be generalized? To do this, please figure out what other conditions are needed.
    \tcblower
    \textit{Solution}: To generalize this equality is to ask whether the equality
    $$ (G/K)\Big/ (H/K) \simeq (G/H) $$
    hold, provided that \textcolor{P}{$K\nles G$, $H\nles G$ and $K\nles H$}. Now is it true that $(H/K)\nles (G/K)$? Obviously yes, and then we know our ways, which is to apply the Clry \{, ID: 1.12.4\}, by considering the following commutative diagram:
    $$ \begin{tikzcd}
        G \arrow[r, two heads] \arrow[d, two heads] & G/K \arrow[d, two heads]\\
        G/H \arrow[r, dashed] & (G/K)\Big/ (H/K) 
    \end{tikzcd} $$
    and letting $f = \pi_K^G$. Then it works for this exercise, and the isomorphism desired is $gH\mapsto \{ghK: h\in H\}$.
\end{Th}

\begin{Th}{Clry 1.12.6 (the 3rd isomorphism theorem) ($ (G/K)\Big/ (H/K) \simeq (G/H) $)}
    Let $G$ be a group. If $K\nles G$, $H\nles G$ and $K\nles H$, then $(H/K)\nles (G/K)$, and
    $$ (G/K)\Big/ (H/K) \simeq (G/H). $$ 
    \tcblower
    \textit{Pf}: This is just the Ex \{, ID: 1.12.6.-1\}, and the isomorphism desired is $gH\mapsto \{ghK: h\in H\}$.
\end{Th}

\begin{Th}{Ex1.12.7.-1}
    Let $f: G\to H$ be a group homomorphism.
    \begin{compactenum}
        \item Does $f(N)\nles H$ hold for any $N\nles G$? What about that $f^{-1}(M) \nles G$ hold for any $M\nles H$?
        \item Consider the map $\varphi$ defined as $K\mapsto f(K)$ for $K<G$. Please study the (1) range, (2) injectivity and (3) surjectivity of $\varphi$.
    \end{compactenum}
    To answer these questions, figure out what other conditions are needed.
    \tcblower
    \textit{Solution}:
    \begin{compactenum}
        \item Obviously $M\nles H$ implies $f^{-1}(M) \nles G$. For the former, $f(N)\nles H$, it suffices to that $f$ is an epimorphism.
        \item Now we study the map $\varphi$.
        \begin{compactenum}
            \item For the range of $\varphi$, we first ask whether $\varphi$ is surjective. Then for $L<H$, is there any $K<G$ s.t. $f(K) = L$? For this to be true, we are first worried about that $f$ is not epimorphic, which would result in that some $L$ beyond the $f(G)$ may fail to have a pre-image. Thus we assume that \textcolor{P}{$f$ is an epimorphism}. Then with this new assumption, $\varphi$ is surjective, since $L = f(f^{-1}(L)) = \varphi(f^{-1}(L))$ given $L<H$.
            \item Is $\varphi$ injective? Does $f(K_1) = f(K_2)$ imply $K_1 = K_2$? This is not so clear, but we notice that $f(K) = f(f^{-1}(f(K)))$, which maybe fail to imply $K = f^{-1}(f(K))$. Actually, the injectivity of $\varphi$ is equivalent to the fact that $K = f^{-1}(f(K))$ holds for every $K < G$. Now we inspect this condition. Clearly $K<f^{-1}(f(K))$. For the other direction, we can easily obtain a sufficient, but extra condition: \textcolor{P}{$\Ker f<K$}. Thus consider restricting $\varphi$ on the set $\mathrm{S}_f(G)\triangleq \{K<G: \,\Ker f<K\}$ we get an injection. 
            \item $\varphi$ is surjective, but how about the restriction of $\varphi$ on $\mathrm{S}_f(G)$? Does the surjectivity of $f$ still imply the surjectivity of (the restricted) $\varphi$? Yes, if $f$ is an epimorphism, then we can still have $L = f(f^{-1}(L)) = \varphi(f^{-1}(L))$ for $L<H$, since always $\Ker f<f^{-1}(L)$.
        \end{compactenum}
    \end{compactenum}
\end{Th}

\begin{Th}{Th1.12.7 ($N\mapsto f(N)$, $K\mapsto f(K)$)}
    Let $f: G\to H$ be a group homomorphism. Then:
    \begin{compactenum}
        \item $M\nles H \implies f^{-1}(M) \nles G$.
        \item If $f$ is an epimorphism, then $N\nles G \implies f(N)\nles H$. 
        \item Consider the map $\varphi: \mathrm{S}(G)\to\mathrm{S}(H)$ where $\mathrm{S}(G)$ (similar to $\mathrm{S}(H)$) is the family of the subgroups of $G$ defined as $K\mapsto f(K)$ for $K<G$. If $f$ is an epimorphism, then $\varphi$ is a surjection.
        \item In the case of 3 (of course assume that $f$ is epimorphic), let $\mathrm{S}_f(G) = \{K\in\mathrm{S}(G): \,\Ker f<K\}$. Then 
        $$ \varphi: \mathrm{S}_f(G)\hooktwoheadrightarrow\mathrm{S}(H) $$
    \end{compactenum}
    \tcblower
    \textit{Pf}: This is just the Ex \{, ID: 1.12.7.-1\}
\end{Th}

\begin{Rmk}{}
    This theorem is useful when we are to study the subgroups of a new group, via group homomorphism. The next theorem is an example.
\end{Rmk}

\begin{Th}{Clry 1.12.8 (the subgroups of $G/N$)}
    Let $G$ be a group and $N\nles G$. Then
    \begin{compactenum}
        \item A subset of $G/N$ is a subgroup of $G/N$ iff it is of the form $K/N$, where $N<K<G$.
        \item A subgroup $K/N$ (where $N<K<G$) of $G/N$ is normal iff $K$ is normal in $G$.
    \end{compactenum}
    \tcblower
    \textit{Pf}: Obvious by applying the Th \{, ID: 1.12.7\} to the canonical epimorphism $G\twoheadrightarrow G/N$.
\end{Th}

\begin{Df}{Df1.13.1 (the parity of a permutation)}
    In $\mathbb{S}_n$ ($n\geq 2$), an \textbf{odd permutation} (resp. \textbf{even permutation}) is a permutation that can be expressed as a product of an odd (resp. even) number of transpositions. 
\end{Df}

\begin{Rmk}{}
    \textcolor{Th}{This definition is well-defined, that is, any permutation cannot be both odd and even.} Actually, this fact was discussed in the 8th chapter of the course 1, where we discussed the parity of a permutation before defining the determinants of square matrices. Then one can be clear that the concept of ``permutation'' is just the same as the one discussed there, and we had an analogue to ``transposition'' there, which is the ``swap'' defined in the Df \{course: 1, ID: 8.1.-1.1\}. Strictly a transposition is not a swap. A transposition exchanges the assigned elements, while a swap exchanges the assigned positions. But these two concepts correspond to each other:
    $$ S(\pmb{p}) = (p, q) \pmb{p} $$
    where $\pmb{p}$ is a permutation, $(p,q)$ is a transposition and $S$ is the swap operator. For $S$ and $(p,q)$ in this equality, if one is $\forall$, then the other is $\exists$. \\
    Therefore things become easy. By those blocks \{course: 1, ID: 8.1.-1.*\}, a permutation is odd (resp. even) $\Longleftrightarrow$ the permutation is a product of an odd (resp. even) number of transpositions $\Longleftrightarrow$ the number of reversals of the permutation is odd (resp. even). As the number of reversals cannot be both odd and even, so does the parity of a permutation. 
\end{Rmk}

\begin{Df}{Df1.13.2 (alternating group)}
    For $n\geq 2$, the \textbf{alternating group on} $n$ \textbf{letters} (or, the \textbf{alternating group of degree} $n$), $\mathbb{A}_n$, is the subset of $\mathbb{S}_n$ consisting of all even permutations. \textcolor{Th}{It is obvious that $\mathbb{A}_n$ is a normal subgroup of $\mathbb{S}_n$, with index 2.} 
\end{Df}

\begin{Rmk}{}
    Clearly \textcolor{Th}{$\mathbb{S}_n = \mathbb{A}_n\sqcup (1,2)\mathbb{A}_n$, where $(1,2)\mathbb{A}_n = \{f\in\mathbb{S}_n: f \text{ is odd}\}$}. And of course \textcolor{Th}{a subgroup of index 2 is always normal}. 
\end{Rmk}

\begin{Df}{Df1.14 (simple group)}
    A group is called \textbf{simple} if it has no proper normal subgroups.
\end{Df}

\begin{Th}{Ex1.14.1.-1 (simple Abelian group)}
    \textcolor{Th}{There are cases of simple non-Abelian groups, such as $\mathbb{A}_3$; and cases of non-simple non-Abelian groups, such as $\langle r\rangle\nles\mathbb{D}_4 = \{1, r, r^2, r^3, s, sr, sr^2, sr^3\}$.} But things seem much easier for Abelian groups, as every subgroup of an Abelian group is normal. Then what?
    \tcblower
    \textit{Solution}: An Abelian group $G$ is simple iff it has no proper subgroup. However, we have a easiest way to contruct a, in most of the case, proper subgroup of $G$, that is, $\langle a\rangle < G$. Thus to make $G$ simple, we must let it to be cyclic. Then, $G\simeq\mathbb{Z}$ or $G\simeq\mathbb{Z}/m\mathbb{Z}$. Since the former case is also impossible (for $G$ to be simple), $G$ must be (isomorphic to) $\mathbb{Z}/m\mathbb{Z}$. Since the subgroups of $\mathbb{Z}/m\mathbb{Z}$ are exactly those $\mathbb{Z}/k\mathbb{Z}$ with $k\mid m$, we get the answer.
\end{Th}

\begin{Th}{Th1.14.1 (simple Abelian groups are exactly those $\mathbb{Z}/p\mathbb{Z}$ with $p$ prime)}
    Simple Abelian groups are exactly those $\mathbb{Z}/p\mathbb{Z}$ with $p$ prime. Here $p$ can be $1$, which means $\mathbb{Z}/p\mathbb{Z}$ is the trivial group, simple of course.
    \tcblower
    \textit{Pf}: This is just the Ex \{, ID: 1.14.1.-1\}.
\end{Th}

\begin{Th}{Lma1.14.2.-2}
    Let $n\geq 3$, and $r,s\in\{1,\cdots, n\}$, $r\neq s$. Then 
    $$ \mathbb{A}_n = \langle \{\text{3-cycles}\} \rangle = \langle \{(rsk): k\neq r,s\} \rangle. $$
    \tcblower
    \textit{Pf}: See the lemma 6.11 in the chapter 1 of the reference book.
\end{Th}

\begin{Th}{Lma1.14.2.-1}
    Let $n\geq 3$. If a normal subgroup $N$ of $\mathbb{A}_n$ contains some 3-cycle, then $N = \mathbb{A}_n$.
    \tcblower
    \textit{Pf}: See the lemma 6.12 in the chapter 1 of the reference book.
\end{Th}

\begin{Th}{Th1.14.2 ($\mathbb{A}_n$ is simple iff $n\neq 4$)}
    The alternating group $\mathbb{A}_n$ is simple iff $n\neq 4$.
    \tcblower
    \textit{Pf}: See the theorem 6.10 in the chapter 1 of the reference book.
\end{Th}

\begin{Th}{Blocks from the P1 file}
\end{Th}

\begin{Th}{Blocks from the P2 file}
\end{Th}

\begin{Th}{Blocks from the P3 file}
\end{Th}

\end{document}