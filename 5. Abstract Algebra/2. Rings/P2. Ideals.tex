\documentclass{article}

    \usepackage{xcolor}
    \definecolor{pf}{rgb}{0.4,0.6,0.4}
    \usepackage[top=1in,bottom=1in, left=0.8in, right=0.8in]{geometry}
    \usepackage{setspace}
    \setstretch{1.2} 
    \setlength{\parindent}{0em}

    \usepackage{paralist}
    \usepackage{cancel}

    % \usepackage{ctex}
    \usepackage{amssymb}
    \usepackage{amsmath}
    \usepackage{extarrows}
    \usepackage{tikz-cd}

    \usepackage{tcolorbox}
    \definecolor{Df}{RGB}{0, 184, 148}
    \definecolor{Th}{RGB}{9, 132, 227}
    \definecolor{Rmk}{RGB}{215, 215, 219}
    \definecolor{P}{RGB}{154, 13, 225}
    \newtcolorbox{Df}[2][]{colbacktitle=Df, colback=white, title={\large\color{white}#2},fonttitle=\bfseries,#1}
    \newtcolorbox{Th}[2][]{colbacktitle=Th, colback=white, title={\large\color{white}#2},fonttitle=\bfseries,#1}
    \newtcolorbox{Rmk}[2][]{colbacktitle=Rmk, colback=white, title={\large\color{black}{Remarks}},fonttitle=\bfseries,#1}
    \newtcolorbox{Rmk_continued}[2][]{colbacktitle=Rmk, colback=white, title={\large\color{black}{Remarks (continued)}},fonttitle=\bfseries,#1}

    \title{\LARGE \textbf{Ideals}}
    \author{\large Jiawei Hu}

    % new commands for formula typying
    \newcommand{\lcm}{\text{lcm}}
    \newcommand{\cycl}{\text{cycl}}
    \newcommand{\nles}{\vartriangleleft}
    \newcommand{\notnles}{\ntriangleleft}
    \newcommand{\Ker}{\text{Ker}\,}
    \newcommand{\Ima}{\text{Im}\,}
    \newcommand{\hooktwoheadrightarrow}{%
        \hookrightarrow\mathrel{\mspace{-15mu}}\rightarrow}
    \newcommand{\End}{\text{End }}
    \newcommand{\Char}{\text{char}\,}

    % terms of rings, similar from those of groups
    \newcommand{\Giso}{\overset{\text{g}}{\simeq}} % group isomorphism
    \newcommand{\Riso}{\overset{\text{r}}{\simeq}} % ring isomorphism
    \newcommand{\subr}{\overset{\text{r}}{<}} % subring
    \newcommand{\ideal}{\overset{\text{r}}{\nles}} % ideal
    \newcommand{\cent}{\text{cent}\,} % center

\begin{document}
\maketitle

This is about the \textbf{Ideals}. By the way, we now pre-claim some commonly-used notations and terms:
\begin{Df}{Notations and Terms}
    \begin{compactenum}
        \item An agreement for the length of a list: if we write $a_1, \dots, a_n$, then we indicate that $n$ is finite and that $n\geq 1$; if we write $a_0, \dots, a_n$, then we indicate that $n$ is finite and that $n\geq 0$.
    \end{compactenum}
\end{Df}

Here is the \textbf{Quick Search} for this chapter:
\begin{Th}{Quick Search}
    \begin{compactdesc}
        \item (2\_P2.4.1.*) Basic concepts of ideals.
        \item (2\_P2.4.2.*) Ring isomorphism theorems.
        \item (2\_P2.4.3.*) Prime ideals and maximal ideals.
        \item (2\_P2.4.4.*) Direct product of rings.
    \end{compactdesc}
\end{Th}

Then with everything prepared, here we go. 

\begin{Df}{Df 2\_P2.4.1 (subrings and ideals)}
    Let $R$ be a ring. 
    \begin{compactenum}
        \item A subset $S$ of $R$ is called a \textbf{subring} of $R$ if $S$ is a ring under the addition and multiplication of $R$, denoted personally by $S\subr R$.
        \item A subring $I$ of $R$ is called a \textbf{left ideal} (resp. \textbf{right ideal}) if
        $$ rx\in I \qquad (\text{resp. }\; xr\in I) $$
        for all $r\in R$ and $x\in I$. If $I$ is both a left ideal and a right ideal of $R$, then $I$ is called an \textbf{ideal} of $R$, denoted personally by $I\ideal R$.
        \item The \textbf{center} of $R$, denoted personally by $\cent R$, is the set
        $$ \cent R = \{c\in R:\, cr = rc \,\text{ for all }\, r\in R\} $$ 
    \end{compactenum}
\end{Df}

\begin{Th}{Th 2\_P2.4.1.1 (judgement of subrings and ideals)}
    Let $(R, +, \times)$ be a ring.
    \begin{compactenum}
        \item A subset $S$ of $R$ is a subring of $R$ iff
        \begin{compactenum}
            \item $(S,+) < (R,+)$ (that is, $S$ is a subgroup of $R$ under the addition of $R$);
            \item $S$ is closed under ``$\times$''.
        \end{compactenum}
        \item A subset $I$ of $R$ is a left (resp. right) ideal of $R$ iff
        \begin{compactenum}
            \item $(I,+) < (R,+)$;
            \item $rx\in I$ (resp. $xr\in I$) for all $r\in R$ and $x\in I$.
        \end{compactenum}
    \end{compactenum}
    \tcblower
    \textit{Pf}: Obvious.
\end{Th}

\begin{Df}{Df 2\_P2.4.1.2 (generated ideal)}
    \begin{compactenum}
        \item For a ring $R$, define the ideal (resp. left ideal) (resp. right ideal) \textbf{generated by} a non-empty subset $X$ (similar to the Df \{, ID: 1.7.2\}).
        \item The ideal generated by $X$ is denoted by $(X)$; the elements of $X$ are called \textbf{generators} of $(X)$. If $X$ is finite, say $X = \{x_1, \dots, x_n\}$, then $(X)$ is said to be \textbf{finitely generated}, and we write $(X) = (x_1, \cdots, x_n)$. 
        \item An ideal $(x)$ generated by a single element $x$ is called a \textbf{principal ideal}. A \textbf{principal ideal ring} is a ring in which every ideal is principal. A principal ideal ring which is an integral domain is called a \textbf{principal ideal domain} (personally abbreviated as PID).
    \end{compactenum}
\end{Df}

\begin{Th}{Th 2\_P2.4.1.3 (which elements are in generated ideals)}
    Let $R$ be a ring. For $a\in R$ and $X\subseteq R$, then:
    \begin{compactenum}
        \item The principal ideal $(a)$ is
        $$ (a) = \left\{na+ra+as+\sum_{i=1}^{m}r_i a s_i:\; n\in \mathbb{Z}, m\in \mathbb{N}^\ast, \; r, s, r_i, s_i\in R \right\}. $$
        \item If $R$ has an identity, then
        $$ (a) = \left\{\sum_{i=1}^{m} r_i a s_i:\; m\in \mathbb{N}^\ast, \; r_i, s_i\in R \right\}. $$
        \item If $a\in\cent R$ (in particular if $R$ is commutative), then
        $$ (a) = \left\{na+ra:\; n\in \mathbb{Z}, r\in R \right\}. $$
        \item $Ra \triangleq \{ra:\; r\in R\}$ (resp. $aR \triangleq \{ar:\; r\in R\}$) is a left ideal (resp. right ideal) of $R$ (It is not necessary that $a\in Ra$ (resp. $a\in aR$)). If $R$ has an identity, then $a\in Ra\cap aR$; if $R$ has an identity and $a\in\cent R$, then $Ra = (a) = aR$. 
        \item If $R$ has an identity and $X\subseteq\cent R$ (in particular if $R$ is commutative), then
        $$ (X) = \left\{ \sum_{i=1}^{m} r_i x_i:\; m\in \mathbb{N}^\ast, r_i\in R, x_i\in X \right\}. $$
    \end{compactenum}
    \tcblower
    \textit{Pf}: Prove the first statement and the rest are straightforward results from it. \\
    The proof is easy, and we just talk about how to derive the first statement. The idea is to gradually expand a subset $I$ containing $a$ to an ideal. An ideal $I$ must be subring, and thus must be closed under addition and multiplication, and thus $\{na\}, \{a^k\}\subseteq I$, and thus
    $\{na+a^k\}\subseteq I$. Then $I$ must be closed under (two-sided) multiplication by $r\in R$, and thus $\{rna+nas+ra^k+a^ls+ra^ms\}\subseteq I$. Then after some reductions for the form we get the first statement.
\end{Th}

\begin{Rmk}{}
    Some examples of subrings and ideals:
    \begin{compactenum}
        \item ($\mathbb{Z}$) \textcolor{Th}{Of the ring $(\mathbb{Z},+,\times)$. Every (additive) subgroup, $n\mathbb{Z}$, is a subring, is a principal ideal ($n\mathbb{Z} = (n) = (-n)$).}
        \item ($\cent R$) \textcolor{Th}{The center of a ring is a subring, but is not necessarily an ideal.} For example, \textcolor{Th}{
        $$ \cent\mathcal{M}_2(\mathbb{R}) = \left\{ \begin{bmatrix}
            a & 0 \\
            0 & a
        \end{bmatrix} : a\in \mathbb{R} \right\} $$
        which is not an ideal of $S$.}
        \item (Intersection) \textcolor{Th}{The intersection of any family of subrings (resp. left ideals) (resp. right ideals) (resp. ideals) of a ring is a subring (resp. left ideal) (resp. right ideal) (resp. ideal) of the ring.} 
        \item (Homomorphism) \textcolor{Th}{A ring homomorphism preserves the subring structure in both directions.} That is, for any ring homomorphism $f: R\to S$, if $Q\subr R$ (resp. $T\subr S$), then $f(Q)\subr S$ (resp. $f^{-1}(T)\subr R$). \textcolor{Th}{For a ring homomorphism $f: R\to S$, we have $\Ker f\ideal R$, but not necessarily $\Ima f\ideal S$.} For example the image of the inclusion map $\mathbb{Z}\mapsto \mathbb{Q}$ is $\mathbb{Z}$, which is not an ideal of $\mathbb{Q}$, as $(1/2)\times 3\notin \mathbb{Z}$.
        \item (Proper ideal) \textcolor{Th}{Let $R$ be a ring. Then $\{0\}$ (called the \textbf{trivial ideal}) and $R$ are both ideals of $R$.} \textcolor{Df}{An ideal (resp. left ideal) (resp. right ideal) $I$ of $R$ is called a \textbf{proper} ideal (resp. left ideal) (resp. right ideal) if $I\neq \{0\}$ and $I\neq R$.} \textcolor{Th}{If $R$ has an identity, then an ideal (resp. left ideal) (resp. right ideal) $I$ of $R$ equals $R$ iff $1\in I$.} Consequently, \textcolor{Th}{a non-zero ideal (resp. left ideal) (resp. right ideal) $I$ of a ring $R$ with identity is proper iff $I$ contains no unit of $R$.}
        \item (Not principal) \textcolor{Th}{In the ring $R$ of all real functions $f: \mathbb{R}\to\mathbb{R}$, the set $I = \{f\in R:\; f(7) = 0\}$ is a principal ideal, generated by the function $g$ s.t. $g(7) = 0$ and $g(x) = 1$ for all $x\neq 7$.} However, \textcolor{Th}{in the ring $R$ of all continuous real functions $f: \mathbb{R}\to\mathbb{R}$, the ideal $I = \{f\in R:\; f(7) = 0\}$ is a not principal.} 
    \end{compactenum}
\end{Rmk}

\begin{Th}{Th 2\_P2.4.1.3.0.1}
    Let $R$ be a ring with identity. Then an ideal (resp. left ideal) (resp. right ideal) $I$ of $R$ equals $R$ iff $1\in I$.
    \tcblower
    \textit{Pf}: Already proved in the Rmk \{, ID: 2\_P2.4.1.3\}. Here we mention it just because this is commonly used in proofs.
\end{Th}

\begin{Th}{Th 2\_P2.4.1.3.1 (division ring $\overset{?}{\Longleftrightarrow}$ no proper ideals)}
    Let $R$ be a non-zero ring. Then
    \begin{compactenum}
        \item If $R$ is a division ring, then $R$ has no proper left (resp. right) ideals.
        \item If $R$ has no proper left (resp. right) ideals, then $R^2 = 0$ or $R$ is a division ring.
    \end{compactenum}
    \tcblower
    \textit{Pf}: Left ideals and right ideals are similar, so we only prove the case of left ideals. 
    \begin{compactenum}
        \item If $R$ is a division ring, then every non-zero left ideal $I$ of $R$ must contain some unit, so that $I = R$.
        \item Prove by the following steps:
        \begin{compactitem}
            \item Clearly the set $\text{Ann}_{\text{r}}(R) \triangleq \{a\in R:\; Ra = 0\}$ is an ideal of $R$ (and thus a left ideal). Then $\text{Ann}_{\text{r}}(R) = 0$ or $\text{Ann}_{\text{r}}(R) = R$. For the latter case we get $R^2 = 0$. Now assume $\text{Ann}_{\text{r}}(R) = 0$, and then for all $a\neq 0$ we have $0\neq Ra\ideal R$, and thus $Ra = R$. 
            \item Now claim that $R$ has no zero divisors so that the two-sided cancellation law holds. If $ab=0$ and $b\neq 0$, then we must have $a=0$. In fact the left ideal $\text{Ann}_{\text{l}}(b) = \{r\in R:\; rb = 0\}$ is $0$, since if otherwise $\text{Ann}_{\text{l}}(b) = R$ we have $rb=0$ for all $r\in R$, contradicting $Rb=R$. Thus $a\in\text{Ann}_{\text{l}}(b) = \{0\}$, the claim is true. 
            \item For some $0\neq a\in R$ we have $Ra = R$, so that $Ra\ni a = ea$ for some $0\neq e\in R$. Then $$ a = ea \implies a^2 = aea \implies a = ae $$
            by the cancellation law. Then we will show that $e$ is the identity of $R$. 
            \item For any $0\neq b\in R$ we have the following (use the cancellation law):
            $$ \begin{aligned}
                ea = a &\implies bea = ba \implies be = b \\
                ae = a &\implies aeb = ab \implies eb = b 
            \end{aligned} $$
            and thus $e$ is the identity of $R$, denoted by $e=1$.
            \item Finally prove that $R$ is a division ring. For any $0\neq a\in R$ we have $Ra\ni 1 = ca$ for some $0\neq c\in R$; for this $0\neq c\in R$ we have $R\ni 1 = dc$ for some $0\neq d\in R$. Then $c$ is both left and right invertible, and thus its left and right inverses coinside, $d=a$. Thus $a$ is a unit, and then $R$ is a division ring.
        \end{compactitem}
    \end{compactenum}
\end{Th}

\begin{Df}{Df 2\_P2.4.1.4 (subset addition and multiplication in a ring)}
    Let $R$ be a ring. 
    \begin{compactenum}
        \item The \textbf{subset addition} of $R$ follows from the Df \{, ID: 1.10\}, treating $R$ as an group under addition.
        \item The \textbf{subset multiplication} of $R$ is defined as (for $A, B\subseteq R$):
        $$ AB = \{a_1b_1 + \cdots + a_nb_n:\; n\in\mathbb{N}^\ast, a_i\in A, b_i\in B\} $$
    \end{compactenum}
\end{Df}

\begin{Rmk}{}
    \textcolor{Th}{The subset multiplication is not defined as $AB = \{ab\}$ because if so the distributive law $A(B+C) = AB + AC$ (talked below) would not hold.}
\end{Rmk}

\begin{Th}{Th 2\_P2.4.1.4.1 (properties of subset addition and multiplication)}
    Let $R$ be a ring and $A$, $B$, $C$ are all ideals (resp. left ideals) (resp. right ideals) of $R$. Then
    \begin{compactenum}
        \item (Closed under addition) $A+B$ is an ideal (resp. left ideal) (resp. right ideal) of $R$.
        \item (Commutativity of addition) $A+B = B+A$.
        \item (Assotiativity of addition) $A+(B+C) = (A+B)+C$. \textcolor{Df}{By this we can define $A_1 + \cdots + A_n$.}
        \item (Closed under multiplication) $AB$ is an ideal (resp. left ideal) (resp. right ideal) of $R$.
        \item (Assotiativity of multiplication) $A(BC) = (AB)C$. \textcolor{Df}{By this we can define $A_1 \cdots A_n$ (in particular, $\underbrace{A\cdots A}_{n} \triangleq A^n$)} And we have
        $$ A_1 \cdots A_n = \left\{\sum_{i=1}^{m} a_1^{(i)} \cdots a_n^{(i)}:\; m\in\mathbb{N}^\ast, a_j^{(i)}\in A_j \right\} $$
        \item ((Two-sided) Distributivity) $A(B+C) = AB + AC$ and $(B+C)A = BA + CA$.
        \item (Inclusion) $A_1\cdots A_n \,\subseteq\, A_1\cap\cdots\cap A_n \,\subseteq\, A_1+\cdots + A_n$.
    \end{compactenum}
    \tcblower
    \textit{Pf}: 
\end{Th}

\begin{Df}{Df 2\_P2.4.2 (quotient rings)}
    Let $R$ be a ring and $I\ideal R$. Then the quotient group of $R$ by $I$ (since as additive group, $I\nles R$), $R/I$, \textcolor{Th}{forms a ring under the following well-defined addition and multiplication,} called the \textbf{quotient ring} (or \textbf{factor ring}) of $R$ by $I$.
    \begin{compactenum}
        \item (Addition) $(a+I)+(b+I) = (a+b)+I$;
        \item (Multiplication) $(a+I)(b+I) = ab+I$.
    \end{compactenum}
\end{Df}

\begin{Rmk}{}
    \textcolor{Th}{For the quotient ring $R/I$
    \begin{compactenum}
        \item $R/I$ is commutative if $R$ is;
        \item $R/I$ has an identity if $R$ does, in which case the identity of $R/I$ is $1+I$.
    \end{compactenum}}
    Later we will see that what an ideal is in a ring is similar to what a normal subgroup is in a group, as those isomorphism theorems for groups are also true for rings (extended to ideals). Then the \textcolor{Df}{\textbf{normal subgroup-ideal analogy} $\mathcal{A}$ needs to be specified:
    \begin{compactenum}
        \item[$\bullet$] [Group terms] $\leftrightarrow$ [ring terms];
        \item Groups $\leftrightarrow$ rings;
        \item ``$\cdot$'' $\leftrightarrow$ ``$+$'';
        \item Subgroups $\leftrightarrow$ subrings;
        \item Group homomorphisms $\leftrightarrow$ ring homomorphisms;
        \item Normal subgroups $\leftrightarrow$ ideals;
        \item (For the notations) $G\leftrightarrow R$, $H\leftrightarrow S$, $N\leftrightarrow I$, $M\leftrightarrow J$, $K\leftrightarrow Q$, $L\leftrightarrow T$.
    \end{compactenum}}
    Many of the following theorems is a direct extension of the corresponding group theorems, under this analogy. In other words, \textbf{just replacing the group terms in the group theorems (including their proofs or solutions) with the ring terms we can get the ring theorems (including their proofs or solutions).} 
\end{Rmk}

\begin{Th}{Th 2\_P2.4.2.1 (ring isomorphism theorems)}
    The following theorems are direct extension to those of groups recorded in the chapter 1.
    \begin{compactenum}
        \item[$\bullet$] [(The mark of the ring theorem)] $\overset{\mathcal{A}}{\Longleftarrow}$ [block ID of the group theorem]
        \item (the canonical epimorphism) $\overset{\mathcal{A}}{\Longleftarrow}$ \verb|{, ID: 1.12.1}|
        \item () $\overset{\mathcal{A}}{\Longleftarrow}$ \verb|{, ID: 1.12.2.-1}|, \\ 
        ($f: R\to S$, $\overline{f}: R/I\to S$) $\overset{\mathcal{A}}{\Longleftarrow}$ \verb|{, ID: 1.12.2}|
        \item () $\overset{\mathcal{A}}{\Longleftarrow}$ \verb|{, ID: 1.12.3.-1}|, \\ 
        (the 1st isomorphism theorem) ($R/\Ker f\Riso \Ima f$) $\overset{\mathcal{A}}{\Longleftarrow}$ \verb|{, ID: 1.12.3}|
        \item () $\overset{\mathcal{A}}{\Longleftarrow}$ \verb|{, ID: 1.12.4.-1}|, \\ 
        ($f: R\to S$, $\overline{f}: R/I\to S/J$) $\overset{\mathcal{A}}{\Longleftarrow}$ \verb|{, ID: 1.12.4}|
        \item () $\overset{\mathcal{A}}{\Longleftarrow}$ \verb|{, ID: 1.12.5.-1}|, \\ 
        (the 2nd isomorphism theorem) ($ Q/(I\cap Q) \Riso (I+Q)/I $) $\overset{\mathcal{A}}{\Longleftarrow}$ \verb|{, ID: 1.12.5}|
        \item () $\overset{\mathcal{A}}{\Longleftarrow}$ \verb|{, ID: 1.12.6.-1}|, \\ 
        (the 3rd isomorphism theorem) ($ (R/Q)\Big/ (S/Q) \Riso (R/S) $) $\overset{\mathcal{A}}{\Longleftarrow}$ \verb|{, ID: 1.12.6}|
        \item () $\overset{\mathcal{A}}{\Longleftarrow}$ \verb|{, ID: 1.12.7.-1}|, \\ 
        ($I\mapsto f(I)$, $Q\mapsto f(Q)$) $\overset{\mathcal{A}}{\Longleftarrow}$ \verb|{, ID: 1.12.7}|
        \item (the subrings of $R/I$) $\overset{\mathcal{A}}{\Longleftarrow}$ \verb|{, ID: 1.12.8}|
    \end{compactenum}
\end{Th}

\begin{Df}{Df 2\_P2.4.3.1 (prime ideals)}
    An ideal $P$ of a ring $R$ is said to be \textbf{prime} if
    \begin{compactenum}
        \item $P\neq R$;
        \item For any $A, B\ideal R$, 
        $$ AB\subseteq P \quad\implies\quad A\subseteq P \;\;\text{or}\;\; B\subseteq P. $$
    \end{compactenum}
\end{Df}

\begin{Rmk}{}
    The definition of prime ideals are extensions of that of the prime integers (talked later).
\end{Rmk}

\begin{Th}{Th 2\_P2.4.3.1.1 (characterization of prime ideals)}
    Let $R$ be a ring, $P\ideal R$ and $P\neq R$. Then
    \begin{compactenum}
        \item[(1)] $P$ is prime if for any $a, b\in R$, 
        $$ ab\in P \quad\implies\quad a\in P \;\;\text{or}\;\; b\in P. $$
        \item[(2)] The converse of (1) is true if $R$ is commutative.
    \end{compactenum}
    \tcblower
    \textit{Pf}: (1) is obvious. For (2), since $ab\in P$ we have $(ab)\subseteq P$, and then
    $$ (a)(b) \subseteq (ab) \subseteq P $$
    and thus $(a)\subseteq P$ or $(b)\subseteq P$, and thus $a\in P$ or $b\in P$. The fact that $(a)(b)\subseteq (ab)$ is true due to the commutativity of $R$ (see the Th \{, ID: 2\_P2.4.1.3 \}).
\end{Th}

\begin{Th}{Th 2\_P2.4.3.1.2 ($P$ is prime $\overset{?}{\Longleftrightarrow}$ $R/P$ is an integral domain)}
    Let $R$ be a commutative ring with identity $1\neq 0$. \\
    Then an ideal $P$ of $R$ is prime iff $R/P$ is an integral domain.
    \tcblower
    \textit{Pf}: Obvious, as $R/P$ is an integral domain iff
    $$ (a+P)(b+P) = 0 \implies a+P = P \;\;\text{or}\;\; b+P = P $$
    iff
    $$ ab\in P \implies a\in P \;\;\text{or}\;\; b\in P. $$
\end{Th}

\begin{Rmk}{}
    \begin{compactenum}
        \item \textcolor{Th}{The zero ideal of any ring with no zero divisors is prime.} 
        \item \textcolor{Th}{Of $\mathbb{Z}$, the (non-zero) prime ideals are exactly the principal ideals of prime integers} (see the Rmk \{, ID: 2.1.2\}).
    \end{compactenum}
\end{Rmk}

\begin{Df}{Df 2\_P2.4.3.2 (maximal ideals)}
    An ideal $M$ of a ring $R$ is said to be \textbf{maximal} if
    \begin{compactenum}
        \item $M\neq R$;
        \item There is no ideal $I$ of $R$ such that $M\subsetneq I\subsetneq R$.
    \end{compactenum}
\end{Df}

\begin{Rmk}{}
    \begin{compactenum}
        \item \textcolor{Df}{Let $R$ be a ring and $p$ be a property about ideals (that is, for any ideal $I$ of $R$, either $p(I) = \text{True}$ or $p(I) = \text{False}$ (and not both)). Consider the poset $\mathcal{I} = \{I\ideal R: I\neq R,\,\, p(I) \text{ is True}\}$ equipped with the partial order $\subseteq$ (the set theoretic inclusion). Then an ideal $M$ of $R$ is said to be \textbf{maximal s.t.} $p$ if $M$ is the maximal element of $\mathcal{I}$.} \textcolor{Th}{Thus the Df \{, ID: 2\_P2.4.3.2\} here is the maximal ideal s.t. True.}
        \item \textcolor{Df}{Similarly we can define the maximal left (resp. right) ideal s.t. $p$.} 
        \item \textcolor{Th}{Of course there can be many maximal ideals in a ring.} 
    \end{compactenum}
    Before characterizing maximal ideals, we can use the condition that $M$ is maximal in this way: \\
    Take $a\in R\setminus M$. Then $(a)+M\supsetneq M$ and thus $(a)+M = R$. If further $R$ has an identity, then $1\in (a)+M$. 
\end{Rmk}

\begin{Th}{Th 2\_P2.4.3.2.1 (existence of maximal ideals)}
    In a ring $R$ with identity $1\neq 0$,
    \begin{compactenum}
        \item[(1)] Maximal ideals (resp. maximal left ideals) (resp. maximal right ideals) always exist.
        \item[(2)] Every proper ideal (resp. proper left ideal) (resp. proper right ideal) is contained in a maximal ideal (resp. maximal left ideal) (resp. maximal right ideal).
    \end{compactenum}
    \tcblower
    \textit{Pf}: We now show the statement (2). Assume $0\neq A\ideal R$. Then let $\mathcal{B} = \{B\ideal R: A\subseteq B\subsetneq R\}$. Clearly $\mathcal{B}\neq\varnothing$, and we just need to check that $\mathcal{B}$ has a maximal element, using the Zorn's lemma. Let $\mathcal{C} = \{C_i\}_{i\in I}$ be a totally ordered subset of $\mathcal{B}$. Then we can check that $C = \bigcup_{i\in I} C_i$ is an upper bound of $\mathcal{C}$ in $\mathcal{B}$. 
\end{Th}

\begin{Th}{Th 2\_P2.4.3.2.2 ($M$ is maximal $\overset{?}{\Longleftrightarrow}$ $R/M$ is an division ring)}
    Let $R$ be a ring with identity $1\neq 0$. Then
    \begin{compactenum}
        \item If $R/M$ ($M\ideal R$) is a division ring, then $M$ is maximal;
        \item If $M$ is a maximal ideal of $R$ and $R$ is commutative, then $R/M$ is a division ring (and thus is a field).
    \end{compactenum}
    \tcblower
    \textit{Pf}: 
    \begin{compactenum}
        \item Clearly $M\neq R$. To prove $M$ is maximal it suffices to show that any ideal $I\supsetneq M$ equals $R$. For some $a\in I\setminus M$, we have $M\subsetneq (a)+M\subseteq I$. Then we claim that $1\in (a)+M$ so that $I\subseteq (a)+M = R\subseteq I$. Then this claim is obvious, as $a+M$ is a unit of $R/M$ and thus $ab+M = 1+M$, i.e. $ab+m = 1$ for some $b\in R$ and $m\in M$. 
        \item For $a\notin M$ we can find some $b\in R$ s.t. $ab+m = 1$ for some $m\in M$. In fact since $M$ is maximal, we have $(a)+M\supsetneq M$, and thus $(a)+M=R$, and thus $1\in (a)+M = \{ra: r\in R\}+M$, and thus $1 = ab+m$ for some $b\in R$ and $m\in M$. 
    \end{compactenum}
\end{Th}

\begin{Th}{Clry 2\_P2.4.3.2.3 (maximal ideals $\overset{?}{\Longrightarrow}$ prime ideals)}
    Let $R$ be a nonzero ring. If
    \begin{compactenum}
        \item $R$ is commutative and
        \item $R^2 = R$ (in particular $R$ has an identity),
    \end{compactenum}
    then every maximal ideal of $R$ is prime.
    \tcblower
    \textit{Pf}: Let $M\ideal R$ be maximal. Then we will show that $M$ is prime by the Th \{, ID: 2\_P2.4.3.1.1\}. Suppose (by contradiction) $ab\in M$ but $a, b\notin M$. Then $M\subsetneq (a)+M, (b)+M$ and thus $(a)+M = (b)+M = R$. Since $R$ is commutative, $(ab)\subseteq (a)(b)$ ($\subseteq M$), and thus we have
    $$ M\supseteq (a)(b) + M(a) + M(b) + M^2 = \textcolor{P}{((a) + M)((b) + M) = R^2} = R = (a)+M\subsetneq M, $$
    which is a contradiction. Thus $M$ is prime.
\end{Th}

\begin{Rmk}{}
    This theorem is a direct result if assuming $R$ has an identity, in which case 
    $$ M \text{ is maximal} \implies R/M \text{ is a field } \implies R/M \text{ is an integral domain } \implies M \text{ is prime}. $$
    But here we can loosen the assumption that $1\in R$, and assume $R^2 = R$ instead.
\end{Rmk}

\begin{Df}{Th 2\_P2.4.4 (direct product of rings)}
    Let $\{R_i\}_{i\in I}$ be a family of rings. The \textbf{direct product} of the rings $R_i$ is their direct product as Abelian groups, $\prod_{i\in I} R_i$, \textcolor{Th}{which is a ring under the ``component-wise'' addition and multiplication.} 
\end{Df}

\begin{Rmk}{}
    Some remarks:
    \begin{compactenum}
        \item The weak (external and internal) direct products of infinitely many rings are not so useful as the weak direct products of groups. Hence we do not inspect them here. But when the family of rings is finite (in which case the direct product and the weak direct product coinside), there are significant results such as the Chinese remainder theorem.
        \item For a family $\{R_i\}_{i\in I}$ of rings, we should denote an element of $\prod_{i\in I} R_i$ by $\{x_i\}_{i\in I}$, but not by $(x_i)_{i\in I}$ as what we did in the group theory, because we may confuse $(x_i)$ with the principal ideal generated by $x_i$.
        \item \textcolor{Df}{Still, if $I$ is finite, we write $\prod_{i\in I} R_i = R_1\times\cdots\times R_n$.}
    \end{compactenum}
    Basic properties of the direct product of rings are similar to those of the direct product of groups. For a family $\{R_i\}_{i\in I}$ of rings:
    \begin{compactenum}
        \item \textcolor{Th}{If each $R_i$ has an identity (resp. each $R_i$ is commutative), then $\prod_{i\in I} R_i$ has an identity (resp. $\prod_{i\in I} R_i$ is commutative).} 
        \item \textcolor{Th}{The} \textcolor{Df}{canonical projections $\pi_i$ and the canonical injections $\iota_i$ defined in the Th \{, ID: 1\_P1.16.2\}} \textcolor{Th}{are all ring homomorphisms.}
    \end{compactenum}
\end{Rmk}

\begin{Th}{Th 2\_P2.4.4.1 (ideal of direct product $\overset{?}{\Longleftrightarrow}$ direct product of ideals)}
    Let $\{R_i\}_{i\in I}$ be a family of rings. Then
    \begin{compactenum}
        \item If $A_i\ideal R_i$ for all $i\in I$, then $\prod_{i\in I} A_i\ideal \prod_{i\in I} R_i$.
        \item If $I$ is finite (denote $I = \{1, \cdots, n\}$) and each $R_i$ has an identity, then each ideal of $R_1\times\cdots\times R_n$ is of the form $A_1\times\cdots\times A_n$, where $A_i\ideal R_i$ for all $i\in I$.
    \end{compactenum}
    \tcblower
    \textit{Pf}: The 1st statement is obvious. Now prove the 2nd statement. First for an ideal $A$ of $R_1\times\cdots\times R_n$ we have (by let $f$ in the 7th statement of Th \{, 2\_P2.4.2.1\} be the canonical projection $\pi_i$) $\pi_i(A)\ideal R_i\triangleq A_i$, and thus $A\subseteq A_1\times\cdots\times A_n$. Then we need to show that $A_1\times\cdots\times A_n\subseteq A$. \\
    For any $(a_1, a_2, \cdots, a_n)\in A_1\times A_2\cdots\times A_n$, there is some $(a_1, \ast, \cdots, \ast)\in A$, and thus
    $$ (a_1, 0, \cdots, 0) = (1_{R_1}, 0, \cdots, 0)(a_1, \ast, \cdots, \ast) \in A. $$
    And so forth, we have $(0, a_2, \cdots, 0)\in A$, \; $\cdots$ \;, $(0, \cdots, 0, a_n)\in A$, and then
    $$ (a_1, a_2, \cdots, a_n) = (a_1, 0, \cdots, 0) + (0, a_2, \cdots, 0) + \cdots + (0, \cdots, 0, a_n) \in A. $$
    Then $A_1\times\cdots\times A_n\subseteq A$. 
\end{Th}

\begin{Rmk}{}
    The 1st statement here is an analogue of the Th \{, ID: 1\_P1.16.4.1\} about groups. But the 2nd statement was not mentioned there.
\end{Rmk}

\begin{Th}{Th 2\_P2.4.4.2 (direct product of rings is the categorical product)}
    In the category $\mathsf{Ring}$, the product of the family $\{R_i\}$ of objects $R_i$ is the direct product $\prod_{i\in I} R_i$ together with the canonical projections $\{\pi_i\}_{i\in I}$.
    \tcblower
    \textit{Pf}: Obvious.
\end{Th}

\begin{Th}{Th 2\_P2.4.4.3 (internal direct product)}
    Let $R$ be a ring and $A_1, \cdots, A_n$ be ideals of $R$. If
    \begin{compactenum}
        \item[(1)] $R = A_1 + \cdots + A_n$;
        \item[(2)] $A_i\cap (A_1 + \cdots + A_{i-1} + A_{i+1} + \cdots + A_n) = 0$ for all $i\in\{1, \cdots, n\}$;
    \end{compactenum}
    Then $R\Riso A_1\times\cdots\times A_n$. \textcolor{Df}{In this case we say that $R$ is the \textbf{internal direct product} of $A_1, \cdots, A_n$, denoted by (an equality mark instead of an isomorphism mark is used here, if no confusion arises)
    $$ R = A_1\times\cdots\times A_n. $$}
    \tcblower
    \textit{Pf}: Construct the ring homomorphism $\varphi: A_1\times\cdots\times A_n\to R$ the same way as the Th \{, ID: 1\_P1.16.3\}. The $\varphi$ is a ring homomorphism due to that $A_i$ are ideals.
\end{Th}

\end{document}