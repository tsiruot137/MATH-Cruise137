\documentclass{article}

    \usepackage{xcolor}
    \definecolor{pf}{rgb}{0.4,0.6,0.4}
    \usepackage[top=1in,bottom=1in, left=0.8in, right=0.8in]{geometry}
    \usepackage{setspace}
    \setstretch{1.2} 
    \setlength{\parindent}{0em}

    \usepackage{paralist}
    \usepackage{cancel}

    % \usepackage{ctex}
    \usepackage{amssymb}
    \usepackage{amsmath}
    \usepackage{extarrows}
    \usepackage{tikz-cd}

    \usepackage{tcolorbox}
    \definecolor{Df}{RGB}{0, 184, 148}
    \definecolor{Th}{RGB}{9, 132, 227}
    \definecolor{Rmk}{RGB}{215, 215, 219}
    \definecolor{P}{RGB}{154, 13, 225}
    \newtcolorbox{Df}[2][]{colbacktitle=Df, colback=white, title={\large\color{white}#2},fonttitle=\bfseries,#1}
    \newtcolorbox{Th}[2][]{colbacktitle=Th, colback=white, title={\large\color{white}#2},fonttitle=\bfseries,#1}
    \newtcolorbox{Rmk}[2][]{colbacktitle=Rmk, colback=white, title={\large\color{black}{Remarks}},fonttitle=\bfseries,#1}
    \newtcolorbox{Rmk_continued}[2][]{colbacktitle=Rmk, colback=white, title={\large\color{black}{Remarks (continued)}},fonttitle=\bfseries,#1}

    \title{\LARGE \textbf{Rings}}
    \author{\large Jiawei Hu}

    % new commands for formula typying
    \newcommand{\lcm}{\text{lcm}}
    \newcommand{\cycl}{\text{cycl}}
    \newcommand{\nles}{\vartriangleleft}
    \newcommand{\notnles}{\ntriangleleft}
    \newcommand{\Ker}{\text{Ker}\,}
    \newcommand{\Ima}{\text{Im}\,}
    \newcommand{\hooktwoheadrightarrow}{%
        \hookrightarrow\mathrel{\mspace{-15mu}}\rightarrow}
    \newcommand{\End}{\text{End }}
    \newcommand{\Char}{\text{char}\,}

    % group isomorphism and ring isomorphism
    \newcommand{\Giso}{\overset{\text{g}}{\simeq}}
    \newcommand{\Riso}{\overset{\text{r}}{\simeq}}
\begin{document}
\maketitle

This is the 1st chapter of Abstract Algebra, which is about the \textbf{Rings}. By the way, we now pre-claim some commonly-used notations and terms:
\begin{Df}{Notations and Terms}
    \begin{compactenum}
        \item An agreement for the length of a list: if we write $a_1, \dots, a_n$, then we indicate that $n$ is finite and that $n\geq 1$; if we write $a_0, \dots, a_n$, then we indicate that $n$ is finite and that $n\geq 0$.
    \end{compactenum}
\end{Df}

Here is the \textbf{Quick Search} for this chapter:
\begin{Th}{Quick Search}
    \begin{compactdesc}
        \item (2.1.*): Basic concepts, properties and examples of rings.
        \item (2.2.*): Polynomial rings, matrix rings and group rings (included in the P1 file).
            \subitem (2.2.1.*) Polynomial rings.
            \subitem (2.2.2.*) Matrix rings.
            \subitem (2.2.3.*) Group rings.
        \item (2.3.*): Ring homomorphisms, characteristics.
        \item (2.4.*): Ideals.
            \subitem (2.4.1.*) Basic concepts of ideals.
            \subitem (2.4.2.*) Ring isomorphism theorems.
            \subitem (2.4.3.*) Prime ideals and maximal ideals.
            \subitem (2.4.4.*) Direct product of rings.
        \item (2.5.*): Factorization in commutative rings.
    \end{compactdesc}
\end{Th}

Then with everything prepared, here we go. 

\begin{Df}{Df2.1 (rings)}
    A \textbf{ring} is a non-empty set $R$ with two binary operations $+$ and $\times$ on it that satisfies the following properties:
    \begin{compactenum}
        \item $(R, +)$ is an Abelian group;
        \item (Multiplicative Assotiativity) For all $a, b, c \in R$, $(ab)c = a(bc)$;
        \item ((Two-sided) Distributivity of $\times$ over $+$) For all $a, b, c \in R$, 
        $$ a(b+c) = ab + ac \quad\text{ and }\quad (b+c)a = ba + ca. $$
    \end{compactenum}
    A ring $R$ is said \textbf{commutative} if
    \begin{compactenum}
        \item[4.] (Multiplicative Commutativity) For all $a, b \in R$, $ab = ba$.
    \end{compactenum}
    A ring $R$ \textbf{with identity} is a ring satisfies
    \begin{compactenum}
        \item[5.] (Multiplicative Identity) There exists $1 \in R$ such that for all $a \in R$, $a1 = 1a = a$.
    \end{compactenum}
\end{Df}

\begin{Rmk}{}
    Basic properties of a ring $R$:
    \begin{compactenum}
        \item \textcolor{Df}{The additive identity of a ring $R$ is denoted as $0$. Then we are certainly clear about the ``integer multiplication'' $na = an$ for $n\in\mathbb{Z}$, $a\in R$ (for example, $na = an = a + \cdots + a$ ($n$ times) for $n>0$).}
        \item \textcolor{Th}{$0a = a0 = 0$ for all $a \in R$.}
        \item \textcolor{Th}{$(na)b = a(nb) = n(ab)$ for all $a, b \in R$ and $n\in\mathbb{Z}$.}
        \item \textcolor{Th}{(Free Combination) For $a_1, \cdots, a_n, b_1, \cdots, b_m \in R$, we have
        $$ \left(\sum_{i=1}^{n} a_i\right) \left(\sum_{j=1}^{m} b_j\right) = \sum_{i=1}^{n} \sum_{j=1}^{m} a_i b_j $$
        } This property is a straightforward consequence of the distributivity of $R$.
        \item \textcolor{Th}{If $R$ has an identity, then the identity is unique.} 
    \end{compactenum}
\end{Rmk}

\begin{Rmk_continued}{}
    Some basic examples of rings:
    \begin{compactenum}
        \item \textcolor{Df}{Given an Abelian group $(G, +)$, we can define the simplest} \textcolor{Th}{ring} \textcolor{Df}{ by $R = G$ and $ab = 0$ for all $a, b \in R$. This ring is said to be \textbf{trivial}. In particular when $R = G = \{0\}$, the ring is called \textbf{zero ring}, denoted as $R = 0$.} \textcolor{Th}{Except for the zero ring, a trivial ring contains no identity; the zero ring is the only ring with an identity $1 = 0$.}
        \item \textcolor{Th}{$\mathbb{Z}$ together with the addition and multiplication in the usual sense, is a commutative ring with identity $1\in\mathbb{Z}$.} 
        \item \textcolor{Th}{$\mathbb{Q}$, $\mathbb{R}$, $\mathbb{C}$, together with the addition and multiplication in the usual sense, are all commutative rings with identity $1\in\mathbb{Q}, \mathbb{R}, \mathbb{C}$.}
        \item \textcolor{Th}{$\mathbb{Z}/n\mathbb{Z}$ together with the addition and multiplication of congruence classes, is a commutative ring with identity $\overline{1}$. If here $n=1$ then $\mathbb{Z}/n\mathbb{Z}$ is the zero ring of course.}
        \item Here is a typical example of a non-commutative ring. \textcolor{Th}{The set $\mathbb{H}$ of \textbf{(real) Hamilton Quaternions} is a non-commutative ring with identity.} Here \textcolor{Df}{$\mathbb{H}$ is defined as $\mathbb{H} = \{a1 + bi + cj + dk: a, b, c, d \in \mathbb{R}\}$, where $1$ is the real number $1$, acting as the identity of the ring $\mathbb{H}$, and:
        \begin{compactenum}
            \item (Addition) The addition is defined ``componentwisely'' by
            $$ (a + bi + cj + dk) + (a' + b'i + c'j + d'k) = (a + a') + (b + b')i + (c + c')j + (d + d')k; $$
            \item (Multiplication) The multiplication is a free combination, that is, is defined following the (two-sided) distributivity by the basic equalities below:
            $$ i^2 = j^2 = k^2 = -1, \quad ij = -ji = k, \quad jk = -kj = i, \quad ki = -ik = j. $$
        \end{compactenum}}
        Also, \textcolor{Th}{the set of \textbf{rational Hamilton Quaternions} is a non-commutative ring with identity.} Here \textcolor{Df}{the rational Hamilton Quaternions are defined similarly as the real Hamilton Quaternions, but with the coefficients $a, b, c, d$ being rational numbers.}
        \item \textcolor{Th}{Let $X$ be a non-empty set and $A$ be a ring. Then the set $\sideset{^X}{} {\mathop{A}}$ of all functions $f: X \to A$ is a ring under the ``pointwise'' addition and multiplication
        $$ (f + g)(x) = f(x) + g(x), \quad (fg)(x) = f(x)g(x) $$}
        \textcolor{Th}{In this case,
        \begin{compactenum}
            \item $\sideset{^X}{} {\mathop{A}}$ is commutative iff $A$ is.
            \item $\sideset{^X}{} {\mathop{A}}$ has an identity iff $A$ does. If $\sideset{^X}{} {\mathop{A}}$ has an identity, then the identity is the constant function $ X\ni x\mapsto 1 \in A $.
        \end{compactenum}}
        \item Here is an example of a ring without identity. \textcolor{Th}{The subset $n\mathbb{Z}$ of $\mathbb{Z}$ ($n\in\mathbb{N}^\ast$) is a ring (under the integer addition and multiplication). This ring contains an identity iff $n=1$.} Another example of a ring without identity is the subset $S$ of real functions:
        $$ S = \{f: \mathbb{R} \to \mathbb{R}: f(x) = 0 \text{ everywhere except for a compact set } X = \{x\} \subseteq\mathbb{R} \}. $$
        \textcolor{Th}{Easy to see that $S$ is a commutative ring without identity, as the identity must be the constant function $f(x) = 1$ but this function is not in $S$.}
    \end{compactenum}
\end{Rmk_continued}

\begin{Df}{Df2.1.1 (zero divisors and cancellation law)}
    \begin{compactenum}
        \item If in a ring $R$ there exist $a, b \in R$ such that 
        $$ a, b \neq 0 \text{ but } ab = 0, $$
        then $a$ and $b$ are called two \textbf{zero divisors} of $R$. In this case, $a$ (resp. $b$) is called a \textbf{left zero divisor} (resp. \textbf{right zero divisor}) 
        \item \textcolor{Th}{A ring $R$ has no zero divisors iff the (two-sided, non-zero) cancellation law holds. That is, for all $a, b, c \in R$, if $a\neq 0$, then
        $$ \left( ab = ac \quad\text{or}\quad ba = ca \right) \quad\Rightarrow\quad b = c. $$}
    \end{compactenum}
\end{Df}

\begin{Df}{Df2.1.2 (invertibility, integral domains, division rings, fields)}
    \begin{compactenum}
        \item Let $R$ be a ring with identity. 
        \begin{compactenum}
            \item[(1)] An element $a \in R$ is called \textbf{left invertible} (resp. \textbf{right invertible}) if there exists $c \in R$ (resp. $b \in R$) such that $ca = 1$ (resp. $ab = 1$); in this case $c$ (resp. $b$) is called a \textbf{left inverse} (resp. \textbf{right inverse}) of $a$.
            \item[(2)] If $a\in R$ is both left and right invertible, then $a$ is said to be \textbf{invertible} or to be a \textbf{unit} of $R$; \textcolor{Th}{in this case the left and right inverses of $a$ coincide as the \textbf{inverse(s)} of $a$, and the inverse is unique,} denoted as $a^{-1}$.
            \item[(3)] The set of all units of $R$ is denoted as $R^\ast$ or $R^\times$. \textcolor{Th}{Clearly $1 \in R^\times$.}
        \end{compactenum}
        \item Let $R$ be a ring with identity $1\neq 0$. Then $R$ is called an \textbf{integral domain} if 
        \begin{compactenum}
            \item[(1)] $R$ is commutative and
            \item[(2)] $R$ has no zero divisors.
        \end{compactenum}
        \item Let $D$ be a ring with identity $1\neq 0$. Then $D$ is called a \textbf{division ring} if
        $$ D^\times = D \setminus \{0\} $$
        \item A \textbf{field} $F$ is a commutative division ring. 
    \end{compactenum}
\end{Df}

\begin{Rmk}{}
    Clearly:
    \begin{compactenum}
        \item \textcolor{Th}{A left invertible element could have more than one left inverse.} For example, consider $A = \{(a_0, a_1, a_2, \cdots)\}$ as the set of sequences in $\mathbb{R}$ (of course $A$ is a group under the componentwise addition). Then the set of endomorphisms of $A$, $\text{End }(A)$, is a ring under the pointwise addition and the composition of functions. Consider $r\in\text{End }(A)$ defined by
        $$ r(a_0, a_1, a_2, \cdots) = (0, a_0, a_1, \cdots), $$
        then $r$ is left invertible, but with two different left inverses $l$ and $l_0$:
        $$ \begin{aligned}
            l(a_0, a_1, a_2, \cdots) &= (a_1, a_2, a_3, \cdots) \\
            l_0(a_0, a_1, a_2, \cdots) &= (a_0+a_1, a_2, a_3, \cdots) 
        \end{aligned} $$ 
        \item \textcolor{Th}{In a ring $R$ with identity, the set $R^\times$ of all units forms a group under the multiplication of $R$.}
        \item \textcolor{Th}{In a ring $R$ with identity, $0$ is not invertible, $1$ is invertible with inverse $1$.}
        \item \textcolor{Th}{For a ring $R$ with identity, $R$ is the following disjoint union:
        $$ R = \{0\} \sqcup \{\text{units}\} \sqcup \{\text{zero divisors}\} \sqcup \{\text{other elements}\} $$
        Consequently, A field is an integral domain}
    \end{compactenum}
    Some examples:
    \begin{compactenum}
        \item For $\mathbb{Z}$, $\mathbb{Q}$, $\mathbb{R}$ and $\mathbb{C}$: \\ \textcolor{Th}{
        % Use \begin{minipage} instead of \begin{table} to avoid the error "Float(s) lost".
        \begin{minipage}{\linewidth}
            \centering
            \begin{tabular}{|c|c|c|c|c|}
                \hline
                \text{Is} & $\mathbb{Z}$ & $\mathbb{Q}$ & $\mathbb{R}$ & $\mathbb{C}$ \\ \hline
                \text{an integral domain} & $\surd$ & $\surd$ & $\surd$ & $\surd$ \\ \hline
                \text{a field} & $\times$ & $\surd$ & $\surd$ & $\surd$ \\ \hline 
            \end{tabular}
        \end{minipage}}
        \textcolor{white}{dddd} % more distance between the table and the text
        \item \textcolor{Th}{A number field (the Df \{course: 1, ID: 1.1\}) is a field}.
        \item \textcolor{Th}{For $\mathbb{Z}/n\mathbb{Z}$ with $n\geq 2$, the set $(\mathbb{Z}/n\mathbb{Z})^\times$ defined previously (defined in Rmk \{course: 0, ID: 2.2.1.1\}) is indeed the set of all units of $\mathbb{Z}/n\mathbb{Z}$, the notation is consistent. } And, \textcolor{Th}{consider the ring $\mathbb{Z}/n\mathbb{Z}$ with $n\geq 2$, written as
        $$ \mathbb{Z}/n\mathbb{Z} = \{\overline{0}, \overline{1}, \cdots, \overline{n-1}\}. $$
        Then the element $\overline{a} \in \mathbb{Z}/n\mathbb{Z}$ ($0\leq a\leq n-1$) satisfies exactly one of the following cases
        \begin{compactenum}
            \item $a = 0$;
            \item $a$ and $n$ are coprime, and thus $\overline{a}$ is a unit;
            \item $a$ and $n$ are not coprime, and thus $\overline{a}$ is a zero divisor.
        \end{compactenum}
        Consequently, 
        $$ \mathbb{Z}/n\mathbb{Z} \,\text{ is an integral domain} \quad \Leftrightarrow \quad \mathbb{Z}/n\mathbb{Z} \,\text{ is a field} \quad \Leftrightarrow \quad n \text{ is a prime integer.} $$ }
    \end{compactenum}
\end{Rmk}

\begin{Rmk_continued}{}
    \begin{compactenum}
        \item[3.] \textcolor{Th}{Consider the ring $R$ of all functions $f: \mathbb{R} \to \mathbb{R}$ (under the pointwise addition and pointwise multiplication). Then $R$ is neither an integral domain nor a field, as $fg = 0$ for 
        $$ 0\neq f(x) = \begin{cases}
            1, & \text{if } x = 1 \\
            0, & \text{otherwise}
        \end{cases} \qquad\text{and}\qquad
        0\neq g(x) = \begin{cases}
            1, & \text{if } x = 2 \\
            0, & \text{otherwise}
        \end{cases} $$ }
        \item[4.] \textcolor{Df}{A nonzero integer $D$ is said to be \textbf{square-free} if the prime factorization of $D$ contains no repeated prime factors (we force each prime factor to be positive and just place the sign of $D$ in front of the product). By convention we say $\pm 1$ are also square-free}. Then \textcolor{Df}{define $$ \mathbb{Q}(\sqrt{D}) = \{a + b\sqrt{D}: a, b \in \mathbb{Q}\} $$
        where $D$ is a square-free integer. Such sets are called \textbf{quadratic fields}, which are classified as \textbf{real quadratic fields} (if $D>0$) and \textbf{imaginary quadratic fields} (if $D<0$).} Of course \textcolor{Th}{$\mathbb{Q}(\sqrt{D})$ is a field.} Actually one can first verify that \textcolor{Th}{for $a+b\sqrt{D}\in\mathbb{Q}(\sqrt{D})$, $a+b\sqrt{D} = 0$ iff $a = b = 0$,} and then \textcolor{Th}{obtain the inverse of $a+b\sqrt{D}$ by
        $$ (a+b\sqrt{D})^{-1} = \frac{1}{a+b\sqrt{D}} = \frac{a-b\sqrt{D}}{(a+b\sqrt{D})(a-b\sqrt{D})} = \frac{a-b\sqrt{D}}{a^2 - b^2 D} \in \mathbb{Q}(\sqrt{D}). $$}
    \end{compactenum}
\end{Rmk_continued}

\begin{Th}{Blocks from the P1 file}
\end{Th}

\begin{Th}{Th2.2.4 (multinomial theorem)}
    Suppose $R$ is a ring with identity. Let $n\in\mathbb{N}^\ast$ and $a_1, \cdots, a_m\in R$. If $a_i a_j = a_j a_i$ for all $1\leq i, j \leq m$, then
    $$ (a_1 + \cdots + a_m)^n = \sum_{k_1 + \cdots + k_m = n} \binom{n}{k_1, \cdots, k_m} \; a_1^{k_1} \cdots a_m^{k_m}, $$
    where the multinomial coefficients are
    $$ \binom{n}{k_1, \cdots, k_m} = \frac{n!}{k_1! \cdots k_m!}. $$
    \tcblower
    \textit{Pf}: Obvious by the free combination property of $R$ (Rmk \{, ID: 2.1\}).
\end{Th}

\begin{Rmk}{}
    The condition that $a_i a_j = a_j a_i$ for all $i, j$ is necessary, but the condition that $R$ has an identity is not. If $R$ has no identity the equation is still valid, as long as we view $a_i^0 a$ as $a$ (although $a_i^0$ is not defined). 
\end{Rmk}

\begin{Df}{Df2.3 (ring homomorphism)}
    Let $R$ and $S$ be two rings. A function $f: R \to S$ is called a \textbf{ring homomorphism} (from $R$ to $S$) if
    $$ f(a+b) = f(a) + f(b), \quad f(ab) = f(a)f(b) $$
    for all $a, b \in R$. 
\end{Df}

\begin{Rmk}{}
    \begin{compactenum}
        \item \textcolor{Th}{A ring homomorphism $R\to S$ is first a group homomorphism $(R, +) \to (S, +)$.}
        \item \textcolor{Df}{Similarly define the terms \textbf{ring monomorphism}, \textbf{ring epimorphism}, \textbf{ring isomorphism}, \textbf{ring endomorphism} and \textbf{ring automorphism}.} And \textcolor{Df}{a ring monomorphism is also called a ring \textbf{embedding}.} \textcolor{Df}{Personally we denote two ring-isomorphic rings $R$ and $S$ as $R \Riso S$.}
        \item \textcolor{Df}{The \textbf{kernel} $\Ker f$ and the \textbf{image} $\Ima f$ of the ring homomorphism $f$ are its kernel and image as a group homomorphism $(R, +) \to (S, +)$.}
        \item \textcolor{Th}{A ring homomorphism between two rings with identity needs not to map the identity to the identity.} For example, consider the ring homomorphism from a ring $R$ with identity to the $2\times 2$ matrix ring $M_2(R)$ over it:
        $$ r\mapsto \begin{bmatrix}
            r & 0 \\
            0 & 0 
        \end{bmatrix} $$
        Then it does not map $1_R$ to the identity matrix in $M_2(R)$. 
    \end{compactenum}
    Some examples of ring homomorphisms:
    \begin{compactenum}
        \item \textcolor{Th}{The map $\mathbb{Z} \to \mathbb{Z}$ defined by $n\mapsto kn$ is generally not a ring homomorphism, unless $k = 1$.}
        \item \textcolor{Th}{The map $\mathbb{Z} \to \mathbb{Z}/n\mathbb{Z}$ defined by $k\mapsto \overline{k}$ is a ring epimorphism; the map $\mathbb{Z}/3\mathbb{Z} \to \mathbb{Z}/6\mathbb{Z}$ defined by $\overline{k}\mapsto \overline{4k}$ is a ring monomorphism.}
        \item \textcolor{Th}{For two groups $G$ and $H$ (written multiplicatively) and a ring $R$, a group homomorphism $f: G \to H$ induces a ring homomorphism $\overline{f}: R(G) \to R(H)$:
        $$ f\left(\sum_{i=1}^n r_i g_i\right) = \sum_{i=1}^n r_i f(g_i). $$}
    \end{compactenum}
\end{Rmk}

\begin{Df}{Df2.3.1 (characteristic)}
    Let $R$ be a ring. If there exists an $n\in\mathbb{N}^\ast$ such that $na = 0$ for all $a \in R$, then we call the smallest such $n$ the \textbf{characteristic} of $R$ (or, say $R$ has characteristic $n$). If no such $n$ exists, we say the characteristic of $R$ is $0$. The characteristic of $R$ is denoted as $\Char R$.
\end{Df}

\begin{Th}{Th2.3.1.1 (properties of characteristic)}
    Let $R$ be a ring with identity $1_R$ and positive characteristic. Then
    \begin{compactenum}
        \item $\Char R = |1_R|$, that is, $\Char R$ is the smallest positive integer $n$ such that $n1_R = 0$;
        \item If $R$ has no zero divisors, then $\Char R$ is a prime integer.
    \end{compactenum}
    \tcblower
    \textit{Pf}: 
    \begin{compactenum}
        \item Obvious.
        \item If otherwise $\Char R$ is not prime, that is, $\Char R = pq$ for some $0 < p, q < \Char R$, then $0 = pq1_R = p1_R q1_R$, indicating that $p1_R = 0$ or $q1_R = 0$, a contradiction to the minimality of $\Char R$. 
    \end{compactenum}
\end{Th}

\begin{Rmk}{}
    \textcolor{Th}{Clearly $\Char\mathbb{Z} = 0$, $\Char\mathbb{Z}_n = n$.} 
\end{Rmk}

\begin{Th}{Th2.3.1.2 (embedding of a rng)}
    Any ring $R$ can be embedded into a ring $S$ with identity. The ring $S$ (which is not unique) can be chosen to be either of characteristic zero or of the same characteristic as $R$.
    \tcblower
    \textit{Pf}: Let $\Char R = n$. Then the solution is
    \begin{compactenum}
        \item ($\Char S = 0$) $S = R\oplus\mathbb{Z}$, where the multiplication on $S$ is defined by (figure out the details of the following formal sums)
        $$ (r_1+k_1)(r_2+k_2) = (r_1r_2 + k_1r_2 + k_2r_1) + k_1k_2. $$
        Obviously $S$ is a ring with identity $0+1$, of characteristic $0$.
        \item ($\Char S = n>0$) If $n>0$, then choose $S = R\oplus\mathbb{Z}_n$, where the multiplication on $S$ is defined by (figure out the details of the following formal sums)
        $$ (r_1+\overline{k_1})(r_2+\overline{k_2}) = (r_1r_2 + k_1r_2 + k_2r_1) + \overline{k_1k_2}. $$
        Obviously $S$ is a ring with identity $0+\overline{1}$, of characteristic $n$.
    \end{compactenum}
\end{Th}

\begin{Th}{Blocks from the P2 file}
\end{Th}

\begin{Th}{Blocks from the P3 file}
\end{Th}

\end{document}