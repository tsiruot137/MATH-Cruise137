\documentclass{article}

    \usepackage{xcolor}
    \definecolor{pf}{rgb}{0.4,0.6,0.4}
    \usepackage[top=1in,bottom=1in, left=0.8in, right=0.8in]{geometry}
    \usepackage{setspace}
    \setstretch{1.2} 
    \setlength{\parindent}{0em}

    \usepackage{paralist}
    \usepackage{cancel}

    % \usepackage{ctex}
    \usepackage{amssymb}
    \usepackage{amsmath}
    \usepackage{extarrows}
    \usepackage{tikz-cd}

    \usepackage{tcolorbox}
    \definecolor{Df}{RGB}{0, 184, 148}
    \definecolor{Th}{RGB}{9, 132, 227}
    \definecolor{Rmk}{RGB}{215, 215, 219}
    \definecolor{P}{RGB}{154, 13, 225}
    \newtcolorbox{Df}[2][]{colbacktitle=Df, colback=white, title={\large\color{white}#2},fonttitle=\bfseries,#1}
    \newtcolorbox{Th}[2][]{colbacktitle=Th, colback=white, title={\large\color{white}#2},fonttitle=\bfseries,#1}
    \newtcolorbox{Rmk}[2][]{colbacktitle=Rmk, colback=white, title={\large\color{black}{Remarks}},fonttitle=\bfseries,#1}
    \newtcolorbox{Rmk_continued}[2][]{colbacktitle=Rmk, colback=white, title={\large\color{black}{Remarks (continued)}},fonttitle=\bfseries,#1}

    \title{\LARGE \textbf{Factorization in Commutative Rings}}
    \author{\large Jiawei Hu}

    % new commands for formula typying
    \newcommand{\lcm}{\text{lcm}}
    \newcommand{\cycl}{\text{cycl}}
    \newcommand{\nles}{\vartriangleleft}
    \newcommand{\notnles}{\ntriangleleft}
    \newcommand{\Ker}{\text{Ker}\,}
    \newcommand{\Ima}{\text{Im}\,}
    \newcommand{\hooktwoheadrightarrow}{%
        \hookrightarrow\mathrel{\mspace{-15mu}}\rightarrow}
    \newcommand{\End}{\text{End }}
    \newcommand{\Char}{\text{char}\,}

    % terms of rings, similar from those of groups
    \newcommand{\Giso}{\overset{\text{g}}{\simeq}} % group isomorphism
    \newcommand{\Riso}{\overset{\text{r}}{\simeq}} % ring isomorphism
    \newcommand{\subr}{\overset{\text{r}}{<}} % subring
    \newcommand{\ideal}{\overset{\text{r}}{\nles}} % ideal
    \newcommand{\cent}{\text{cent}\,} % center

\begin{document}
\maketitle

This file is about the \textbf{Factorization in Commutative Rings}.

Here is the \textbf{Quick Search} for this chapter:
\begin{Th}{Quick Search}
    \begin{compactdesc}
        \item (2\_P3.5.1.*) Division, associates.
        \item (2\_P3.5.2.*) Irreducible and prime elements.
        \item (2\_P3.5.3.*) PID $\implies$ UFD.
        \item (2\_P3.5.4.*) Euclidean domain.
        \item (2\_P3.5.5.*) GCD.
    \end{compactdesc}
\end{Th}

Then with everything prepared, here we go.

\begin{Df}{Df 2\_P3.5.1 (division)}
    Let $R$ be a nonzero commutative ring and $a, b \in R$.
    \begin{compactenum}
        \item $a$ is said to \textbf{divide} $b$, denoted by $a\mid b$, if $a\neq 0$ and $b=ax$ for some $x \in R$.
        \item $a$ and $b$ are said to be \textbf{associates}, denoted personally by $a\mid b\mid a$, if $a\mid b$ and $b\mid a$.
    \end{compactenum}
\end{Df}

\begin{Th}{Th 2\_P3.5.1.1 (divisibility rephrased in principal ideal, unit)}
    Let $R$ be a nonzero commutative ring with identity. Then (the following, $a$, $b$,\dots, are all elements of $R$):
    \begin{compactenum}
        \item $a\mid b$ iff $(a)\supseteq (b)$.
        \item $a\mid b\mid a$ iff $(a)=(b)$.
        \item $u$ is a unit \quad $\iff$\quad $(u)=R$ \quad $\iff$\quad $u\mid r$ for all $r\in R$.
        \item The relation ``$a$ is an associate of $b$'' is an equivalence relation on $R$.
        \item If $b=au$ for some unit $u$, then $a\mid b\mid a$. If $R$ is an integral domain, then the converse is true.
    \end{compactenum}
    \tcblower
    \textit{Pf}: Obvious.
\end{Th}

\begin{Df}{Df 2\_P3.5.2 (irreducible and prime elements)}
    Let $R$ be a commutative ring with identity $1\neq 0$. Then (the following, $a$, $b$,\dots, are all elements of $R$):
    \begin{compactitem}
        \item $c$ is said to be \textbf{irreducible} if
        \begin{compactenum}
            \item[(1)] $c$ is nonzero and nonunit;
            \item[(2)] $c=ab$ \quad $\implies$\quad $a$ or $b$ is a unit.
        \end{compactenum}
        \item $p$ is said to be \textbf{prime} if
        \begin{compactenum}
            \item[(1)] $p$ is nonzero and nonunit;
            \item[(2)] $p\mid ab$ \quad $\implies$\quad $p\mid a$ or $p\mid b$.
        \end{compactenum}
    \end{compactitem}
\end{Df}

\begin{Rmk}{}
    Some examples of irreducible and prime elements:
    \begin{compactenum}
        \item \textcolor{Th}{In $\mathbb{Z}$, $\{\text{prime elements}\} = \{\text{irreducible elements}\} = \{\text{prime numbers}\}$ (see the Df \{course: 0, ID: 2.1.2\}).}
        \item \textcolor{Th}{In the polynomial ring $\mathbb{F}[x]$ over a number field $\mathbb{F}$, the definition of irreducible elements coincides with the one of Df \{course: 1, ID: 1.8\}.}
        \item \textcolor{Th}{In general $\{\text{prime elements}\} \subsetneq, \supsetneq \{\text{irreducible elements}\}$.} In fact, in $\mathbb{Z}_6$, $\bar{2}$ is prime but not irreducible ($\bar{2}=\bar{2}\cdot\bar{4}$). There are also irreducible elements that are not prime in other rings.
    \end{compactenum}
    And there is a close connection between prime elements (resp. irreducible elements) and prime principal ideals (resp. maximal principal ideals).
\end{Rmk}

\begin{Th}{Th 2\_P3.5.2.1 (prime (resp. irreducible) elements $\overset{?}{\Longleftrightarrow}$ prime (resp. maximal) principal ideals)}
    Let $R$ be an integral domain. Then
    \begin{compactenum}
        \item Let $p$ be nonzero. Then $p$ is prime iff $(p)$ is a prime ideal.
        \item Let $c$ be nonzero nonunit. Then 
        $$ \begin{aligned}
            & c \text{ is irreducible} \quad\Longleftrightarrow\quad (c) \text{ is a maximal principal ideal} \quad\Longleftrightarrow\quad \\
            & \text{The only divisors of } c \text{ are its associates and the units.}
        \end{aligned} $$
        \item Every prime element is irreducible. If $R$ is a principal ideal domain, then the converse is true.
        \item Every associate of a prime (resp. irreducible) element is also prime (resp. irreducible).
    \end{compactenum}
    \tcblower
    \textit{Pf}: Obvious.
\end{Th}

\begin{Df}{Df 2\_P3.5.3 (unique factorization domain)}
    A \textbf{unique factorization domain} (personally abbreviated as UFD) is an integral domain $R$ such that
    \begin{compactenum}
        \item (Existence) Every nonzero nonunit element $a$ can be written as $a=c_1\cdots c_n$ with the $c_i$'s irreducible.
        \item (Uniqueness) If $a=c_1\cdots c_n=d_1\cdots d_m$ with the $c_i$'s and $d_j$'s irreducible, then
        \begin{compactenum}
            \item $n=m$;
            \item for some permutation $\sigma$ of $\{1,\cdots,n\}$, \quad $c_i\mid d_{\sigma(i)}\mid c_i$ for all $i$.
        \end{compactenum}
    \end{compactenum}
\end{Df}

\begin{Th}{Lma 2\_P3.5.3.1.-1 (principal ideals grow infinitely?)}
    In a principal ideal ring $R$, any sequence of ideals
    $$ (a_1)\subseteq (a_2)\subseteq (a_3)\subseteq \cdots $$
    is of the form
    $$ (a_1)\subseteq (a_2)\subseteq \cdots \subseteq (a_n)=(a_{n+1})=(a_{n+2})=\cdots $$
    \tcblower
    \textit{Pf}: $A = \bigcup_{i=1}^{\infty} (a_i) = (a) = (a_n)$.
\end{Th}

\begin{Rmk}{}
    This lemma is introduced to show the essential result that PID $\implies$ UFD. To prove the existence of the unique factorization, we still factor an element $a$ inductively. 
    \begin{compactenum}
        \item If $a$ itself is irreducible then it is done; otherwise $a$ is reducible and then $a = c_1c_2$.
        \item If $c_1$, $c_2$ are irreducible then it is done; otherwise, say $c_1$ is reducible, and then $c_1 = c_{11}c_{12}$. And by the same discussion about $c_2$ we get (if $c_2$ is reducible) $a = c_{11}c_{12}c_{21}c_{22}$.
    \end{compactenum}
    Then by this inductive algorithm we get a binary tree 
    $$a=c_1c_2=c_{11}c_{12}c_{21}c_{22}=c_{111}c_{112}c_{121}c_{122}c_{211}c_{212}c_{221}c_{222}=\cdots,$$
    which probably grows infinitely. \\
    To prove the unique factorization, we need to cut off this tree. Then a necessary condition is that any ``divisor chain'' $\cdots\mid a_3\mid a_2\mid a_1$, namely, any ``principal ideal inclusion chain'' $(a_1)\subseteq (a_2)\subseteq (a_3)$, would not extend infinitely. This introduces us to this lemma.
\end{Rmk}

\begin{Th}{Th 2\_P3.5.3.1 (PID $\implies$ UFD)}
    Every principal ideal domain is a unique factorization domain.
    \tcblower
    \textit{Pf}: Let $R$ be a principal ideal domain. 
    \begin{compactenum}
        \item (Existence) Let $\mathcal{S}$ consists of the elements $a$ such that
        \begin{compactenum}
            \item $a$ is nonzero and nonunit;
            \item $a$ can not be written as a finite product of irreducible elements.
        \end{compactenum}
        Then we would show that $\mathcal{S} = \varnothing$. If otherwise $a\in \mathcal{S}$, then $(a)\neq R$ and thus $(a)\subseteq (c)$ for some irreducible $c$ (Th \{, ID: 2\_P2.4.3.2.1\}). By the Axiom of Choice we select a $c_a$ in such $c$'s, for which there is an $x$ s.t. $a = c_a x$. Since $R$ is an integral domain, such $x$ is unique, denoted $a = c_a x_a$. Of course $x_a\in\mathcal{S}$, otherwise 
        \begin{compactenum}
            \item if $x_a=0$, contradiction;
            \item if $x_a$ is a unit, then $a$ is an associate of the irreducible $c_a$, and thus is irreducible, contradiction;
            \item if $x_a$ can be written as a finite product of irreducibles, then so can $a$, contradiction.
        \end{compactenum}
        Also, $(a)\subsetneq (x_a)$ (if otherwise $(a)=(x_a)$, then $a\mid x_a\mid a$ and thus $a$ and $x_a$ are associates, and thus $c_a$ is a unit, contradiction). 
        Then we well-define a function $f:\mathcal{S}\to\mathcal{S}$ by $f(a) = x_a$. By the inductive definition of $f$ we get a sequence
        $$ a_1 = a, a_2 = f(a_1), a_3 = f(a_2), \cdots $$
        such that $(a_1)\subsetneq (a_2)\subsetneq (a_3)\subsetneq \cdots $. This contradicts the lemma \{, ID: 2\_P3.5.3.1.-1\}.
        \item (Uniqueness) Let $a = c_1\cdots c_n = d_1\cdots d_m$ with the $c_i$'s and $d_j$'s irreducible. Then $c_1\mid d_1\cdots d_m$ and thus (since $c$ is prime) $c_1\mid d_j$ for some $j$, and thus $d_j = c_1 u$. Since $d_j$ is irreducible and $c_1$ is nonunit, we have that $u$ is a unit, and thus $c_1\mid d_j\mid c_1$. By an inductive argument we can show the uniqueness of the factorization.
    \end{compactenum}
\end{Th}

\begin{Rmk}{}
    With the lemma \{, ID: 2\_P3.5.3.1.-1\} we still cannot directly prove the existence of the factorization in the way of cutting the binary tree (see the Rmk \{, ID: 2\_P3.5.3.1.-1\}). At least this is not obvious, if we cannot show that we can cut the tree in a certain, uniform depth. In the proof of the unique factorization of a polynomial ring (the Th \{course: 1, ID: 1.8.3\}) things are much easier, since each time we move down the tree the degree of the polynomial decreases strictly so that we can cut the tree in a uniform depth. The proof here handles this problem by preventing the tree from growing, in one branch of each fork (the choose of $c_a$), so that the binary tree is reduced to a chain (a unary tree). 
\end{Rmk}

\begin{Th}{Th 2\_P3.5.4 (Euclidean domain $\implies$ PID)}
    \begin{compactitem}
        \item \textcolor{Df}{Let $R$ be a commutative ring. Then $R$ is called an \textbf{Euclidean ring} if there is a function $\varphi:R\setminus\{0\}\to\mathbb{N}$ s.t.
        \begin{compactenum}
            \item If $a, b\in R$ and $ab\neq 0$, then $\varphi(a), \varphi(b) \leq \varphi(ab)$.
            \item (Division with remainder) If $a, b\in R$ and $b\neq 0$, then there are $q, r\in R$ s.t. $a = qb + r$ with
            \begin{compactenum}
                \item $r=0$ or
                \item $r\neq 0$ and $\varphi(r) < \varphi(b)$.
            \end{compactenum}
        \end{compactenum}}
        \item \textcolor{Df}{A Euclidean ring is an \textbf{Euclidean domain} if it is an integral domain.}
        \item Every Euclidean ring is a principal ideal ring with identity (consequently, every Euclidean domain is a principal ideal domain, and thus a unique factorization domain).
    \end{compactitem}
    \tcblower
    \textit{Pf}: Let $R$ be a Euclidean ring. For $0\neq I\ideal R$, let $a = \arg\min_{x\in I}\varphi(x)$, then we can check that $I = (a)$. In fact, for any $b\in I$ we have $b = qa + r$ with $q\in R$ and $r\in R$. Clearly $r = b-qa\in I$. If $r\neq 0$, then $\varphi(r) < \varphi(a)$, which contradicts the definition of $a$. So $b = qa$, and thus $b\in Ra\subseteq (a)$. Then $I\subseteq Ra\subseteq (a)\subseteq I$, $R$ is a principal ideal ring. \\
    Now for $R\ideal R$ we choose the $a$ as above, and then $R = Ra$. Thus $a = ea = ae$ for some $e\in R$. Then for any $b\in R$ we have $b = xa$ for some $x\in R$. Since $eb = be = xae = xa = b$, the element $e$ is the identity of $R$. 
\end{Th}

\begin{Rmk}{}
    Some examples of Euclidean rings:
    \begin{compactenum}
        \item \textcolor{Th}{$\mathbb{Z}$ is a Euclidean domain with $\varphi(n) = |n|$.}
        \item \textcolor{Th}{Any field $F$ is a Euclidean domain with $\varphi(a) = 1$.}
        \item (Proved later) The polynomial ring $F[x]$ over a field $F$ is a Euclidean domain with $\varphi(f) = \deg(f)$.
        \item \textcolor{Th}{The} \textcolor{Df}{\textbf{Gaussian integers} $\mathbb{Z}[i] = \{a+bi: a,b\in\mathbb{Z}\}$ ($i$ is the imaginary unit)} \textcolor{Th}{is a Euclidean domain with $\varphi(a+bi) = a^2+b^2$.}
    \end{compactenum}
\end{Rmk}

\begin{Df}{Df 2\_P3.5.5 (gcd)}
    Let $R$ be a commutative ring. Suppose $\varnothing\neq X\subseteq R$, then $d\in R$ is said to be a \textbf{greatest common divisor} (or, \textbf{gcd} personally) of $X$ if
    \begin{compactenum}
        \item $d\mid x$ for all $x\in X$;
        \item $c\mid x$ for all $x\in X$ \quad $\implies$\quad $c\mid d$.
    \end{compactenum}
\end{Df}

\begin{Rmk}{}
    \begin{compactenum}
        \item \textcolor{Th}{Let $X$ be a non-empty subset of a commutative ring $R$. The gcd of $X$ does not always exist; even if it exists, it maybe not unique.} For example, in $2\mathbb{Z}$, $\{2\}$ does not even have a divisor; in $\mathbb{Z}$, $\{2,4\}$ has both $2$ and $-2$ as gcd's. \textcolor{Th}{In $\mathbb{Z}$, the gcd of $a_1,\cdots,a_n$ defined in Df \{course: 0, ID: 2.1.3\} is just the gcd defined here, but only considering the positive integers.}
        \item \textcolor{Th}{Let $X$ be a non-empty subset of a commutative ring $R$. If $d$ is a gcd of $X$, then $\{\text{gcd's of } X\} = \{\text{all associates of } d\}$.}
        \item \textcolor{Df}{Let $R$ be a commutative ring with identity. If $\{a_1,\cdots, a_n\}\subseteq R$ have $1$ as their gcd, then $a_1,\cdots, a_n$ are said to be \textbf{relatively prime} (or, \textbf{coprime} personally).}
    \end{compactenum}
\end{Rmk}

\begin{Th}{Th 2\_P3.5.5.1 (the existence of gcd and its form)}
    Let $R$ be a commutative ring with identity. For $a_1,\cdots,a_n\in R$,
    \begin{compactenum}
        \item $d$ is a gcd of $a_1,\cdots,a_n$ such that $d = r_1a_1+\cdots+r_na_n$ (for some $r_i\in R$) iff $(d) = (a_1)+\cdots+(a_n)$.
        \item If $R$ is a principal ideal ring, then the gcd $d$ of $a_1,\cdots,a_n$ always exist, and it must be of the form $d = r_1a_1+\cdots+r_na_n$.
        \item If $R$ is a UFD, then the gcd $d$ of $a_1,\cdots,a_n$ always exist.
    \end{compactenum}
    \tcblower
    \textit{Pf}:
    \begin{compactenum}
        \item Obvious.
        \item Obvious from the statement 1.
        \item Factorize the $a_i$'s, and it is clear that the product of the common irreducible divisors (along with the multiplicity) is a gcd of $a_1,\cdots,a_n$.
    \end{compactenum}
\end{Th}

\end{document}